\documentclass[conference]{IEEEtran}
\IEEEoverridecommandlockouts
% The preceding line is only needed to identify funding in the first footnote. If that is unneeded, please comment it out.
\usepackage{cite}
\usepackage{amsmath,amssymb,amsfonts}
\usepackage{algorithmic}
\usepackage{graphicx}
\usepackage{textcomp}
\usepackage{xcolor}
\def\BibTeX{{\rm B\kern-.05em{\sc i\kern-.025em b}\kern-.08em
    T\kern-.1667em\lower.7ex\hbox{E}\kern-.125emX}}
\begin{document}

\title{3DPredictor Web Interface Performance}

\author{
\IEEEauthorblockN{1\textsuperscript{st} Emil Valeev}
\IEEEauthorblockA{\textit{ICG SB RAS, Novosibirsk, Russia}\\
emil@bionet.nsc.ru}
\and
\IEEEauthorblockN{2\textsuperscript{nd} Polina Belokopytova}
\IEEEauthorblockA{\textit{ICG SB RAS, Novosibirsk, Russia}\\
belokopytova@bionet.nsc.ru}
\and
\IEEEauthorblockN{3\textsuperscript{rd} Veniamin Fishman}
\IEEEauthorblockA{\textit{ICG SB RAS, Novosibirsk, Russia}\\
\textit{NSU, Novosibirsk, Russia}\\
minja@bionet.nsc.ru}
}

\maketitle

\begin{abstract}
Abstract here.
\end{abstract}

\begin{IEEEkeywords}
Hi-C, chromatin organization
\end{IEEEkeywords}

\section{Motivation and Aim}

\subsection{Motivation}

Understand mechanisms behind chromatine 3D organization.

Conditions, resulting from chromosome rearrangements, may relate to changes in chromatine 3D organization.

% 1) понять, какие механизмы в основе 3д орг
% 2) патологии, связанные с хр перестройками, могут быть связаны с 3д -орг.

\subsection{Aim}

Earlier an algorythm was created.
There appeared a need in a quick and easy-to-use tool which could predict chromatine 3D organization using epigenetic data as input.

% Создать ресурс, позволяющий моделировать 3д орг генома на основе эпиген данных

\section{Methods}

For prediction we used algorythm detailed in \cite{b1}.

Datasets for model learning are from \cite{b2}.

For web service we used:
\begin{itemize}
\item Front-end: PureCSS framework
\item Back-end:
\begin{itemize}
\item Pipeline script: Bash
\item Preprocessing CTCF data: gimmemotifs\footnote{gimmemotifs by group of Simon van Heeringen, Radboud University. GitHub: vanheeringen-lab/gimmemotifs}
\item Hi-C map preparation: Juicer Tools\footnote{Juicer Tools by Aiden Lab, The Center for Genome Architecture, Baylor College of Medicine. Github: aidenlab/juicer}
\end{itemize}
\end{itemize}

% 1) Предсказания сделаны на основе алгоритма в статье Белокопытовой ет ал.
% 2) Датасеты были использованы из статьи MNA HiC (из статьи)
% 3) Описание проекта -- инструменты, хост етц.

\section{Results}

Prediction can be significantly improved if dataset has higher resolution (smaller bin size).

Also, we made a web service which performs prediction of chromatine structure for a region of interest.

% 1) использование в качестве датасета данных с более высоким разрешением (размер бина) улучшает предсказание
% 2) мы сделали веб-сервер, на котором есть модели, позволяющие предсказывать 3д орг интересующих районов генома.

\section*{Aknowledgement}

Supported by the RFBR (18-29-13021).

\begin{thebibliography}{00}
\bibitem{b1}
Polina S. Belokopytova, Miroslav A. Nuriddinov, Evgeniy A. Mozheiko, Daniil Fishman and Veniamin Fishman,
``Quantitative prediction of enhancer–promoter interactions'',
Genome Research, 2019.\\
doi: 10.1101/gr.249367.119
\bibitem{b2} Nils Krietenstein, Sameer Abraham, Sergey V. Venev, Nezar Abdennur, Johan Gibcus, Tsung-Han S. Hsieh, Krishna Mohan Parsi, Liyan Yang, René Maehr, Leonid A. Mirny, Job Dekker, Oliver J. Rando, ``Ultrastructural details of mammalian chromosome architecture'', 2019.\\ doi: 10.1101/639922 
\end{thebibliography}

\end{document}
