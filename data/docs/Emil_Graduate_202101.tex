\documentclass[a4paper,12pt]{article}
\usepackage{polyglossia} 
\setdefaultlanguage[babelshorthands=true]{russian}
\setotherlanguage{english}
\defaultfontfeatures{Ligatures=TeX,Mapping=tex-text}

\usepackage{amsmath}
\setmainfont{Liberation Serif}

\usepackage{xltxtra}
\usepackage{microtype}

\usepackage{xcolor}
\definecolor{darkblue}{HTML}{003153}
\usepackage{graphicx}
\usepackage {textcomp}
\usepackage[twoside,left=2.5cm,right=3cm,top=3cm,bottom=4cm,bindingoffset=0cm]{geometry}
\usepackage{epigraph}
\renewcommand{\epigraphsize}{\footnotesize}
\epigraphrule=0pt
\epigraphwidth=8cm
\usepackage{soul, xspace}
\xspaceaddexceptions{ < ) }
\sodef\so{}{.1em}{1em}{.3em plus.05em minus.05em}
\newcommand{\ldotst}{\so{...}\xspace}
\newcommand{\ldotsq}{\so{?\hbox{\hspace{-.212em}}..}\xspace}
\newcommand{\ldotse}{\so{!..}\xspace}
\newcommand{\razd}{~\\{\centering\Large\bfseries$\ast \ast \ast$\par}~\\}
\newcommand{\spacing}{\textcolor{red}{\textbf{(разрыв)}}}
\newcommand{\authornote}{\emph{"--- Прим. авт.}}

\renewcommand{\texttt}[1]{\text}

\usepackage{etoolbox}
\AtBeginEnvironment{quote}{}
\makeatletter
\newlength\episourceskip
\pretocmd{\@episource}{\em}{}{}
\apptocmd{\@episource}{\em}{}{}
\patchcmd{\epigraph}{\@episource{#1}\\}{\@episource{#1}\\[\episourceskip]}{}{}
\makeatother

\usepackage{hyperref}
\hypersetup{colorlinks=true, linkcolor=darkblue, citecolor=darkblue, filecolor=darkblue, urlcolor=darkblue}

\usepackage{fancyhdr}
\pagestyle{fancy}
\fancyhead[LE,RO]{\thepage}
\fancyhead[LO]{{\small
Дипломная работа
}}
\fancyhead[RE]{{\small
Эмиль Валеев
}} 
\fancyfoot{}
\fancypagestyle{plain}
{
\fancyhead{}
\renewcommand{\headrulewidth}{0mm}
\fancyfoot{}
}
\fancypagestyle{requisit}
{
\fancyhead{}
\renewcommand{\headrulewidth}{0mm}
}

\begin{document}

\begin{titlepage}

\begin{center}
 \begin{small}
ФЕДЕРАЛЬНОЕ ГОСУДАРСТВЕННОЕ АВТОНОМНОЕ ОБРАЗОВАТЕЛЬНОЕ УЧРЕЖДЕНИЕ ВЫСШЕГО ОБРАЗОВАНИЯ <<НОВОСИБИРСКИЙ НАЦИОНАЛЬНЫЙ ИССЛЕДОВАТЕЛЬСКИЙ ГОСУДАРСТВЕННЫЙ УНИВЕРСИТЕТ>> (НОВОСИБИРСКИЙ ГОСУДАРСТВЕННЫЙ УНИВЕРСИТЕТ, НГУ)

~

Институт медицины и психологии В. Зельмана НГУ
\end{small}

\vspace{5cm}

\begin{LARGE}{\textbf{КУРСОВАЯ РАБОТА}}\end{LARGE}

\vspace{0.5cm}

Валеев Эмиль Салаватович

Группа 12452

\vspace{0.5cm}

Тема работы: <<Разработка инструментов для поиска клинически значимых полиморфизмов в геноме человека на основе данных секвенирования 3C-библиотек>>

\end{center}

\vfill

\hfill
\begin{minipage}{0.47\textwidth}

\textbf{Научный руководитель:}

Фишман Вениамин Семенович,

к.б.н., ведущий научный сотрудник,

заведующий Сектором геномных

механизмов онтогенеза, ИЦиГ СО РАН

\vspace{1cm}

ФИО: \hrulefill/\hrulefill

<<\rule{2em}{0.4pt}>>\hrulefill20\rule{2em}{0.4pt} г.

\vspace{1cm}

Оценка: \hrulefill

\end{minipage}

\vfill

{\centering \begin{small}Новосибирск, 2020\end{small}\par}

\end{titlepage}

\tableofcontents

\section{Введение}

\subsection{Актуальность}

\subsection{Цели}

\subsection{Задачи}

\section{Обзор литературы}

Генетическое детерминирование патологий

Частые и редкие (орфанные) патологии

\subsection{Механизмы развития патологий}

\paragraph{Структура белка.}

\paragraph{Эпигенетика.}
Также патологии могут развиваться из-за изменения экспрессии, вызванных эпигенетическими механизмами, не затрагивающими непосредственно последовательность ДНК генов.
К таким механизмам можно отнести, например, метилирование ДНК, ацетилирование гистонов.
Кроме того, на экспрессию в значительной степени влияет трёхмерная структура хроматина, регулируемая механизмами loop extrusion, block copolymers, фазовая сепарация.
ТАДы, петли, етц.

\subsection{Типы генетических аномалий, лежащих в основе патологий}

\paragraph{Хромосомные аномалии.}
Анэуплоидии (изменение числа хромосом), перестройки (крупные делеции, дупликации, инверсии и транслокации).

\paragraph{Вариации числа копий (CNV).}

\paragraph{Точечные полиморфизмы (SNV).}

\paragraph{Короткие инверсии и делеции (indels).}

\subsection{Функциональные классы вариантов}

Внутригенные SNP могут находиться в:

\begin{itemize}
\item Нетранслируемых областях (3' и 5' UTR), вовлечённых в регуляцию транскрипции, трансляции и деградации транскрипта.
\item Экзонах, непосредственно отвечающих за последовательность белка.
SNP могут быть синонимичными (без замены аминокислоты) и несинонимичными "--- миссенс (замена на другую АК), нонсенс (замена на стоп-кодон) либо сдвиг рамки считывания, приводящий к изменению значительной части белковой молекулы.
\item Интронах, которые содержат регуляторные области и сплайс-сайты, необходимые для процессинга транскрипта в готовую мРНК.
\end{itemize}

Внегенные SNP могут приходиться на различные регуляторные последовательности, например, энхансеры, сайленсеры.
Также известно, что за трёхмерную структуру хроматина отвечают в том числе и специфические белки, связывающиеся с ДНК "--- например, CTCF\cite{wutz}.
Варианты, приходящиеся на сайты связывания CTCF, могут разрушать границы ТАДов и вызывать изменения экспрессии.

\subsection{Методы детектирования}

\paragraph{Кариотипирование.}

\paragraph{CGH.}

\paragraph{FISH.}

\paragraph{STS.}

\paragraph{MLPA.}

\paragraph{Секвенирование по Сэнгеру.}
Метод, позволяющий с высокой точностью анализировать короткий (до 1kb) фрагмент ДНК\cite{sanger}.
В настоящее время используется для подтверждения вариантов, найденных с помощью описанных выше методов.

\paragraph{Хромосомный микроматричный анализ (ХМА)}

\paragraph{Микрочиповая гибридизация.}
Гаплотипы.

\paragraph{NGS.}
Секвенирование нового поколения (NGS) "--- это комплекс технологий, позволяющих прочитать за сравнительно небольшое время миллионы коротких последовательностей ДНК.

Проблемы данных NGS:

Ошибки секвенирования.

Неоднородность покрытия генома.

Ошибки ПЦР и ПЦР-дубликаты

Неточное выравнивание инделов и повторяющихся последовательностей (например, поли-A трактов).

В настоящее время NGS используется и для выявления крупных перестроек (метод Hi-C).

\subsection{Виды NGS}

\paragraph{Полногеномное секвенирование (WGS).}

WGS также может быть использовано в диагностике микробиома с целью определения источника хронической инфекции, реконструкции путей передачи инфекции, а также выявления антибиотикорезистентных штаммов\cite{balloux}.

\paragraph{Полноэкзомное секвенирование (WES)}

\paragraph{Таргетные панели}

\paragraph{Hi-C}

\subsection{Базовая схема обработки результатов секвенирования}

\paragraph{Удаление адаптерных последовательностей.}
В процессе секвенирования к целевым фрагментам ДНК могут пришиваться так называемые адаптерные последовательности.
Если целевая ДНК короче длины прочтения, то фрагменты адаптера могут попасть в готовые данные, и встаёт вопрос об их удалении.
Также присутствие адаптера в прочтениях может быть признаком контаминации, и такие прочтения следует исключить из дальнейшего анализа\cite{cutadapt}.

\paragraph{Картирование.}
Прочтения необходимо картировать на некую референсную геномную последовательность.

Проблемы картирования:

\begin{itemize}
\item Высоковариативные регионы
\item Вырожденные (неуникальные) регионы
\item Регионы с инделами
\end{itemize}

Отличие генома образца от референсного генома называется вариантом (синонимичные термины <<мутация>> и <<полиморфизм>> не рекомендованы к употреблению\cite{richards}).
В настоящее время <<золотым стандартом>> являются утилиты, использующие алгоритм Берроуса--Уиллера\cite{burrows}.

\paragraph{Удаление ПЦР-дубликатов.}
Тем не менее, было показано, что для WGS-данных удаление ПЦР-дубликатов имеет минимальный эффект на улучшение поиска полиморфизмов "--- приблизительно 92\% из более чем 17 млн вариантов были найдены вне зависимости от наличия этапа удаления дубликатов и использованных инструментов\cite{ebbert}.
Учитывая, что удаление ПЦР-дубликатов может занимать значительную часть потраченного на обработку данных времени и ресурсов компьютера, следует взвесить пользу и затраты данного этапа для конкретной прикладной задачи.

\paragraph{Рекалибровка качества прочтений (BQSR).}
Приборная оценка качества оснований не соответствует эмпирической.
Первоочередное влияние на эту разницу оказывают цикл секвенирования и нуклеотидный контекст.
Решением является рекалибровка качества, исходя из известных паттернов ковариации.

\paragraph{Поиск точечных полиморфизмов.}

\subsection{Аннотация, фильтрация и интерпретация результатов}

\paragraph{Номер экзона, функциональный класс варианта.}

\paragraph{Частота аллеля по основным базам данных.}
Несмотря на то, что были созданы базы данных для всех рас, очень часто этого недостаточно и необходимо учитывать частоты в популяциях отдельных народов и стран.
Такими базами данных являются GME\cite{gme}, в которой отражены частоты по популяции Ближнего Востока, ABraOM\cite{abraom}, предоставляющая частоты вариантов среди практически здорового пожилого населения Бразилии.
Также для анализа берутся популяции, в которых велика доля близкородственных связей, например, пакистанская\cite{saleheen}.


\paragraph{Loss-of-function.}
Различные показатели, отражающие устойчивость функции гена, основанные на данных о стоп-кодонах, сдвигах рамки считывания и сплайс-вариантах (pLi).

Основные проблемы pLI:

\begin{itemize}
\item Плохо приспособлен к распознаванию AR вариантов (из-за того, что частота повреждающих вариантов в популяции может быть высокой) и XR вариантов (из-за наличия в популяции здоровых гетерозиготных носителей).
\item Плохо приспособлен к распознаванию вариантов в генах, ответственных за патологии, не влияющие на взросление и воспроизводство.
Их частота в популяции также может быть высокой.
К таким относятся варианты в генах BRCA1-2.
\item Сплайс-варианты рассматриваются как повреждающие, несмотря на то, что вариант в сплайс-сайте может не иметь эффекта на сплайсинг, либо приводить к появлению изоформы белка без потери функции.
\item Высокая частота распространения заболевания в контрольной группе.
Пример "--- шизофрения.
\item К миссенс-вариантам pLI применять следует с осторожностью, и без функциональной пробы следует исключить из анализа.
\item Также следует отнестись с осторожностью к нонсенс-вариантам и сдвигам рамки считывания в последнем экзоне либо в C-терминальной части предпоследнего.
Такие транскрипты избегают нонсенс-индуцированного разложения РНК и могут в результате как не привести к каким-либо функциональным изменениям, так и привести к образованию мутантного белка, обладающего меньшей активностью по сравнению с исходным, либо токсичного для клетки.
\item В некоторых случаях соотношение pLI с гаплонедостаточностью конкретного гена в принципе сложно объяснить\cite{ziegler}.
\end{itemize}

Таким образом, высокое значение pLI можно считать хорошим показателем LoF, низкое "--- с осторожностью.

\paragraph{Анализ и предсказание функционального эффекта in silico.}

\paragraph{Клинические данные из бд и статей.}

\paragraph{Семейный анализ, анализ de novo вариантов.}

\subsection{Когортный анализ}

Помимо того, что когортный анализ необходим для получения информации о частоте аллеля, существует необходимость детекции систематических отклонений покрытия и артефактов выравнивания, связанных с конкретными районами генома и/или особенностями приготовления библиотек. Также анализ нескольких родственных образцов помогает определить зиготность варианта либо импутировать район с недостаточным покрытием.

\subsection{Случайные находки}

\subsection{Exo-C: суть метода}

\section{Материалы и методы}

Данные секвенирования клеточной линии K562 (Hi-C\cite{rao}, WGS\cite{zhou}) были взяты из публичных источников

Контроль качества "--- FastQC\cite{fastqc}.

Удаление адаптерных последовательностей производилось с помощью cutadapt\cite{cutadapt}.

Для картирования был взят геном GRCh37/hg19.
Из него были удалены так называемые неканоничные хромосомы, что позволило улучшить качество выравнивания и значительно упростить работу с готовыми данными.

Картирование производилось с помощью инструментов Bowtie2\cite{bowtie2} и BWA\cite{bwa}.
BWA показал лучшие показатели;
кроме того, он значительно лучше работает с химерными ридами, что немаловажно для метода Exo-C.

Сбор статистики производился с помощью samtools flagstat.

Так как мы использовали данные экзомного секвенирования, а количество образцов у нас было относительно небольшим и мы были заинтересованы в максимально качественной подготовке данных, в пайплайн был включён этап удаления ПЦР-дубликатов.
Удаление дубликатов "--- MarkDuplicates от Picard\cite{picard}, интегрированный в GATK.
Оптимальные показатели скорости MarkDuplicates достигаются при запуске Java с параллелизацией сборщиков мусора и количеством сборщиков мусора равным двум\cite{heldenbrand}.

Рекалибровка qual'ов

Для обучения модели требуются вариации в VCF формате (для человеческого генома - dbSNP >132). Нативная база данных с NCBI требует перепарсинг - другие контиги, а также удаление точек в Ref/Alt. Обжать базу нужно bgzip.

Далее выполняется индексирование (и одновременно проверка на пригодность).

Рекалибровка.
В \cite{heldenbrand} было показано, что оптимальные показатели скорости BaseRecalibrator достигаются, как и в случае с MarkDuplicates, запуском Java с двумя параллельными сборщиками мусора;
кроме того, BaseRecalibrator поддаётся внешнему распараллеливанию путём разделения картированных ридов на хромосомные группы.
Хромосомные группы формировались вручную для используемой сборки генома, каждая запускалась с помощью bash-скрипта.
Нам удалось усовершенствовать данный этап "--- запуск BaseRecalibrator производился с помощью библиотек Python3 subprocess, а параллелизация осуществлялась библиотекой multiprocessing, таким образом, можно было делить файл с картированными прочтениями по хромосомам и обрабатывать их отдельно, так как multiprocessing автоматически распределяет процессы по имеющимся потокам.
Всего для генома GRCh37/hg19 удалось достичь максимально возможное ускорение "--- в 10 раз (по сравнению с запуском на одном потоке).

Покрытие и обогащение в экзоме оценивалось с помощью скрипта на основе bedtools\cite{bedtools}.

Поиск вариантов производился с помощью GATK HaplotypeCaller.
Как и в случае с BaseRecalibrator, HaplotypeCaller поддаётся внешнему распараллеливанию\cite{heldenbrand}.
Мы также осуществили параллелизацию с помощью сочетания subprocess и multiprocessing, достигнув 10-12-кратного ускорения по сравнению с запуском на одном потоке.

Аннотация вариантов производилась вначале с помощью инструмента Ensembl VEP\cite{vep}, затем мы мигрировали на ANNOVAR\cite{annovar}.

Используемые базы данных:

\begin{enumerate}
\item Human Gene Mutation Database (HGMD\textregistered)\cite{hgmd}
\item Online Mendelian Inheritance in Man (OMIM\textregistered)\cite{omim}
\item GeneCards\textregistered: The Human Gene Database --- \href{https://www.genecards.org/}{https://www.genecards.org/}
\item ClinVar --- \href{https://www.ncbi.nlm.nih.gov/clinvar/}{https://www.ncbi.nlm.nih.gov/clinvar/}
\item dbSNP --- \href{https://www.ncbi.nlm.nih.gov/snp/}{https://www.ncbi.nlm.nih.gov/snp/}
\item Genome Aggregation Database (gnomAD)\cite{gnomad}
\item 1000 Genomes Project --- \href{https://www.internationalgenome.org/}{https://www.internationalgenome.org/}
\item Great Middle East allele frequencies (GME)\cite{gme}
\item dbNSFP: Exome Predictions\cite{dbnsfp}
\item dbscSNV: Splice site prediction\cite{dbscsnv}
\item RegSNPIntron: intronic SNVs prediction\cite{regsnpintron}
\end{enumerate}

Интерпретация данных и составление отчёта производилось в соответствии с рекомендациями Американского колледжа медицинской генетики и геномики (ACMG) и Ассоциации молекулярной патологии\cite{richards}.

Пограничным значением pLI было взято 0.9, согласно рекомендациям в оригинальной статье\cite{lek}.

\section{Результаты}

\section{Обсуждение результатов}

\section{Предварительные выводы}

\begin{thebibliography}{100}

\bibitem{balloux}
Balloux F, Brønstad Brynildsrud O, van Dorp L, et al. From Theory to Practice: Translating Whole-Genome Sequencing (WGS) into the Clinic. Trends Microbiol. 2018;26(12):1035-1048. doi:10.1016/j.tim.2018.08.004

\bibitem{heldenbrand}
Heldenbrand JR, Baheti S, Bockol MA, et al. Recommendations for performance optimizations when using GATK3.8 and GATK4 [published correction appears in BMC Bioinformatics. 2019 Dec 17;20(1):722]. BMC Bioinformatics. 2019;20(1):557. Published 2019 Nov 8. doi:10.1186/s12859-019-3169-7

\bibitem{regsnpintron}
Lin, H., Hargreaves, K.A., Li, R. et al. RegSNPs-intron: a computational framework for predicting pathogenic impact of intronic single nucleotide variants. Genome Biol 20, 254 (2019). doi:10.1186/s13059-019-1847-4

\bibitem{dbnsfp}
Liu X, Wu C, Li C, Boerwinkle E. dbNSFP v3.0: A One-Stop Database of Functional Predictions and Annotations for Human Nonsynonymous and Splice-Site SNVs. Hum Mutat. 2016;37(3):235-241. doi:10.1002/humu.22932

\bibitem{gme}
Scott EM, Halees A, Itan Y, et al. Characterization of Greater Middle Eastern genetic variation for enhanced disease gene discovery. Nat Genet. 2016;48(9):1071-1076. doi:10.1038/ng.3592

\bibitem{gnomad}
Karczewski, K.J., Francioli, L.C., Tiao, G. et al. The mutational constraint spectrum quantified from variation in 141,456 humans. Nature 581, 434–443 (2020). doi:10.1038/s41586-020-2308-7

\bibitem{hgmd}
Stenson PD, Mort M, Ball EV, et al. The Human Gene Mutation Database: towards a comprehensive repository of inherited mutation data for medical research, genetic diagnosis and next-generation sequencing studies. Hum Genet. 2017;136(6):665-677. doi:10.1007/s00439-017-1779-6

\bibitem{omim}
Amberger JS, Bocchini CA, Schiettecatte F, Scott AF, Hamosh A. OMIM.org: Online Mendelian Inheritance in Man (OMIM®), an online catalog of human genes and genetic disorders. Nucleic Acids Res. 2015;43(Database issue):D789-D798. doi:10.1093/nar/gku1205

\bibitem{dbscsnv}
Jian X, Boerwinkle E, Liu X. In silico tools for splicing defect prediction: a survey from the viewpoint of end users. Genet Med. 2014;16(7):497-503. doi:10.1038/gim.2013.176

\bibitem{annovar}
Wang K, Li M, Hakonarson H. ANNOVAR: functional annotation of genetic variants from high-throughput sequencing data. Nucleic Acids Res. 2010;38(16):e164. doi:10.1093/nar/gkq603

\bibitem{vep}
McLaren, W., Gil, L., Hunt, S.E. et al. The Ensembl Variant Effect Predictor. Genome Biol 17, 122 (2016). doi: 10.1186/s13059-016-0974-4

\bibitem{cutadapt}
MARTIN, Marcel. Cutadapt removes adapter sequences from high-throughput sequencing reads. EMBnet.journal, [S.l.], v. 17, n. 1, p. pp. 10-12, may 2011. ISSN 2226-6089. doi: 10.14806/ej.17.1.200. 

\bibitem{fastqc}
Andrews, S. (2010). FastQC:  A Quality Control Tool for High Throughput Sequence Data [Online]. Available online at: http://www.bioinformatics.babraham.ac.uk/projects/fastqc/

\bibitem{bowtie2}
Langmead, B., Salzberg, S. Fast gapped-read alignment with Bowtie 2. Nat Methods 9, 357–359 (2012). doi: 10.1038/nmeth.1923

\bibitem{bwa}
Li H, Durbin R. Fast and accurate short read alignment with Burrows-Wheeler transform. Bioinformatics. 2009;25(14):1754-1760. doi:10.1093/bioinformatics/btp324

\bibitem{picard}
''Picard Toolkit.'' 2019. Broad Institute, GitHub Repository. http://broadinstitute.github.io/picard/; Broad Institute

\bibitem{bedtools}
Quinlan AR and Hall IM, 2010. BEDTools: a flexible suite of utilities for comparing genomic features. Bioinformatics. 26, 6, pp. 841–842.

\bibitem{rao}
Rao SS, Huntley MH, Durand NC, et al. A 3D map of the human genome at kilobase resolution reveals principles of chromatin looping [published correction appears in Cell. 2015 Jul 30;162(3):687-8]. Cell. 2014;159(7):1665-1680. doi:10.1016/j.cell.2014.11.021

\bibitem{zhou}
Zhou B, Ho SS, Greer SU, et al. Comprehensive, integrated, and phased whole-genome analysis of the primary ENCODE cell line K562. Genome Res. 2019;29(3):472-484. doi:10.1101/gr.234948.118

\bibitem{sanger}
Sanger F, Nicklen S, Coulson AR. DNA sequencing with chain-terminating inhibitors. Proc Natl Acad Sci U S A. 1977;74(12):5463-5467. doi:10.1073/pnas.74.12.5463

\bibitem{burrows}
Burrows M, Wheeler DJ. Technical report 124. Palo Alto, CA: Digital Equipment Corporation; 1994. A block-sorting lossless data compression algorithm.

\bibitem{ebbert}
Ebbert MT, Wadsworth ME, Staley LA, et al. Evaluating the necessity of PCR duplicate removal from next-generation sequencing data and a comparison of approaches. BMC Bioinformatics. 2016;17 Suppl 7(Suppl 7):239. Published 2016 Jul 25. doi:10.1186/s12859-016-1097-3

\bibitem{richards}
Richards S, Aziz N, Bale S, et al. Standards and guidelines for the interpretation of sequence variants: a joint consensus recommendation of the American College of Medical Genetics and Genomics and the Association for Molecular Pathology. Genet Med. 2015;17(5):405-424. doi:10.1038/gim.2015.30

\bibitem{lek}
Lek M, Karczewski KJ, Minikel EV, et al. Analysis of protein-coding genetic variation in 60,706 humans. Nature. 2016;536(7616):285-291. doi:10.1038/nature19057

\bibitem{abraom}
Naslavsky MS, Yamamoto GL, de Almeida TF, Ezquina SAM, Sunaga DY, Pho N, Bozoklian D, Sandberg TOM, Brito LA, Lazar M, Bernardo DV, Amaro E Jr, Duarte YAO, Lebrão ML, Passos-Bueno MR, Zatz M. Exomic variants of an elderly cohort of Brazilians in the ABraOM database. Hum Mutat. 2017 Jul;38(7):751-763. doi: 10.1002/humu.23220.

\bibitem{ziegler}
Ziegler A, Colin E, Goudenège D, Bonneau D. A snapshot of some pLI score pitfalls. Hum Mutat. 2019 Jul;40(7):839-841. doi: 10.1002/humu.23763

\bibitem{saleheen}
Saleheen D, Natarajan P, Armean IM, et al. Human knockouts and phenotypic analysis in a cohort with a high rate of consanguinity. Nature. 2017;544(7649):235-239. doi:10.1038/nature22034

\bibitem{wutz}
Wutz G, Várnai C, Nagasaka K, et al. Topologically associating domains and chromatin loops depend on cohesin and are regulated by CTCF, WAPL, and PDS5 proteins. EMBO J. 2017;36(24):3573-3599. doi:10.15252/embj.201798004
\end{thebibliography}

\end{document}
