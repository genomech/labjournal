\documentclass[a4paper,12pt]{article}
\usepackage{polyglossia} 
\setdefaultlanguage[babelshorthands=true]{russian}
\setotherlanguage{english}
\defaultfontfeatures{Ligatures=TeX,Mapping=tex-text}

\usepackage{amsmath}
\setmainfont{Liberation Serif}

\newsavebox{\uniquecontrols}

\usepackage{calc}
\usepackage{float}
\usepackage[square,sort,comma,numbers]{natbib}
\bibliographystyle{plainnat}
\usepackage{xltxtra}
\usepackage{microtype}
\usepackage{multirow,tabularx}

\usepackage[table]{xcolor}
\definecolor{darkblue}{HTML}{003153}
\definecolor{grayhead}{HTML}{A5A5A5}
\definecolor{grayrow}{HTML}{D1D1D1}
\usepackage{graphicx}
\usepackage {textcomp}
\usepackage[twoside,left=2.5cm,right=3cm,top=3cm,bottom=4cm,bindingoffset=0cm]{geometry}
\usepackage{epigraph}
\renewcommand{\epigraphsize}{\footnotesize}
\epigraphrule=0pt
\epigraphwidth=8cm
\usepackage{soul, xspace}
\xspaceaddexceptions{ < ) }
\sodef\so{}{.1em}{1em}{.3em plus.05em minus.05em}
\newcommand{\ldotst}{\so{...}\xspace}
\newcommand{\ldotsq}{\so{?\hbox{\hspace{-.212em}}..}\xspace}
\newcommand{\ldotse}{\so{!..}\xspace}
\newcommand{\razd}{~\\{\centering\Large\bfseries$\ast \ast \ast$\par}~\\}
\newcommand{\spacing}{\textcolor{red}{\textbf{(разрыв)}}}
\newcommand{\authornote}{\emph{"--- Прим. авт.}}

\newcommand{\boldbigrow}[2]{\parbox[c][1.3cm]{\widthof{\textbf{#1}}}{\textbf{#2}}}
\newcommand{\bigrow}[2]{\parbox[c][1.3cm]{\widthof{#1}}{#2}}

\renewcommand{\texttt}[1]{\text}

\usepackage{etoolbox}
\AtBeginEnvironment{quote}{}
\makeatletter
\newlength\episourceskip
\pretocmd{\@episource}{\em}{}{}
\apptocmd{\@episource}{\em}{}{}
\patchcmd{\epigraph}{\@episource{#1}\\}{\@episource{#1}\\[\episourceskip]}{}{}
\makeatother

\usepackage{hyperref}
\hypersetup{colorlinks=true, linkcolor=darkblue, citecolor=darkblue, filecolor=darkblue, urlcolor=darkblue}

\usepackage{fancyhdr}
\pagestyle{fancy}
\fancyhead[LE,RO]{\thepage}
\fancyhead[LO]{{\small
Дипломная работа
}}
\fancyhead[RE]{{\small
Эмиль Валеев
}} 
\fancyfoot{}
\fancypagestyle{plain}
{
\fancyhead{}
\renewcommand{\headrulewidth}{0mm}
\fancyfoot{}
}
\fancypagestyle{requisit}
{
\fancyhead{}
\renewcommand{\headrulewidth}{0mm}
}

\newenvironment{mytable}[2]
{\begin{table}[H]
\caption{#1}
\label{#2}\vspace{0.5em}
\setlength\arrayrulewidth{1pt}
\begin{lrbox}{\uniquecontrols}
\bgroup
\def\arraystretch{1.5}
\rowcolors{2}{grayrow}{white}}
{\egroup
\end{lrbox}
\resizebox{\textwidth}{!}{\usebox{\uniquecontrols}}
\end{table}}

\begin{document}

\begin{titlepage}

\begin{center}
 \begin{small}
ФЕДЕРАЛЬНОЕ ГОСУДАРСТВЕННОЕ АВТОНОМНОЕ ОБРАЗОВАТЕЛЬНОЕ УЧРЕЖДЕНИЕ ВЫСШЕГО ОБРАЗОВАНИЯ <<НОВОСИБИРСКИЙ НАЦИОНАЛЬНЫЙ ИССЛЕДОВАТЕЛЬСКИЙ ГОСУДАРСТВЕННЫЙ УНИВЕРСИТЕТ>> (НОВОСИБИРСКИЙ ГОСУДАРСТВЕННЫЙ УНИВЕРСИТЕТ, НГУ)

~

Институт медицины и психологии В. Зельмана НГУ
\end{small}

\vspace{5cm}

\begin{LARGE}{\textbf{КУРСОВАЯ РАБОТА}}\end{LARGE}

\vspace{0.5cm}

Валеев Эмиль Салаватович

Группа 12452

\vspace{0.5cm}

Тема работы: <<Разработка инструментов для поиска клинически значимых полиморфизмов в геноме человека на основе данных секвенирования 3C-библиотек>>

\end{center}

\vfill

\hfill
\begin{minipage}{0.47\textwidth}

\textbf{Научный руководитель:}

Фишман Вениамин Семенович,

к.б.н., ведущий научный сотрудник,

заведующий Сектором геномных

механизмов онтогенеза, ИЦиГ СО РАН

\vspace{1cm}

ФИО: \hrulefill/\hrulefill

<<\rule{2em}{0.4pt}>>\hrulefill20\rule{2em}{0.4pt} г.

\vspace{1cm}

Оценка: \hrulefill

\end{minipage}

\vfill

{\centering \begin{small}Новосибирск, 2020\end{small}\par}

\end{titlepage}

\tableofcontents

\section*{Список сокращений}

\begin{description}
\item[3C] (Chromosome Conformation Capture) "--- определение конформации хромосом
\item[BAM]
\item[BQSR] (Base Quality Score Recalibration) "---
\item[cfDNA] (Circulating Free DNA) "--- 
\item[cffDNA] (Cell-Free Fetal DNA) "---
\item[CGH] (Comparative Genomic Hybridization) "--- сравнительная геномная гибридизация
\item[CNV] (Copy Number Variation) "--- вариация числа копий
\item[Exo-C]
\item[FISH] (Fluorescence In Situ Hybridization) "--- флуоресцентная \textit{in situ} гибридизация
\item[GATK] (Genome Analysis ToolKit)
\item[Hi-C] (all-vs-all chromosome conformation capture) "--- 
\item[LoF] (Loss of Function) "---
\item[MAPQ] (MAPping Quality) "--- качество картирования
молекулярно импринтированных полимеров (MIP) 
\item[MLPA] (Multiplex Ligation-dependent Probe Amplification) "--- мультиплексная лигат-зависимая амплификация зонда
\item[NGS] (New Generation Sequencing) "--- секвенирование нового поколения
\item[NIPT] (Non-Invasive Prenatal Testing) "---
\item[NOR] (Nucleolus Organizer Region) "--- ядрышковый организатор
захвата с помощью расширения праймера (PEC)
\item[RG] (Read Group)
анализа транскриптома одной клетки (SMART)
\item[SNV] (Single Nucleotide Variant)
\item[ТАД] (Topologically Associated Domain) "--- топологически ассоциированные домены
\item[UTR] (UnTranslated Regions) "--- нетранслируемая область
\item[VCF] (Variant Call Format) "---
\item[WES] (Whole Exome Sequencing) "--- полноэкзомное секвенирование
\item[WGS] (Whole Genome Sequencing) "--- полногеномное секвенирование
\item[ДНК] "--- дезоксирибонуклеиновая кислота
\item[ПЦР] "--- полимеразная цепная реакция
\item[РНК] "--- рибонуклеиновая кислота
\item[ХМА] "--- хромосомный микроматричный анализ
\end{description}

\section{Введение}

\subsection{Актуальность}

\subsection{Цели}

\subsection{Задачи}

\section{Обзор литературы}

% Генетика играет роль во всех патологических состояниях человека, в большей или меньшей степени.
Генетические варианты, их взаимодействие друг с другом и со средой определяет течение болезней.
Существуют генетические варианты, которые определяют предрасположенность и проявляются только во взаимодействии со средой; примером могут служить варианты, определяющие предрасположенность к аддикциям (никотин, героин, алкоголь и пр.)\cite{hiroi}.
Бывают и такие генетические варианты, которые повышают восприимчивость к одному фактору среды и повышают устойчивость к другому, либо дают позитивный эффект в сочетании и негативный по отдельности.
Примером может служить бета-талассемия\cite{galanello}.
Особняком стоят те варианты, которые вне зависимости от средового компонента и генетического окружения приводят к развитию заболевания (например, нейрофиброматоз I типа, который наследуется по аутосомно-доминантному типу и имеет 100\% пенетрантность\cite{jett}).

Генетические заболевания остаются одной из основных причин младенческой и детской смертности в развитых странах.
От врождённых аномалий в младенческом возрасте умирают около 20\% от числа всех случаев смертности, а также порядка 10\% детей в возрасте 1--4 года и 6\% детей в возрасте 5--9 лет.
Злокачественные новообразования являются причиной смерти 8\% детей в возрасте 1--4 лет, 15\% детей в возрасте 5--9 лет.
Порядка 3\% детей в возрасте 1--9 лет умирают от сердечных патологий\cite{field}.
Взрослые люди с генетическими патологиями требуют огромных затрат средств "--- на радикальные и паллиативные операции, медикаментозную поддержку (иногда пожизненную), создание условий, учреждений и обучение персонала для обеспечения специализированного ухода.

Таким образом, доступные и точные методы диагностики генетических заболеваний могут помочь в сокращении заболеваемости и смертности, а также повысить экономическое благополучие населения.

\paragraph{Частые и редкие (орфанные) патологии.}
Генетические патологии делятся на группы по частоте встречаемости в популяции.
Выделяют частые и редкие (орфанные) заболевания.
Определения орфанных заболеваний могут различаться "--- например, в США, согласно ``Health Promotion and Disease Prevention Amendments of 1984'', редкими считаются патологии, поражающие менее 200 тыс. населения страны (примерно 1 : 1630 при текущей численности населения в 326 млн человек)\cite{herder}.
Европейское Медицинское Агентство определяет границу как 1 : 2000.
Систематический анализ показал, что существует более 290 определений, и среднее значение находится в интервале 40--50 на 100 тыс. населения\cite{richter}.

Также сложности в определении орфанных заболеваний представляет неравномерность их распространённости в тех или иных регионах.
Некоторые заболевания могут быть орфанными в одной популяции и частыми в другой (эффект основателя, а также сверхдоминирование).
Частным случаем эффекта основателя является атаксия Каймановых островов, связанная с гипоплазией мозжечка и сопутствующими неврологическими проявлениями (задержка развития, дизартрия, нистагм, интенционное дрожание).
Это аутосомно-рецессивное заболевание распространено исключительно в одном регионе "--- Большой Кайманов остров, гетерозиготные носители составляют около 18\% местного населения\cite{bomar}.
Примером позитивного отбора является бета-талассемия "--- заболевание, связанное с нарушением структуры гемоглобина.
Несмотря на то, что у эритроцитов носителей в значительной степени снижена способность переносить кислород, дефектный гемоглобин представляет сложность для развития малярийного плазмодия и таким образом повышает устойчивость носителя бета-талассемии к малярии\cite{galanello}.
Соответственно, бета-талассемия распространена в эпидемически опасных по малярии регионах "--- Средиземноморье и Юго-Восточная Азия, наибольшая частота встречаемости наблюдается на Кипре (14\%) и Сардинии (10,3\%) при средней частоте по земному шару в 1,5\%.

Несмотря на то, что каждое из орфанных заболеваний само по себе встречается редко, в сумме они поражают значительный процент населения (предположительно 5--8\% европейской популяции).
Общее число орфанных болезней неизвестно по причине недостатков стандартизации, наиболее частая оценка "--- 5000--8000\cite{lancet}.
Существуют различные базы данных, собирающие информацию по орфанным заболеваниям, наиболее известными и часто используемыми из них являются:

\begin{enumerate}
\item Global Genes;
\item Online Mendelian Inheritance in Man (OMIM)\cite{omim};
\item Orphadata.
\end{enumerate}

Около 80\% редких болезней имеют генетическую природу и начинаются в раннем детстве\cite{lancet}.
Таким образом, ключевым моментом для изучения данных заболеваний является понимание механизмов, лежащих в основе их развития.
Количество орфанных заболеваний делает эту задачу крайне непростой.
Тем не менее, многие механизмы на сегодняшний момент достаточно хорошо изучены.
О них речь пойдёт далее.

\subsection{Механизмы развития генетических патологий}

Механизмы развития генетических патологий делятся на две большие группы.
В первую относят изменения белок-кодирующей последовательности гена, приводящие к прекращению синтеза белка либо к синтезу изменённого полипептида.
Ко второй группе относятся эпигенетические механизмы, не затрагивающие непосредственно белок-кодирующие последовательности генов.

% \paragraph{Изменения белок-кодирующей последовательности гена.}
Изменения белок-кодирующей последовательности гена (экзонов и сплайсинг-сайтов) могут приводить к замене аминокислот, сдвигам рамки считывания, появлению преждевременных стоп-кодонов и нарушениям сплайсинга.
Прекращение синтеза белка снижает дозу гена, а изменённый полипептид способен как потерять свою функцию, снизив таким образом дозу гена, так и приобрести новые свойства (токсичность).
Классическим примером приобретения белком токсичности являются некоторые наследственные нейродегенеративные заболевания "--- в частности, аутосомно-доминантный вариант болезни Альцгеймера.
Другое нейродегенеративное заболевание "--- аутосомно-рецессивная болезнь Паркинсона "--- может служить примером потери белком протективной функции\cite{winklhofer}.

% \paragraph{Эпигенетика.}
Также генетические патологии могут развиваться из-за эпигенетических механизмов, приводящих к изменению экспрессии генов.
К таким механизмам можно отнести метилирование ДНК "--- изменение молекулы ДНК без изменения нуклеотидной последовательности, а также ацетилирование гистонов.

В частности, нарушение метилирования ДНК ответственно за развитие синдрома Беквита---Видемана.
Экспрессия генов CDKN1C и IGF2 регулируется в зависимости от того, на материнской или отцовской хромосоме они находятся (явление геномного импринтинга).
Потеря импринтинга, вызванная изменениями регуляторного района, ведёт к изменению экспрессии этих генов и, как следствие, к тяжёлым порокам развития, включающим висцеромегалию, виcцеральные грыжи, эмбриональные опухоли, пороки сердца и почек\cite{jin}.
Изменение ацетилирования гистонов некоторых генов в клетках головного мозга связано с развитием такого заболевания, как шизофрения\cite{tang}.

Кроме того, на экспрессию в значительной степени влияет трёхмерная структура хроматина.
К примеру, энхансерный район не обязательно находится в непосредственной близости от гена, для его работы необходим физический контакт с промотором гена за счёт выпетливания ДНК.
Белковый комплекс, связанный с энхансером, привлекает в эту область РНК-полимеразу и увеличивает вероятность её связывания с промотором.
Известно, что большая часть промотор-энхансерных взаимодействий находится внутри топологически ассоциированных доменов (ТАД).
В результате разрушения старых или образования новых границ ТАД формируются структурные варианты, характеризующиеся иными промотор-энхансерными взаимодействиями.
Это является причиной таких состояний, как FtM-инверсия пола (ген SOX9) и синдром Кукса (ген KCNJ2)\cite{spielmann}.

В основе как генетических, так и эпигенетических механизмов чаще всего\footnote{Почему <<чаще всего>>, а не всегда?} лежат изменения тех или иных участков ДНК.
Эти варианты существенно различаются по размеру, характеру изменения, а также функциональному значению, которое напрямую зависит от затрагиваемых вариантом районов генома.

\subsection{Типы генетических аномалий, лежащих в основе генетических патологий}

Самыми крупными являются хромосомные перестройки.
Они делятся на две основных группы "--- сбалансированные (без изменения количества генетической информации) и несбалансированные (с изменением количества генетической информации).

Несбалансированные перестройки в большинстве своём приводят к летальному исходу (в эмбриональном или детском периодах) и грубым изменениям фенотипа.
К несбалансированным относятся:

\begin{itemize}
\item Анэуплоидии "--- изменение числа хромосом.
Примерами анэуплоидий могут служить синдром Дауна (трисомия 21 хромосомы), Эдвардса (трисомия 18 хромосомы), Патау (трисомия 13 хромосомы), а также вариации числа половых хромосом (синдромы Тёрнера, Клайнфельтера и другие).
Частичная моносомия "--- синдром кошачьего крика (связан с утратой плеча 5 хромосомы).
Прочие анэуплоидии ведут к несовместимым с жизнью нарушениям эмбрионального развития и, как следствие, спонтанным абортам.
\item Несбалансированные транслокации "--- перемещение фрагмента хромосомы с одного места на другое с изменением количества генетической информации.
Несбалансированные транслокации могут приводить к значимым изменениям фенотипа (например, 46,XX инверсия пола\cite{rizvi}) и служить онкогенами\cite{o-connor}.
\item Вариации числа копий (CNV) "--- дупликации (мультипликации) и делеции хромосомных сегментов размером от тысячи до нескольких миллионов пар оснований.
Могут возникнуть из несбалансированных транслокаций, амплификаций и собственно делеций.
CNV способны увеличивать или уменьшать дозу гена, в значительной степени влияя на его экспрессию.
Различия в количестве копий могут носить как положительный характер, так и отрицательный "--- в частности, дупликации в гене CCL3L1 способны увеличить устойчивость к ВИЧ\cite{gonzalez}, а крупные CNV в разных частях генома ассоциированы с расстройствами аутического спектра\cite{sebat}.
\end{itemize}

Сбалансированные перестройки чаще всего характеризуются более мягкими фенотипическими проявлениями, а иногда и их отсутствием.
К сбалансированным перестройкам относятся

\begin{itemize}
\item Инверсии "--- переворот фрагмента хромосомы.
Крупные инверсии могут быть причиной изменения границы ТАД, а также запирания кроссинговера и образования гаплогрупп.
\item Сбалансированные транслокации "--- перемещение фрагмента хромосомы с одного места на другое без изменения количества генетической информации.
В свою очередь они делятся на реципрокные (взаимный обмен участками между негомологичными хромосомами) и Робертсоновские (слияние акроцентрических хромосом с образованием метацентрической или субметацентрической).
Сбалансированные транслокации могут как не проявляться в фенотипе (сказываясь только на фертильности\cite{dong}), так и приводить к серьёзным последствиям "--- например, синдрому Дауна (робертсоновская транслокация является причиной синдрома Дауна в 2--4\% случаев\cite{asim}).
\end{itemize}

Самыми небольшими "--- но не менее важными "--- являются точечные полиморфизмы (SNV) и короткие инсерции и делеции (indels) размером 20--50bp.
Чаще всего эти генетические варианты нейтральные и не имеют фенотипических проявлений, но некоторые могут приводить как к генетическим, так и к эпигенетическим изменениям.
Также варианты делятся на наследуемые, которые передаются от родителей к детям, и варианты de novo.
Согласно оценкам, предоставленным \cite{acuna-higaldo}, в среднем в каждом поколении у человека возникают 44--82 SNV de novo, из них 1--2 приходятся на белок-кодирующие регионы.
Число небольших инсерций и делеций оценивается в 2.9--9 на геном, крупные перестройки встречаются значительно реже.
Также известно, что количество генетических вариантов de novo непрерывно растёт в течение жизни человека.

\subsection{Функциональные классы генетических вариантов}

Как уже было упомянуто выше, значение генетических вариантов напрямую зависит от их положения относительно функциональных частей генома.
Варианты могут находиться как внутри генов, так и вне их.

Области гена, в которые может попасть генетический вариант:

\begin{itemize}
\item Экзоны, непосредственно отвечающие за последовательность белка.
Генетические варианты в экзонах могут быть синонимичными (без замены аминокислоты) и несинонимичными "--- миссенс (замена на другую аминокислоту), нонсенс (замена на стоп-кодон) либо сдвиг рамки считывания, приводящий к изменению значительной части белковой молекулы.
Миссенс-варианты редко приводят к утрате функции белка, но они могут повлиять на экспрессию гена, если замена пришлась на регуляторный мотив\cite{brea-fernandez}.
\item Интроны, которые содержат регуляторные области и сплайсинг-сайты, необходимые для процессинга транскрипта в готовую мРНК, а также 3'-нетранслируемая область (3' UTR) и 5'-нетранслируемая область (5' UTR), вовлечённые в регуляцию транскрипции, трансляции и деградации транскрипта.
Влияние генетических вариантов в этих областях недостаточно изучено, и их связь с конкретной патологией у пациента порой достаточно трудно доказать.
Тем не менее, существуют специальные инструменты, позволяющие оценить патогенность таких вариантов.
Интронные и UTR генетические варианты обычно рассматриваются в случае, если иного объяснения фенотипу пациента не было найдено.
\end{itemize}

Внегенные варианты могут приходиться на различные регуляторные последовательности, например, энхансеры, сайленсеры, а также сайты связывания белков, отвечающих за процессы метилирования или трёхмерную организацию хроматина.

Как мы видим, типов генетических вариантов существует огромное множество, они в значительной степени различаются между собой, и их определение может представлять трудность для исследователя.
На сегодняшний день разработано множество методик, облегчающих эту задачу.
О них речь пойдёт ниже.

\subsection{Методы детекции генетических вариантов}

\paragraph{Кариотипирование.}
Данный метод представляет собой микроскопическое исследование клеток, синхронизированных на стадии метафазы митоза.
Однако простое микроскопическое исследование хромосом плохо подходит для обнаружения генетических вариантов, поэтому были разработаны различные методы окраски (бэндинга), позволяющие отдифференцировать отдельные хромосомы и хромосомные регионы\cite{schreck}:

\begin{enumerate}
\item Q-окрашивание "--- позволяет отдифференцировать все хромосомы, применяется для исследования Y-хромосомы (быстрое определение генетического пола, выявление мозаицизма по Y-хромосоме, транслокаций между Y-хромосомой и другими хромосомами).
Окрашивание легко снимается, что позволяет использовать этот метод для последовательной окраски и изучения хромосом;
\item G-окрашивание "--- наиболее часто используемый метод.
Позволяет отдифференцировать все хромосомы, гарантирует стойкое окрашивание, легко поддаётся фотографированию.
\item R-окрашивание "--- визуализирует концы хромосом, а также специфические именно для этого окрашивания бэнды (так называемые R-позитивные бэнды).
\item C-окрашивание "--- применяется для анализа вариабельной дистальной части Y-хромосомы, а также центромерных регионов прочих хромосом, содержащих конститутивный гетерохроматин.
Хорошо подходит для выявления перестроек, затрагивающих гетерохроматиновые регионы.
Кроме того, C-окрашиванием хорошо определяются кольцевые и дицентрические хромосомы;
\item NOR-окрашивание "--- визуализирует ядрышковые организаторы (NOR), богатые рибосомальными генами;
\item DA--DAPI-окрашивание "--- применяется для идентификации центромерных гетерохроматизированных районов.
\end{enumerate}

Окрашенные хромосомы далее изучаются на предмет формы, количества и наличия перестроек.

Кариотипирование "--- рутинная методика при диагностике врождённых патологий, аутопсии мертворожденных и злокачественных образований кроветворного ряда.
Преимущества кариотипирования в том, что данным методом можно охватить весь геном, визуализации поддаются отдельные клетки и отдельные хромосомы.
Ограничения "--- обязательно требуются живые клетки, также на эффективность влияет размер перестроек (не менее 1--5 миллионов п.о.) и процент поражённых клеток в образце (минимум 5--10\%)\cite{sampson}.

В целом классический метод кариотипирования, достаточно дешёвый и простой в исполнении, требует от исследователя значительного опыта при интерпретации.
Более поздние методы изучения хромосом, как будет показано далее, развивались не только в направлении увеличения разрешающей способности, но и облегчения интерпретации полученных данных.

\paragraph{Флуоресцентная \textit{in situ} гибридизация (FISH).}
Основой является гибридизация нуклеиновых кислот образца и комплементарных им проб, содержащих флуоресцентную метку.
Гибридизация может производиться с ДНК (метафазные или интерфазные хромосомы) или с РНК.
FISH позволяет определить число исследуемых локусов в геноме (при использовании метода 3D-FISH) или последовательность расположения на метафазной хромосоме.
Метод является <<золотым стандартом>> в определении хромосомных патологий "--- как в клетках с врождёнными перестройками, так и в клетках опухолей.

Данные при помощи метода FISH можно получить, анализируя отсутствие или присутствие сигналов от использованных флюорофоров.
Количество различимых цветовых меток равно $(2^x - 1)$, где  $x$ "--- количество флюорофоров.
Это позволяет реализовать, например, спектральное кариотипирование (SKY), при котором каждая хромосома окрашивается в свой собственный цвет и межхромосомные перестройки видны даже начинающему специалисту\cite{guo}.
Тем не менее, лимитирующими факторами остаются:

\begin{itemize}
\item потребность в хорошо обученном персонале.
Относительная простота интерпретации результатов сочетается со сложностью протокола приготовления образца, который зависит от характера пробы и образца, и должен быть настроен эмпирически;
\item цена реактивов;
\item время гибридизации.
Кинетика реакций гибридизации в ядре изучена недостаточно, и требуется достаточно долгое время, чтобы получить сигналы, которые можно измерить и сравнить между собой.
\end{itemize}

В настоящее время методика FISH значительно усложнилась.
Биотехнологические компании предлагают панели олигонуклеотидов, определяющие специфические участки размером от десятков тысяч до миллиона пар оснований, а также олигонуклеотиды с высокой чувствительностью, позволяющие определить сплайс-варианты и даже SNV.
Разрабатываются технологии micro-FISH ($\mu$FISH), сочетающие FISH с микрофлюидными технологиями (проведение реакций в микроскопических объёмах жидкости).
При этом процесс удешевляется, автоматизируется, ускоряется (за счёт уменьшения объёмов, а соответственно, и времени гибридизации) и упрощается для использования в обширных исследованиях и для внедрения в клинику\cite{huber}.

\paragraph{Сравнительная геномная гибридизация (CGH).}
Как и в случае с методом FISH, основой данного метода является флуоресцентная гибридизация.
Однако CGH использует два образца генома "--- тестовый и контрольный, каждый из которых метится флюорофором, а затем гибридизуется в соотношении 1 : 1.
Таким образом в тестовом образце можно обнаружить CNV и перестройки.

В отличие от FISH, CGH проверяет весь геном на наличие перестроек и не требует знаний о целевом регионе.
% Однако, как и у описанных выше методов, разрешение CGH ограничено 5--10миллионами п.о..
К ограничениям анализа относится невозможность выявления полиплоидии, мозаицизма и сбалансированных транслокаций.

В настоящее время CGH используется в виде array-CGH (aCGH), или хромосомного микроматричного анализа (ХМА), при котором CGH комбинируется с микрочиповой гибридизацией\cite{theisen}.
Гибридизация между комплементарными цепями ДНК позволяет исследовать неизвестную ДНК путем сравнения с ДНК известной последовательности.
Для этой цели были созданы ДНК-микрочипы, или микроматрицы.
Они представляют собой сотни тысяч или миллионы однонитевых фрагментов ДНК (зондов), которые ковалентно пришиты к основанию (микрочипу).
При ХМА на микрочип наносятся контрольные фрагменты генома либо контрольные последовательности генов, которые могут быть связаны с конкретной патологией.
Порядок зондов на чипе строго определён, что упрощает локализацию и определение характера перестройки.
    
С помощью сравнительной гибридизации геномов могут быть обнаружены самые разные структурные вариации "--- CNV, инверсии, хромосомные транслокации и анэуплоидии.
Для этого используются длинные зонды, которые позволяют проводить гибридизацию последовательностей, имеющих некоторые различия.
Когда пробы ДНК короткие, эффективность гибридизации очень чувствительна к несовпадениям; такие зонды облегчают сравнение геномов на нуклеотидном уровне (поиск SNV).

Микроматрицы предлагают относительно недорогие и эффективные средства сравнения всех известных типов генетических вариаций.
Однако для таких целей, как обнаружение неизвестных или часто повторяющихся последовательностей, эти методы не подходят\cite{gresham}.

\paragraph{Мультиплексная лигаза-зависимая амплификация зонда (MLPA).}
Основой MLPA является ПЦР-амплификация специальных проб, гибридизующихся с целевыми районами ДНК.
Каждая проба представляет собой пару полу-проб;
каждая полу-проба имеет комплементарную геному часть и технические последовательности "--- праймер для ПЦР и вставки, обеспечивающие большой размер продукта амплификации.
Если полу-пробы гибридизуются с геномом без зазора, они лигируются и впоследствии амплифицируются;
лигированные пробы отличаются от полу-проб с праймером по длине.
Длину готового ПЦР-продукта определяют методом электрофореза.

Данная методика подходит для определения CNV, включающих целые гены, а также аномалий метилирования ДНК.
Во втором случае используют метил-чувствительные рестриктазы "--- ферменты, которые по определённым сайтам гидролизуют исключительно метилированную ДНК.
Для определения этих участков также применяют электрофорез, т.к. не подвергшаяся гидролизу ДНК по длине значительно превосходит фрагменты гидролизованных рестриктазой метилированных регионов.

Слабым местом MLPA остаётся интерпретация результатов.
Определение гомозиготных CNV не представляет труда "--- их разпознают по наличию/отсутствию пика в сравнении с контрольным образцом.
Гетерозиготные CNV видны как пики отличающейся высоты, и их поиск требует серьёзную биоинформационную обработку с учётом особенностей конкретной ПЦР-реакции и различий между образцами\cite{stuppia}.

~

Как мы видим, перечисленные методы имеют один серьёзный недостаток "--- они могут определить наличие или отсутствие, совпадение или несовпадение, но не способны прочитать априори неизвестную последовательность ДНК.
Специально для этого были разработаны методы секвенирования.

\paragraph{Секвенирование по Сэнгеру.}
Исторический метод, позволяющий с высокой точностью анализировать короткие (до 1 тысячи п.о.) фрагмент ДНК\cite{sanger}.
Суть его состоит в проведении обычной реакции амплификации ДНК, только в смесь дезоксирибонуклеотидов (dNTP) добавлены дидезоксирибонуклеотиды (ddNTP), которые при присоединении к ДНК обрывают синтез и имеют флуоресцентную или радиоактивную метку (соотношение примерно 100 : 1 соответственно).
Таким образом, в процессе амплификации в пробирках образуется смесь из меченых цепей разной длины.
При разделении этой смеси на электрофорезе проявляется характерная <<лестница>>, последовательность флуоресцентных сигналов в которой совпадает с последовательностью исследуемой ДНК.

Основным недостатком секвенирования по Сэнгеру является ограничение длины исследуемого фрагмента ДНК.
% Также метод чувствителен к контаминации; в случае, если исследуемая матрица была клонирована, а не амплифицирована методом ПЦР, она может быть контаминирована вектором клонирования, что затрудняет прочтение последовательности.

В настоящее время метод Сэнгера используется для подтверждения вариантов, найденных с помощью методов секвенирования нового поколения (NGS).

\paragraph{Секвенирование нового поколения NGS.}
Это комплекс технологий, позволяющих прочитать за сравнительно небольшое время миллионы последовательностей ДНК.
Благодаря этому единовременно можно проанализировать несколько генов, либо весь геном.

В методах NGS наблюдается развитие двух основных парадигм, различающихся по длине прочтений.
Секвенирование короткими прочтениями характеризуется меньшей ценой и более качественными данными, что позволяет применять данные методы в популяционных исследованиях и клинической практике (поиск патогенных вариантов).
Секвенирование длинными прочтениями хорошо подходит для сборки новых геномов и изучения отдельных изоформ генов\cite{goodwin}.
Количество различных методов в настоящее время значительно, но самым часто используемым является метод Illumina (короткие прочтения).

Основные проблемы данных NGS:

\begin{itemize}
\item Финансовые вложения и время, затраченные на секвенирование и анализ данных.
По-прежнему остаются лимитирующим фактором применения NGS в клинической практике;
\item Ошибки секвенирования и ПЦР.
Их значимость уменьшается с увеличением покрытия, но не исчезает полностью;
\item Неоднородность покрытия генома прочтениями.
Это может быть связано как с недостатками приготовления библиотеки, так и с проблемами картирования;
\item Неточное картирование прочтений, содержащих инсерции и делеции, а также повторяющихся последовательностей (например, поли-A трактов).
\end{itemize}

\subsection{Виды NGS}

\paragraph{Полногеномное секвенирование (WGS).}
Приготовление библиотек при полногеномном секвенировании производится из всего клеточного материала, либо только из ядер.
ДНК фрагментируется таким образом, что достигается относительно ровное покрытие генома.

WGS при достаточной глубине покрытия вполне пригодно для поиска SNV, небольших делеций и инсерций.
Полногеномное секвенирование со слабым покрытием может быть использовано для определения CNV "--- например, при неинвазивном пренатальном тестировании (NIPT), когда используется свободная ДНК плода (cffDNA), циркулирующая в крови матери\cite{yu}.

\paragraph{Таргетные панели.}
Основой данных методов является обогащение целевых регионов генома.
Методов обогащения существует достаточно много, но все они делятся на 4 основные категории\cite{teer}:

\begin{enumerate}
\item Твердофазная гибридизация.
Для этого используют комплементарные целевым регионам короткие ДНК-пробы, зафиксированные на твёрдом основании (микрочипе).
Далее нецелевую ДНК вымывают, а целевые фрагменты остаются на чипе.
\item Жидкофазная гибридизация.
Эти методы характеризуются тем, что ДНК-пробы находятся в растворе и помечены специальной молекулой (например, биотином).
После гибридизации с целевой ДНК пробы вылавливают бусинами, поверхность которых способна связывать молекулы биотина.
\item Полимеразно-опосредованный захват.
В этих методах ПЦР производят на стадии обогащения.
Например, методы молекулярно импринтированных полимеров (MIP) и анализа транскриптома одной клетки (SMART) используют длинные пробы, содержащие как праймер, так и регион для остановки элонгации и инициации лигирования.
После элонгации и лигирования получаются кольцевые молекулы, содержащие целевой регион;
линейные молекулы в последующем удаляют из раствора.
Метод захвата с помощью расширения праймера (PEC) использует биотинилированные праймеры, которые гибридизуются с целевыми регионами и элонгируются;
далее их вылавливают бусинами, как в методах жидкофазной гибридизации.
\item Захват регионов.
Включает в себя сортировку и микродиссекцию хромосом, благодаря чему можно обогатить библиотеку фрагментов последовательностями отдельной хромосомы или даже её части.
Это методы, требующие чрезвычайно сложных техник и хорошо обученный персонал, но очень полезные в отдельных ситуациях.
\end{enumerate}

Данный вид тестов позволяет анализировать гены, ответственные за отдельные группы заболеваний "--- например, существуют таргетные панели для иммунодефицитов, почечных, неврологических болезней, болезней соединительной ткани, сетчатки, а также предрасположенности к отдельным видам онкологических заболеваний.
Таргетные панели позволяют анализировать и клетки опухолей "--- некоторые приспособлены к выявлению общих для многих раковых линий мутаций, другие же разработаны для специфического типа опухолей\cite{yohe}.

\paragraph{Полноэкзомное секвенирование (WES).}
Техника заключается в секвенировании обогащённого экзома "--- совокупности белок-кодирующих последовательностей клетки.
Для этого используют специальные экзомные таргетные панели.
Несмотря на то, что существует множество методов таргетного обогащения, конкретно для WES могут быть использованы лишь немногие из них, а именно "--- твердофазная и жидкофазная гибридизация\cite{teer}.

У человека экзом составляет примерно 1\% от генома, или примерно 30 миллионов п.о. (суммарно).
При этом более 80\% генетических вариантов, которые представлены в базе данных известных геномных вариантов CLINVAR, и из них более 89\% вариантов, которые отмечены как <<патогенные>>, относятся к белок-кодирующим областям генома;
эта цифра приближается к 99\%, если учитывать ближайшие окрестности экзонов\cite{barbitoff}.
Таким образом, полноэкзомное секвенирование намного лучше подходит для обычной клинической практики, нежели полногеномное.
Кроме того, полноэкзомное секвенирование значительно дешевле, что увеличивает его доступность и позволяет, например, произвести тестирование ребёнка и родителей (так называемый трио-тест) и, как следствие, улучшить интерпретацию вариантов\cite{yohe}.

\paragraph{Технологии захвата конформации хромосом (3C).}
Данные методики позволяют определить расстояние в 3D-пространстве ядра между двумя точками генома.
Принцип состоит в том, что интактное ядро фиксируют формальдегидом, ДНК гидролизуют, лигируют, затем продукты лигазной реакции секвенируют при помощи NGS.
Во время лигирования ковалентно связанными могут оказаться только те участки, которые физически находятся близко друг от друга.
Картирование химерных прочтений с помощью специальных инструментов позволяет узнать, какие именно участки генома были связаны\cite{dekker}\footnote{Ссылка неудачная}.
При обработке большого количества 3C данных геном разделяют на районы фиксированной длины, называемые бинами.
Длина бинов называется разрешением; чем меньше длина, тем более высоким считается разрешение.
Прочтение, части которого были картированы на два разных бина, называется контактом между этими районами.
Практическое значение имеет информация об относительной частоте контактов между бинами.

В настоящее время существует множество вариантов протокола 3C.
Самым известным и широко применяемым является метод Hi-C, сочетающий 3C с методами массового параллельного секвенирования.
С его помощью можно подсчитать количество контактов во всём геноме "--- как внутри-, так и межхромосомные контакты\cite{oluwadare}.

~

Результаты NGS представляют собой гигантские блоки данных, содержащие всевозможные ошибки.
Обработка данных секвенирования "--- это высокотехнологичная отрасль, которая позволяет получить из этих данных практически значимую информацию и минимизировать влияние ошибок на эту информацию.

\subsection{Базовая схема обработки результатов секвенирования}

\paragraph{Демультиплексикация.}
В процессе приготовления NGS-библиотеки к целевым фрагментам ДНК лигируют так называемые адаптерные последовательности, или адаптеры.
Очень часто потенциальное количество прочтений, которое способен выдать секвенатор за один запуск, значительно превышает требуемое количество прочтений для отдельной библиотеки, поэтому из соображений экономии и повышения производительности на одном чипе секвенируют сразу несколько библиотек.
Для этого в адаптеры вставляют баркоды "--- последовательности, с помощью которых можно отличить прочтения, относящиеся к разным библиотекам или образцам.
Процесс сортировки данных секвенирования по баркодам называется демультиплексикацией.

\paragraph{Удаление адаптерных последовательностей.}
Если целевой фрагмент ДНК короче длины прочтения, то фрагменты адаптерной последовательности на 3'-конце могут попасть в готовые данные.
Это замедляет работу алгоритма картирования, а порой в значительной степени ухудшает его результаты, поэтому встаёт вопрос об удалении адаптерных последовательностей.
Также присутствие адаптера в прочтениях может быть признаком контаминации библиотеки, и такие прочтения следует исключить из дальнейшего анализа\cite{cutadapt}.

\begin{Huge}Рисунок\end{Huge}

\paragraph{Картирование прочтений.}
Как уже упоминалось выше, результаты NGS "--- это прочтения, содержащие небольшие (до 150bp) фрагменты генома.
Извлечение информации из необработанных результатов секвенирования затруднительно, так как эти фрагменты содержат много ошибок (как в результате ПЦР-реакции, так и допущенные в процессе секвенирования) и не имеют никакой информации о регионе, из которого они произошли.
Поэтому прочтения необходимо картировать на некую референсную геномную последовательность.
Алгоритм картирования представляет собой очень сложную систему, которая учитывает последовательность букв в прочтении и качество прочтения.
Качество прочтения отражает вероятность того, что буква, прочитанная секвенатором, совпадает с реальным нуклеотидом в данной позиции.
Обычно качество прочтения записывается в шкале Phred, к которой приводится формулой \begin{equation}Q = -10\log_{10}P,\label{eq:qual}\end{equation} где $P$ "--- вероятность того, что нуклеотид был прочтен правильно.
Разработано множество алгоритмов картирования, но в настоящее время <<золотым стандартом>> являются утилиты, использующие алгоритм Берроуса--Уиллера\cite{burrows}.

Обычно алгоритм картирования выставляет выравниванию коэффициент, называемый качеством выравнивания (MAPQ).
MAPQ отражает вероятность правильности картирования и также записывается в шкале Phred (формула \ref{eq:qual}).
В силу размеров референсной последовательности в ней существует огромное множество повторов и похожих регионов.
Современные алгоритмы могут находить несколько потенциальных мест картирования для одного прочтения, и их количество влияет на качество выравнивания.

Также алгоритмы способны разделять прочтение на участки, которые могут быть картированы в разные места генома.
По этому признаку прочтения делятся на линейные и химерные.
В линейных прочтениях не может быть изменения направления картирования, т.е. картированная часть может иметь только прямое направление, либо только обратное направление относительно генома.
Химерные прочтения имеют картированные части с разным направлением.
Эти участки могут перекрываться, и количество перекрытий также влияет на MAPQ.

Исходя из особенностей алгоритмов картирования, выравнивания делятся на следующие классы:

\begin{itemize}
\item Первичное выравнивание (primary) "--- выравнивание наиболее крупного (и содержащего наименьшее количество перекрытий, в случае химерного прочтения) фрагмента прочтения с наиболее высоким MAPQ.
Первичное выравнивание только одно.
Первичное выравнивание химерного прочтения называется репрезентативным;
\item Вторичное выравнивание (secondary) "--- выравнивание наиболее крупного фрагмента прочтения с меньшим MAPQ.
Вторичных выравниваний может быть несколько (в зависимости от выставленного нижнего порога MAPQ);
\item Добавочное выравнивание (supplementary) "--- выравнивание менее крупных (либо содержащих большее количество перекрытий) фрагментов прочтения.
Добавочные выравнивания характерны только для химерных прочтений.
\end{itemize}

Картированный участок может содержать в себе несовпадения с референсной последовательностью, инсерции и делеции.
Это могут быть как ошибки, так и генетические варианты, поэтому данная информация безусловно важна при анализе данных.
Также в частично картированных прочтениях могут присутствовать некартируемые участки с 3' или 5' конца.
В отличие от делеций внутри картированных участков, некартированные концы обычно подвергаются так называемому клипированию и в дальнейшем не учитываются при анализе.
Клипирование бывает двух типов:

\begin{itemize}
\item Мягкое клипирование (soft-clip) "--- отсечение невыравненного конца прочтения с сохранением полной последовательности прочтения.
В отсечённых методом мягкого клипирования регионах могут быть адаптерные последовательности, а также часть химерного прочтения (в репрезентативном выравнивании).
\item Жёсткое клипирование (hard-clip) "--- отсечение невыравненного конца прочтения без сохранения его последовательности.
В регионах, подвергшихся жёсткому клипированию, обычно находятся репрезентативные участки химерных прочтений (в добавочных выравниваниях).
\end{itemize}

Основные проблемы картирования:

\begin{itemize}
\item Высоковариативные регионы.
Алгоритм картирования разработан для поиска наиболее полных соответствий, и при большом количестве несовпадений прочтение просто не сможет быть картировано на нужный регион генома;
\item Вырожденные (неуникальные) регионы.
Соответствие между регионами может привести к неправильному распределению прочтений между ними, а значит "--- и неправильному картированию генетических вариаций.
Кроме того, генетические варианты в регионах с короткими повторами в принципе невозможно картировать точно, поэтому обычной практикой является левое смещение (left-align).
\item Регионы с инсерциями и делециями.
Помимо того, что сами по себе эти варианты сильно ухудшают картирование, края инсерций и делеций могут быть смещены в зависимости от особенностей конкретного региона.

\begin{Huge}Рисунок\end{Huge}
\end{itemize}

\paragraph{Удаление дубликатов.}
Так как молекулы ДНК очень малы, вероятность их разрушения или возникновения в них ошибок велика, а полученные от них сигналы находятся за пределами чувствительности многих современных приборов.
Решением этих проблем является амплификация молекул ДНК.
Амплификация может быть как на стадии приготовления библиотеки (ПЦР), так и на стадии секвенирования.
Амплицикация и последующее объединение ампликонов в кластер производится для усиления сигнала и нивелирования ошибок, происходящих на каждом цикле секвенирования с отдельными молекулами.
Соответственно, в процессе секвенирования возникают дубликатные прочтения, которые могут быть как ПЦР-дубликатами библиотеки, так и возникать из-за ошибок распознавания кластеров амплификации (оптические дубликаты).
Согласно принятой практике, дубликаты должны быть удалены или помечены для улучшения поиска генетических вариантов\cite{gatk}.

Однако, было показано, что для WGS-данных удаление дубликатов имеет минимальный эффект на улучшение поиска полиморфизмов "--- приблизительно 92\% из более чем 17 млн вариантов были найдены вне зависимости от наличия этапа удаления дубликатов и использованных инструментов для поиска дубликатов\cite{ebbert}.
Учитывая, что удаление дубликатов может занимать значительную часть потраченного на обработку данных времени, следует взвесить пользу и затраты данного этапа для конкретной прикладной задачи.

\paragraph{Рекалибровка качества прочтений (BQSR).}
В приборной оценке качества прочтений всегда имеют место систематические ошибки.
Это связано как с особенностями физико-химических реакций в секвенаторе, так и с техническими недостатками оборудования.
Вычисление качества прочтения "--- сложный алгоритм, защищённый авторскими правами производителя секвенатора.
Вместе с тем от качества прочтений напрямую зависит алгоритм поиска вариантов "--- он использует данный коэффициент как вес в пользу присутствия или отсутствия генетического варианта в конкретной точке генома.

Решением является рекалибровка качества прочтений, представляющая собой корректировку систематических ошибок, исходя из известных паттернов ковариации случайных величин\footnote{НЁХ, могут спросить}.
Следует заметить, что рекалибровка не помогает определить, какой нуклеотид в реальности находится в данной позиции "--- она лишь указывает алгоритму поиска генетических вариантов, выше или ниже вероятность правильного прочтения нуклеотида секвенатором.

Первоочередное влияние на ошибки оказывают:

\begin{enumerate}
\item Собственно прибор (секвенатор) и номер запуска.
Большая часть секвенаторов выставляет прочтению более высокое качество прочтения по сравнению с ожидаемым, гораздо реже встречаются модели, занижающие качество прочтения\cite{gatk}.
Каждый отдельный запуск может различаться по параметрам чипа и химических реагентов;
\item Цикл секвенирования.
Качество прочтения уменьшается с каждым циклом за счёт накопления ошибок в кластере амплификации;
\item Нуклеотидный контекст.
Систематические ошибки, связанные с физико-химическими процессами, влияют на качество прочтения нуклеотида в зависимости от предшествующего ему динуклеотида.
\end{enumerate}

Кроме того, алгоритм рекалибровки учитывает изменчивость каждого отдельного сайта, используя базы данных известных генетических вариантов.
Высокая изменчивость повышает вероятность правильного прочтения нуклеотида, не совпадающего с референсным в данной позиции генома.

Broad Institute of MIT and Harvard рекомендует BQSR к использованию для любых данных секвенирования\cite{gatk}.

\paragraph{Поиск генетических вариантов.}
Невозможно точно сказать, какой нуклеотид находится в каждой позиции генома.
Анализ производит специальный алгоритм, который оценивает качество прочтения, качество выравнивания и процент букв в данной позиции на картированных прочтениях.
Отличие генома образца от референсного генома называется генетическим вариантом (синонимичные термины <<мутация>> и <<полиморфизм>> не рекомендованы к употреблению\cite{richards}).
Алгоритм выставляет каждому генетическому варианту коэффициент качества варианта (VCF Qual), записываемый в шкале Phred (формула \ref{eq:qual}).
Помимо определения генетического варианта, алгоритм может определять его зиготность.

Также важным этапом поиска вариантов является уже упомянутое выше левое выравнивание.
Варианты в повторяющихся последовательностях с длиной менее длины одного прочтения невозможно точно локализовать, поэтому они всегда сдвигаются как можно левее относительно последовательности генома.
Это чрезвычайно важно при аннотации генетических вариантов, так как все БД используют данные с левым выравниванием, и неправильная локализация может привести к отсеиванию потенциально патогенного варианта.

~

После того, как генетические варианты найдены, можно приступать к поиску тех, которые связаны с конкретной патологией у пациента.
Однако только в кодирующих областях генома количество генетических вариантов достигает 100 тыс. (из них около 86\% SNV, 7\% инсерций и 7\% делеций)\cite{supernat}, из них с патологиями связаны единицы.
Даже после жёсткой фильтрации приходится работать минимум с сотней подходящих генетических вариантов.
Это делает серьёзной проблемой поиск нужного варианта и интерпретацию полученных результатов.

\subsection{Аннотация, фильтрация и интерпретация результатов}

Первое, что следует сделать "--- это определить, насколько генетический вариант значим для нашего исследования, то есть аннотировать его.
Существуют две основных парадигмы аннотации генетического варианта "--- это аннотация по региону и аннотация по координате.

Основные методы аннотации по региону:

\begin{enumerate}
\item Функциональный класс.
Для определения функционального класса генетического варианта существуют три основных базы данных: knownGene, refGene и ensGene.
Они содержат информацию о генах, их частях и транскриптах "--- координаты, направление, а также номера экзонов и интронов.
Координаты в этих базах данных могут незначительно различаться.
Каждая БД имеет свою классификацию, поэтому использование той или иной "--- вопрос вкуса и привычки.

\item Клиническая значимость гена.
Количество генетических вариантов для поиска можно сузить, зная, какие именно гены могут быть связаны с наблюдаемым у пациента фенотипом.
Для поиска генов по клинической значимости существуют такие базы данных, как OMIM и OrphaData.

\item Потеря функции (LoF).
Различные показатели, отражающие устойчивость функции гена, основанные на данных о стоп-кодонах, сдвигах рамки считывания и сплайс-вариантах.
Одним из таких показателей является pLI.

Основные проблемы pLI\cite{ziegler}:

\begin{itemize}
\item Плохо приспособлен к распознаванию аутосомно-рецессивных вариантов (из-за того, что частота повреждающих вариантов в популяции может быть высокой) и X-сцепленных рецессивных вариантов (из-за наличия в популяции здоровых гетерозиготных носителей).
\item Плохо приспособлен к распознаванию генетических вариантов в генах, ответственных за патологии, не влияющие на взросление и воспроизводство.
Их частота в популяции также может быть высокой.
К таким относятся варианты в генах BRCA1 и BRCA2, ответственных за рак молочной железы.
\item Сплайс-варианты априори рассматриваются как повреждающие, несмотря на то, что вариант в сайте сплайсинга может не иметь эффекта на сплайсинг, либо приводить к появлению изоформы белка без потери функции.
\item Высокая частота распространения заболевания в контрольной группе.
Пример "--- шизофрения.
\item К миссенс-вариантам pLI применять следует с осторожностью, и без клинических данных следует исключить из анализа.
\item Также следует отнестись с осторожностью к нонсенс-вариантам и сдвигам рамки считывания в последнем экзоне либо в C-терминальной части предпоследнего.
Такие транскрипты избегают нонсенс-индуцированной деградации РНК и могут в результате как не привести к каким-либо функциональным изменениям, так и привести к образованию мутантного белка, обладающего меньшей активностью по сравнению с исходным, либо токсичного для клетки.
\item В некоторых случаях соотношение pLI с гаплонедостаточностью конкретного гена в принципе сложно объяснить.
\end{itemize}

Таким образом, высокое значение pLI можно считать хорошим показателем LoF, низкое "--- с осторожностью.                                                                                                       \end{enumerate}

Аннотация по координате обычно предназначена для миссенс-, интронных и сплайс-вариантов, связь которых с патологическим состоянием значительно сложнее выявить и доказать.

\begin{enumerate}
\item Частота аллеля в популяции.
% Как уже говорилось, генетические патологии делятся по частоте встречаемости на частые и редкие (орфанные) заболевания.
Многие тяжёлые генетические патологии испытывают на себе давление отбора, а значит, вызывающие их генетические варианты не могут иметь высокую частоту в популяции.
Фильтрация по частоте является одним из базовых способов фильтрации генетических вариантов.
Следует заметить, однако, что низкая частота генетического варианта далеко не всегда связана с его патогенностью, поэтому рассматривать низкую частоту как доказательство патогенности некорректно.

По мере развития методов NGS и увеличения их доступности, начали появляться базы данных, агрегирующие результаты секвенирования различных популяций, а значит "--- способные определить частоту генетических вариантов в популяции.
В настоящее время наиболее крупной является gnomAD\cite{gnomad}, поглотившая существовавший ранее ExAC, содержавший исключительно экзомные данные.
Она содержит частоты генетических вариантов для всех основных рас, а также некоторых условно-здоровых групп.

Несмотря на то, что были созданы базы данных для всех рас, очень часто этого недостаточно и необходимо учитывать частоты в популяциях отдельных народов и стран.
Такими базами данных являются GME\cite{gme}, в которой отражены частоты по популяции Ближнего Востока, ABraOM\cite{abraom}, предоставляющая частоты генетических вариантов среди практически здорового пожилого населения Бразилии.
Также для анализа берутся популяции, в которых велика доля близкородственных связей, например, пакистанская\cite{saleheen}.

\item Клинические данные из БД и статей.
Наиболее достоверным источником данных о патогенности генетического варианта являются семейные и популяционные исследования конкретной патологии, а также базы данных, агрегирующие информацию из подобных статей.
Наиболее используемыми в настоящее время являются HGMD\cite{hgmd} и CLINVAR.
Тем не менее, CLINVAR считается лишь дополнительным источником, так как часто содержит информацию низкого качества\cite{ryzhkova}.

\item Анализ и предсказание функционального эффекта in silico.
In silico методы появились в ответ на необходимость как-то классифицировать генетические варианты, по которым недостаточно клинической информации.
Существует множество способов проверить патогенность таких вариантов in vitro, но проверять таким образом все нецелесообразно, а иногда и невозможно.
Даже в хорошо изученных генах варианты с неопределённой клинической значимостью могут занимать большую долю "--- например, в BRCA1 и BRCA2 это 33\% и 50\% соответственно.
Менее изученные гены, а также пациенты, принадлежащие к популяциям с плохо изученным составом генетических вариантов, представляют ещё большую проблему.

Поэтому были разработаны инструменты на основе машинного обучения, предсказывающие консервативность районов и патогенность генетических вариантов на основе имеющихся данных "--- положения относительно гена и его функциональных элементов, характера замены, а также клинической информации об известных заменах\cite{brea-fernandez}.
Предсказательная способность отдельных инструментов оставляет желать лучшего, поэтому чаще всего в клинической практике используются агрегаторы, собирающие предсказания с большого числа известных in silico инструментов.
\end{enumerate}

Значимость вклада каждого отдельного фактора достаточно сложно оценить.
Эту проблему решают калькуляторы патогенности, которые по специальным критериям присваивают генетическому варианту ранг, отражающий вероятность повреждающего действия\cite{ryzhkova}.

\paragraph{Когортный и семейный анализ.}
В случае, если исследователь имеет доступ к группе, представители которой связаны узами крови с пациентом, есть возможность провести семейный анализ.
Семейный анализ нужен для установления путей наследования тех или иных генетических вариантов в родословной.
Это позволяет уточнить их связь с фенотипом.
Также анализ нескольких родственных образцов помогает определить зиготность варианта, обнаружить генетические варианты de novo, либо импутировать район с недостаточным покрытием.
 
Если же в распоряжении исследователя находится группа, связанная одной патологией или вариантом фенотипа, можно провести когортный анализ.
Когортный анализ позволяет, например, оценить частоты генетических вариантов в исследуемой и контрольной группе.
Кроме того, когортный анализ образцов в конкретной лаборатории помогает детектировать систематические отклонения покрытия и артефакты выравнивания, связанные с конкретными районами генома и/или особенностями приготовления библиотек.

\paragraph{Cлучайные находки.}
Несмотря на то, что точность определения патогенности вариантов достаточно невысокая, этические правила, регламентирующие работу врача-генетика, рекомендуют сообщать о потенциально патогенных вариантах в некоторых генах, даже если они не связаны с текущим состоянием пациента.
К таким генам относятся, например, BRCA1 и BRCA2, связанные с раком молочной железы.

~

Описанная выше схема характерна для поиска генетических вариантов во всех видах NGS-данных.
Тем не менее, частности могут различаться.
Это связано с особенностями покрытия генома, наличием технических последовательностей в результатах секвенирования и многими другими факторами.
Таким образом, новые методы приготовления NGS-библиотек часто требуют соответствующей доработки биоинформационных методов, а иногда и разработки новых\footnote{значимость раздела поставлена под сомнение}.

\subsection{Exo-C: суть метода}

Как уже упоминалось выше, одним из основных ограничений NGS-технологий в настоящее время является их цена, напрямую зависящая от глубины секвенирования библиотеки.
Есть ограничения и по возможностям поиска тех или иных генетических вариантов.
3C методы на сегодняшний момент являются наиболее эффективным способом обнаружения хромосомных перестроек, но при небольшой глубине секвенирования обнаружение точечных вариантов затруднительно.
WGS способно обнаруживать большую часть SNV, небольших инстерций и делеций, но требует большую глубину секвенирования; WES, с другой стороны, позволяет выявить генетические варианты при небольшой глубине, но только в экзоме.
Возможности обнаружения хромосомных перестроек для последних двух методов ограничены.

Компромиссом между ценой и возможностями поиска генетических вариантов может служить новейший метод Exo-C, сочетающий технологии таргетного обогащения с 3C.
Суть его заключается в приготовлении Hi-C библиотеки и последующем обогащении только тех последовательностей, которые связаны с экзомом.
Таким образом, с его помощью можно как искать точечные варианты в обогащённых регионах (за счёт большой глубины покрытия в них), так и хромосомные перестройки во всём геноме (за счёт Hi-C, дающей относительно небольшое, но доступное для анализа покрытие всего генома)\cite{mozheiko}.

Тем не менее, как выяснилось, уже существующие биоинформационные методы не подходят для обработки данных Exo-C.
Разработка новых методов в настоящее время двигается в двух направлениях:

\begin{enumerate}
\item Поиск крупных хромосомных перестроек во всём геноме\cite{mozheiko};
\item Поиск SNV, небольших инсерций и делеций в экзомных районах, чему и посвящена данная работа.
\end{enumerate}

\section{Материалы и методы}

\paragraph{Данные секвенирования.}
Результаты секвенирования клеточной линии K562 были взяты из публичных источников (см. Приложение \ref{appendix:accession}).
Из данных Banaszak et al. в дальнейшем были исключены все генетические варианты в интервале \verb|chr2:25455845-25565459| с фланкированием 1 тысячи п.о. (ген DNMT3A), так как в статье использовали генетически модифицированную линию с вариантами в данном гене.
% В качестве тестовых образцов мы использовали данные пациентов, а также клеточной линии K562, имеющейся в Институте Цитологии и Генетики.

\paragraph{Контроль качества NGS-данных.}
Для контроля качества прочтений мы использовали утилиту FastQC\cite{fastqc}, способную оценивать наличие адаптерных последовательностей, распределение прочтений по длине, GC-состав прочтений, а также производить анализ зависимости нуклеотидного состава от позиции в прочтении.
Критерии качества были использованы согласно протоколу разработчика\cite{fastqc}.

\paragraph{Удаление адаптерных последовательностей.}
Удаление адаптерных последовательностей производилось с помощью утилиты cutadapt\cite{cutadapt}.
В \cite{gatk} рекомендуется использовать в качестве входных данных некартированный BAM-файл (uBAM), а для удаления адаптеров использовать их собственный инструмент "--- MarkIlluminaAdapters, так как это позволяет сохранить важные метаданные.
Тем не менее, был сделан акцент на том, что uBAM должен использоваться как выходной формат на уровне секвенатора, что не является общепринятой практикой.

Мы использовали сторонние данные в формате FastQ.
Пребразование FastQ файлов в uBAM не предотвращает потерю метаданных, но значительно увеличивает время обработки данных.
Сравнение эффективности cutadapt и MarkIlluminaAdapters в процессе удаления адаптеров не показало каких-либо значимых различий.

\paragraph{Картирование.}
Картирование производилось с помощью инструментов Bowtie2\cite{bowtie2} и BWA\cite{bwa}.
BWA показал лучшие результаты;
кроме того, он значительно более эффективно работает с химерными ридами, что немаловажно для используемого нами метода Exo-C.

Для картирования был взят геном GRCh37/hg19.
Из него были удалены так называемые неканоничные хромосомы (некартированные/вариативные референсные последовательности), что позволило улучшить качество выравнивания и значительно упростить работу с готовыми данными.

Кроме того, для правильного функционирования инструментов на дальнейших этапах был разработан скрипт, создающий метку группы прочтений (RG tag) для каждого файла.
Конкретных рекомендаций по составлению RG не существует, поэтому мы разработали собственные, основанные на следующих требованиях\cite{gatk}:

\begin{itemize}
\item Поле SM является уникальным для каждого биологического образца и используется при поиске вариантов.
Несколько SM в одном файле могут быть использованы при когортном анализе.
\item Поле ID является уникальным для каждого RG в BAM-файле.
BQSR использует ID как идентификатор самой базовой технической единицы секвенирования.
\item Поле PU не является обязательным.
Рекомендации GATK советуют помещать в него информацию о чипе секвенирования (баркод), ячейке и баркоде (номере) образца.
Во время BQSR, при наличии поле PU является приоритетным по отношению к ID.
\item Поле LB является уникальным для каждой библиотеки, приготовленной из биологического образца.
Оно отражает различия в количестве ПЦР-дубликатов и потому используется инструментом MarkDuplicates.
\end{itemize}

Объединение BAM-файлов производилось инструментом MergeSamFiles.
Сбор статистики по картированию мы осуществляли с помощью инструмента samtools flagstat.

\paragraph{Удаление ПЦР-дубликатов.}
Для улучшения данных экзомного секвенирования в пайплайн был включён этап удаления ПЦР-дубликатов.
Обычно этот процесс занимает много времени, но количество образцов у нас было относительно небольшим, и мы были заинтересованы в максимально качественной подготовке данных.

Удаление дубликатов производилось инструментом MarkDuplicates от Picard\cite{picard}, интегрированным в GATK.
Оптимальные показатели скорости MarkDuplicates достигаются при запуске Java с параллелизацией сборщиков мусора и количеством сборщиков мусора равным двум\cite{heldenbrand}.
Также, согласно рекомендациям разработчиков, прочтения были предварительно отсортированы по именам, чтобы удалению подверглись не только первичные, но и добавочные выравнивания\cite{gatk}.

\paragraph{Рекалибровка качества прочтений (BQSR).}
Рекалибровка производилась с помощью инструментов GATK "--- BaseRecalibrator и ApplyBQSR.
Для обучения машинной модели требуются варианты в VCF формате (согласно рекомендациям для Homo sapiens "--- dbSNP >132).

К сожалению, предоставленная Broad Institute база данных оказалась сильно устаревшей и не вполне подходила для сделанной нами геномной сборки, поэтому было решено подвергнуть обработке dbSNP v150, скачанную с NCBI\cite{dbsnp}.
База данных потребовала замену и сортировку контигов в соответствии с референсным геномом, а также удаление <<пустых>> вариантов, содержащих точки в полях REF и ALT.
Далее база данных была архивирована с помощью bgzip, а затем проиндексирована IndexFeatureFile от GATK (этот же инструмент одновременно проверяет БД на пригодность для BQSR).

В \cite{heldenbrand} было показано, что оптимальные показатели скорости BaseRecalibrator достигаются, как и в случае с MarkDuplicates, запуском Java с двумя параллельными сборщиками мусора;
кроме того, BaseRecalibrator поддаётся внешнему распараллеливанию путём разделения картированных ридов на хромосомные группы.
Хромосомные группы формировались вручную для используемой сборки генома, каждая запускалась с помощью bash-скрипта.
Нам удалось усовершенствовать данный этап "--- запуск BaseRecalibrator производился с помощью библиотеки Python3 subprocess, а параллелизация осуществлялась библиотекой multiprocessing, таким образом, можно было делить файл с картированными прочтениями по хромосомам и обрабатывать их отдельно, так как multiprocessing автоматически распределяет процессы по имеющимся потокам.
Также для повышения отказоустойчивости скрипта у BaseRecalibrator и ApplyBQSR была устранена разница в фильтрации прочтений, из-за которой при малых размерах библиотек пайплайн экстренно завершал работу.

\paragraph{Оценка покрытия и обогащения.}
Покрытие и обогащение в экзоме оценивались с помощью скрипта на основе bedtools\cite{bedtools}.

\paragraph{Поиск вариантов.}
Поиск вариантов производился с помощью инструмента HaplotypeCaller от GATK.
Инструмент запускался с дополнительным параметром \verb|--dont-use-soft-clipped-bases|, который не позволял использовать для поиска генетических вариантов невыравненные химерные части и адаптеры.

Как и в случае с BaseRecalibrator, HaplotypeCaller поддаётся внешнему распараллеливанию\cite{heldenbrand}.
Мы также осуществили параллелизацию с помощью сочетания subprocess и multiprocessing, достигнув 10--12-кратного ускорения по сравнению с запуском на одном потоке.

\paragraph{Рекалибровка и ранжирование вариантов.}
В GATK также присутствуют инструменты для рекалибровки и ранжирования вариантов, с использованием моделей машинного обучения и баз данных с частыми вариантами (CNNScoreVariants и FilterVariantTranches).

Анализ показал, что при наличии этапа рекалибровки вариантов время обработки результатов секвенирования увеличивается почти вдвое.
Между тем, рекалибровка и ранжирование с помощью инструментов GATK не исключает необходимость проверки вариантов вручную.
Таким образом, от этого этапа решено было отказаться.

\paragraph{Аннотация вариантов.}
Аннотация вариантов производилась вначале с помощью инструмента Ensembl VEP\cite{vep}, затем мы мигрировали на ANNOVAR\cite{annovar}.

Используемые базы данных:

\begin{enumerate}
\item Human Gene Mutation Database (HGMD\textregistered)\cite{hgmd}
\item Online Mendelian Inheritance in Man (OMIM\textregistered)\cite{omim}
\item GeneCards\textregistered: The Human Gene Database --- \href{https://www.genecards.org/}{https://www.genecards.org/}
\item ClinVar --- \href{https://www.ncbi.nlm.nih.gov/clinvar/}{https://www.ncbi.nlm.nih.gov/clinvar/}
\item dbSNP --- \href{https://www.ncbi.nlm.nih.gov/snp/}{https://www.ncbi.nlm.nih.gov/snp/}
\item Genome Aggregation Database (gnomAD)\cite{gnomad}
\item 1000 Genomes Project --- \href{https://www.internationalgenome.org/}{https://www.internationalgenome.org/}
\item Great Middle East allele frequencies (GME)\cite{gme}
\item dbNSFP: Exome Predictions\cite{dbnsfp}
\item dbscSNV: Splice site prediction\cite{dbscsnv}
\item RegSNPIntron: intronic SNVs prediction\cite{regsnpintron}
\end{enumerate}

\paragraph{Фильтрация генетических вариантов.}
Аннотации были агрегированы для удобства использования.
Так, агрегации подверглись:

\begin{itemize}
\item Имена генов по разным БД "--- для облегчения поиска;
\item Описания функциональных классов из разных БД "--- для устранения несоответствий между ними;
\item Ранги инструментов, предсказывающих патогенность генетического варианта.
Трёхранговые системы (патогенный, вероятно патогенный и безвредный) были сведены к двухранговой (патогенный и безвредный).
Отдельно были агрегированы предсказательные инструменты для экзонов, инструменты для интронов и сплайс-вариантов также учитывались отдельно;
\item Ранги инструментов, предсказывающих консервативность нуклеотида.
Эмпирическим путём было подобрано пороговое значение 0.7 "--- нуклеотид считался консервативным, если его предсказанная консервативность была выше, чем у 70\% всех нуклеотидов.
Это максимальное пороговое значение, которое обеспечивает распределение балла агрегатора от минимального до максимального (от 0 до 7 баз данных, считающих данный нуклеотид консервативным);
\item Популяционные частоты "--- из всех имеющихся в базах данных по конкретному генетическому варианту была выбрана максимальная частота.
\end{itemize}

Фильтрация происходила в две стадии:
\begin{enumerate}
\item Фильтрация отдельных генетических вариантов на основе имеющихся аннотаций.
Самая жёсткая фильтрация, которой подвергались все варианты:
\begin{itemize}
\item По глубине покрытия.
Генетический вариант считался существующим, если он присутствовал в двух перекрывающихся парных прочтениях, либо в чётырёх независимых прочтениях;
\item Частота генетического варианта в популяции не более 3\%\cite{ryzhkova}.
\end{itemize}

Прочие фильтры были мягкими "--- генетический вариант отсеивался только в случае несоответствия всем указанным критериям:

\begin{itemize}
\item Присутствие описания связанной с геном патологии в базе данных OMIM;
\item Присутствие генетического варианта в базе данных HGMD;
\item Балл агрегатора патогенности экзомных вариантов не менее 3\cite{ryzhkova};
\item Ранг <<патогенный>> у агрегаторов интронных или сплайс-вариантов;
\item Ранги <<патогенный>> и <<возможно патогенный>> по базе данных CLINVAR;
\item По функциональному классу: сдвиги рамки считывания, потери стоп- и старт-кодонов, нонсенс- и сплайс-варианты.
\end{itemize}

\item Фильтрация значимых вариантов на основе аннотаций гена.
Все эти фильтры были мягкими "--- ген мог соответствовать одному любому из перечисленных критериев:

\begin{itemize}
\item Значение pLI более 0.9, согласно рекомендациям в оригинальной статье\cite{lek};
\item Наследование в гене значится как <<доминантное>> по базе данных OMIM, либо информации о доминантности нет;
\item Любой значимый вариант в гене является гомозиготным;
\item В гене более одного значимого варианта (вероятность цис-транс-положения).
\end{itemize}
\end{enumerate}

\paragraph{Интерпретация.}
Интерпретация данных и составление отчёта производилось в соответствии с рекомендациями Американского колледжа медицинской генетики и геномики (ACMG) и Ассоциации молекулярной патологии\cite{richards}.
В среднем на каждый образец в данных Exo-C приходилось порядка 1--2 тыс. значимых вариантов, затрагивающих около 100--150 генов.
Порядка 100--200 вариантов были результатом систематических ошибок, возникших в ходе приготовления библиотеки или обработки данных.

\paragraph{Пайплайн.}
Для автоматизации и отказоустойчивости протокола был создан пайплайн, интегрирующий все стадии обработки данных секвенирования.
Код доступен на GitHub.

\section{Результаты}

\subsection{Сравнение данных секвенирования клеточной линии K562}

Всего в контрольных образцах было выявлено 5~496~486 генетических вариантов.
Из них 75~328 (1,37\%) найдены в данных из всех восьми статей "--- их было решено использовать как <<золотой стандарт>>.
Наибольший процент уникальных вариантов найден в данных Banaszak et al.


\begin{mytable}{Доля уникальных вариантов в контрольных данных секвенирования}{tab:unique-controls}
\begin{tabular}{| l | l | r | r | r | r |}
\hline
\rowcolor{grayhead}
\textbf{Статья} & 
\textbf{Протокол} & 
\boldbigrow{Глубина секвенирования,}{Глубина секвенирования, прочтений} & 
\boldbigrow{Общее число}{Общее число вариантов} & 
\boldbigrow{Уникальные}{Уникальные варианты} & 
\boldbigrow{Доля уникальных}{Доля уникальных вариантов, \%}
\\
\hline
Banaszak et al. & WES & 254~983~225 & 408~008 & 41~830 & 10,25 \\
Belaghzal et al. & Hi-C & 72~914~268 & 1~399~457 & 27~365 & 1,95 \\
Dixon et al. & WGS & 366~291~496 & 4~649~012 & 327~184 & 7,03 \\
Moquin et al. & Hi-C & 256~500~659 & 2~365~361 & 67~678 & 2,86 \\
Rao et al. & Hi-C & 1~366~228~845 & 4~218~233 & 320~508 & 7,59 \\
Ray et al. & Hi-C & 428~306~794 & 1~789~324 & 89~624 & 5,00 \\
Wang et al. & Repli-seq & 301~663~640 & 2~207~451 & 37~578 & 1,70 \\
Zhou et al. & WGS & 2~621~311~293 & 4~412~455 & 166~451 & 3,77 \\
\hline
\end{tabular}
\end{mytable}

При исключении из выборки данных Banaszak et al. и Belaghzal et al. общими являются 1~091~331 (19,85\%) вариантов.
Их решено было использовать как добавочный (<<серебряный>>) стандарт.

Далее был произведён поиск стандартных вариантов в наших библиотеках.

\begin{mytable}{Данные по наличию стандартных вариантов в наших Exo-C библиотеках}{tab:standard-vs-ours}
\begin{tabular}{| l | r | r | r | r |}
\hline
\rowcolor{grayhead}
\textbf{Параметр} &
\textbf{ExoCBel} &
\textbf{Quarantine} &
\textbf{В обеих} &
\textbf{Ни в одной}
\\
\hline
Общее число вариантов в библиотеке & 3~173~343 & 3~750~319 & --- & --- \\
Вариантов <<золотого стандарта>> & 62~335 & 72~705 & 60~728 & 1~016 \\
Вариантов <<серебряного стандарта>> & 616~375 & 982~858 & 580~351 & 72~449 \\
Доля вариантов <<золотого стандарта>>, \% & 82,75 & 96,52 & 80,62 & 1,35 \\
Доля вариантов <<серебряного стандарта>>, \% & 56,48 & 90,06 & 53,18 & 6,64 \\
\bigrow{Доля вариантов <<серебряного стандарта>>, \%}{Доля <<золотого стандарта>>\newline от всех вариантов библиотеки, \%} & 1,96 & 1,94 & --- & --- \\
\bigrow{Доля вариантов <<серебряного стандарта>>, \%}{Доля <<серебряного стандарта>>\newline от всех вариантов библиотеки, \%} & 19,42 & 26,21 & --- & --- \\
\hline
\end{tabular}
\end{mytable}

При удалении из наших библиотек всех вариантов, в которых менее 4 альтернативных прочтений (использованный нами базовый фильтр), получились следующие результаты:

\begin{mytable}{Данные по наличию стандартных вариантов в наших Exo-C библиотеках (фильтрация по глубине альтернативного варианта)}{tab:standard-vs-ours-4alt}
\begin{tabular}{| l | r | r | r | r |}
\hline
\rowcolor{grayhead}
\textbf{Параметр} &
\textbf{ExoCBel} &
\textbf{Quarantine} &
\textbf{В обеих} &
\textbf{Ни в одной}
\\
\hline
Общее число вариантов в библиотеке & 1~396~525 & 2~577~934 & --- & --- \\
Вариантов <<золотого стандарта>> & 52~732 & 67~270 & 48~840 & 4~166 \\
Вариантов <<серебряного стандарта>> & 391~273 & 821~991 & 340~833 & 218~900 \\
Доля вариантов <<золотого стандарта>>, \% & 70,00 & 89,30 & 64,84 & 5,53 \\
Доля вариантов <<серебряного стандарта>>, \% & 35,85 & 75,32 & 31,23 & 20,06 \\
\bigrow{Доля вариантов <<серебряного стандарта>>, \%}{Доля <<золотого стандарта>>\newline от всех вариантов библиотеки, \%} & 3,78 & 2,61 & --- & --- \\
\bigrow{Доля вариантов <<серебряного стандарта>>, \%}{Доля <<серебряного стандарта>>\newline от всех вариантов библиотеки, \%} & 28,02 & 31,89 & --- & --- \\
\hline
\end{tabular}
\end{mytable}

Сравнение параметров результатов фильтрации по глубине альтернативного аллеля производилось с помощью коэффициента, рассчитываемого по формуле \[\log_{10}\frac{P_{f+}}{P_{f-}},\] где $P_{f+}$ и $P_{f-}$ "--- значения параметра после и до фильтрации соответственно.

\begin{mytable}{Значения коэффициента фильтрации}{tab:filtration-efficiency}
\begin{tabular}{| l | r | r | r | r |}
\hline
\rowcolor{grayhead}
\textbf{Параметр} &
\textbf{ExoCBel} &
\textbf{Quarantine} &
\textbf{В обеих} &
\textbf{Ни в одной}
\\
\hline
Общее число вариантов в библиотеке & --0,3565 & --0,1628 & --- & --- \\
Вариантов <<золотого стандарта>> & --0,0726 & --0,0337 & --0,0946 & +0,6128 \\
Вариантов <<серебряного стандарта>> & --0,1974 & --0,0776 & --0,2311 & +0,4802 \\
\bigrow{Доля вариантов <<серебряного стандарта>>, \%}{Доля <<золотого стандарта>>\newline от всех вариантов библиотеки, \%} & +0,2852 & +0,1288 & --- & --- \\
\bigrow{Доля вариантов <<серебряного стандарта>>, \%}{Доля <<серебряного стандарта>>\newline от всех вариантов библиотеки, \%} & +0,1592 & +0,0852 & --- & --- \\
\hline
\end{tabular}
\end{mytable}

% \subsection{Поиск генетических вариантов у пациентов, связанных с диагнозом}
% 
% Всего были проанализированы Exo-C данные 10 пациентов.
% Из них два образца принадлежали двум сибсам (P2, P4), ещё два "--- матери и ребёнку (P7 и P6 соответственно); эти образцы мы подвергли семейному анализу.
% 
% \begin{small}
% \begin{table}[H]
% \caption{Результаты поиска генетических вариантов в данных пациентов}
% \label{tab:patients-results}
% \begin{tabular}{| p{1.25cm} | p{5cm} | p{\widthof{Гены-кандидаты}} | p{\widthof{Подтверждение}} | p{3cm} |}
% \hline
% Код пациента & Фенотип & Гены-кандидаты со значимыми вариантами & Подтверждение по Сэнгеру & Случайные находки \\
% \hline
% P1 & Врождённые патологии почек, подозрение на муковисцидоз & Не найдены & --- & Не найдены \\
% P2, P4 & Тяжёлая задержка психо-речевого развития, агрессивное поведение & ZMYND1 & Да & Не найдены \\
% P3 &  Задержка психо-речевого развития, ДЦП & GRIN2A & Да & Не найдены \\
% P5 & Задержка психо-речевого развития & TCF20 & Да & Не найдены \\
% P7, P6 & Подозрение на нейрофиброматоз I типа & Не найдены & --- & Не найдены \\
% P8 & Неуточнённый иммунодефицит, нарушения свёртываемости & Не найдены & --- & 4 варианта, связанных с~аритмией, гиперхолестеринемией и~мигренью \\
% P9 & Задержка психомоторного развития, спастический тетрапарез, микроцефалия & NGN & Да & Не найдены \\
% P10 & Задержка психомоторного развития, пороки сердца, микроцефалия, грубые нарушения скелета лицевого черепа и конечностей & Транслокация (данные Exo-C) & --- & Не найдены \\
% \hline
% \end{tabular}
% \end{table}
% \end{small}

\section{Обсуждение результатов}

\subsection{Контрольные образцы}

Как видно из представленных выше данных, образец Banaszak et al. содержит наибольшее число уникальных вариантов (10,25\%).
Это может быть связано с тем, что в этой работе использовались линии клеток, в значительной степени отличающиеся от классической линии K562, либо с тем, что трансгенез затронул не только область гена DNMT3A.

Прослеживается ожидаемая положительная связь между глубиной секвенирования Hi-C библиотек и количеством уникальных вариантов в них.
В двух WGS-библиотеки подобной связи не наблюдается.
Это может быть связано как с ошибками протокола, так и с отличиями использованной линии K562.

\subsection{Оценка наших библиотек}

<<Золотой стандарт>> покрыт нашими библиотеками на 82,75\% и 96,51\%, <<серебряный стандарт>> "--- на 59,51\% и 92,28\% (библиотеки ExoCBel и Quarantine соответственно).
Различия объясняются протоколами приготовления: у библиотеки Quarantine в 6 раз выше обогащение в экзомных районах (критерий Манна--Уитни $U = 0.0, p = 0.0003$).
Кроме того, в библиотеке ExoCBel были использованы адаптерные последовательности, дающие большое количество шума.
Таким образом, если использовать библиотеку Quarantine как хороший пример, можно сказать, что метод ExoC эффективно находит более 95\% вариантов в экзомных районах.

Одним из базовых методов фильтрации генетических вариантов является фильтрация по глубине альтернативного аллеля.
В библиотеке ExoCBel количество вариантов понизилось на 15,4\% и 36,67\%, а в библиотеке  Quarantine "--- на 7,47\% и 41,44\% (для <<золотого стандарта>> и <<серебряного стандарта>> соответственно).
Собственно

\section{Предварительные выводы}

\section{План работы}

\newpage
\appendix

\section{Данные секвенирования K562}

\begin{mytable}{Библиотеки данных секвенирования клеточной линии K562}{appendix:control-libs}
\begin{tabular}{| l | l | l | l | l | l | r | r | r | r | r | r | r | r | r | r | r |}
\hline
\rowcolor{grayhead}
\textbf{Библиотека} &
\textbf{Статья} &
\textbf{Репозиторий} &
\textbf{Коды доступа} &
\textbf{Тип данных} &
\textbf{Тип прочтений} &
\boldbigrow{ прочтений}{Глубина, прочтений} &
\boldbigrow{Общее число}{Общее число прочтений} &
\boldbigrow{Доля картированных,}{Доля картированных, \% от общего числа} &
\boldbigrow{\% от общего числа}{Доля добавочных, \% от общего числа} &
\boldbigrow{Картированные}{Картированные PE прочтения} &
\boldbigrow{Картированные}{Картированные синглетоны} &
\boldbigrow{PE прочтений}{Дубликаты\newline PE прочтений} &
\boldbigrow{синглетонов}{Дубликаты синглетонов} &
\textbf{Доля дубликатов, \%} &
\boldbigrow{Оценка размера}{Оценка размера библиотеки}
\\
\hline
\multicolumn{16}{| c |}{\boldbigrow{\Large Контрольные данные}{\Large Контрольные данные}} \\
\hline
GSM1551618\_HIC069 & Rao et al. & GEO & SRR1658693 & Hi-C & PE & 456~757~799 & 1~001~169~248 & 96,57 & 8,755 & 424~945~100 & 29~290~805 & 17~848~021 & 13~182~626 & 5,56 & 4~916~114~832 \\
GSM1551619\_HIC070 & Rao et al. & GEO & SRR1658694 & Hi-C & PE & 591~854~553 & 1~314~487~595 & 98,7 & 9,949 & 575~565~379 & 15~452~072 & 98~778~796 & 8~811~532 & 17,69 & 1~478~944~337 \\
GSM1551620\_HIC071 & Rao et al. & GEO & \parbox[c][1.3cm]{\widthof{ENCFF004THU  }}{SRR1658695\newline SRR1658696} & Hi-C & PE & 79~905~895 & 173~931~529 & 98,81 & 8,118 & 77~880~938 & 1~975~600 & 486~893 & 269~138 & 0,79 & 6~202~732~721 \\
GSM1551621\_HIC072 & Rao et al. & GEO & \parbox[c][1.3cm]{\widthof{ENCFF004THU  }}{SRR1658697\newline SRR1658698} & Hi-C & PE & 79~578~049 & 159~160~116 & 98,38 & 0,003 & 77~155~821 & 2~265~995 & 366~805 & 285~395 & 0,65 & 8~088~955~029 \\
GSM1551622\_HIC073 & Rao et al. & GEO & \parbox[c][1.3cm]{\widthof{ENCFF004THU  }}{SRR1658699\newline SRR1658700} & Hi-C & PE & 77~353~816 & 154~710~364 & 98,33 & 0,002 & 74~866~287 & 2~383~970 & 240~304 & 293~115 & 0,51 & 11~637~260~975 \\
GSM1551623\_HIC074 & Rao et al. & GEO & \parbox[c][1.3cm]{\widthof{ENCFF004THU  }}{SRR1658702\newline SRR1658701} & Hi-C & PE & 80~778~733 & 175~291~763 & 98,65 & 7,835 & 78~467~294 & 2~254~814 & 644~986 & 321~965 & 1,01 & 4~746~870~162 \\
ENCSR025GPQ & Zhou et al. & ENCODE & \parbox[c][1.8cm]{\widthof{ENCFF004THU  }}{ENCFF574YLG\newline ENCFF921AXL\newline ENCFF590SSX} & WGS & SE & 258~022~356 & 260~044~021 & 85,39 & 0,777 & --- & 220~029~156 & --- & 50~689~083 & 23,04 & --- \\
ENCSR053AXS & Zhou et al. & ENCODE & \parbox[c][2.9cm]{\widthof{ENCFF004THU  }}{ENCFF004THU\newline ENCFF066GQD\newline ENCFF313MGL\newline ENCFF506TKC\newline ENCFF080MQF} & WGS & SE & 1~472~492~722 & 1~592~540~515 & 91,19 & 7,538 & --- & 1~332~175~586 & --- & 496~237~198 & 37,25 & --- \\
ENCSR711UNY & Zhou et al. & ENCODE & \parbox[c][1.8cm]{\widthof{ENCFF004THU  }}{ENCFF471WSA\newline ENCFF826SYZ\newline ENCFF590SSX} & WGS & SE & 890~796~215 & 899~473~769 & 99,72 & 0,965 & --- & 888~239~055 & --- & 203~498~352 & 22,91 & --- \\
SRX3358201 & Dixon et al. & GEO & SRR6251264 & WGS & PE & 366~291~496 & 737~534~099 & 99,72 & 0,671 & 364~794~328 & 923~254 & 73~018~048 & 406~066 & 20,05 & 785~091~005 \\
GSE148362\_G1 & Wang et al. & GEO & SRR11518301 & Repli-seq & SE & 24~804~095 & 24~804~396 & 96,39 & 0,001 & --- & 23~909~072 & --- & 921~353 & 3,85 & --- \\
GSE148362\_G2 & Wang et al. & GEO & SRR11518308 & Repli-seq & SE & 33~032~314 & 33~033~010 & 97,61 & 0,002 & --- & 32~241~907 & --- & 3~881~991 & 12,04 & --- \\
GSE148362\_S1 & Wang et al. & GEO & SRR11518302 & Repli-seq & SE & 30~884~788 & 30~885~298 & 98,7 & 0,002 & --- & 30~481~936 & --- & 2~156~480 & 7,07 & --- \\
GSE148362\_S2 & Wang et al. & GEO & SRR11518303 & Repli-seq & SE & 45~359~273 & 45~360~305 & 98,39 & 0,002 & --- & 44~630~884 & --- & 1~939~846 & 4,35 & --- \\
GSE148362\_S3 & Wang et al. & GEO & SRR11518304 & Repli-seq & SE & 49~807~076 & 49~807~988 & 98,79 & 0,002 & --- & 49~205~535 & --- & 2~889~464 & 5,87 & --- \\
GSE148362\_S4 & Wang et al. & GEO & SRR11518305 & Repli-seq & SE & 44~149~029 & 44~149~770 & 98,46 & 0,002 & --- & 43~469~002 & --- & 2~678~091 & 6,16 & --- \\
GSE148362\_S5 & Wang et al. & GEO & SRR11518306 & Repli-seq & SE & 38~424~060 & 38~424~835 & 97,96 & 0,002 & --- & 37~640~056 & --- & 3~600~260 & 9,57 & --- \\
GSE148362\_S6 & Wang et al. & GEO & SRR11518307 & Repli-seq & SE & 35~203~005 & 35~203~676 & 97,51 & 0,002 & --- & 34~324~742 & --- & 4~177~438 & 12,17 & --- \\
INSITU\_HS1 & Ray et al. & GEO & SRR9019504 & Hi-C & PE & 86~294~895 & 172~589~790 & 93,3 & 0 & 75~521~119 & 9~982~274 & 1~841~061 & 1~615~286 & 3,29 & 1~523~677~153 \\
INSITU\_HS2 & Ray et al. & GEO & SRR9019505 & Hi-C & PE & 127~093~919 & 254~187~838 & 93,36 & 0 & 111~730~240 & 13~858~195 & 1~923~146 & 3~048~273 & 2,91 & 3~208~280~267 \\
INSITU\_NHS1 & Ray et al. & GEO & SRR9019506 & Hi-C & PE & 86~445~594 & 172~891~188 & 93,43 & 0 & 75~893~138 & 9~737~847 & 1~903~981 & 1~649~376 & 3,38 & 1~487~154~386 \\
INSITU\_NHS2 & Ray et al. & GEO & SRR9019507 & Hi-C & PE & 128~472~386 & 256~944~772 & 93,27 & 0 & 112~615~319 & 14~417~076 & 1~961~996 & 3~196~535 & 2,97 & 3~194~317~878 \\
PDDE\_TRANSIENT & Moquin et al. & GEO & \parbox[c][1.3cm]{\widthof{ENCFF004THU  }}{SRR5470541\newline SRR5470540} & Hi-C & PE & 55~158~049 & 110~319~638 & 95,6 & 0,003 & 51~158~920 & 3~140~556 & 3~917~308 & 721~938 & 8,11 & 316~780~447 \\
PD\_STABLE\_REP1 & Moquin et al. & GEO & \parbox[c][1.3cm]{\widthof{ENCFF004THU  }}{SRR5470535\newline SRR5470534} & Hi-C & PE & 67~172~619 & 134~347~099 & 97,58 & 0,001 & 64~767~511 & 1~565~427 & 5~573~966 & 376~260 & 8,79 & 354~373~851 \\
PD\_STABLE\_REP2 & Moquin et al. & GEO & \parbox[c][1.3cm]{\widthof{ENCFF004THU  }}{SRR5470536\newline SRR5470537} & Hi-C & PE & 52~872~167 & 105~745~908 & 98,23 & 0,001 & 51~442~087 & 993~483 & 2~058~449 & 217~598 & 4,17 & 625~522~723 \\
PD\_TRANSIENT & Moquin et al. & GEO & \parbox[c][1.3cm]{\widthof{ENCFF004THU  }}{SRR5470539\newline SRR5470538} & Hi-C & PE & 81~297~824 & 162~600~928 & 95,28 & 0,003 & 75~141~163 & 4~639~787 & 7~298~377 & 1~339~404 & 10,29 & 361~336~652 \\
GSM2588815\_R1 & Belaghzal et al. & GEO & SRR5479813 & Hi-C & PE & 72~914~268 & 172~533~452 & 99,39 & 15,478 & 72~067~575 & 648~294 & 9~694~590 & 210~273 & 13,54 & 243~264~112 \\
GSM2536769\_WT & Banaszak et al. & GEO & SRR5345331 & WES\footnotemark[1] & PE & 39~211~303 & 78~464~649 & 99,46 & 0,054 & 38~914~993 & 171~253 & 7~821~960 & 91~145 & 20,17 & 83~342~746 \\
GSM2536770\_WT\_TF & Banaszak et al. & GEO & SRR5345332 & WES\footnotemark[1] & PE & 49~394~206 & 98~820~633 & 99,54 & 0,033 & 49~068~605 & 193~565 & 10~478~814 & 114~795 & 21,43 & 97~869~629 \\
GSM2536771\_MT2 & Banaszak et al. & GEO & SRR5345333 & WES\footnotemark[1] & PE & 42~020~936 & 84~093~776 & 99,63 & 0,062 & 41~772~436 & 189~177 & 8~755~216 & 104~927 & 21,04 & 85~177~326 \\
GSM2536772\_MT3 & Banaszak et al. & GEO & SRR5345334 & WES\footnotemark[1] & PE & 43~669~613 & 87~375~385 & 99,6 & 0,041 & 43~414~109 & 164~448 & 9~489~133 & 93~601 & 21,92 & 84~242~110 \\
GSM2536773\_MT4 & Banaszak et al. & GEO & SRR5345335 & WES\footnotemark[1] & PE & 39~879~263 & 79~788~847 & 99,53 & 0,038 & 39~609~943 & 166~651 & 8~590~165 & 90~809 & 21,76 & 77~577~055 \\
GSM2536774\_MT5 & Banaszak et al. & GEO & SRR5345336 & WES\footnotemark[1] & PE & 40~807~904 & 81~649~292 & 99,59 & 0,041 & 40~559~969 & 163~957 & 8~801~283 & 91~545 & 21,77 & 79~383~290 \\
\hline
\multicolumn{16}{| c |}{\boldbigrow{\Large Тестовые данные}{\Large Тестовые данные}} \\
\hline
FG\_ExoCBel-001 & ExoCBel & --- & --- & Exo-C & PE & 136~609~179 & 359~215~777 & 99,31 & 23,940 & 135~150~334 & 443~409 & 25~453~568 & 159~152 & 18,86 & 319~784~450 \\
FG\_Quarantine-A & Quarantine & --- & --- & Exo-C & PE & 53~598~130 & 140~214~460 & 99,79 & 23,150 & 53~598~130 & 259~561 & 7~809~282 & 68~779 & 14,60 & 193~853~459 \\
FG\_Quarantine-B & Quarantine & --- & --- & Exo-C & PE & 55~279~173 & 144~641~130 & 99,76 & 23,108 & 55~279~173 & 310~369 & 8~808~307 & 90~489 & 15,97 & 177~375~163 \\
\hline
\end{tabular}
\end{mytable}

\begin{mytable}{Образцы данных секвенирования клеточной линии K562}{appendix:control-samples}
\begin{tabular}{| l | l | l | r | r | r | r | r | r | r | r | r | r |}
\hline
\rowcolor{grayhead}
\textbf{Статья} &
\textbf{Тип данных} &
\textbf{Тип прочтений} &
\boldbigrow{ прочтений}{Глубина, прочтений} &
\boldbigrow{Общее число}{Общее число прочтений} &
\boldbigrow{Доля картированных,}{Доля картированных, \% от общего числа} &
\boldbigrow{\% от общего числа}{Доля добавочных, \% от общего числа} &
\boldbigrow{\% от картированных}{FR PE прочтения,\newline \% от картированных} &
\boldbigrow{Картированные}{Картированные PE прочтения} &
\boldbigrow{Картированные}{Картированные синглетоны} &
\boldbigrow{Картированные на разные хромосомы}{Картированные на разные хромосомы пары, \% от картированных} &
\boldbigrow{Картированные на разные хромосомы пары (QMAP 4+),}{Картированные на разные хромосомы пары (QMAP 4+), \% от картированных на разные хромосомы} 
\\
\hline
\multicolumn{12}{| c |}{\boldbigrow{\Large Контрольные данные}{\Large Контрольные данные}} \\
\hline
Rao et al. & Hi-C & PE & 1~366~228~845 & 2~978~750~615 & 97,95 & 8,268 & 27,04 & 2~617~761~638 & 53~623~256 & 21,03 & 84,23 \\
Zhou et al. & WGS & SE & 2~621~311~293 & 2~752~058~305 & 93,43 & 4,751 & --- & --- & --- & --- & --- \\
Dixon et al. & WGS & PE & 366~291~496 & 737~534~099 & 99,72 & 0,671 & 97,16 & 729~588~656 & 923~254 & 1,25 & 51,22 \\
Wang et al. & Repli-seq & SE & 301~663~640 & 301~669~278 & 98,09 & 0,002 & --- & --- & --- & --- & --- \\
Ray et al. & Hi-C & PE & 428~306~794 & 856~613~588 & 93,33 & 0 & 35,92 & 751~519~632 & 47~995~392 & 22,77 & 76,00 \\
Moquin et al. & Hi-C & PE & 256~500~659 & 513~013~573 & 96,56 & 0,002 & 46,64 & 485~019~362 & 10~339~253 & 17,76 & 75,56 \\
Belaghzal et al. & Hi-C & PE & 72~914~268 & 172~533~452 & 99,39 & 15,478 & 24,77 & 144~135~150 & 648~294 & 34,02 & 88,02 \\
Banaszak et al. & WES & PE & 254~983~225 & 510~192~582 & 99,56 & 0,044 & 99,41 & 506~680~110 & 1~049~051 & 0,11 & 81,38 \\
\hline
\rowcolor{white}
\multicolumn{12}{| c |}{\boldbigrow{\Large Тестовые данные}{\Large Тестовые данные}} \\
\hline
ExoCBel & Exo-C & PE & 136~609~179 & 359~215~777 & 99,31 & 23,94 & 89,22 & 270~300~668 & 443~409 & 5,01 & 66,93 \\
Quarantine & Exo-C & PE & 109~486~529 & 284~855~590 & 99,77 & 23,13 & 70,02 & 217~754~606 & 569~930 & 5,87 & 78,00 \\
\hline
\end{tabular}
\end{mytable}


\footnotetext[1]{Варианты в гене DNMT3A были исключены из выборки.}

% На крайняк - https://www.encodeproject.org/search/?type=Experiment&assay_term_name=ChIA-PET&replicates.library.biosample.donor.organism.scientific_name=Homo%20sapiens&biosample_ontology.term_name=K562&status=released 

\begin{thebibliography}{300}

% ---- GATK RECOMMENDATIONS ----

\bibitem{gatk}
Van der Auwera GA, Carneiro MO, Hartl C, et al. From FastQ data to high confidence variant calls: the Genome Analysis Toolkit best practices pipeline. Curr Protoc Bioinformatics. 2013;43(1110):11.10.1-11.10.33. doi:10.1002/0471250953.bi1110s43

\bibitem{heldenbrand}
Heldenbrand JR, Baheti S, Bockol MA, et al. Recommendations for performance optimizations when using GATK3.8 and GATK4 [published correction appears in BMC Bioinformatics. 2019 Dec 17;20(1):722]. BMC Bioinformatics. 2019;20(1):557. Published 2019 Nov 8. doi:10.1186/s12859-019-3169-7

% ---- K562 DATA ----

\bibitem{rao}
Rao SS, Huntley MH, Durand NC, et al. A 3D map of the human genome at kilobase resolution reveals principles of chromatin looping [published correction appears in Cell. 2015 Jul 30;162(3):687-8]. Cell. 2014;159(7):1665-1680. doi:10.1016/j.cell.2014.11.021

\bibitem{zhou}
Zhou B, Ho SS, Greer SU, et al. Comprehensive, integrated, and phased whole-genome analysis of the primary ENCODE cell line K562. Genome Res. 2019;29(3):472-484. doi:10.1101/gr.234948.118

\bibitem{dixon}
Dixon JR, Xu J, Dileep V, et al. Integrative detection and analysis of structural variation in cancer genomes. Nat Genet. 2018;50(10):1388-1398. doi:10.1038/s41588-018-0195-8

\bibitem{wang}
Yuchuan Wang, Yang Zhang, et al. SPIN reveals genome-wide landscape of nuclear compartmentalization. bioRxiv 2020.03.09.982967; doi: https://doi.org/10.1101/2020.03.09.982967

\bibitem{ray}
Ray J, Munn PR, Vihervaara A, et al. Chromatin conformation remains stable upon extensive transcriptional changes driven by heat shock. Proc Natl Acad Sci U S A. 2019;116(39):19431-19439. doi:10.1073/pnas.1901244116

\bibitem{moquin}
Moquin SA, Thomas S, Whalen S, et al. The Epstein-Barr Virus Episome Maneuvers between Nuclear Chromatin Compartments during Reactivation. J Virol. 2018;92(3):e01413-17. Published 2018 Jan 17. doi:10.1128/JVI.01413-17

\bibitem{belaghzal}
Belaghzal H, Dekker J, Gibcus JH. Hi-C 2.0: An optimized Hi-C procedure for high-resolution genome-wide mapping of chromosome conformation. Methods. 2017;123:56-65. doi:10.1016/j.ymeth.2017.04.004

\bibitem{banaszak}
Banaszak LG, Giudice V, Zhao X, et al. Abnormal RNA splicing and genomic instability after induction of DNMT3A mutations by CRISPR/Cas9 gene editing. Blood Cells Mol Dis. 2018;69:10-22. doi:10.1016/j.bcmd.2017.12.002

% ---- TOOLS ----

\bibitem{annovar}
Wang K, Li M, Hakonarson H. ANNOVAR: functional annotation of genetic variants from high-throughput sequencing data. Nucleic Acids Res. 2010;38(16):e164. doi:10.1093/nar/gkq603

\bibitem{vep}
McLaren, W., Gil, L., Hunt, S.E. et al. The Ensembl Variant Effect Predictor. Genome Biol 17, 122 (2016). doi: 10.1186/s13059-016-0974-4

\bibitem{cutadapt}
MARTIN, Marcel. Cutadapt removes adapter sequences from high-throughput sequencing reads. EMBnet.journal, [S.l.], v. 17, n. 1, p. pp. 10-12, may 2011. ISSN 2226-6089. doi: 10.14806/ej.17.1.200. 

\bibitem{fastqc}
Andrews, S. (2010). FastQC:  A Quality Control Tool for High Throughput Sequence Data [Online]. Available online at: http://www.bioinformatics.babraham.ac.uk/projects/fastqc/

\bibitem{bowtie2}
Langmead, B., Salzberg, S. Fast gapped-read alignment with Bowtie 2. Nat Methods 9, 357–359 (2012). doi: 10.1038/nmeth.1923

\bibitem{bwa}
Li H, Durbin R. Fast and accurate short read alignment with Burrows-Wheeler transform. Bioinformatics. 2009;25(14):1754-1760. doi:10.1093/bioinformatics/btp324

\bibitem{picard}
''Picard Toolkit.'' 2019. Broad Institute, GitHub Repository. http://broadinstitute.github.io/picard/; Broad Institute

\bibitem{bedtools}
Quinlan AR and Hall IM, 2010. BEDTools: a flexible suite of utilities for comparing genomic features. Bioinformatics. 26, 6, pp. 841–842.

% ---- DATABASES ----

\bibitem{regsnpintron}
Lin, H., Hargreaves, K.A., Li, R. et al. RegSNPs-intron: a computational framework for predicting pathogenic impact of intronic single nucleotide variants. Genome Biol 20, 254 (2019). doi:10.1186/s13059-019-1847-4

\bibitem{dbnsfp}
Liu X, Wu C, Li C, Boerwinkle E. dbNSFP v3.0: A One-Stop Database of Functional Predictions and Annotations for Human Nonsynonymous and Splice-Site SNVs. Hum Mutat. 2016;37(3):235-241. doi:10.1002/humu.22932

\bibitem{gme}
Scott EM, Halees A, Itan Y, et al. Characterization of Greater Middle Eastern genetic variation for enhanced disease gene discovery. Nat Genet. 2016;48(9):1071-1076. doi:10.1038/ng.3592

\bibitem{gnomad}
Karczewski, K.J., Francioli, L.C., Tiao, G. et al. The mutational constraint spectrum quantified from variation in 141,456 humans. Nature 581, 434–443 (2020). doi:10.1038/s41586-020-2308-7

\bibitem{hgmd}
Stenson PD, Mort M, Ball EV, et al. The Human Gene Mutation Database: towards a comprehensive repository of inherited mutation data for medical research, genetic diagnosis and next-generation sequencing studies. Hum Genet. 2017;136(6):665-677. doi:10.1007/s00439-017-1779-6

\bibitem{omim}
Amberger JS, Bocchini CA, Schiettecatte F, Scott AF, Hamosh A. OMIM.org: Online Mendelian Inheritance in Man (OMIM®), an online catalog of human genes and genetic disorders. Nucleic Acids Res. 2015;43(Database issue):D789-D798. doi:10.1093/nar/gku1205

\bibitem{dbscsnv}
Jian X, Boerwinkle E, Liu X. In silico tools for splicing defect prediction: a survey from the viewpoint of end users. Genet Med. 2014;16(7):497-503. doi:10.1038/gim.2013.176

\bibitem{abraom}
Naslavsky MS, Yamamoto GL, de Almeida TF, Ezquina SAM, Sunaga DY, Pho N, Bozoklian D, Sandberg TOM, Brito LA, Lazar M, Bernardo DV, Amaro E Jr, Duarte YAO, Lebrão ML, Passos-Bueno MR, Zatz M. Exomic variants of an elderly cohort of Brazilians in the ABraOM database. Hum Mutat. 2017 Jul;38(7):751-763. doi: 10.1002/humu.23220.

% ---- OTHER ----

\bibitem{yohe}
Yohe S, Thyagarajan B. Review of Clinical Next-Generation Sequencing. Arch Pathol Lab Med. 2017 Nov;141(11):1544-1557. doi: 10.5858/arpa.2016-0501-RA

\bibitem{balloux}
Balloux F, Brønstad Brynildsrud O, van Dorp L, et al. From Theory to Practice: Translating Whole-Genome Sequencing (WGS) into the Clinic. Trends Microbiol. 2018;26(12):1035-1048. doi:10.1016/j.tim.2018.08.004

\bibitem{sanger}
Sanger F, Nicklen S, Coulson AR. DNA sequencing with chain-terminating inhibitors. Proc Natl Acad Sci U S A. 1977;74(12):5463-5467. doi:10.1073/pnas.74.12.5463

\bibitem{burrows}
Burrows M, Wheeler DJ. Technical report 124. Palo Alto, CA: Digital Equipment Corporation; 1994. A block-sorting lossless data compression algorithm.

\bibitem{ebbert}
Ebbert MT, Wadsworth ME, Staley LA, et al. Evaluating the necessity of PCR duplicate removal from next-generation sequencing data and a comparison of approaches. BMC Bioinformatics. 2016;17 Suppl 7(Suppl 7):239. Published 2016 Jul 25. doi:10.1186/s12859-016-1097-3

\bibitem{richards}
Richards S, Aziz N, Bale S, et al. Standards and guidelines for the interpretation of sequence variants: a joint consensus recommendation of the American College of Medical Genetics and Genomics and the Association for Molecular Pathology. Genet Med. 2015;17(5):405-424. doi:10.1038/gim.2015.30

\bibitem{lek}
Lek M, Karczewski KJ, Minikel EV, et al. Analysis of protein-coding genetic variation in 60,706 humans. Nature. 2016;536(7616):285-291. doi:10.1038/nature19057

\bibitem{ziegler}
Ziegler A, Colin E, Goudenège D, Bonneau D. A snapshot of some pLI score pitfalls. Hum Mutat. 2019 Jul;40(7):839-841. doi: 10.1002/humu.23763

\bibitem{saleheen}
Saleheen D, Natarajan P, Armean IM, et al. Human knockouts and phenotypic analysis in a cohort with a high rate of consanguinity. Nature. 2017;544(7649):235-239. doi:10.1038/nature22034

\bibitem{wutz}
Wutz G, Várnai C, Nagasaka K, et al. Topologically associating domains and chromatin loops depend on cohesin and are regulated by CTCF, WAPL, and PDS5 proteins. EMBO J. 2017;36(24):3573-3599. doi:10.15252/embj.201798004

\bibitem{yu}
Yu D, Zhang K, Han M, et al. Noninvasive prenatal testing for fetal subchromosomal copy number variations and chromosomal aneuploidy by low-pass whole-genome sequencing. Mol Genet Genomic Med. 2019;7(6):e674. doi:10.1002/mgg3.674

\bibitem{herder}
Herder M. What Is the Purpose of the Orphan Drug Act?. PLoS Med. 2017;14(1):e1002191. Published 2017 Jan 3. doi:10.1371/journal.pmed.1002191

\bibitem{richter}
Richter T, Nestler-Parr S, Babela R, Khan ZM, Tesoro T, Molsen E, Hughes DA; International Society for Pharmacoeconomics and Outcomes Research Rare Disease Special Interest Group. Rare Disease Terminology and Definitions-A Systematic Global Review: Report of the ISPOR Rare Disease Special Interest Group. Value Health. 2015 Sep;18(6):906-14. doi: 10.1016/j.jval.2015.05.008

\bibitem{lancet}
The Lancet Neurology. Rare neurological diseases: a united approach is needed. Lancet Neurol. 2011 Feb;10(2):109. doi: 10.1016/S1474-4422(11)70001-1. Erratum in: Lancet Neurol. 2011 Mar;10(3):205.

\bibitem{huber}
D. Huber, L. Voith von Voithenberg, G.V. Kaigala. Fluorescence in situ hybridization (FISH): History, limitations and what to expect from micro-scale FISH? Micro and Nano Engineering, Volume 1, 2018, Pages 15-24, ISSN 2590-0072. doi: 10.1016/j.mne.2018.10.006.

\bibitem{theisen}
Theisen, A. (2008) Microarray-based Comparative Genomic Hybridization (aCGH). Nature Education 1(1):45

\bibitem{stuppia}
Stuppia L, Antonucci I, Palka G, Gatta V. Use of the MLPA assay in the molecular diagnosis of gene copy number alterations in human genetic diseases. Int J Mol Sci. 2012;13(3):3245-3276. doi:10.3390/ijms13033245

\bibitem{dekker}
Dekker J, Rippe K, Dekker M, Kleckner N. Capturing chromosome conformation. Science. 2002 Feb 15;295(5558):1306-11. doi: 10.1126/science.1067799. PMID: 11847345.

\bibitem{oluwadare}
Oluwadare, O., Highsmith, M. \& Cheng, J. An Overview of Methods for Reconstructing 3-D Chromosome and Genome Structures from Hi-C Data. Biol Proced Online 21, 7 (2019). doi: 10.1186/s12575-019-0094-0

\bibitem{brea-fernandez}
Alejandro j. Brea-Fernandez, Marta Ferro, Ceres Fernandez-Rozadilla, Ana Blanco, Laura Fachal, Marta Santamarina, Ana Vega, Alejandro Pazos, Angel Carracedo and Clara Ruiz-Ponte, An Update of In Silico Tools for the Prediction of Pathogenesis in Missense Variants, Current Bioinformatics (2011) 6: 185. https://doi.org/10.2174/1574893611106020185 

\bibitem{dbsnp}
https://www.ncbi.nlm.nih.gov/snp/

\bibitem{mozheiko}
Mozheiko, E.A., Fishman, V.S. Detection of Point Mutations and Chromosomal Translocations Based on Massive Parallel Sequencing of Enriched 3C Libraries. Russ J Genet 55, 1273–1281 (2019). https://doi.org/10.1134/S1022795419100089

\bibitem{sampson}
B. Sampson, A. McGuire. Genetics and the Molecular Autopsy. Editor(s): Linda M. McManus, Richard N. Mitchell. Pathobiology of Human Disease, Academic Press, 2014. Pages 3459--3467. ISBN 9780123864574. doi: 10.1016/B978-0-12-386456-7.06707-1.

\bibitem{gresham}
Gresham, D., Dunham, M. \& Botstein, D. Comparing whole genomes using DNA microarrays. Nat Rev Genet 9, 291–-302 (2008). https://doi.org/10.1038/nrg2335

\bibitem{ryzhkova}
Ryzhkova O.P., Kardymon O.L., Prohorchuk E.B., Konovalov F.A., Maslennikov A.B., Stepanov V.A., Afanasyev A.A., Zaklyazminskaya E.V., Kostareva A.A., Pavlov A.E., Golubenko M.V., Polyakov A.V., Kutsev S.I. Guidelines for the interpretation of massive parallel sequencing variants. Medical Genetics. 2017;16(7):4-17. (In Russ.) 

\bibitem{galanello}
Galanello R, Origa R. Beta-thalassemia. Orphanet J Rare Dis. 2010;5:11. Published 2010 May 21. doi:10.1186/1750-1172-5-11

\bibitem{bomar}
Bomar, J., Benke, P., Slattery, E. et al. Mutations in a novel gene encoding a CRAL-TRIO domain cause human Cayman ataxia and ataxia/dystonia in the jittery mouse. Nat Genet 35, 264–269 (2003). https://doi.org/10.1038/ng1255

\bibitem{jin}
Jin Z, Liu Y. DNA methylation in human diseases. Genes Dis. 2018;5(1):1-8. Published 2018 Jan 31. doi:10.1016/j.gendis.2018.01.002

\bibitem{tang}
Tang, B., Dean, B. \& Thomas, E. Disease- and age-related changes in histone acetylation at gene promoters in psychiatric disorders. Transl Psychiatry 1, e64 (2011). https://doi.org/10.1038/tp.2011.61

\bibitem{spielmann}
Spielmann, M., Lupiáñez, D.G. \& Mundlos, S. Structural variation in the 3D genome. Nat Rev Genet 19, 453–467 (2018). https://doi.org/10.1038/s41576-018-0007-0

\bibitem{winklhofer}
Winklhofer KF, Tatzelt J, Haass C. The two faces of protein misfolding: gain- and loss-of-function in neurodegenerative diseases. EMBO J. 2008;27(2):336-349. doi:10.1038/sj.emboj.7601930

\bibitem{gonzalez}
Gonzalez E, Kulkarni H, Bolivar H, Mangano A, Sanchez R, Catano G, Nibbs RJ, Freedman BI, Quinones MP, Bamshad MJ, Murthy KK, Rovin BH, Bradley W, Clark RA, Anderson SA, O'connell RJ, Agan BK, Ahuja SS, Bologna R, Sen L, Dolan MJ, Ahuja SK. The influence of CCL3L1 gene-containing segmental duplications on HIV-1/AIDS susceptibility. Science. 2005 Mar 4;307(5714):1434-40. doi: 10.1126/science.1101160.

\bibitem{sebat}
Sebat J, Lakshmi B, Malhotra D, Troge J, Lese-Martin C, Walsh T, Yamrom B, Yoon S, Krasnitz A, Kendall J, Leotta A, Pai D, Zhang R, Lee YH, Hicks J, Spence SJ, Lee AT, Puura K, Lehtimäki T, Ledbetter D, Gregersen PK, Bregman J, Sutcliffe JS, Jobanputra V, Chung W, Warburton D, King MC, Skuse D, Geschwind DH, Gilliam TC, Ye K, Wigler M. Strong association of de novo copy number mutations with autism. Science. 2007 Apr 20;316(5823):445-9. doi: 10.1126/science.1138659.

\bibitem{acuna-higaldo}
Acuna-Hidalgo, R., Veltman, J.A. \& Hoischen, A. New insights into the generation and role of de novo mutations in health and disease. Genome Biol 17, 241 (2016). https://doi.org/10.1186/s13059-016-1110-1

\bibitem{schreck}
Schreck RR, Distèche CM. Chromosome banding techniques. Curr Protoc Hum Genet. 2001 May;Chapter 4:Unit4.2. doi: 10.1002/0471142905.hg0402s00. PMID: 18428280.

\bibitem{guo}
Guo B, Han X, Wu Z, Da W, Zhu H. Spectral karyotyping: an unique technique for the detection of complex genomic rearrangements in leukemia. Transl Pediatr. 2014;3(2):135-139. doi:10.3978/j.issn.2224-4336.2014.01.02

\bibitem{goodwin}
Goodwin, S., McPherson, J. \& McCombie, W. Coming of age: ten years of next-generation sequencing technologies. Nat Rev Genet 17, 333–351 (2016). https://doi.org/10.1038/nrg.2016.49

\bibitem{supernat}
Supernat, A., Vidarsson, O.V., Steen, V.M. et al. Comparison of three variant callers for human whole genome sequencing. Sci Rep 8, 17851 (2018). https://doi.org/10.1038/s41598-018-36177-7

\bibitem{field}
Institute of Medicine (US) Committee on Palliative and End-of-Life Care for Children and Their Families; Field MJ, Behrman RE, editors. When Children Die: Improving Palliative and End-of-Life Care for Children and Their Families. Washington (DC): National Academies Press (US); 2003. CHAPTER 2, PATTERNS OF CHILDHOOD DEATH IN AMERICA. Available from: https://www.ncbi.nlm.nih.gov/books/NBK220806/

\bibitem{teer}
Teer JK, Mullikin JC. Exome sequencing: the sweet spot before whole genomes. Hum Mol Genet. 2010;19(R2):R145-R151. doi:10.1093/hmg/ddq333

\bibitem{barbitoff}
Barbitoff, Y.A., Polev, D.E., Glotov, A.S. et al. Systematic dissection of biases in whole-exome and whole-genome sequencing reveals major determinants of coding sequence coverage. Sci Rep 10, 2057 (2020). https://doi.org/10.1038/s41598-020-59026-y

\bibitem{hiroi}
Hiroi, N., Agatsuma, S. Genetic susceptibility to substance dependence. Mol Psychiatry 10, 336–344 (2005). https://doi.org/10.1038/sj.mp.4001622

\bibitem{jett}
Jett, K., Friedman, J. Clinical and genetic aspects of neurofibromatosis 1. Genet Med 12, 1–11 (2010). https://doi.org/10.1097/GIM.0b013e3181bf15e3

\bibitem{o-connor}
O'Connor, C. (2008) Human chromosome translocations and cancer. Nature Education 1(1):56

\bibitem{rizvi}
Rizvi AA. 46, XX man with SRY gene translocation: cytogenetic characteristics, clinical features and management. Am J Med Sci. 2008 Apr;335(4):307-9. doi: 10.1097/MAJ.0b013e31811ec1b4

\bibitem{dong}
Dong Y, Du RC, Jiang YT, Wu J, Li LL, Liu RZ. Impact of chromosomal translocations on male infertility, semen quality, testicular volume and reproductive hormone levels. J Int Med Res. 2012;40(6):2274-83. doi: 10.1177/030006051204000625

\bibitem{asim}
Asim A, Kumar A, Muthuswamy S, Jain S, Agarwal S. "Down syndrome: an insight of the disease". J Biomed Sci. 2015 Jun 11;22(1):41. doi: 10.1186/s12929-015-0138-y. PMID: 26062604

\end{thebibliography}

\end{document}
