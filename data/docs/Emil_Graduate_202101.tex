\documentclass[a4paper,12pt]{article}
\usepackage{polyglossia} 
\setdefaultlanguage[babelshorthands=true]{russian}
\setotherlanguage{english}
\defaultfontfeatures{Ligatures=TeX,Mapping=tex-text}

\usepackage{amsmath}
\setmainfont{Liberation Serif}

\usepackage[square,sort,comma,numbers]{natbib}
\bibliographystyle{plainnat}
\usepackage{xltxtra}
\usepackage{microtype}
\usepackage{multirow,tabularx}

\usepackage{xcolor}
\definecolor{darkblue}{HTML}{003153}
\usepackage{graphicx}
\usepackage {textcomp}
\usepackage[twoside,left=2.5cm,right=3cm,top=3cm,bottom=4cm,bindingoffset=0cm]{geometry}
\usepackage{epigraph}
\renewcommand{\epigraphsize}{\footnotesize}
\epigraphrule=0pt
\epigraphwidth=8cm
\usepackage{soul, xspace}
\xspaceaddexceptions{ < ) }
\sodef\so{}{.1em}{1em}{.3em plus.05em minus.05em}
\newcommand{\ldotst}{\so{...}\xspace}
\newcommand{\ldotsq}{\so{?\hbox{\hspace{-.212em}}..}\xspace}
\newcommand{\ldotse}{\so{!..}\xspace}
\newcommand{\razd}{~\\{\centering\Large\bfseries$\ast \ast \ast$\par}~\\}
\newcommand{\spacing}{\textcolor{red}{\textbf{(разрыв)}}}
\newcommand{\authornote}{\emph{"--- Прим. авт.}}

\renewcommand{\texttt}[1]{\text}

\usepackage{etoolbox}
\AtBeginEnvironment{quote}{}
\makeatletter
\newlength\episourceskip
\pretocmd{\@episource}{\em}{}{}
\apptocmd{\@episource}{\em}{}{}
\patchcmd{\epigraph}{\@episource{#1}\\}{\@episource{#1}\\[\episourceskip]}{}{}
\makeatother

\usepackage{hyperref}
\hypersetup{colorlinks=true, linkcolor=darkblue, citecolor=darkblue, filecolor=darkblue, urlcolor=darkblue}

\usepackage{fancyhdr}
\pagestyle{fancy}
\fancyhead[LE,RO]{\thepage}
\fancyhead[LO]{{\small
Дипломная работа
}}
\fancyhead[RE]{{\small
Эмиль Валеев
}} 
\fancyfoot{}
\fancypagestyle{plain}
{
\fancyhead{}
\renewcommand{\headrulewidth}{0mm}
\fancyfoot{}
}
\fancypagestyle{requisit}
{
\fancyhead{}
\renewcommand{\headrulewidth}{0mm}
}

\begin{document}

\begin{titlepage}

\begin{center}
 \begin{small}
ФЕДЕРАЛЬНОЕ ГОСУДАРСТВЕННОЕ АВТОНОМНОЕ ОБРАЗОВАТЕЛЬНОЕ УЧРЕЖДЕНИЕ ВЫСШЕГО ОБРАЗОВАНИЯ <<НОВОСИБИРСКИЙ НАЦИОНАЛЬНЫЙ ИССЛЕДОВАТЕЛЬСКИЙ ГОСУДАРСТВЕННЫЙ УНИВЕРСИТЕТ>> (НОВОСИБИРСКИЙ ГОСУДАРСТВЕННЫЙ УНИВЕРСИТЕТ, НГУ)

~

Институт медицины и психологии В. Зельмана НГУ
\end{small}

\vspace{5cm}

\begin{LARGE}{\textbf{КУРСОВАЯ РАБОТА}}\end{LARGE}

\vspace{0.5cm}

Валеев Эмиль Салаватович

Группа 12452

\vspace{0.5cm}

Тема работы: <<Разработка инструментов для поиска клинически значимых полиморфизмов в геноме человека на основе данных секвенирования 3C-библиотек>>

\end{center}

\vfill

\hfill
\begin{minipage}{0.47\textwidth}

\textbf{Научный руководитель:}

Фишман Вениамин Семенович,

к.б.н., ведущий научный сотрудник,

заведующий Сектором геномных

механизмов онтогенеза, ИЦиГ СО РАН

\vspace{1cm}

ФИО: \hrulefill/\hrulefill

<<\rule{2em}{0.4pt}>>\hrulefill20\rule{2em}{0.4pt} г.

\vspace{1cm}

Оценка: \hrulefill

\end{minipage}

\vfill

{\centering \begin{small}Новосибирск, 2020\end{small}\par}

\end{titlepage}

\tableofcontents

\section{Введение}

\subsection{Актуальность}

\subsection{Цели}

\subsection{Задачи}

\section{Обзор литературы}

Генетические болезни (моногенные заболевания, геномные структурные дефекты, вариации числа копий) "--- это основная причина смертности детей до 10 лет.

Генетическое детерминирование патологий

\paragraph{Частые и редкие (орфанные) патологии.}
Генетические патологии делятся на группы по частоте встречаемости в популяции "--- частые и редкие (орфанные).
Определения орфанных заболеваний могут различаться "--- например, в США, согласно ``Health Promotion and Disease Prevention Amendments of 1984'', редкими считаются патологии, поражающие менее 200 тыс. населения страны (примерно 1 : 1630 при текущей численности населения в 326 млн человек)\cite{herder}.
Европейское Медицинское Агентство определяет границу как 1 : 2000.
Систематический анализ показал, что существует более 290 определений, и среднее значение находится в интервале 40--50 на 100 тыс. населения\cite{richter}.

Некоторые заболевания могут быть орфанными в одной популяции и частыми в другой (эффект основателя).
Например, бета-талассемия в Средиземноморье.

Несмотря на то, что каждое из орфанных заболеваний само по себе встречается редко, в сумме они поражают значительный процент населения (предположительно 5--8\% европейской популяции).
Общее число орфанных болезней неизвестно по причине недостатков стандартизации, наиболее частая оценка "--- 5000--8000.
Около 80\% редких болезней имеют генетическую природу и начинаются в раннем детстве\cite{lancet}.

Основные источники информации по орфанным заболеваниям:

\begin{enumerate}
\item Global Genes
\item Online Mendelian Inheritance in Man (OMIM)\cite{omim}
\item Orphadata
\end{enumerate}

\subsection{Механизмы развития патологий}

\paragraph{Структура белка.}
Самым очевидным способом является изменение структуры белка, вызванное заменами аминокислот, сдвигами рамок считывания и нарушением сплайсинга.

\paragraph{Эпигенетика.}
Также патологии могут развиваться из-за изменения экспрессии, вызванных эпигенетическими механизмами, не затрагивающими непосредственно последовательность ДНК генов.
К таким механизмам можно отнести, например, метилирование ДНК, ацетилирование гистонов.
Кроме того, на экспрессию в значительной степени влияет трёхмерная структура хроматина, регулируемая механизмами loop extrusion, block copolymers, фазовая сепарация.

\subsection{Типы генетических аномалий, лежащих в основе патологий}

\paragraph{Хромосомные аномалии.}
Основными типами являются:
\begin{itemize}
\item Анэуплоидии "--- изменение числа хромосом.
Самые известные "--- синдром Дауна, Эдвардса, Патау, Тёрнера.
\item Инверсии "--- переворот фрагмента хромосомы на 180 градусов.
Крупные инверсии могут быть причиной изменения экспрессии генов, а также запирания кроссинговера и образования гаплогрупп.
\item Транслокации "--- перемещение фрагмента хромосомы на другое плечо хромосомы, либо на другую хромосому.
\end{itemize}

\paragraph{Вариации числа копий (CNV).}
К ним относятся дупликации (мультипликации) и делеции.

\paragraph{Точечные полиморфизмы (SNV).}
Изменения отдельных букв в геноме.

\paragraph{Короткие инсерции и делеции (indels).}
Вставки и потери 20--50 нуклеотидов.

\subsection{Функциональные классы вариантов}

Внутригенные варианты могут находиться в:

\begin{itemize}
\item Нетранслируемых областях (3' и 5' UTR), вовлечённых в регуляцию транскрипции, трансляции и деградации транскрипта.
\item Экзонах, непосредственно отвечающих за последовательность белка.
SNP могут быть синонимичными (без замены аминокислоты) и несинонимичными "--- миссенс (замена на другую АК), нонсенс (замена на стоп-кодон) либо сдвиг рамки считывания, приводящий к изменению значительной части белковой молекулы.

Миссенс-варианты редко приводят к поломке белка, но они могут повлиять на экспрессию гена, если замена пришлась на регуляторный мотив\cite{brea-fernandez}.

\item Интронах, которые содержат регуляторные области и сплайс-сайты, необходимые для процессинга транскрипта в готовую мРНК.
\end{itemize}

Внегенные SNP могут приходиться на различные регуляторные последовательности, например, энхансеры, сайленсеры.
Также известно, что за трёхмерную структуру хроматина отвечают в том числе и специфические белки, связывающиеся с ДНК "--- например, CTCF\cite{wutz}.
Варианты, приходящиеся на сайты связывания CTCF, могут разрушать границы ТАДов и вызывать изменения экспрессии.

\subsection{Методы детектирования}

\paragraph{Кариотипирование.}
Данный метод представляет собой микроскопическое исследование клеток, остановленных на стадии метафазы.
Их ядра растворяются, хромосомы окрашиваются различными красителями, разделяются и изучаются на предмет формы, количества и наличия перестроек.
Кариотипирование "--- рутинная методика при диагностике врождённых патологий, аутопсии мертворожденных и злокачественных образований кроветворного ряда.

Ограничения кариотипирования "--- обязательно требуются живые клетки, также на эффективность влияет размер перестроек и процент поражённых клеток в образце (минимум 5--10\%)\cite{sampson}.

\paragraph{Флуоресцентная \textit{in situ} гибридизация (FISH).}
Основой является гибридизация содержащих флуоресцентную метку последовательностей с комплементарными ими участками НК.
Гибридизация может производиться на ДНК (метафазные или интерфазные хромосомы) и на РНК.
FISH позволяет определить количественные характеристики НК и их пространственное расположение в ядре.
Метод является <<золотым стандартом>> в определении хромосомных патологий "--- как в клетках с врождёнными перестройками, так и в клетках опухолей.

Данные при FISH можно получить как за счёт спектрального анализа сигналов, так и за счёт их отсутствия/присутствия.
Всего 7 флюорофоров дают 127 вариантов цветовых меток; это позволяет реализовать, например, спектральное кариотипирование.
В методике MER-FISH количество цветовых меток увеличено до 1001.
Тем не менее, лимитирующими факторами остаются:

\begin{itemize}
\item потребность в хорошо обученном персонале. 
Протокол FISH зависит от характера пробы и образца, и должен быть настроен эмпирически;
\item цена реактивов;
\item время гибридизации.
Кинетика реакций гибридизации в ядре изучена недостаточно, и требуется достаточно долгое время, чтобы получить сигналы, которые можно измерить и сравнить между собой.
\end{itemize}

В настоящее время методика FISH значительно усложнилась.
Биотехнологические компании предлагают панели олигонуклеотидов для определённых целей, определяющие участки от десятков килобаз до мегабаз, а также олиги с высокой чувствительностью, позволяющие определить сплайс-варианты и даже SNP.
Разрабатываются технологии micro-FISH ($\mu$FISH), сочетающие FISH с микрофлюидными технологиями.
При этом процесс удешевляется, автоматизируется, ускоряется (за счёт уменьшения объёмов, а соответственно, времени гибридизации) и упрощается для использования в обширных исследованиях и для внедрения в клинику\cite{huber}.

\paragraph{CGH.}
Как и в случае с FISH, основой метода является флуоресцентная гибридизация.
Однако CGH использует два образца генома "--- тест и контроль, каждый из которых метится флюорофором, а затем гибридизуется 1 : 1.
Таким образом в тестовом образце можно обнаружить CNV и перестройки.

В отличие от FISH, CGH проверяет весь геном на наличие перестроек, не требует знаний о целевом регионе и может быть использован на интерфазных клетках.
Однако, как и у классических методов, разрешение CGH ограничено 5--10 Мб.

В настоящее время CGH используется в виде array-CGH (aCGH), комбинируя этот метод с микрочипами\cite{theisen}.

% \paragraph{STS.}
% \spacing

\paragraph{Мультиплексная лигат-зависимая амплификация зонда (MLPA).}
Основой MLPA является ПЦР-амплификация специальных проб, гибридизующихся с целевыми районами ДНК.
Каждая проба представляет собой пару полу-проб;
каждая полу-проба имеет комплементарную геному часть и технические последовательности "--- праймер для ПЦР и вставки, обеспечивающие большой размер продукта амплификации.
Если полу-пробы гибридизуются с геномом без зазора, они лигируются и впоследствии амплифицируются;
лигированные пробы отличаются от полу-проб с праймером по длине.
Готовый ПЦР-продукт разделяется на электрофорезе и анализируется.

Данная методика подходит для определения CNV целых генов, а также аномалий метилирования ДНК.
Во втором случае используются метил-чувствительные рестриктазы.

Слабым местом MLPA остаётся интерпретация результатов.
Гомозиготные CNV разпознаются просто "--- по наличию/отсутствию пика по сравнению с контрольным образцом.
Гетерозиготные CNV видны как пики отличающейся высоты, и их поиск требует серьёзную биоинформационную обработку с учётом особенностей конкретной ПЦР-реакции и различий между образцами\cite{stuppia}.


\paragraph{Секвенирование по Сэнгеру.}
Метод, позволяющий с высокой точностью анализировать короткий (до 1kb) фрагмент ДНК\cite{sanger}.
В настоящее время используется для подтверждения вариантов, найденных с помощью описанных выше методов.

\paragraph{Микрочиповая гибридизация.}
Гибридизация между комплементарными цепями ДНК позволяет исследовать неизвестную ДНК путем сравнения с ДНК известной последовательности.
Для этой цели были созданы ДНК-микрочипы, содержащие сотни тысяч или миллионы зондов.
    
С помощью сравнительной гибридизации геномов могут быть обнаружены структурные вариации "--- делеции и дупликации, CNV, инсерции, инверсии и хромосомные транслокации.
Для этого используются длинные зонды, которые позволяют проводить гибридизацию последовательностей, имеющих некоторые различия.
Когда пробы ДНК короткие, эффективность гибридизации очень чувствительна к несовпадениям; поэтому такие зонды облегчают сравнение геномов на нуклеотидном уровне (поиск SNV).

Микроматрицы предлагают относительно недорогие и эффективные средства сравнения всех известных типов вариаций.
Однако для таких целей, как обнаружение неизвестных или часто повторяющихся последовательностей, эти методы не годятся\cite{gresham}.

% Сравнение методов\cite{stuppia}:
% 
% \begin{tabular}{| p{0.3\textwidth}  | p{0.3\textwidth} | p{0.3\textwidth} |}
% \hline
% Method & Advantages & Disadvantages \\
% \hline
% MLPA & Detects small rearrangement. Up to 40 target. High throughput. Low cost & Cannot detect copy neutral loss of heterozygosity. May have problems with mosaicism, tumor heterogeneity, or contamination with normal cells. \\
% \hline
% FISH & Detects balanced rearrangement. Detects mosaicis. Detects tumor heterogeneity. Can quantify multiple copies. & Cannot detect copy neutral loss of heterozygosity. Cannot detect small rearrangements (e.g., deletions <100 kb or duplications >500 kb). Limited number of targets and throughput. \\
% \hline
% Quantitative/Sq-PCR & Detects small rearrangements and even point mutation. Can quantify multiple copie. Low cost & Test optimization and efficiency is a concern. Limited number of targets. May have problems with mosaicism, tumor heterogeneity, or contamination with normal cells. \\
% \hline
% Southern blot & Detects small rearrangement. Detects mosaicism & Cannot detect copy neutral loss of heterozygosity. Not quantitative. Laborious and time consumin. Limited number of targets and throughput. \\
% \hline
% CGH array & Can detect very small rearrangement. Can probe entire genom. Low cost per data point & Cannot detect copy neutral loss of heterozygosity. Costly equipment and reagent. Low throughput \\
% \hline
% SNP array & Can detect copy neutral loss or heterozygosit. Can probe entire genom. Low cost per data point & Cannot detect small rearrangements (e.g., deletions or duplications <100 kb). Costly equipment and reagent. Low throughput \\
% \hline
% \end{tabular}

\paragraph{NGS.}
Секвенирование нового поколения (NGS) "--- это комплекс технологий, позволяющих прочитать за сравнительно небольшое время миллионы коротких последовательностей ДНК.
Благодаря этому единовременно можно проанализировать несколько генов, либо весь геном (в отличие от традиционных методов).

Проблемы данных NGS:

\begin{itemize}
\item Ошибки секвенирования.
\item Неоднородность покрытия генома.
\item Ошибки ПЦР и ПЦР-дубликаты.
\item Неточное выравнивание инделов и повторяющихся последовательностей (например, поли-A трактов).
\end{itemize}

В настоящее время NGS используется и для выявления крупных перестроек (метод Hi-C).

\subsection{Виды NGS}

\paragraph{Полногеномное секвенирование (WGS).}
Приготовление библиотек при полногеномном секвенировании производится из всего клеточного материала, либо только из ядер.
ДНК фрагментируется с помощью специфических рестриктаз либо точно подобранных доз ДНКаз.
Таким образом достигается относительно ровное покрытие генома.

WGS при достаточной глубине покрытия вполне пригоден для поиска SNP и небольших инделов.
Полногеномное секвенирование со слабым покрытием может быть использовано для определения CNV "--- например, NIPT, когда используется свободная ДНК плода (cfDNA), циркулирующая в крови матери\cite{yu}.
WGS также может быть использовано в диагностике микробиома с целью определения источника хронической инфекции, реконструкции путей передачи инфекции, а также выявления антибиотикорезистентных штаммов\cite{balloux}.

\paragraph{Полноэкзомное секвенирование (WES)}
Техника заключается в секвенировании экзома "--- совокупности белок-кодирующих последовательностей клетки.
Это достигается с помощью обогащения последовательностей, гибридизованных с олигонуклеотидами, фланкирующими экзоны.

В помощью WES нельзя производить поиск вариантов в некодирующих регионах, но, тем не менее, этот метод неплохо себя зарекомендовал, так как значительная часть патогенных вариантов располагается в экзонах и вблизи них.
Кроме того, полноэкзомное секвенирование значительно дешевле, что увеличивает его доступность и позволяет, например, произвести тестирование ребёнка и родителей (так называемый трио-тест) и, как следствие, улучшить интерпретацию вариантов\cite{yohe}.

\paragraph{Таргетные панели.}
Как и в случае с WES, основой является обогащение целевых регионов.
Отличие заключается в том, что целевые регионы подбираются специфически для конкретной цели.
Данный вид тестов позволяет анализировать гены, ответственные за отдельные группы заболеваний "--- например, существуют таргетные панели для иммунодефицитов, почечных, неврологических болезней, болезней соединительной ткани, сетчатки, а также предрасположенности к отдельным видам онкологических заболеваний.
Также таргетные панели позволяют анализировать клетки опухолей "--- некоторые приспособлены к выявлению общих для многих раковых линий мутаций, другие же разработаны для специфического типа опухолей\cite{yohe}.

\paragraph{Технологии захвата конформации хромосом (3C).}
Данные методики позволяют определить расстояние между двумя точками генома, используя информацию об относительной частоте контактов между ними.
Принцип состоит в том, что интактное ядро (бактериальная хромосома) фиксируется, ДНК разрезается, лигируется, затем снова разрезается и секвенируется.
Во время лигирования связанными могут оказаться участки, которые физически находятся близко друг от друга.
Картирование химерных прочтений с помощью специальных инструментов позволяет узнать, какие именно участки генома были связаны\cite{dekker}.

В настоящее время существует множество вариантов протокола 3C "--- 4C, 5C, Hi-C, TCC, ChIA-PET и scHi-C.
Самым известным и широко применяемым является метод Hi-C, сочетающий 3C с методами массового параллельного секвенирования.
С его помощью можно подсчитать количество контактов во всём геноме "--- как внутри-, так и межхромосомные контакты\cite{oluwadare}.

\subsection{Базовая схема обработки результатов секвенирования}

\paragraph{Демультиплексирование.}
В процессе секвенирования к целевым фрагментам ДНК могут пришиваться так называемые адаптерные последовательности.
Эти адаптеры могут содержать так называемые баркоды "--- последовательности, с помощью которых можно отличить ДНК различных образцов.
Процесс сортировки данных секвенирования по баркодам называется демультиплексированием.
Чаще всего демультиплексирование производится самим секвенатором, но иногда его приходится производить вручную.

\paragraph{Удаление адаптерных последовательностей.}
Если целевая ДНК короче длины прочтения, то фрагменты адаптера на 3' конце могут попасть в готовые данные, и встаёт вопрос об их удалении.
Также присутствие адаптера в прочтениях может быть признаком контаминации, и такие прочтения следует исключить из дальнейшего анализа\cite{cutadapt}.

\paragraph{Картирование.}
Прочтения необходимо картировать на некую референсную геномную последовательность.
Алгоритм картирования представляет собой очень сложную систему, которая учитывает последовательность букв в прочтении и их качество.
В настоящее время <<золотым стандартом>> являются утилиты, использующие алгоритм Берроуса--Уиллера\cite{burrows}.

Основные термины картирования:

\begin{description}
\item[Качество выравнивания] (MAPQ) "--- коэффициент, выставляемый конкретному выравниванию алгоритмом.
\item[Первичное выравнивание] (primary) "--- выравнивание наиболее крупного фрагмента прочтения с наиболее высоким MAPQ.
Первичное выравнивание только одно.
\item[Вторичное выравнивание] (secondary) "--- выравнивание наиболее крупного фрагмента прочтения с меньшим MAPQ.
Вторичных выравниваний может быть несколько.
\item[Добавочное выравнивание] (supplementary) "--- выравнивание менее крупных фрагментов прочтения.
\item[Мягкое клипирование] (soft-clip) "--- отсечение невыравненного конца прочтения с сохранением полной последовательности прочтения.
Мягкому клипированию подвергаются прочтения, содержащие адаптеры, а также химерные первичные выравнивания.
\item[Жёсткое клипирование] (hard-clip) "--- отсечение невыравненного конца прочтения без сохранения его последовательности.
Жёсткому клипированию подвергаются добавочные выравнивания.
\end{description}

Проблемы картирования:

\begin{itemize}
\item Высоковариативные регионы
\item Вырожденные (неуникальные) регионы
\item Регионы с инделами
\end{itemize}

\paragraph{Удаление дубликатов.}
Дубликатные прочтения могут появляться как в результате особенностей приготовления библиотеки (ПЦР-дубликаты), так и из-за ошибок распознавания кластеров амплификации (оптические дубликаты).
Согласно принятой практике, дубликаты должны быть удалены или помечены для улучшения поиска вариантов\cite{gatk}.

Тем не менее, было показано, что для WGS-данных удаление дубликатов имеет минимальный эффект на улучшение поиска полиморфизмов "--- приблизительно 92\% из более чем 17 млн вариантов были найдены вне зависимости от наличия этапа удаления дубликатов и использованных инструментов\cite{ebbert}.
Учитывая, что удаление дубликатов может занимать значительную часть потраченного на обработку данных времени и ресурсов компьютера, следует взвесить пользу и затраты данного этапа для конкретной прикладной задачи.

\paragraph{Рекалибровка качества прочтений (BQSR).}
Приборная оценка качества оснований не соответствует эмпирической, имеют место систематические ошибки прочтения.
Вычисление качества каждой отдельной буквы "--- сложный алгоритм, защищённый авторскими правами производителя секвенатора.
Вместе с тем от качества букв напрямую зависит алгоритм поиска вариантов "--- он использует качество как вес в пользу присутствия или отсутствия варианта в конкретной точке генома.

Решением является рекалибровка качества прочтений, представляющая собой корректировку систематических ошибок, исходя из известных паттернов ковариации.
Первоочередное влияние на ошибки оказывают цикл секвенирования и нуклеотидный контекст (динуклеотид).
Кроме того, алгоритм рекалибровки учитывает ещё и изменчивость каждого отдельного сайта, используя базы данных известных вариантов.

Разработчики GATK настоятельно рекомендуют использовать BQSR для любых данных секвенирования\cite{gatk}.

\paragraph{Поиск вариантов.}
Невозможно точно сказать, какая буква находится в каждой позиции генома;
анализ производит специальный алгоритм, который оценивает качество оснований (т.е. вероятность, в которой буква присутствует в прочтении) и процент букв в позиции на картированных прочтениях.
Отличие генома образца от референсного генома называется вариантом (синонимичные термины <<мутация>> и <<полиморфизм>> не рекомендованы к употреблению\cite{richards}).
Помимо определения буквы, алгоритм может определять зиготность.

Также важным этапом поиска вариантов является так называемое выравнивание по левому краю (left-alignment).
Варианты в повторяющихся последовательностях с длиной менее длины одного прочтения невозможно точно локализовать, поэтому они всегда сдвигаются как можно левее.
Это чрезвычайно важно при аннотации вариантов, так как все базы данных используют варианты с left-alignment, и неправильная локализация может привести к отсеиванию потенциально патогенного варианта.

\subsection{Аннотация, фильтрация и интерпретация результатов}

\paragraph{Номер экзона, функциональный класс варианта.}
\spacing

\paragraph{Частота аллеля по основным базам данных.}
По мере развития методов NGS и увеличения их доступности, начали появляться базы данных, агрегирующие результаты секвенирования различных популяций, а значит "--- способные определить частоту вариантов в популяции.
В настоящее время наиболее крупной является gnomAD\cite{gnomad}, поглотившая существовавший ранее ExAC, содержавший исключительно экзомные данные.
Она содержит частоты вариантов для всех основных рас, а также некоторых условно-здоровых групп (non-neuro, non-topmed и т.п.)

Несмотря на то, что были созданы базы данных для всех рас, очень часто этого недостаточно и необходимо учитывать частоты в популяциях отдельных народов и стран.
Такими базами данных являются GME\cite{gme}, в которой отражены частоты по популяции Ближнего Востока, ABraOM\cite{abraom}, предоставляющая частоты вариантов среди практически здорового пожилого населения Бразилии.
Также для анализа берутся популяции, в которых велика доля близкородственных связей, например, пакистанская\cite{saleheen}.

\paragraph{Loss-of-function (LoF).}
Различные показатели, отражающие устойчивость функции гена, основанные на данных о стоп-кодонах, сдвигах рамки считывания и сплайс-вариантах.
Одним из таких показателей является pLi.

Основные проблемы pLI\cite{ziegler}:

\begin{itemize}
\item Плохо приспособлен к распознаванию AR вариантов (из-за того, что частота повреждающих вариантов в популяции может быть высокой) и XR вариантов (из-за наличия в популяции здоровых гетерозиготных носителей).
\item Плохо приспособлен к распознаванию вариантов в генах, ответственных за патологии, не влияющие на взросление и воспроизводство.
Их частота в популяции также может быть высокой.
К таким относятся варианты в генах BRCA1-2.
\item Сплайс-варианты рассматриваются как повреждающие, несмотря на то, что вариант в сплайс-сайте может не иметь эффекта на сплайсинг, либо приводить к появлению изоформы белка без потери функции.
\item Высокая частота распространения заболевания в контрольной группе.
Пример "--- шизофрения.
\item К миссенс-вариантам pLI применять следует с осторожностью, и без функциональной пробы следует исключить из анализа.
\item Также следует отнестись с осторожностью к нонсенс-вариантам и сдвигам рамки считывания в последнем экзоне либо в C-терминальной части предпоследнего.
Такие транскрипты избегают нонсенс-индуцированного разложения РНК и могут в результате как не привести к каким-либо функциональным изменениям, так и привести к образованию мутантного белка, обладающего меньшей активностью по сравнению с исходным, либо токсичного для клетки.
\item В некоторых случаях соотношение pLI с гаплонедостаточностью конкретного гена в принципе сложно объяснить.
\end{itemize}

Таким образом, высокое значение pLI можно считать хорошим показателем LoF, низкое "--- с осторожностью.

\paragraph{Клинические данные из бд и статей.}
Наиболее достоверным источником данных о патогенности варианта являются исследования, которые изучали конкретный вариант, а также базы данных, агрегирующие информацию из этих статей.
Наиболее используемыми в настоящее время являются HGMD\cite{hgmd} и CLINVAR.
Тем не менее, CLINVAR считается лишь дополнительным источником, так как часто содержит информацию низкого качества\cite{ryzhkova}.

\paragraph{Анализ и предсказание функционального эффекта in silico.}
In silico методы появились в ответ на необходимость как-то классифицировать варианты, по которым недостаточно клинической информации.
Существует множество способов проверить патогенность таких вариантов in vitro, но проверять таким образом все варианты нецелесообразно, а иногда и невозможно.
Даже в хорошо изученных генах варианты с неопределённой клинической значимостью могут занимать большую долю "--- например, в BRCA1 и BRCA2 это 33\% и 50\% соответственно.
Менее изученные гены, а также пациенты, принадлежащие к популяциям с плохо изученным составом вариантов, представляют ещё большую проблему.

Поэтому существуют инструменты на основе машинного обучения, предсказывающие консервативность районов и патогенность вариантов на основе имеющихся данных "--- положения относительно гена и его функциональных элементов, характера замены, а также клинической информации об известных заменах\cite{brea-fernandez}.

\subsection{Когортный и семейный анализ}

Анализ группы, представители которой связаны узами крови (семейный анализ), либо патологией или вариантом фенотипа (когортный анализ).

Семейный анализ нужен для установления путей наследования тех или иных вариантов в родословной.
Это позволяет уточнить степень их корреляции с фенотипом.
Также с помощью семейного анализа можно находить мутации de novo.

Когортный анализ позволяет, например, оценить частоты вариантов в исследуемой и контрольной группе.
Помимо этого существует необходимость детекции систематических отклонений покрытия и артефактов выравнивания, связанных с конкретными районами генома и/или особенностями приготовления библиотек.
Также анализ нескольких родственных образцов помогает определить зиготность варианта либо импутировать район с недостаточным покрытием.

\subsection{Случайные находки}

Несмотря на то, что точность определения патогенности вариантов достаточно невысокая, этические правила, регламентирующие работу врача-генетика, рекомендуют сообщать о потенциально патогенных вариантах в некоторых генах, даже если они не связаны с текущим состоянием пациента.
К таким генам относятся, например, BRCA1 и BRCA2, связанные с раком молочной железы.

\subsection{Exo-C: суть метода}

Основным ограничением NGS-технологий в настоящее время является их цена, напрямую зависящая от глубины секвенирования библиотеки.
3C методы на сегодняшний момент являются наиболее эффективным способом обнаружения хромосомных перестроек, но при небольшой глубине секвенирования обнаружение точечных вариантов затруднено.
WGS при соответствующей глубине способно обнаруживать большую часть маленьких инделов и SNP, WES, с другой стороны, позволяет выявить варианты при небольшой глубине, но только в экзоме.
Возможности обнаружения хромосомных перестроек для этих двух методов ограничены.

Неплохим компромиссом является метод Exo-C, сочетающий технологии таргетного обогащения с 3C.
С его помощью можно как искать точечные варианты в таргетных регионах, так и перестройки во всём геноме\cite{mozheiko}.

\section{Материалы и методы}

\paragraph{Данные секвенирования.}
Результаты секвенирования клеточной линии K562 были взяты из публичных источников (см. Приложение \ref{appendix:accession}).

\paragraph{Контроль качества.}
Для контроля качества прочтений использовалась утилита FastQC\cite{fastqc}, способная оценивать наличие адаптерных последовательностей, распределение прочтений по длине, долю букв по позициям, а также GC-состав.
Критерии качества были взяты согласно протоколу разработчика.

\paragraph{Удаление адаптерных последовательностей.}
Удаление адаптерных последовательностей производилось с помощью утилиты cutadapt\cite{cutadapt}.
GATK Best Practices рекомендуют использовать в качестве входных данных некартированный BAM-файл (uBAM), а для удаления адаптеров использовать собственный инструмент "--- MarkIlluminaAdapters, так как это позволяет сохранить важные метаданные.
Тем не менее, разработчики делают акцент на том, что uBAM должен использоваться как выходной формат на уровне секвенатора, что не является общепринятой практикой\cite{gatk}.

Мы использовали сторонние данные в формате FastQ.
Пребразование FastQ файлов в uBAM не позволяет предотвратить потерю метаданных, но значительно увеличивает время обработки данных.
Сравнение эффективности cutadapt и MarkIlluminaAdapters не показало каких-либо значимых различий.

\paragraph{Картирование.}
Картирование производилось с помощью инструментов Bowtie2\cite{bowtie2} и BWA\cite{bwa}.
BWA показал лучшие результаты;
кроме того, он значительно более эффективно работает с химерными ридами, что немаловажно для используемого нами метода Exo-C.

Для картирования был взят геном GRCh37/hg19.
Из него были удалены так называемые неканоничные хромосомы (некартированные/вариативные референсные последовательности), что позволило улучшить качество выравнивания и значительно упростить работу с готовыми данными.

Кроме того, для правильного функционирования инструментов на дальнейших этапах был разработан скрипт, создающий ReadGroup tag для каждого файла.
Конкретных рекомендаций по составлению RG не существует, поэтому мы разработали собственные, основанные на следующих требованиях\cite{gatk}:

\begin{itemize}
\item Поле SM является уникальным для каждого биологического образца и используется при поиске вариантов.
Несколько SM в одном файле могут быть использованы при когортном анализе.
\item Поле ID является уникальным для каждого RG в BAM-файле.
BQSR использует ID как идентификатор самой базовой технической единицы секвенирования.
\item Поле PU не является обязательным.
Рекомендации GATK советуют помещать в него информацию о чипе секвенирования (баркод), ячейке и баркоде (номере) образца.
Во время BQSR, при наличии поле PU является приоритетным по отношению к ID.
\item Поле LB является уникальным для каждой библиотеки, приготовленной из биологического образца.
Оно отражает различия в количестве ПЦР-дубликатов и потому используется инструментом MarkDuplicates.
\end{itemize}

Объединение BAM-файлов производилось инструментом MergeSamFiles.
Сбор статистики по картированию производился с помощью инструмента samtools flagstat.

\paragraph{Удаление ПЦР-дубликатов.}
Так как мы использовали данные экзомного секвенирования, а количество образцов у нас было относительно небольшим и мы были заинтересованы в максимально качественной подготовке данных, в пайплайн был включён этап удаления ПЦР-дубликатов.
Удаление дубликатов производилось инструментом MarkDuplicates от Picard\cite{picard}, интегрированным в GATK.
Оптимальные показатели скорости MarkDuplicates достигаются при запуске Java с параллелизацией сборщиков мусора и количеством сборщиков мусора равным двум\cite{heldenbrand}.
Также, согласно рекомендациям разработчиков, прочтения были предварительно отсортированы по именам, чтобы удалению подверглись не только первичные, но и добавочные выравнивания\cite{gatk}.

\paragraph{Рекалибровка качества прочтений (BQSR).}
Рекалибровка производилась с помощью инструментов GATK "--- BaseRecalibrator и ApplyBQSR.
Для обучения машинной модели требуются варианты в VCF формате (согласно рекомендациям для Homo sapiens "--- dbSNP >132).

К сожалению, предоставленная Broad Institute база данных оказалась сильно устаревшей и не вполне подходила для сделанной нами геномной сборки, поэтому было решено подвергнуть обработке dbSNP v150, скачанную с NCBI\cite{dbsnp}.
База данных потребовала замену и сортировку контигов в соответствии с референсным геномом, а также удаление <<пустых>> вариантов, содержащих точки в полях REF и ALT.
Далее база данных была архивирована с помощью bgzip, а затем проиндексирована IndexFeatureFile от GATK (этот же инструмент одновременно проверяет БД на пригодность для BQSR).

В \cite{heldenbrand} было показано, что оптимальные показатели скорости BaseRecalibrator достигаются, как и в случае с MarkDuplicates, запуском Java с двумя параллельными сборщиками мусора;
кроме того, BaseRecalibrator поддаётся внешнему распараллеливанию путём разделения картированных ридов на хромосомные группы.
Хромосомные группы формировались вручную для используемой сборки генома, каждая запускалась с помощью bash-скрипта.
Нам удалось усовершенствовать данный этап "--- запуск BaseRecalibrator производился с помощью библиотек Python3 subprocess, а параллелизация осуществлялась библиотекой multiprocessing, таким образом, можно было делить файл с картированными прочтениями по хромосомам и обрабатывать их отдельно, так как multiprocessing автоматически распределяет процессы по имеющимся потокам.
Также для повышения отказоустойчивости скрипта у BaseRecalibrator и ApplyBQSR была устранена разница в фильтрации прочтений, из-за которой при малых размерах библиотек пайплайн экстренно завершал работу.

\paragraph{Оценка покрытия и обогащения.}
Покрытие и обогащение в экзоме оценивались с помощью скрипта на основе bedtools\cite{bedtools}.

\paragraph{Поиск вариантов.}
Поиск вариантов производился с помощью GATK HaplotypeCaller.
Инструмент запускался с дополнительным параметром \verb|--dont-use-soft-clipped-bases|, который не позволял использовать для поиска вариантов невыравненные химерные части.

Как и в случае с BaseRecalibrator, HaplotypeCaller поддаётся внешнему распараллеливанию\cite{heldenbrand}.
Мы также осуществили параллелизацию с помощью сочетания subprocess и multiprocessing, достигнув 10-12-кратного ускорения по сравнению с запуском на одном потоке.

\paragraph{Рекалибровка и ранжирование вариантов.}
В арсенале GATK также присутствуют инструменты для рекалибровки и ранжирования вариантов, с использованием моделей машинного обучения и баз данных с частыми вариантами (CNNScoreVariants и FilterVariantTranches).

Анализ показал, что при наличии этапа рекалибровки вариантов время обработки результатов секвенирования увеличивается почти вдвое.
Между тем, рекалибровка и ранжирование с помощью инструментов GATK не исключает необходимость проверки вариантов вручную.
Таким образом, от этого этапа решено было отказаться.

\paragraph{Аннотация вариантов.}
Аннотация вариантов производилась вначале с помощью инструмента Ensembl VEP\cite{vep}, затем мы мигрировали на ANNOVAR\cite{annovar}.

Используемые базы данных:

\begin{enumerate}
\item Human Gene Mutation Database (HGMD\textregistered)\cite{hgmd}
\item Online Mendelian Inheritance in Man (OMIM\textregistered)\cite{omim}
\item GeneCards\textregistered: The Human Gene Database --- \href{https://www.genecards.org/}{https://www.genecards.org/}
\item ClinVar --- \href{https://www.ncbi.nlm.nih.gov/clinvar/}{https://www.ncbi.nlm.nih.gov/clinvar/}
\item dbSNP --- \href{https://www.ncbi.nlm.nih.gov/snp/}{https://www.ncbi.nlm.nih.gov/snp/}
\item Genome Aggregation Database (gnomAD)\cite{gnomad}
\item 1000 Genomes Project --- \href{https://www.internationalgenome.org/}{https://www.internationalgenome.org/}
\item Great Middle East allele frequencies (GME)\cite{gme}
\item dbNSFP: Exome Predictions\cite{dbnsfp}
\item dbscSNV: Splice site prediction\cite{dbscsnv}
\item RegSNPIntron: intronic SNVs prediction\cite{regsnpintron}
\end{enumerate}

\paragraph{Фильтрация вариантов.}
\spacing
Пограничным значением pLI было взято 0.9, согласно рекомендациям в оригинальной статье\cite{lek}.

Интерпретация данных и составление отчёта производилось в соответствии с рекомендациями Американского колледжа медицинской генетики и геномики (ACMG) и Ассоциации молекулярной патологии\cite{richards}.

\section{Результаты}

\section{Обсуждение результатов}

\section{Предварительные выводы}

\newpage
\appendix

\section{Данные секвенирования K562}

\label{appendix:accession}

\bgroup
\def\arraystretch{1.5}
\begin{tabular}{| l | l | l | l | p{3cm} |}
\hline
\textbf{Name} & \textbf{Article} & \textbf{Type} & \textbf{Reads, M} & \textbf{Accession Codes} \\
\hline
GSM1551618\_HIC069 & Rao et al.\cite{rao} & Hi-C & 456.8 & SRR1658693 \\
\hline
GSM1551619\_HIC070 & Rao et al.\cite{rao} & Hi-C & 591.9 & SRR1658694 \\
\hline
GSM1551620\_HIC071 & Rao et al.\cite{rao} & Hi-C & 79.9 & SRR1658695 SRR1658696 \\
\hline
GSM1551621\_HIC072 & Rao et al.\cite{rao} & Hi-C & 79.6 & SRR1658697 SRR1658698 \\
\hline
GSM1551622\_HIC073 & Rao et al.\cite{rao} & Hi-C & 77.4 & SRR1658699 SRR1658700 \\
\hline
GSM1551623\_HIC074 & Rao et al.\cite{rao} & Hi-C & 80.8 & SRR1658702 SRR1658701 \\
\hline
ENCSR025GPQ & Zhou et al.\cite{zhou} & WGS & 130.0 & ENCFF574YLG ENCFF921AXL ENCFF590SSX \\
\hline
ENCSR053AXS & Zhou et al.\cite{zhou} & WGS & 796.2 & ENCFF004THU ENCFF066GQD ENCFF313MGL ENCFF506TKC ENCFF080MQF \\
\hline
ENCSR711UNY & Zhou et al.\cite{zhou} & WGS & 449.7 & ENCFF471WSA ENCFF826SYZ ENCFF590SSX \\
\hline
SRX3358201 & Dixon et al.\cite{dixon} & WGS & 366.3 & SRR6251264 \\
\hline
GSE148362\_G1 & Wang et al.\cite{wang} & Repli-seq & 24.8 & SRR11518301 \\
\hline
GSE148362\_S1 & Wang et al.\cite{wang} & Repli-seq & 30.9 & SRR11518302 \\
\hline
GSE148362\_S2 & Wang et al.\cite{wang} & Repli-seq & 45.4 & SRR11518303 \\
\hline
GSE148362\_S3 & Wang et al.\cite{wang} & Repli-seq & 49.8 & SRR11518304 \\
\hline
GSE148362\_S4 & Wang et al.\cite{wang} & Repli-seq & 44.1 & SRR11518305 \\
\hline
GSE148362\_S5 & Wang et al.\cite{wang} & Repli-seq & 38.4 & SRR11518306 \\
\hline
GSE148362\_S6 & Wang et al.\cite{wang} & Repli-seq & 35.2 & SRR11518307 \\
\hline
GSE148362\_G2 & Wang et al.\cite{wang} & Repli-seq & 33.0 & SRR11518308 \\
\hline
\end{tabular}

\begin{tabular}{| l | l | l | l | p{3cm} |}
\hline
\textbf{Name} & \textbf{Article} & \textbf{Type} & \textbf{Reads, M} & \textbf{Accession Codes} \\
\hline
INSITU\_HS1 & Ray et al.\cite{ray} & Hi-C & 86.3 & SRR9019504 \\
\hline
INSITU\_HS2 & Ray et al.\cite{ray} & Hi-C & 127.1 & SRR9019505 \\
\hline
INSITU\_NHS1 & Ray et al.\cite{ray} & Hi-C & 86.4 & SRR9019506 \\
\hline
INSITU\_NHS2 & Ray et al.\cite{ray} & Hi-C & 128.5 & SRR9019507 \\
\hline
PD\_STABLE\_REP1 & Moquin et al.\cite{moquin} & Hi-C & 67.2 & SRR5470535 SRR5470534 \\
\hline
PD\_STABLE\_REP2 & Moquin et al.\cite{moquin} & Hi-C & 52.9 & SRR5470536 SRR5470537 \\
\hline
PD\_TRANSIENT & Moquin et al.\cite{moquin} & Hi-C & 81.3 & SRR5470539 SRR5470538 \\
\hline
PDDE\_TRANSIENT & Moquin et al.\cite{moquin} & Hi-C & 55.2 & SRR5470541 SRR5470540 \\
\hline
GSM2588815\_R1 & Belaghzal et al.\cite{belaghzal} & Hi-C & 72.9 & SRR5479813 \\
\hline
GSM2536769\_WT & Banaszak et al.\cite{banaszak} & WES\footnotemark[1] & 39.2 & SRR5345331 \\
\hline
GSM2536770\_WT\_TF & Banaszak et al.\cite{banaszak} & WES\footnotemark[1] & 49.4 & SRR5345332 \\
\hline
GSM2536771\_MT2 & Banaszak et al.\cite{banaszak} & WES\footnotemark[1] & 42.0 & SRR5345333 \\
\hline
GSM2536772\_MT3 & Banaszak et al.\cite{banaszak} & WES\footnotemark[1] & 43.7 & SRR5345334 \\
\hline
GSM2536773\_MT4 & Banaszak et al.\cite{banaszak} & WES\footnotemark[1] & 39.9 & SRR5345335 \\
\hline
GSM2536774\_MT5 & Banaszak et al.\cite{banaszak} & WES\footnotemark[1] & 40.8 & SRR5345336 \\
\hline
\end{tabular}
\egroup

\footnotetext[1]{Варианты в гене DNMT3A были исключены из выборки.}

% На крайняк - https://www.encodeproject.org/search/?type=Experiment&assay_term_name=ChIA-PET&replicates.library.biosample.donor.organism.scientific_name=Homo%20sapiens&biosample_ontology.term_name=K562&status=released 

\begin{thebibliography}{100}

% ---- GATK RECOMMENDATIONS ----

\bibitem{gatk}
Van der Auwera GA, Carneiro MO, Hartl C, et al. From FastQ data to high confidence variant calls: the Genome Analysis Toolkit best practices pipeline. Curr Protoc Bioinformatics. 2013;43(1110):11.10.1-11.10.33. doi:10.1002/0471250953.bi1110s43

\bibitem{heldenbrand}
Heldenbrand JR, Baheti S, Bockol MA, et al. Recommendations for performance optimizations when using GATK3.8 and GATK4 [published correction appears in BMC Bioinformatics. 2019 Dec 17;20(1):722]. BMC Bioinformatics. 2019;20(1):557. Published 2019 Nov 8. doi:10.1186/s12859-019-3169-7

% ---- K562 DATA ----

\bibitem{rao}
Rao SS, Huntley MH, Durand NC, et al. A 3D map of the human genome at kilobase resolution reveals principles of chromatin looping [published correction appears in Cell. 2015 Jul 30;162(3):687-8]. Cell. 2014;159(7):1665-1680. doi:10.1016/j.cell.2014.11.021

\bibitem{zhou}
Zhou B, Ho SS, Greer SU, et al. Comprehensive, integrated, and phased whole-genome analysis of the primary ENCODE cell line K562. Genome Res. 2019;29(3):472-484. doi:10.1101/gr.234948.118

\bibitem{dixon}
Dixon JR, Xu J, Dileep V, et al. Integrative detection and analysis of structural variation in cancer genomes. Nat Genet. 2018;50(10):1388-1398. doi:10.1038/s41588-018-0195-8

\bibitem{wang}
Yuchuan Wang, Yang Zhang, et al. SPIN reveals genome-wide landscape of nuclear compartmentalization. bioRxiv 2020.03.09.982967; doi: https://doi.org/10.1101/2020.03.09.982967

\bibitem{ray}
Ray J, Munn PR, Vihervaara A, et al. Chromatin conformation remains stable upon extensive transcriptional changes driven by heat shock. Proc Natl Acad Sci U S A. 2019;116(39):19431-19439. doi:10.1073/pnas.1901244116

\bibitem{moquin}
Moquin SA, Thomas S, Whalen S, et al. The Epstein-Barr Virus Episome Maneuvers between Nuclear Chromatin Compartments during Reactivation. J Virol. 2018;92(3):e01413-17. Published 2018 Jan 17. doi:10.1128/JVI.01413-17

\bibitem{belaghzal}
Belaghzal H, Dekker J, Gibcus JH. Hi-C 2.0: An optimized Hi-C procedure for high-resolution genome-wide mapping of chromosome conformation. Methods. 2017;123:56-65. doi:10.1016/j.ymeth.2017.04.004

\bibitem{banaszak}
Banaszak LG, Giudice V, Zhao X, et al. Abnormal RNA splicing and genomic instability after induction of DNMT3A mutations by CRISPR/Cas9 gene editing. Blood Cells Mol Dis. 2018;69:10-22. doi:10.1016/j.bcmd.2017.12.002

% ---- TOOLS ----

\bibitem{annovar}
Wang K, Li M, Hakonarson H. ANNOVAR: functional annotation of genetic variants from high-throughput sequencing data. Nucleic Acids Res. 2010;38(16):e164. doi:10.1093/nar/gkq603

\bibitem{vep}
McLaren, W., Gil, L., Hunt, S.E. et al. The Ensembl Variant Effect Predictor. Genome Biol 17, 122 (2016). doi: 10.1186/s13059-016-0974-4

\bibitem{cutadapt}
MARTIN, Marcel. Cutadapt removes adapter sequences from high-throughput sequencing reads. EMBnet.journal, [S.l.], v. 17, n. 1, p. pp. 10-12, may 2011. ISSN 2226-6089. doi: 10.14806/ej.17.1.200. 

\bibitem{fastqc}
Andrews, S. (2010). FastQC:  A Quality Control Tool for High Throughput Sequence Data [Online]. Available online at: http://www.bioinformatics.babraham.ac.uk/projects/fastqc/

\bibitem{bowtie2}
Langmead, B., Salzberg, S. Fast gapped-read alignment with Bowtie 2. Nat Methods 9, 357–359 (2012). doi: 10.1038/nmeth.1923

\bibitem{bwa}
Li H, Durbin R. Fast and accurate short read alignment with Burrows-Wheeler transform. Bioinformatics. 2009;25(14):1754-1760. doi:10.1093/bioinformatics/btp324

\bibitem{picard}
''Picard Toolkit.'' 2019. Broad Institute, GitHub Repository. http://broadinstitute.github.io/picard/; Broad Institute

\bibitem{bedtools}
Quinlan AR and Hall IM, 2010. BEDTools: a flexible suite of utilities for comparing genomic features. Bioinformatics. 26, 6, pp. 841–842.

% ---- DATABASES ----

\bibitem{regsnpintron}
Lin, H., Hargreaves, K.A., Li, R. et al. RegSNPs-intron: a computational framework for predicting pathogenic impact of intronic single nucleotide variants. Genome Biol 20, 254 (2019). doi:10.1186/s13059-019-1847-4

\bibitem{dbnsfp}
Liu X, Wu C, Li C, Boerwinkle E. dbNSFP v3.0: A One-Stop Database of Functional Predictions and Annotations for Human Nonsynonymous and Splice-Site SNVs. Hum Mutat. 2016;37(3):235-241. doi:10.1002/humu.22932

\bibitem{gme}
Scott EM, Halees A, Itan Y, et al. Characterization of Greater Middle Eastern genetic variation for enhanced disease gene discovery. Nat Genet. 2016;48(9):1071-1076. doi:10.1038/ng.3592

\bibitem{gnomad}
Karczewski, K.J., Francioli, L.C., Tiao, G. et al. The mutational constraint spectrum quantified from variation in 141,456 humans. Nature 581, 434–443 (2020). doi:10.1038/s41586-020-2308-7

\bibitem{hgmd}
Stenson PD, Mort M, Ball EV, et al. The Human Gene Mutation Database: towards a comprehensive repository of inherited mutation data for medical research, genetic diagnosis and next-generation sequencing studies. Hum Genet. 2017;136(6):665-677. doi:10.1007/s00439-017-1779-6

\bibitem{omim}
Amberger JS, Bocchini CA, Schiettecatte F, Scott AF, Hamosh A. OMIM.org: Online Mendelian Inheritance in Man (OMIM®), an online catalog of human genes and genetic disorders. Nucleic Acids Res. 2015;43(Database issue):D789-D798. doi:10.1093/nar/gku1205

\bibitem{dbscsnv}
Jian X, Boerwinkle E, Liu X. In silico tools for splicing defect prediction: a survey from the viewpoint of end users. Genet Med. 2014;16(7):497-503. doi:10.1038/gim.2013.176

\bibitem{abraom}
Naslavsky MS, Yamamoto GL, de Almeida TF, Ezquina SAM, Sunaga DY, Pho N, Bozoklian D, Sandberg TOM, Brito LA, Lazar M, Bernardo DV, Amaro E Jr, Duarte YAO, Lebrão ML, Passos-Bueno MR, Zatz M. Exomic variants of an elderly cohort of Brazilians in the ABraOM database. Hum Mutat. 2017 Jul;38(7):751-763. doi: 10.1002/humu.23220.

% ---- OTHER ----

\bibitem{yohe}
Yohe S, Thyagarajan B. Review of Clinical Next-Generation Sequencing. Arch Pathol Lab Med. 2017 Nov;141(11):1544-1557. doi: 10.5858/arpa.2016-0501-RA

\bibitem{balloux}
Balloux F, Brønstad Brynildsrud O, van Dorp L, et al. From Theory to Practice: Translating Whole-Genome Sequencing (WGS) into the Clinic. Trends Microbiol. 2018;26(12):1035-1048. doi:10.1016/j.tim.2018.08.004

\bibitem{sanger}
Sanger F, Nicklen S, Coulson AR. DNA sequencing with chain-terminating inhibitors. Proc Natl Acad Sci U S A. 1977;74(12):5463-5467. doi:10.1073/pnas.74.12.5463

\bibitem{burrows}
Burrows M, Wheeler DJ. Technical report 124. Palo Alto, CA: Digital Equipment Corporation; 1994. A block-sorting lossless data compression algorithm.

\bibitem{ebbert}
Ebbert MT, Wadsworth ME, Staley LA, et al. Evaluating the necessity of PCR duplicate removal from next-generation sequencing data and a comparison of approaches. BMC Bioinformatics. 2016;17 Suppl 7(Suppl 7):239. Published 2016 Jul 25. doi:10.1186/s12859-016-1097-3

\bibitem{richards}
Richards S, Aziz N, Bale S, et al. Standards and guidelines for the interpretation of sequence variants: a joint consensus recommendation of the American College of Medical Genetics and Genomics and the Association for Molecular Pathology. Genet Med. 2015;17(5):405-424. doi:10.1038/gim.2015.30

\bibitem{lek}
Lek M, Karczewski KJ, Minikel EV, et al. Analysis of protein-coding genetic variation in 60,706 humans. Nature. 2016;536(7616):285-291. doi:10.1038/nature19057

\bibitem{ziegler}
Ziegler A, Colin E, Goudenège D, Bonneau D. A snapshot of some pLI score pitfalls. Hum Mutat. 2019 Jul;40(7):839-841. doi: 10.1002/humu.23763

\bibitem{saleheen}
Saleheen D, Natarajan P, Armean IM, et al. Human knockouts and phenotypic analysis in a cohort with a high rate of consanguinity. Nature. 2017;544(7649):235-239. doi:10.1038/nature22034

\bibitem{wutz}
Wutz G, Várnai C, Nagasaka K, et al. Topologically associating domains and chromatin loops depend on cohesin and are regulated by CTCF, WAPL, and PDS5 proteins. EMBO J. 2017;36(24):3573-3599. doi:10.15252/embj.201798004

\bibitem{yu}
Yu D, Zhang K, Han M, et al. Noninvasive prenatal testing for fetal subchromosomal copy number variations and chromosomal aneuploidy by low-pass whole-genome sequencing. Mol Genet Genomic Med. 2019;7(6):e674. doi:10.1002/mgg3.674

\bibitem{herder}
Herder M. What Is the Purpose of the Orphan Drug Act?. PLoS Med. 2017;14(1):e1002191. Published 2017 Jan 3. doi:10.1371/journal.pmed.1002191

\bibitem{richter}
Richter T, Nestler-Parr S, Babela R, Khan ZM, Tesoro T, Molsen E, Hughes DA; International Society for Pharmacoeconomics and Outcomes Research Rare Disease Special Interest Group. Rare Disease Terminology and Definitions-A Systematic Global Review: Report of the ISPOR Rare Disease Special Interest Group. Value Health. 2015 Sep;18(6):906-14. doi: 10.1016/j.jval.2015.05.008

\bibitem{lancet}
The Lancet Neurology. Rare neurological diseases: a united approach is needed. Lancet Neurol. 2011 Feb;10(2):109. doi: 10.1016/S1474-4422(11)70001-1. Erratum in: Lancet Neurol. 2011 Mar;10(3):205.

\bibitem{huber}
D. Huber, L. Voith von Voithenberg, G.V. Kaigala. Fluorescence in situ hybridization (FISH): History, limitations and what to expect from micro-scale FISH? Micro and Nano Engineering, Volume 1, 2018, Pages 15-24, ISSN 2590-0072. doi: 10.1016/j.mne.2018.10.006.

\bibitem{theisen}
Theisen, A. (2008) Microarray-based Comparative Genomic Hybridization (aCGH). Nature Education 1(1):45

\bibitem{stuppia}
Stuppia L, Antonucci I, Palka G, Gatta V. Use of the MLPA assay in the molecular diagnosis of gene copy number alterations in human genetic diseases. Int J Mol Sci. 2012;13(3):3245-3276. doi:10.3390/ijms13033245

\bibitem{dekker}
Dekker J, Rippe K, Dekker M, Kleckner N. Capturing chromosome conformation. Science. 2002 Feb 15;295(5558):1306-11. doi: 10.1126/science.1067799. PMID: 11847345.

\bibitem{oluwadare}
Oluwadare, O., Highsmith, M. \& Cheng, J. An Overview of Methods for Reconstructing 3-D Chromosome and Genome Structures from Hi-C Data. Biol Proced Online 21, 7 (2019). doi: 10.1186/s12575-019-0094-0

\bibitem{brea-fernandez}
Alejandro j. Brea-Fernandez, Marta Ferro, Ceres Fernandez-Rozadilla, Ana Blanco, Laura Fachal, Marta Santamarina, Ana Vega, Alejandro Pazos, Angel Carracedo and Clara Ruiz-Ponte, An Update of In Silico Tools for the Prediction of Pathogenesis in Missense Variants, Current Bioinformatics (2011) 6: 185. https://doi.org/10.2174/1574893611106020185 

\bibitem{dbsnp}
https://www.ncbi.nlm.nih.gov/snp/

\bibitem{mozheiko}
Mozheiko, E.A., Fishman, V.S. Detection of Point Mutations and Chromosomal Translocations Based on Massive Parallel Sequencing of Enriched 3C Libraries. Russ J Genet 55, 1273–1281 (2019). https://doi.org/10.1134/S1022795419100089

\bibitem{sampson}
B. Sampson, A. McGuire. Genetics and the Molecular Autopsy. Editor(s): Linda M. McManus, Richard N. Mitchell. Pathobiology of Human Disease, Academic Press, 2014. Pages 3459--3467. ISBN 9780123864574. doi: 10.1016/B978-0-12-386456-7.06707-1.

\bibitem{gresham}
Gresham, D., Dunham, M. \& Botstein, D. Comparing whole genomes using DNA microarrays. Nat Rev Genet 9, 291–-302 (2008). https://doi.org/10.1038/nrg2335

\bibitem{ryzhkova}
Ryzhkova O.P., Kardymon O.L., Prohorchuk E.B., Konovalov F.A., Maslennikov A.B., Stepanov V.A., Afanasyev A.A., Zaklyazminskaya E.V., Kostareva A.A., Pavlov A.E., Golubenko M.V., Polyakov A.V., Kutsev S.I. Guidelines for the interpretation of massive parallel sequencing variants. Medical Genetics. 2017;16(7):4-17. (In Russ.) 
\end{thebibliography}

\end{document}
