\documentclass[12pt, twoside, a4paper]{article}
\usepackage{calc,etoolbox,float,microtype,soul,xspace,textcomp,xltxtra}
\usepackage[table]{xcolor}
\usepackage{polyglossia}
\setdefaultlanguage[babelshorthands=true]{russian}
\usepackage{mbb}
\usepackage{ulem}
\usepackage{hhline}
\usepackage{color,colortbl}

\usepackage{numprint}
\npthousandsep{\,}
\npdecimalsign{.}
\newcommand{\thousands}{тыс.}
\newcommand{\mln}{млн}
\newcommand{\genename}[1]{\textit{#1}}
\newcommand{\utilname}[1]{\textenglish{#1}}

\newcommand{\engterm}[1]{англ. \textenglish{\textit{#1}}}
\newcommand{\picref}[1]{рис.~\ref{#1}}
\newcommand{\tableref}[1]{табл.~\ref{#1}}
\newcommand{\formularef}[1]{формула~\ref{#1}}

\usepackage{hyphenat}

\usepackage[square,sort,comma,numbers]{natbib}
\bibliographystyle{naturemag}
% \bibliographystyle{ugost2008ns}

\definecolor{mbbcolor}{RGB}{219,229,241}
\usepackage{multirow,tabularx,rotating,wrapfig}

\definecolor{tableoddrowcolor}{RGB}{238,238,238}
\definecolor{tableevenrowcolor}{gray}{1.0}

\newsavebox{\defaultsavebox}

\newcommand{\headerbigrow}[2]{\parbox[c][3.8em]{\widthof{#1}}{#2}}
\newcommand{\bigrow}[2]{\parbox[c][3.8em]{\widthof{#1}}{#2}}

\newenvironment{booktable}[2]
{\begin{table}[H]\caption{\label{#2}#1}\setlength\tabcolsep{10pt}\vspace{0.5em}\setlength\arrayrulewidth{1pt}\begin{lrbox}{\defaultsavebox}\bgroup\def\arraystretch{1}}
			{\egroup\end{lrbox}\resizebox{\textwidth}{!}{\usebox{\defaultsavebox}}\end{table}}

\newenvironment{albumtable}[2]
{\begin{sidewaystable}[h]\caption{\label{#2}#1}\setlength\tabcolsep{10pt}\vspace{0.5em}\setlength\arrayrulewidth{1pt}\begin{lrbox}{\defaultsavebox}\bgroup\def\arraystretch{1}}
			{\egroup\end{lrbox}\resizebox{1\textheight}{!}{\usebox{\defaultsavebox}}\end{sidewaystable}}
			
			
			\usepackage{graphicx}

\newcommand{\intextfigure}[5]
{\begin{wrapfigure}{#1}{#2\textwidth}\centering\includegraphics[width=#2\textwidth]{#3}\caption{\label{#4}#5}\end{wrapfigure}}

\newcommand{\centerfigure}[5]
{\begin{figure}[#1]\centering\includegraphics[width=#5\textwidth]{#2}\caption{\label{#3}#4}\end{figure}}
%%------------------------FOR-EDITORIAL-STAFF------------------------------

\lang{rus}                               %\lang{eng}
\artclass{Информационные и вычислительные технологии в биологии и медицине}                        %\artclass{SECTION}
% \artclassB{Continuation of SECTION}
\year{2020}
\artvolume{15}
\artnumber{1}
\setcounter{page}{1}
\DOI{doi: 10.17537/2020.15.***}
\UDK{616-074+57.087.1}
\artreceived{01.01.2020.}                 %\artreceived{01 January 2016.}
\artaccepted{07.03.2020.}                 %\artaccepted{01 March 2016.} 
\artpublished{07.05.2020.}                %\artpublished{07 May 2016.}


%%----------------------   ARTICLE METADATA BLOCK  -----------------------------
%%---- THE FIELDS SHOULD BE FILLED IN THE ORIGINAL LANGUAGE OF THE ARTICLE ----
%%------ THE EXAMPLE IS GIVEN FOR AN ARTICLE TO BE PUBLISHED IN RUSSIAN -------

\title{Разработка инструментов для поиска клинически значимых полиморфизмов в геноме человека на основе данных секвенирования 3C-библиотек}

\authors{
  Валеев Э.С.\footnotemark[1]$^{1,2}$,
  Гридина М.М.$^{1}$,
  Фишман В.С.\footnotemark[7]$^{1,2}$,
}

\affiliations{
  $^1$Институт Цитологии и Генетики СО РАН, Новосибирск, Россия\\
  $^2$Новосибирский Государственный Университет, Новосибирск, Россия\
}

\abstract{
    Здесь приводится краткое содержание статьи. Обязательно должны быть сформулированы основные результаты работы. Текст аннотации должен быть самодостаточным, без ссылок на список литературы, с понятными обозначениями, без аббревиатур. Обращаем внимание русскоязычных авторов на необходимость качественного перевода описания статьи на английский язык, поскольку весь зарубежный научный мир будет иметь представление о работе именно по этому описанию. Аннотация на английском языке не должна быть точным переводом русскоязычной аннотации. Она призвана отражать содержание статьи, которая недоступна для читателей, не владеющих русским языком. Длина аннотации должна быть 200--250 слов (не менее 25 строк).
    }

\keywords{Exo-C, ...}

%% Фамилии авторов, если много,то так: ИВАНОВ и др.
\runningAuthors{ВАЛЕЕВ, ГРИДИНА, ФИШМАН}
%% Короткое название статьи для колонтитулов
\runningTitle{РАЗРАБОТКА ИНСТРУМЕНТОВ ДЛЯ ПОИСКА ГЕНЕТИЧЕСКИХ ВАРИАНТОВ НА ОСНОВЕ ДАННЫХ 3C-СЕКВЕНИРОВАНИЯ}

%%----  ARTICLE METADATA BLOCK ENDS HERE -----------------------------------


\begin{document}

\footnotetext[1]{emil@bionet.nsc.ru}
\footnotetext[7]{minja@bionet.nsc.ru}

\maketitle

\section*{ВВЕДЕНИЕ}

Наследственные заболевания являются одной из основных причин младенческой и детской смертности в развитых странах\cite{Field_2003}.
Взрослые люди с такими патологиями требуют огромных затрат средств на медикаменты, оперативные вмешательства, специальный уход и социальные льготы.
Таким образом, доступные и точные методы диагностики наследственных заболеваний могут помочь в сокращении заболеваемости и смертности, а также повысить экономическое благополучие населения.

Несмотря на то, что в развитии наследственных заболеваний играют роль множество механизмов, в основе их всегда лежат изменения тех или иных участков ДНК.
Эти генетические варианты существенно различаются по размеру, характеру изменения, а также функциональному значению.
Существует множество методов выявления генетических вариантов, каждый метод имеет свои преимущества и границы применения.

Наиболее перспективными в диагностическом и исследовательском плане в настоящее время являются методы секвенирования "--- например, полногеномное и полноэкзомное секвенирование.
В Секторе геномных механизмов онтогенеза ИЦиГ~СО~РАН был разработан новейший метод секвенирования "--- Exo-C, сочетающий технологии экзомного обогащения с захватом конформации хромосом.
Потенциальным преимуществом данного метода может быть возможность поиска как крупных перестроек, так и точечных полиморфизмов в экзоме при относительно небольшой глубине секвенирования, от которой напрямую зависит цена секвенирования.
Широкий спектр применения метода и доступность в финансовом аспекте делают метод Exo-C привлекательным как для медико-биологических научных исследований, так и для внедрения в клиническую практику.

Целью нашей работы является сравнение эффективности методов Exo-C, полногеномного секвенирования и экзомного секвенирования для поиска точечных полиморфизмов в геномах клеток человека.

Основные задачи, которые необходимо решить для достижения поставленной нами цели:

\begin{enumerate}
	\item Разработать биоинформационный протокол анализа данных секвенирования Exo-C-библиотек.
	\item Проанализировать доступные данные полногеномного, полноэкзомного, Hi-C и Exo-C-секвенирования для иммортализованной клеточной линии человека K562.
	\item Сравнить точечные генетические варианты в геноме клеток K562, детектируемые при использовании полногеномного и экзомного секвенирования, с таковыми, найденными методом Exo-C.
\end{enumerate} 

\section*{МАТЕРИАЛЫ И МЕТОДЫ}

\subsection{Данные секвенирования}
Поиск данных секвенирования производился в базах данных NCBI (GEO DataSets, SRA, PubMed) и ENCODE с использованием ключевых слов ``K562'', ``K562+WGS'', ``K562+WES'', ``K562+Hi-C''.

\subsection{Контроль качества NGS\hyp{}данных}
Для контроля качества прочтений мы использовали утилиту \utilname{FastQC}\,\cite{FastQC}, способную оценивать наличие адаптерных последовательностей, распределение прочтений по длине, GC-состав прочтений, а также производить анализ зависимости нуклеотидного состава от позиции в прочтении.
Критерии качества были использованы согласно протоколу разработчика\,\cite{FastQC}.

\subsection{Удаление адаптерных последовательностей}
Удаление адаптерных последовательностей производилось с помощью утилиты \utilname{cutadapt}\,\cite{Martin_2011}.
В \cite{Auwera_2013} рекомендуется использовать в качестве входных данных некартированный BAM-файл (\engterm{Unmapped Binary sequence Alignment\/Map, uBAM}), а для удаления адаптеров использовать их собственный инструмент "--- \utilname{MarkIlluminaAdapters}, так как это позволяет сохранить важные метаданные.
Тем не менее, был сделан акцент на том, что uBAM должен использоваться как выходной формат на уровне секвенатора, что не является общепринятой практикой.

Мы использовали данные секвенирования в формате FastQ.
Пребразование FastQ-файлов в uBAM не предотвращает потерю метаданных, но значительно увеличивает время обработки данных.
Сравнение эффективности \utilname{cutadapt} и \utilname{MarkIlluminaAdapters} в процессе удаления адаптеров не показало каких-либо значимых различий.

\subsection{Картирование}
Картирование производилось с помощью инструментов \utilname{Bowtie2}\,\cite{Langmead_2012} и \utilname{BWA}\,\cite{Li_2009}.
\utilname{BWA} показал лучшие результаты;
кроме того, он значительно более эффективно работает с химерными ридами, что немаловажно для используемого нами метода Exo-C.

Для картирования был взят геном GRCh37/hg19, предоставленный NCBI.
Из него были удалены так называемые неканоничные хромосомы (некартированные/вариативные референсные последовательности), что позволило улучшить качество выравнивания и значительно упростить работу с готовыми данными.

Кроме того, для правильного функционирования инструментов на дальнейших этапах был разработан скрипт, создающий метку группы прочтений (\engterm{Read Group tag, RG}) для каждого файла.
Конкретных рекомендаций по составлению RG не существует, поэтому мы разработали собственные, основанные на требованиях Broad Institute\,\cite{Auwera_2013}.

Объединение BAM-файлов производилось инструментом \utilname{MergeSamFiles}.
Сбор статистики по картированию мы осуществляли с помощью инструмента \utilname{SAMTools flagstat}\,\cite{Li_2009_SAMTools}.

\subsection{Удаление ПЦР\hyp{}дубликатов}
Для улучшения данных экзомного секвенирования в пайплайн был включён этап удаления ПЦР\hyp{}дубликатов.
Обычно этот процесс занимает много времени, но количество образцов у нас было относительно небольшим, и мы были заинтересованы в максимально качественной подготовке данных.

Удаление дубликатов производилось инструментом \utilname{MarkDuplicates} от Picard\,\cite{PicardTools}, интегрированным в \utilname{GATK}.
Оптимальные показатели скорости \utilname{MarkDuplicates} достигаются при запуске \utilname{Java} с параллелизацией сборщиков мусора и количеством сборщиков мусора равным двум\,\cite{Heldenbrand_2019}.
Также, согласно рекомендациям разработчиков, прочтения были предварительно отсортированы по именам, чтобы удалению подверглись не только первичные, но и добавочные выравнивания\,\cite{Auwera_2013}.

\subsection{Рекалибровка качества прочтений (BQSR)}
Рекалибровка производилась с помощью инструментов \utilname{GATK BaseRecalibrator} и \utilname{GATK ApplyBQSR}.
Для обучения машинной модели требуются генетические варианты в VCF-формате (согласно рекомендациям для \textit{Homo sapiens} "--- dbSNP v132+).

К сожалению, предоставленная Broad Institute база данных оказалась сильно устаревшей и не вполне подходила для сделанной нами геномной сборки, поэтому было решено подвергнуть обработке dbSNP v150, предоставленную NCBI\,\cite{Sherry_2001}.
База данных потребовала замену и сортировку контигов в соответствии с референсным геномом, а также удаление <<пустых>> вариантов, содержащих точки в полях REF и ALT.
Далее база данных была архивирована с помощью \utilname{bgzip}, а затем проиндексирована \utilname{GATK IndexFeatureFile} (этот же инструмент одновременно проверяет БД на пригодность для BQSR).

В \cite{Heldenbrand_2019} было показано, что оптимальные показатели скорости \utilname{BaseRecalibrator} достигаются, как и в случае с \utilname{MarkDuplicates}, запуском \utilname{Java} с двумя параллельными сборщиками мусора;
кроме того, \utilname{BaseRecalibrator} поддаётся внешнему распараллеливанию путём разделения картированных прочтений на хромосомные группы.
Хромосомные группы формировались вручную для используемой сборки генома, каждая запускалась с помощью \utilname{bash}-скрипта.
Нам удалось усовершенствовать данный этап "--- запуск \utilname{BaseRecalibrator} производился с помощью библиотеки \utilname{Python} \utilname{subprocess}, а параллелизация осуществлялась библиотекой \utilname{multiprocessing}, таким образом, можно было делить файл с картированными прочтениями по хромосомам и обрабатывать их отдельно, так как \utilname{multiprocessing} автоматически распределяет процессы по имеющимся потокам.
Также для повышения отказоустойчивости скрипта у \utilname{BaseRecalibrator} и \utilname{ApplyBQSR} была устранена разница в фильтрации прочтений, из-за которой при малых размерах библиотек пайплайн экстренно завершал работу.

\subsection{Оценка покрытия и обогащения}
Покрытие и обогащение в экзоме оценивались с помощью скрипта на основе \utilname{BEDTools}\,\cite{Quinlan_2010}.

\subsection{Поиск вариантов}
Поиск вариантов производился с помощью инструмента \utilname{GATK HaplotypeCaller}.
Инструмент запускался с дополнительным параметром \verb|--dont-use-soft-clipped-bases|, который не позволял использовать для поиска генетических вариантов клипированные химерные части и адаптеры.

Как и в случае с \utilname{BaseRecalibrator}, \utilname{HaplotypeCaller} поддаётся внешнему распараллеливанию\,\cite{Heldenbrand_2019}.
Мы также осуществили параллелизацию с помощью сочетания \utilname{subprocess} и \utilname{multiprocessing}, достигнув 10--12-кратного ускорения по сравнению с запуском на одном потоке.

\subsection{Аннотация вариантов}
Аннотация вариантов производилась вначале с помощью инструмента \utilname{ANNOVAR}\,\cite{Wang_2010}.

Используемые базы данных:

\begin{enumerate}
	\item Human Gene Mutation Database (HGMD\textregistered)\,\cite{Stenson_2017}
	\item Online Mendelian Inheritance in Man (OMIM\textregistered)\,\cite{Amberger_2014}
	\item GeneCards\textregistered: The Human Gene Database\,\cite{Stelzer_2016}
	\item CLINVAR\,\cite{Landrum_2017}
	\item dbSNP\,\cite{Sherry_2001}
	\item Genome Aggregation Database (gnomAD)\,\cite{Karczewski_2020}
	\item 1000Genomes Project\,\cite{Auton_2015}
	\item Great Middle East allele frequencies (GME)\,\cite{Scott_2016}
	\item dbNSFP: Exome Predictions\,\cite{Liu_2016}
	\item dbscSNV: Splice site prediction\,\cite{Jian_2013}
	\item RegSNPIntron: intronic SNVs prediction\,\cite{Lin_2019}
\end{enumerate}

\subsection{Фильтрация генетических вариантов}
Аннотации были агрегированы для удобства использования.
Так, агрегации подверглись:

\begin{itemize}
	\item Имена генов по разным БД "--- для облегчения поиска;
	\item Описания функциональных классов из разных БД "--- для устранения несоответствий между ними;
	\item Ранги инструментов, предсказывающих патогенность генетического варианта.
	      Трёхранговые системы (патогенный, вероятно патогенный и безвредный) были сведены к двухранговой (патогенный и безвредный).
	      Отдельно были агрегированы предсказательные инструменты для экзонов, инструменты для интронов и сплайс-вариантов также учитывались отдельно;
	\item Ранги инструментов, предсказывающих консервативность нуклеотида.
	      Эмпирическим путём было подобрано пороговое значение \numprint{0.7} "--- нуклеотид считался консервативным, если его предсказанная консервативность была выше, чем у \numprint[\%]{70} всех нуклеотидов.
	      Это максимальное пороговое значение, которое обеспечивает распределение балла агрегатора от минимального до максимального (от 0 до 7 баз данных, считающих данный нуклеотид консервативным);
	\item Популяционные частоты "--- из всех имеющихся в базах данных по конкретному генетическому варианту была выбрана максимальная частота.
\end{itemize}

Фильтрация происходила в две стадии:
\begin{enumerate}
	\item Фильтрация отдельных генетических вариантов на основе имеющихся аннотаций.
	      Самая жёсткая фильтрация, которой подвергались все варианты:
	      \begin{itemize}
		      \item По глубине покрытия.
		            Генетический вариант считался существующим, если он присутствовал в двух перекрывающихся парных прочтениях, либо в чётырёх независимых прочтениях;
		      \item Частота генетического варианта в популяции не более \numprint[\%]{3}\,\cite{Ryzhkova_2017}.
	      \end{itemize}

	      Прочие фильтры были мягкими "--- генетический вариант отсеивался только в случае несоответствия всем указанным критериям:

	      \begin{itemize}
		      \item Присутствие описания связанной с геном патологии в базе данных OMIM;
		      \item Присутствие генетического варианта в базе данных HGMD;
		      \item Балл агрегатора патогенности экзомных вариантов не менее 3\,\cite{Ryzhkova_2017};
		      \item Ранг <<патогенный>> у агрегаторов интронных или сплайс-вариантов;
		      \item Ранги <<патогенный>> и <<возможно патогенный>> по базе данных CLINVAR;
		      \item По функциональному классу: сдвиги рамки считывания, потери стоп- и старт-кодонов, нонсенс- и сплайс-варианты.
	      \end{itemize}

	\item Фильтрация значимых вариантов на основе аннотаций гена.
	      Все эти фильтры были мягкими "--- ген мог соответствовать одному любому из перечисленных критериев:

	      \begin{itemize}
		      \item Значение pLI более \numprint{0.9}, согласно рекомендациям в оригинальной статье\,\cite{Lek_2016};
		      \item Наследование в гене значится как <<доминантное>> по базе данных OMIM, либо информации о доминантности нет;
		      \item Любой значимый вариант в гене является гомозиготным;
		      \item В гене более одного значимого варианта (вероятность цис-транс-положения).
	      \end{itemize}
\end{enumerate}

\subsection{Интерпретация}
Интерпретация данных и составление отчёта производилось в соответствии с рекомендациями Американского колледжа медицинской генетики и геномики (\engterm{American College of Medical Genetics, Bethesda, MD, USA}) и Ассоциации молекулярной патологии\,\cite{Richards_2015}.
% В среднем на каждый образец в данных Exo-C приходилось порядка 1--2 \thousands значимых вариантов, затрагивающих около 100--150 генов.
% Порядка 100--200 вариантов были результатом систематических ошибок, возникших в ходе приготовления библиотеки или обработки данных.

\section*{РЕЗУЛЬТАТЫ}

\subsection{Результаты секвенирования Exo-C\hyp{}библиотек}

Несмотря на то, что составляющие протокола Exo-C "--- таргетное обогащение и Hi-C "--- в настоящее время достаточно отработаны, сочетание этих методик имеет свои подводные камни.
Было разработано две вариации протокола Exo-C (ExoC-19 и ExoC-20), обе этих вариации были использованы для приготовления библиотек клеточной линии K562\,\citep{Ma_2018,Ramani_2016,Gridina_Forthcoming}.
Критическим различием протоколов является использование дополнительных адаптеров в протоколе ExoC-19.
Результаты секвенирования этих библиотек проверялись биоинформационными методами.

Базовыми параметрами качества библиотек были приняты:

\begin{itemize}
	\item Доля дубликатов, отражающая качество стадии ПЦР;
	\item Доля участков, в которых покрытие прочтениями отсутствует, а также тех, в которых оно превышает минимальный порог для анализа (10 прочтений);
	\item Отношение среднего покрытия вне и внутри экзома, которое можно считать показателем качества таргетного обогащения.
\end{itemize}

Данные по качеству Exo-C\hyp{}библиотек представлены в \tableref{tab:exoc-enrichment}.

\begin{booktable}{Данные по обогащению Exo-C\hyp{}библиотек}{tab:exoc-enrichment}
	\begin{tabular}{| l | r | r | r | r | r | r | r | r |}
		\hline
		\rowcolor{mbbcolor}
		\textbf{Название}                                                                   &
		\headerbigrow{ прочтений}{Глубина, прочтений}                                       &
		\textbf{Доля дубликатов, \%}                                                        &
		\headerbigrow{Доля экзома с глубиной}{Доля экзома с глубиной покрытия более 10, \%} &
		\headerbigrow{Среднее покрытие}{Среднее покрытие в экзоме}                          &
		\headerbigrow{Среднее покрытие}{Среднее покрытие вне экзома}                        &
		\headerbigrow{Обогащение}{Обогащение экзома, раз}                                   &
		\headerbigrow{регионов в экзоме, \%}{Доля непокрытых\newline регионов в экзоме, \%} &
		\headerbigrow{регионов вне экзома, \%}{Доля непокрытых регионов вне экзома, \%}                                                                                                                                                               \\
		\hline
		ExoC-19                                                                             & \numprint{136609179} & \numprint{18.86} & \numprint{91.68} & \numprint{60.51} & \numprint{5.56} & \numprint{10.89} & \numprint{1.75} & \numprint{28.12} \\\hline
		ExoC-20                                                                             & \numprint{109486529} & \numprint{15.00} & \numprint{72.58} & \numprint{14.88} & \numprint{7.74} & \numprint{1.92}  & \numprint{1.66} & \numprint{11.62} \\\hline
	\end{tabular}
\end{booktable}

\subsection{Автоматизация обработки данных секвенирования}

При обработке данных секвенирования приходится сталкиваться с проблемами различного характера.
Одними из ключевых являются проблемы использования ресурсов компьютера.
Результаты секвенирования даже в сжатом виде занимают десятки и сотни гигабайт дискового пространства, и многие инструменты создают файлы с промежуточными результатами, которые занимают дисковое пространство, не неся никакой практической пользы для исследования.
Кроме того, из-за вычислительной сложности обработка таких больших блоков данных может занимать дни, недели и даже месяцы работы вычислительного кластера.

Вторая, не менее важная группа проблем, связана с используемыми для обработки инструментами.
Как было показано выше, стадий у обработки значительное количество, и не все стадии нужны при обработке конкретного блока данных секвенирования.
Ручная настройка и контроль процесса отнимают значительное количество времени исследователя;
таким образом, встаёт вопрос стандартизации и автоматизации процесса обработки данных секвенирования.

Существующие инструменты для обработки данных секвенирования были разработаны независимыми группами людей.
Эти инструменты различаются по многим аспектам.
% 
% \begin{itemize}
% \item Язык программирования.
% Каждый язык имеет свои особенности взаимодействия с вычислительной техникой "--- использование памяти, потребность в специальных окружениях и т.п.;
% \item Консольный интерфейс.
% Каждый инструмент имеет свои собственный интерфейс взаимодействия с пользователем.
% Этот интерфейс может содержать недокументированные или неправильно документированные возможности;
% \item Способность к взаимодействию с потоками данных.
% Большая часть инструментов может использовать стандартные потоки ввода/вывода (\verb|stdin|, \verb|stdout|), но некоторые в силу особенностей алгоритма (например, требующего обработку не одной строки, а всего файла целиком) взаимодействуют исключительно с файловой системой.
% \item Настраиваемость.
% Отдельные инструменты не предоставляют пользователю возможность настроить нужные параметры, и для этого требуются дополнительные надстройки.
% \item Требования к входным данным.
% Несмотря на то, что большая часть используемых форматов стандартизированы, в них могут быть вариабельные и необязательные блоки данных, к которым у конкретного инструмента могут быть свои требования.
% \end{itemize}
%
Так как разработка каждого отдельного инструмента является сложным и трудоёмким процессом, целесообразно использовать их как есть, а несоответствия устранять с помощью специально разработанной надстройки.
Таким образом, для нами был создан пайплайн, интегрирующий все стадии обработки данных секвенирования.
Блок-схема пайплайна представлена на \picref{fig:pipeline}.

\centerfigure{h}{BlockScheme.pdf}{fig:pipeline}{Принципиальная схема пайплайна для обработки Exo-C-данных}{1}

Решённые задачи:

\begin{itemize}
	\item Отказоустойчивость: максимально устранены несоответствия форматов входных и выходных данных; процесс разделён на стадии, и в случае экстренного прерывания вычислений (программного или аппаратного) предусмотрен автоматический откат.
	\item Оптимизация, параллелизация и масштабируемость: все процессы, которые способны использовать стандартные потоки ввода/вывода, объединены вместе, поддающиеся внешнему распараллеливанию были распараллелены, также были подобраны оптимальные параметры запуска приложений, использующих машину Java.
	      Пайплайн может быть использован как на кластерах с большим количеством ядер и оперативной памяти, так и на относительно небольших мощностях офисных компьютеров;
	\item Значительно упрощены процессы развёртки и использования пайплайна: автоматизировано индексирование референсной последовательности, настройки вынесены в специальный конфигурационный файл, есть возможность обработки пула данных, используя один короткий сценарий;
\end{itemize}

Код пайплайна доступен на GitHub\,\citep{Scissors}.

\subsection{Сравнение данных секвенирования клеточной линии K562}

Следующим важным этапом работы была проверка эффективности поиска генетических вариантов в Exo-C\hyp{}библиотеках.
Было решено использовать для этого распространённую иммортализованную клеточную линию K562, полученную от пациентки с хроническим миелолейкозом\,\citep{Lozzio_1975}.
Данная клеточная линия была многократно секвенирована различными лабораториями с использованием различных методик приготовления библиотек.
Таким образом, несмотря на то, что в этой клеточной линии наблюдается некоторая гетерогенность между лабораториями из-за большого количества пассажей, несмотря на наличие систематических ошибок при использовании разных методов секвенирования и приготовления библиотек, по K562 существует достаточное количество данных, чтобы использовать эту клеточную линию как стандарт для поиска генетических вариантов.

Результаты секвенирования клеточной линии K562 были взяты из публичных источников\,\citep{Banaszak_2018,Belaghzal_2017,Dixon_2018,Moquin_2017,Rao_2014,Ray_2019,Wang_2020,Zhou_2019}.
Использованные в этих статьях методики включают WGS, WES, Hi-C и Repli-seq.
Из данных полноэкзомного секвенирования в дальнейшем были исключены все генетические варианты в интервале \numprint{chr2:25455845-25565459} с фланкированием \numprint[kbp]{1} (ген \genename{DNMT3A}), так как в одной из работ использовали генетически модифицированную линию с вариантами в данном гене\,\citep{Banaszak_2018}.
В качестве тестовых Exo-C-образцов мы использовали данные, полученные на основе клеточной линии K562, имеющейся в Институте Цитологии и Генетики СО~РАН.
Технические данные контроля качества по тестовым и контрольным образцам представлены в \tableref{appendix:control-libs} и \tableref{appendix:control-samples}.

В общей сложности, объединив варианты из всех контрольных образцов, мы получили \numprint{5496486} различных генетических вариантов.
Также в библиотеках было найдено некоторое количество уникальных генетических вариантов, встречающихся в одной библиотеке и не встречающихся в остальных (\tableref{tab:unique-controls}).
Наибольший процент уникальных вариантов найден в данных Banaszak~et~al.\,\cite{Banaszak_2018}

\begin{booktable}{Уникальные генетические варианты в данных секвенирования контрольных образцов клеточной линии K562}{tab:unique-controls}
	\begin{tabular}{| l | l | r | r | r | r |}
		\hline
		\rowcolor{mbbcolor}
		\textbf{Название}                                                         &
		\textbf{Протокол}                                                         &
		\headerbigrow{Глубина секвенирования,}{Глубина секвенирования, прочтений} &
		\headerbigrow{Общее число}{Общее число вариантов}                         &
		\headerbigrow{Уникальные}{Уникальные варианты}                            &
		\headerbigrow{Доля уникальных}{Доля уникальных вариантов, \%}
		\\\hline
		Banaszak~et~al.\,\cite{Banaszak_2018}                                                & WES       & \numprint{254983225}  & \numprint{408008}  & \numprint{41830}  & \numprint{10.25} \\\hline
		Belaghzal~et~al.\,\cite{Belaghzal_2017}                                               & Hi-C      & \numprint{72914268}   & \numprint{1399457} & \numprint{27365}  & \numprint{1.95}  \\\hline
		Dixon~et~al.\,\cite{Dixon_2018}                                                   & WGS       & \numprint{366291496}  & \numprint{4649012} & \numprint{327184} & \numprint{7.03}  \\\hline
		Moquin~et~al.\,\cite{Moquin_2017}                                                  & Hi-C      & \numprint{256500659}  & \numprint{2365361} & \numprint{67678}  & \numprint{2.86}  \\\hline
		Rao~et~al.\,\cite{Rao_2014}                                                     & Hi-C      & \numprint{1366228845} & \numprint{4218233} & \numprint{320508} & \numprint{7.59}  \\\hline
		Ray~et~al.\,\cite{Ray_2019}                                                     & Hi-C      & \numprint{428306794}  & \numprint{1789324} & \numprint{89624}  & \numprint{5.00}  \\\hline
		Wang~et~al.\,\cite{Wang_2020}                                                    & Repli-seq & \numprint{301663640}  & \numprint{2207451} & \numprint{37578}  & \numprint{1.70}  \\\hline
		Zhou~et~al.\,\cite{Zhou_2019}                                                    & WGS       & \numprint{2621311293} & \numprint{4412455} & \numprint{166451} & \numprint{3.77}  \\\hline
	\end{tabular}
\end{booktable}

\numprint{75328} генетических вариантов были найдены в данных из всех восьми статей "--- их было решено использовать как <<золотой стандарт>>.
Сразу можно внимание на то, что это составляет лишь \numprint[\%]{1.37} геномных SNV клеток K562.
Такая ситуация может возникнуть в следующих случаях:

\begin{enumerate}
	\item В одной или нескольких работах обнаружено очень много уникальных вариантов, которые дают существенный вклад в общее число вариантов, но не пересекаются с результатами других исследований;
	\item В одной или нескольких работах не найдено подавляющее большинство вариантов, найденных во всех остальных работах;
	\item Распределение уникальных вариантов и число общих вариантов между парами работ относительно равномерно, и низкое число общих для всех восьми работ вариантов не может объясняться особенностями какого-то одного или нескольких исследований.
\end{enumerate}

Чтобы проверить, не связана ли низкая доля общих генетических вариантов с особенностями какого-то одного из использованных наборов данных, мы протестировали все комбинации из семи и шести работ.
Результаты представлены на \picref{fig:exclusion}.

\centerfigure{hp!}{Exclusion_6.pdf}{fig:exclusion}{Исключение отдельных образцов из контрольной выборки позволило увеличить количество генетических вариантов, которые можно использовать как стандарт. На рисунке показано суммарное количество вариантов в выборке (светло-зелёный) и процент общих для этой выборки вариантов (тёмно-зелёный) для выборок размером в 6 и 7 образцов. Слева указаны названия исключённых из выборки образцов.}{0.7}

При исключении из выборки данных Banaszak~et~al.\,\cite{Banaszak_2018} и Belaghzal~et~al.\,\cite{Belaghzal_2017} общими являются \numprint{1091331} (\numprint[\%]{19.85}) вариантов.
Их решено было использовать как добавочный (<<серебряный>>) стандарт.
Далее мы использовали варианты <<серебряного>> и <<золотого>> стандартов для того, чтобы определить точность поиска генетических вариантов в наших Exo-C\hyp{}библиотеках.
Для этого мы оценили количество генетических вариантов, являющихся общими для <<серебряного>> и <<золотого>> стандартов и наших Exo-C\hyp{}библиотек, их долю от общего числа вариантов в Exo-C\hyp{}библиотеках, а также количество и долю ложноположительных (отсутствующих в контрольных образцах) генетических вариантов.
Также было решено проверить эффективность использованного нами базового фильтра "--- удаление всех генетических вариантов, в которых глубина альтернативного аллеля составляет менее 4.
Поиск вариантов <<серебряного>> и <<золотого>> стандартов в наших библиотеках был произведён до и после фильтрации.
Результаты показаны в \tableref{tab:filtration-efficiency}.

\begin{booktable}{Параметры Exo-C\hyp{}библиотек. (F--) "--- до фильтрации по глубине альтернативного аллеля, (F+) "--- после фильтрации, ($\Delta$) "--- изменение параметра после фильтрации в процентах}{tab:filtration-efficiency}
	\begin{tabular}{| l | r | r | r | r | r | r | r | r | r | r | r | r |}
		\hline
		\rowcolor{mbbcolor}
		\textbf{Параметр}                                                                                                                        &
		\multicolumn{3}{ c |}{\textbf{ExoC-19}}                                                                                                  &
		\multicolumn{3}{ c |}{\textbf{ExoC-20}}                                                                                                  &
		\multicolumn{3}{ c |}{\textbf{В обеих}}                                                                                                  &
		\multicolumn{3}{ c |}{\textbf{Ни в одной}}
		\\
		\rowcolor{mbbcolor}
		~                                                                                                                                        &
		\textbf{F--}                                                                                                                             &
		\textbf{F+}                                                                                                                              &
		\textbf{$\Delta$, \%}                                                                                                                    &
		\textbf{F--}                                                                                                                             &
		\textbf{F+}                                                                                                                              &
		\textbf{$\Delta$, \%}                                                                                                                    &
		\textbf{F--}                                                                                                                             &
		\textbf{F+}                                                                                                                              &
		\textbf{$\Delta$, \%}                                                                                                                    &
		\textbf{F--}                                                                                                                             &
		\textbf{F+}                                                                                                                              &
		\textbf{$\Delta$, \%}                                                                                                                                                                                                                                                                                                                                                                        \\
		\hline
		Общее число вариантов в библиотеке                                                                                                       & \numprint{3173343} & \numprint{1396525} & \numprint{-55.99} & \numprint{3750319} & \numprint{2577934} & \numprint{-31.26} & ---               & ---               & ---               & ---              & ---               & ---                \\\hline
		Вариантов <<золотого стандарта>>                                                                                                         & \numprint{62335}   & \numprint{52732}   & \numprint{-15.41} & \numprint{72705}   & \numprint{67270}   & \numprint{-7.48}  & \numprint{60728}  & \numprint{48840}  & \numprint{-19.58} & \numprint{1016}  & \numprint{4166}   & \numprint{+310.04} \\\hline
		Доля вариантов <<золотого стандарта>>, \%                                                                                                & \numprint{82.75}   & \numprint{70.00}   & ---               & \numprint{96.52}   & \numprint{89.30}   & ---               & \numprint{80.62}  & \numprint{64.84}  & ---               & \numprint{1.35}  & \numprint{5.53}   & ---                \\\hline
		Вариантов <<серебряного стандарта>>                                                                                                      & \numprint{616375}  & \numprint{391273}  & \numprint{-36.52} & \numprint{982858}  & \numprint{821991}  & \numprint{-16.37} & \numprint{580351} & \numprint{340833} & \numprint{-41.27} & \numprint{72449} & \numprint{218900} & \numprint{+202.14} \\\hline
		Доля вариантов <<серебряного стандарта>>, \%                                                                                             & \numprint{56.48}   & \numprint{35.85}   & ---               & \numprint{90.06}   & \numprint{75.32}   & ---               & \numprint{53.18}  & \numprint{31.23}  & ---               & \numprint{6.64}  & \numprint{20.06}  & ---                \\\hline
		\bigrow{Доля вариантов <<серебряного стандарта>>, \%}{Количество вариантов библиотеки,\newline отсутствующих в контрольных образцах}     & \numprint{1130049} & \numprint{84770}   & \numprint{-92.50} & \numprint{354044}  & \numprint{41719}   & \numprint{-88.22} & \numprint{14455}  & \numprint{2981}   & \numprint{-79.38} & ---              & ---               & ---                \\\hline
		\bigrow{Доля вариантов <<серебряного стандарта>>, \%}{Доля вариантов, отсутствующих в контрольных образцах, от вариантов библиотеки, \%} & \numprint{35.61}   & \numprint{6.07}    & \numprint{-82.95} & \numprint{9.44}    & \numprint{1.62}    & \numprint{-82.86} & ---               & ---               & ---               & ---              & ---               & ---                \\
		\hline
	\end{tabular}
\end{booktable}

\section*{ОБСУЖДЕНИЕ}

\subsection{Контрольные образцы}

<<Золотой стандарт>> с учётом подбора библиотек скорее всего является набором генетических вариантов, относящихся к экзомным регионам, так как одна из библиотек представляла собой результаты WES.
Их было обнаружено \numprint[\thousands]{75}, что соответствует оценкам среднего количества генетических вариантов в кодирующих регионах у человека "--- \numprint[\thousands]{100}\,\citep{Supernat_2018}.
Общее число несоответствий с референсным геномом у среднего человека составляет \numprint[\mln]{4.1--5}\,\citep{Auton_2015}, что с учётом гетерогенности клеточной линии K562 перекликается с общим количеством найденных нами генетических вариантов (\numprint[\mln]{5.5}).

Как видно из представленных выше данных, образец Banaszak~et~al.\,\cite{Banaszak_2018} содержит наибольшее число уникальных вариантов (\numprint[\%]{10.25}).
Это может быть связано с тем, что это данные полноэкзомного секвенирования, с высоким покрытием в экзонах, где и были найдены уникальные варианты.
В качестве дополнительной гипотезы можно предположить, что в этой работе использовались линии клеток, в значительной степени отличающиеся от классической линии K562.

Прослеживается ожидаемая положительная связь между глубиной секвенирования Hi-C\hyp{}библиотек и количеством уникальных вариантов в них.
В двух WGS\hyp{}библиотеках подобной связи не наблюдается.
Вероятнее всего, это также связано с отличиями использованных линий K562.

\subsection{Оценка результатов секвенирования Exo-C\hyp{}библиотек}

В Exo-C\hyp{}библиотеках глубина секвенирования составляет \numprint[\mln]{136.6} прочтений (\numprint[bp]{2.05e10}) и \numprint[\mln]{109.4} прочтений (\numprint[bp]{1.64e10}), а среднее покрытие в экзоме "--- \numprint{60.51} и \numprint{14.88} прочтений для ExoC-19 и ExoC-20 соответственно.
Глубину покрытия более 10 прочтений имеют \numprint[\%]{91.68} и \numprint[\%]{72.58} экзома для ExoC-19 и ExoC-20 соответственно.
Согласно \cite{Sims_2014}, для репрезентативных результатов экзомного секвенирования необходима глубина секвенирования не менее чем в \numprint[bp]{e10}, а для Hi-C "--- не менее чем \numprint[\mln]{100} прочтений.
Минимальным порогом глубины для возможности поиска генетических вариантов считается 10 прочтений, практически все гомозиготные SNV могут быть найдены при глубине в 15 прочтений, а гетерозиготные требуют глубину прочтений не менее 33.
Приемлемая доля экзома с репрезентативным покрытием (более 10 прочтений) составляет \numprint[\%]{90}.
Таким образом, можно утверждать, что ExoC-19 отвечает требованиям для поиска SNV, а ExoC-20, во-первых, пригодна к поиску только гомозиготных генетических вариантов, а во-вторых, имеет недостаточно хорошее покрытие в экзоме.

<<Золотой стандарт>> покрыт нашими библиотеками на \numprint[\%]{82.75} и \numprint[\%]{96.52}, <<серебряный стандарт>> "--- на \numprint[\%]{56.48} и \numprint[\%]{90.06} (библиотеки ExoC-19 и ExoC-20 соответственно).
Различия объясняются протоколами приготовления: у библиотеки ExoC-20 выше глубина покрытия в экзоме, в 6 раз выше обогащение в экзомных районах (критерий Манна---Уитни $p = 0.0003$).
Кроме того, в библиотеке ExoC-19 были использованы адаптерные последовательности, дающие большое количество шума.

Одним из базовых методов фильтрации генетических вариантов является фильтрация по глубине альтернативного аллеля.
Сразу можно обратить внимание на следующее:

\begin{itemize}
	\item В библиотеке ExoC-19 потеряна большая доля вариантов, чем в ExoC-20 "--- как относительно общего числа, так и относительно вариантов <<золотого>> и <<серебряного>> стандартов.
	\item Доля ложноположительных (отсутствующих в контрольных образцах) генетических вариантов снизилась в 5 раз.
\end{itemize}

Всё это можно объяснить наличием в библиотеке ExoC-19 большого количества регионов с низким покрытием, генетические варианты в которых были отсеяны фильтрацией по глубине.
То есть, фильтрация по глубине является эффективным способом улучшения данных низкого качества.

\section*{ВЫВОДЫ}

Таким образом, из приведённых нами данных можно сделать следующие выводы:

\begin{enumerate}
	\item Пайплайн, созданный нами с учётом актуальных рекомендаций для биоинформационной обработки, позволяет обрабатывать данные Exo-C\hyp{}секвенирования, а также находить в этих данных SNV.
	\item Использование разработанного конвейера биоинформационных инструментов позволило обнаружить около \numprint[\mln]{5.5} генетических вариантов в контрольных данных клеточной линии K562 (что сопоставимо со средним количеством точечных полиморфизмов в геноме человека), из которых наличие \numprint[\thousands]{75} подтвердилось в восьми независимых исследованиях, а \numprint[\mln]{1} "--- в шести независимых исследованиях, не включающих экзомные данные.
	\item Сравнение генетических вариантов, полученных из контрольных образцов и Exo-C\hyp{}библиотек, позволяет утверждать, что метод Exo-C способен детектировать около \numprint[\%]{75--90} SNV, обнаруживаемых другими методами.
\end{enumerate}

\section*{КОНФЛИКТ ИНТЕРЕСОВ}

Авторы декларируют отсутствие конфликта интересов.

\acknowledgment{
  Благодарности и ссылки на гранты размещаются здесь. Thanks and references to fundings are written here.
}

\clearpage

\newpage

\begin{albumtable}{Библиотеки данных секвенирования клеточной линии K562}{appendix:control-libs}
	\begin{tabular}{| l | l | l | l | l | l | r | r | r | r | r | r | r | r | r | r | r |}
		\hline
		\rowcolor{mbbcolor}
		Библиотека                                                        &
		Статья                                                            &
		Репозиторий                                                       &
		Коды доступа                                                      &
		Тип данных                                                        &
		Тип прочтений                                                     &
		\headerbigrow{ прочтений}{Глубина, прочтений}                              &
		\headerbigrow{Общее число}{Общее число прочтений}                          &
		\headerbigrow{Доля картированных,}{Доля картированных, \% от общего числа} &
		\headerbigrow{\% от общего числа}{Доля добавочных, \% от общего числа}     &
		\headerbigrow{Картированные}{Картированные PE прочтения}                   &
		\headerbigrow{Картированные}{Картированные синглетоны}                     &
		\headerbigrow{PE прочтений}{Дубликаты\newline PE прочтений}                &
		\headerbigrow{синглетонов }{Дубликаты синглетонов}                         &
		Доля дубликатов, \%                                               &
		\headerbigrow{Оценка размера}{Оценка размера библиотеки}
		\\
		\hline
		\multicolumn{16}{| c |}{\headerbigrow{\Large Контрольные данные}{\Large Контрольные данные}}                                                                                                                                                                                                                                                                                                                                                                                                                          \\
		\hline
		GSM1551618\_HIC069                                                         & Rao~et~al.\,\cite{Rao_2014}       & GEO    & SRR1658693                                                                                                                              & Hi-C                & PE & \numprint{456757799}  & \numprint{1001169248} & \numprint{96.57} & \numprint{8.755}  & \numprint{424945100} & \numprint{29290805}   & \numprint{17848021} & \numprint{13182626}  & \numprint{5.56}  & \numprint{4916114832}  \\\hline
		GSM1551619\_HIC070                                                         & Rao~et~al.\,\cite{Rao_2014}       & GEO    & SRR1658694                                                                                                                              & Hi-C                & PE & \numprint{591854553}  & \numprint{1314487595} & \numprint{98.7}  & \numprint{9.949}  & \numprint{575565379} & \numprint{15452072}   & \numprint{98778796} & \numprint{8811532}   & \numprint{17.69} & \numprint{1478944337}  \\\hline
		GSM1551620\_HIC071                                                         & Rao~et~al.\,\cite{Rao_2014}       & GEO    & \parbox[c][3.8em]{\widthof{ENCFF004THU  }}{SRR1658695\newline SRR1658696}                                                               & Hi-C                & PE & \numprint{79905895}   & \numprint{173931529}  & \numprint{98.81} & \numprint{8.118}  & \numprint{77880938}  & \numprint{1975600}    & \numprint{486893}   & \numprint{269138}    & \numprint{0.79}  & \numprint{6202732721}  \\\hline
		GSM1551621\_HIC072                                                         & Rao~et~al.\,\cite{Rao_2014}       & GEO    & \parbox[c][3.8em]{\widthof{ENCFF004THU  }}{SRR1658697\newline SRR1658698}                                                               & Hi-C                & PE & \numprint{79578049}   & \numprint{159160116}  & \numprint{98.38} & \numprint{0.003}  & \numprint{77155821}  & \numprint{2265995}    & \numprint{366805}   & \numprint{285395}    & \numprint{0.65}  & \numprint{8088955029}  \\\hline
		GSM1551622\_HIC073                                                         & Rao~et~al.\,\cite{Rao_2014}       & GEO    & \parbox[c][3.8em]{\widthof{ENCFF004THU  }}{SRR1658699\newline SRR1658700}                                                               & Hi-C                & PE & \numprint{77353816}   & \numprint{154710364}  & \numprint{98.33} & \numprint{0.002}  & \numprint{74866287}  & \numprint{2383970}    & \numprint{240304}   & \numprint{293115}    & \numprint{0.51}  & \numprint{11637260975} \\\hline
		GSM1551623\_HIC074                                                         & Rao~et~al.\,\cite{Rao_2014}       & GEO    & \parbox[c][3.8em]{\widthof{ENCFF004THU  }}{SRR1658702\newline SRR1658701}                                                               & Hi-C                & PE & \numprint{80778733}   & \numprint{175291763}  & \numprint{98.65} & \numprint{7.835}  & \numprint{78467294}  & \numprint{2254814}    & \numprint{644986}   & \numprint{321965}    & \numprint{1.01}  & \numprint{4746870162}  \\\hline
		ENCSR025GPQ                                                                & Zhou~et~al.\,\cite{Zhou_2019}      & ENCODE & \parbox[c][5.5em]{\widthof{ENCFF004THU  }}{ENCFF574YLG\newline ENCFF921AXL\newline ENCFF590SSX}                                         & WGS                 & SE & \numprint{258022356}  & \numprint{260044021}  & \numprint{85.39} & \numprint{0.777}  & ---                  & \numprint{220029156}  & ---                 & \numprint{50689083}  & \numprint{23.04} & ---                    \\\hline
		ENCSR053AXS                                                                & Zhou~et~al.\,\cite{Zhou_2019}      & ENCODE & \parbox[c][8.8em]{\widthof{ENCFF004THU  }}{ENCFF004THU\newline ENCFF066GQD\newline ENCFF313MGL\newline ENCFF506TKC\newline ENCFF080MQF} & WGS                 & SE & \numprint{1472492722} & \numprint{1592540515} & \numprint{91.19} & \numprint{7.538}  & ---                  & \numprint{1332175586} & ---                 & \numprint{496237198} & \numprint{37.25} & ---                    \\\hline
		ENCSR711UNY                                                                & Zhou~et~al.\,\cite{Zhou_2019}      & ENCODE & \parbox[c][5.5em]{\widthof{ENCFF004THU  }}{ENCFF471WSA\newline ENCFF826SYZ\newline ENCFF590SSX}                                         & WGS                 & SE & \numprint{890796215}  & \numprint{899473769}  & \numprint{99.72} & \numprint{0.965}  & ---                  & \numprint{888239055}  & ---                 & \numprint{203498352} & \numprint{22.91} & ---                    \\\hline
		SRX3358201                                                                 & Dixon~et~al.\,\cite{Dixon_2018}     & GEO    & SRR6251264                                                                                                                              & WGS                 & PE & \numprint{366291496}  & \numprint{737534099}  & \numprint{99.72} & \numprint{0.671}  & \numprint{364794328} & \numprint{923254}     & \numprint{73018048} & \numprint{406066}    & \numprint{20.05} & \numprint{785091005}   \\\hline
		GSE148362\_G1                                                              & Wang~et~al.\,\cite{Wang_2020}      & GEO    & SRR11518301                                                                                                                             & Repli-seq           & SE & \numprint{24804095}   & \numprint{24804396}   & \numprint{96.39} & \numprint{0.001}  & ---                  & \numprint{23909072}   & ---                 & \numprint{921353}    & \numprint{3.85}  & ---                    \\\hline
		GSE148362\_G2                                                              & Wang~et~al.\,\cite{Wang_2020}      & GEO    & SRR11518308                                                                                                                             & Repli-seq           & SE & \numprint{33032314}   & \numprint{33033010}   & \numprint{97.61} & \numprint{0.002}  & ---                  & \numprint{32241907}   & ---                 & \numprint{3881991}   & \numprint{12.04} & ---                    \\\hline
		GSE148362\_S1                                                              & Wang~et~al.\,\cite{Wang_2020}      & GEO    & SRR11518302                                                                                                                             & Repli-seq           & SE & \numprint{30884788}   & \numprint{30885298}   & \numprint{98.7}  & \numprint{0.002}  & ---                  & \numprint{30481936}   & ---                 & \numprint{2156480}   & \numprint{7.07}  & ---                    \\\hline
		GSE148362\_S2                                                              & Wang~et~al.\,\cite{Wang_2020}      & GEO    & SRR11518303                                                                                                                             & Repli-seq           & SE & \numprint{45359273}   & \numprint{45360305}   & \numprint{98.39} & \numprint{0.002}  & ---                  & \numprint{44630884}   & ---                 & \numprint{1939846}   & \numprint{4.35}  & ---                    \\\hline
		GSE148362\_S3                                                              & Wang~et~al.\,\cite{Wang_2020}      & GEO    & SRR11518304                                                                                                                             & Repli-seq           & SE & \numprint{49807076}   & \numprint{49807988}   & \numprint{98.79} & \numprint{0.002}  & ---                  & \numprint{49205535}   & ---                 & \numprint{2889464}   & \numprint{5.87}  & ---                    \\\hline
		GSE148362\_S4                                                              & Wang~et~al.\,\cite{Wang_2020}      & GEO    & SRR11518305                                                                                                                             & Repli-seq           & SE & \numprint{44149029}   & \numprint{44149770}   & \numprint{98.46} & \numprint{0.002}  & ---                  & \numprint{43469002}   & ---                 & \numprint{2678091}   & \numprint{6.16}  & ---                    \\\hline
		GSE148362\_S5                                                              & Wang~et~al.\,\cite{Wang_2020}      & GEO    & SRR11518306                                                                                                                             & Repli-seq           & SE & \numprint{38424060}   & \numprint{38424835}   & \numprint{97.96} & \numprint{0.002}  & ---                  & \numprint{37640056}   & ---                 & \numprint{3600260}   & \numprint{9.57}  & ---                    \\\hline
		GSE148362\_S6                                                              & Wang~et~al.\,\cite{Wang_2020}      & GEO    & SRR11518307                                                                                                                             & Repli-seq           & SE & \numprint{35203005}   & \numprint{35203676}   & \numprint{97.51} & \numprint{0.002}  & ---                  & \numprint{34324742}   & ---                 & \numprint{4177438}   & \numprint{12.17} & ---                    \\\hline
		INSITU\_HS1                                                                & Ray~et~al.\,\cite{Ray_2019}       & GEO    & SRR9019504                                                                                                                              & Hi-C                & PE & \numprint{86294895}   & \numprint{172589790}  & \numprint{93.3}  & 0                 & \numprint{75521119}  & \numprint{9982274}    & \numprint{1841061}  & \numprint{1615286}   & \numprint{3.29}  & \numprint{1523677153}  \\\hline
		INSITU\_HS2                                                                & Ray~et~al.\,\cite{Ray_2019}       & GEO    & SRR9019505                                                                                                                              & Hi-C                & PE & \numprint{127093919}  & \numprint{254187838}  & \numprint{93.36} & 0                 & \numprint{111730240} & \numprint{13858195}   & \numprint{1923146}  & \numprint{3048273}   & \numprint{2.91}  & \numprint{3208280267}  \\\hline
		INSITU\_NHS1                                                               & Ray~et~al.\,\cite{Ray_2019}       & GEO    & SRR9019506                                                                                                                              & Hi-C                & PE & \numprint{86445594}   & \numprint{172891188}  & \numprint{93.43} & 0                 & \numprint{75893138}  & \numprint{9737847}    & \numprint{1903981}  & \numprint{1649376}   & \numprint{3.38}  & \numprint{1487154386}  \\\hline
		INSITU\_NHS2                                                               & Ray~et~al.\,\cite{Ray_2019}       & GEO    & SRR9019507                                                                                                                              & Hi-C                & PE & \numprint{128472386}  & \numprint{256944772}  & \numprint{93.27} & 0                 & \numprint{112615319} & \numprint{14417076}   & \numprint{1961996}  & \numprint{3196535}   & \numprint{2.97}  & \numprint{3194317878}  \\\hline
		PDDE\_TRANSIENT                                                            & Moquin~et~al.\,\cite{Moquin_2017}    & GEO    & \parbox[c][3.8em]{\widthof{ENCFF004THU  }}{SRR5470541\newline SRR5470540}                                                               & Hi-C                & PE & \numprint{55158049}   & \numprint{110319638}  & \numprint{95.6}  & \numprint{0.003}  & \numprint{51158920}  & \numprint{3140556}    & \numprint{3917308}  & \numprint{721938}    & \numprint{8.11}  & \numprint{316780447}   \\\hline
		PD\_STABLE\_REP1                                                           & Moquin~et~al.\,\cite{Moquin_2017}    & GEO    & \parbox[c][3.8em]{\widthof{ENCFF004THU  }}{SRR5470535\newline SRR5470534}                                                               & Hi-C                & PE & \numprint{67172619}   & \numprint{134347099}  & \numprint{97.58} & \numprint{0.001}  & \numprint{64767511}  & \numprint{1565427}    & \numprint{5573966}  & \numprint{376260}    & \numprint{8.79}  & \numprint{354373851}   \\\hline
		PD\_STABLE\_REP2                                                           & Moquin~et~al.\,\cite{Moquin_2017}    & GEO    & \parbox[c][3.8em]{\widthof{ENCFF004THU  }}{SRR5470536\newline SRR5470537}                                                               & Hi-C                & PE & \numprint{52872167}   & \numprint{105745908}  & \numprint{98.23} & \numprint{0.001}  & \numprint{51442087}  & \numprint{993483}     & \numprint{2058449}  & \numprint{217598}    & \numprint{4.17}  & \numprint{625522723}   \\\hline
		PD\_TRANSIENT                                                              & Moquin~et~al.\,\cite{Moquin_2017}    & GEO    & \parbox[c][3.8em]{\widthof{ENCFF004THU  }}{SRR5470539\newline SRR5470538}                                                               & Hi-C                & PE & \numprint{81297824}   & \numprint{162600928}  & \numprint{95.28} & \numprint{0.003}  & \numprint{75141163}  & \numprint{4639787}    & \numprint{7298377}  & \numprint{1339404}   & \numprint{10.29} & \numprint{361336652}   \\\hline
		GSM2588815\_R1                                                             & Belaghzal~et~al.\,\cite{Belaghzal_2017} & GEO    & SRR5479813                                                                                                                              & Hi-C                & PE & \numprint{72914268}   & \numprint{172533452}  & \numprint{99.39} & \numprint{15.478} & \numprint{72067575}  & \numprint{648294}     & \numprint{9694590}  & \numprint{210273}    & \numprint{13.54} & \numprint{243264112}   \\\hline
		GSM2536769\_WT                                                             & Banaszak~et~al.\,\cite{Banaszak_2018}  & GEO    & SRR5345331                                                                                                                              & WES\footnotemark[1] & PE & \numprint{39211303}   & \numprint{78464649}   & \numprint{99.46} & \numprint{0.054}  & \numprint{38914993}  & \numprint{171253}     & \numprint{7821960}  & \numprint{91145}     & \numprint{20.17} & \numprint{83342746}    \\\hline
		\footnotetext[1]{Варианты в гене DNMT3A были исключены из выборки.}
		GSM2536770\_WT\_TF                                                         & Banaszak~et~al.\,\cite{Banaszak_2018}  & GEO    & SRR5345332                                                                                                                              & WES\footnotemark[1] & PE & \numprint{49394206}   & \numprint{98820633}   & \numprint{99.54} & \numprint{0.033}  & \numprint{49068605}  & \numprint{193565}     & \numprint{10478814} & \numprint{114795}    & \numprint{21.43} & \numprint{97869629}    \\\hline
		GSM2536771\_MT2                                                            & Banaszak~et~al.\,\cite{Banaszak_2018}  & GEO    & SRR5345333                                                                                                                              & WES\footnotemark[1] & PE & \numprint{42020936}   & \numprint{84093776}   & \numprint{99.63} & \numprint{0.062}  & \numprint{41772436}  & \numprint{189177}     & \numprint{8755216}  & \numprint{104927}    & \numprint{21.04} & \numprint{85177326}    \\\hline
		GSM2536772\_MT3                                                            & Banaszak~et~al.\,\cite{Banaszak_2018}  & GEO    & SRR5345334                                                                                                                              & WES\footnotemark[1] & PE & \numprint{43669613}   & \numprint{87375385}   & \numprint{99.6}  & \numprint{0.041}  & \numprint{43414109}  & \numprint{164448}     & \numprint{9489133}  & \numprint{93601}     & \numprint{21.92} & \numprint{84242110}    \\\hline
		GSM2536773\_MT4                                                            & Banaszak~et~al.\,\cite{Banaszak_2018}  & GEO    & SRR5345335                                                                                                                              & WES\footnotemark[1] & PE & \numprint{39879263}   & \numprint{79788847}   & \numprint{99.53} & \numprint{0.038}  & \numprint{39609943}  & \numprint{166651}     & \numprint{8590165}  & \numprint{90809}     & \numprint{21.76} & \numprint{77577055}    \\\hline
		GSM2536774\_MT5                                                            & Banaszak~et~al.\,\cite{Banaszak_2018}  & GEO    & SRR5345336                                                                                                                              & WES\footnotemark[1] & PE & \numprint{40807904}   & \numprint{81649292}   & \numprint{99.59} & \numprint{0.041}  & \numprint{40559969}  & \numprint{163957}     & \numprint{8801283}  & \numprint{91545}     & \numprint{21.77} & \numprint{79383290}    \\
		\hline
		\multicolumn{16}{| c |}{\headerbigrow{\Large Тестовые данные}{\Large Тестовые данные}}                                                                                                                                                                                                                                                                                                                                                                                                                                \\
		\hline
		FG\_ExoCBel-001                                                            & ExoC-19                     & ---    & ---                                                                                                                                     & Exo-C               & PE & \numprint{136609179}  & \numprint{359215777}  & \numprint{99.31} & \numprint{23.940} & \numprint{135150334} & \numprint{443409}     & \numprint{25453568} & \numprint{159152}    & \numprint{18.86} & \numprint{319784450}   \\\hline
		FG\_Quarantine-A                                                           & ExoC-20                     & ---    & ---                                                                                                                                     & Exo-C               & PE & \numprint{53598130}   & \numprint{140214460}  & \numprint{99.79} & \numprint{23.150} & \numprint{53598130}  & \numprint{259561}     & \numprint{7809282}  & \numprint{68779}     & \numprint{14.60} & \numprint{193853459}   \\\hline
		FG\_Quarantine-B                                                           & ExoC-20                     & ---    & ---                                                                                                                                     & Exo-C               & PE & \numprint{55279173}   & \numprint{144641130}  & \numprint{99.76} & \numprint{23.108} & \numprint{55279173}  & \numprint{310369}     & \numprint{8808307}  & \numprint{90489}     & \numprint{15.97} & \numprint{177375163}   \\\hline
	\end{tabular}
\end{albumtable}

\begin{albumtable}{Образцы данных секвенирования клеточной линии K562}{appendix:control-samples}
	\begin{tabular}{| l | l | l | r | r | r | r | r | r | r | r | r | r |}
		\hline
		\rowcolor{mbbcolor}
		Образец                                                                                              &
		Тип данных                                                                                           &
		Тип прочтений                                                                                        &
		\headerbigrow{ прочтений}{Глубина, прочтений}                                                                 &
		\headerbigrow{Общее число}{Общее число прочтений}                                                             &
		\headerbigrow{Доля картированных,}{Доля картированных, \% от общего числа}                                    &
		\headerbigrow{\% от общего числа}{Доля добавочных, \% от общего числа}                                        &
		\headerbigrow{\% от картированных}{FR PE прочтения,\newline \% от картированных}                              &
		\headerbigrow{Картированные}{Картированные PE прочтения}                                                      &
		\headerbigrow{Картированные}{Картированные синглетоны}                                                        &
		\headerbigrow{Картированные на разные хромосомы}{Картированные на разные хромосомы пары, \% от картированных} &
		\headerbigrow{Картированные на разные хромосомы пары (QMAP 4+),}{Картированные на разные хромосомы пары (QMAP 4+), \% от картированных на разные хромосомы}
		\\
		\hline
		\multicolumn{12}{| c |}{\headerbigrow{\Large Контрольные данные}{\Large Контрольные данные}}                                                                                                                                                                                                                                 \\
		\hline
		Rao~et~al.\,\cite{Rao_2014}                                                                                         & Hi-C      & PE & \numprint{1366228845} & \numprint{2978750615} & \numprint{97.95} & \numprint{8.268}  & \numprint{27.04} & \numprint{2617761638} & \numprint{53623256} & \numprint{21.03} & \numprint{84.23} \\\hline
		Zhou~et~al.\,\cite{Zhou_2019}                                                                                        & WGS       & SE & \numprint{2621311293} & \numprint{2752058305} & \numprint{93.43} & \numprint{4.751}  & ---              & ---                   & ---                 & ---              & ---              \\\hline
		Dixon~et~al.\,\cite{Dixon_2018}                                                                                       & WGS       & PE & \numprint{366291496}  & \numprint{737534099}  & \numprint{99.72} & \numprint{0.671}  & \numprint{97.16} & \numprint{729588656}  & \numprint{923254}   & \numprint{1.25}  & \numprint{51.22} \\\hline
		Wang~et~al.\,\cite{Wang_2020}                                                                                        & Repli-seq & SE & \numprint{301663640}  & \numprint{301669278}  & \numprint{98.09} & \numprint{0.002}  & ---              & ---                   & ---                 & ---              & ---              \\\hline
		Ray~et~al.\,\cite{Ray_2019}                                                                                         & Hi-C      & PE & \numprint{428306794}  & \numprint{856613588}  & \numprint{93.33} & 0                 & \numprint{35.92} & \numprint{751519632}  & \numprint{47995392} & \numprint{22.77} & \numprint{76.00} \\\hline
		Moquin~et~al.\,\cite{Moquin_2017}                                                                                      & Hi-C      & PE & \numprint{256500659}  & \numprint{513013573}  & \numprint{96.56} & \numprint{0.002}  & \numprint{46.64} & \numprint{485019362}  & \numprint{10339253} & \numprint{17.76} & \numprint{75.56} \\\hline
		Belaghzal~et~al.\,\cite{Belaghzal_2017}                                                                                   & Hi-C      & PE & \numprint{72914268}   & \numprint{172533452}  & \numprint{99.39} & \numprint{15.478} & \numprint{24.77} & \numprint{144135150}  & \numprint{648294}   & \numprint{34.02} & \numprint{88.02} \\\hline
		Banaszak~et~al.\,\cite{Banaszak_2018}                                                                                    & WES       & PE & \numprint{254983225}  & \numprint{510192582}  & \numprint{99.56} & \numprint{0.044}  & \numprint{99.41} & \numprint{506680110}  & \numprint{1049051}  & \numprint{0.11}  & \numprint{81.38} \\\hline
		\rowcolor{white}
		\multicolumn{12}{| c |}{\headerbigrow{\Large Тестовые данные}{\Large Тестовые данные}}                                                                                                                                                                                                                                       \\
		\hline
		ExoC-19                                                                                                       & Exo-C     & PE & \numprint{136609179}  & \numprint{359215777}  & \numprint{99.31} & \numprint{23.94}  & \numprint{89.22} & \numprint{270300668}  & \numprint{443409}   & \numprint{5.01}  & \numprint{66.93} \\\hline
		ExoC-20                                                                                                       & Exo-C     & PE & \numprint{109486529}  & \numprint{284855590}  & \numprint{99.77} & \numprint{23.13}  & \numprint{70.02} & \numprint{217754606}  & \numprint{569930}   & \numprint{5.87}  & \numprint{78.00} \\\hline
	\end{tabular}
\end{albumtable}

\newpage

\bibliography{Sources}

\acceptdate

%%----------------   TRANSLATED ARTICLE METADATA BLOCK  ------------------------
%%---- THE FIELDS SHOULD BE FILLED IN THE SECOND LANGUAGE OF THE ARTICLE ----

\renewcommand{\runningreference}{}
%\lang{rus}                               
\lang{eng}
\artclass{SECTION}
%\artclassB{Continuation of SECTION}
\title{This is a manuscript template for "Mathematical Biology and Bioinformatics"}

\authors{
	Ivanov I.\footnotemark[1]$^{1,2}$,
	Petrov I.\footnotemark[7]$^{1}$
}

\affiliations{
	$^1$Institution 1, Sity1, Country1\\
	$^2$Institution 2, Sity2, Country2
}

\abstract{
		The abstract should be a single subsection in the length of about  200--250 words (at least 25 lines). It should contain a brief description of used approaches. The main results of the work should be formulated here. URLs must be included only in one case. These must be the links to the websites where data, software or tools referred to in the article are hosted. References should not be included in the abstract. 
}

\keywords{the first one, the second, ..., the last.}
\newpage
\maketitle

\end{document}

