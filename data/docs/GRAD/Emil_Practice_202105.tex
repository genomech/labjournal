\documentclass[a4paper,12pt]{extarticle}

% ------ GLOBAL ------
\usepackage{indentfirst}
\usepackage{calc,etoolbox,float,microtype,soul,xspace,textcomp,xltxtra}
\usepackage[table]{xcolor}

% ------ LANGUAGE ------

\usepackage{polyglossia}
\setdefaultlanguage[babelshorthands=true]{russian}
\setotherlanguage{english}
\defaultfontfeatures{Ligatures=TeX,Mapping=tex-text}

% ------ PAGE VIEW ------

\usepackage[left=3cm,right=1.5cm,top=2cm,bottom=2cm]{geometry}
\usepackage{hyphenat}

\usepackage{enumitem}
\setlist[itemize]{label={---}}

\setmainfont[Ligatures=TeX]{Liberation Serif}
\setsansfont[Ligatures=TeX]{Liberation Sans}
\setlength{\parindent}{0cm}

\linespread{1.3}

\renewcommand{\labelitemi}{$\bullet$}

% ------ HEADER & FOOTER ------

\usepackage{fancyhdr}
\pagestyle{fancy}
\fancyhead{}\renewcommand{\headrulewidth}{0mm}\fancyfoot[CE,CO]{\thepage}
\fancypagestyle{plain}{\fancyhead{}\renewcommand{\headrulewidth}{0mm}\fancyfoot{}}

% ------ TABLES ------

\usepackage{multirow,tabularx,rotating,wrapfig}

\definecolor{tableheadcolor}{RGB}{200,200,200}
\definecolor{tableoddrowcolor}{RGB}{238,238,238}
\definecolor{tableevenrowcolor}{gray}{1.0}

\newsavebox{\defaultsavebox}

\newcommand{\headerbigrow}[2]{\parbox[c][3.8em]{\widthof{\textbf{#1}}}{\textbf{#2}}}
\newcommand{\bigrow}[2]{\parbox[c][3.8em]{\widthof{#1}}{#2}}

\newenvironment{booktable}[2]
{\begin{table}[H]\setlength\arrayrulewidth{1pt}\begin{lrbox}{\defaultsavebox}\bgroup\def\arraystretch{1}}
{\egroup\end{lrbox}\resizebox{\textwidth}{!}{\usebox{\defaultsavebox}}\end{table}}

\newcommand{\neworrenewcommand}[1]{\providecommand{#1}{}\renewcommand{#1}}

\newcommand{\variant}[9]{\neworrenewcommand{\vvariant}[4]{\begin{table}[H]\setlength\arrayrulewidth{1pt}\begin{lrbox}{\defaultsavebox}\bgroup\def\arraystretch{1}\begin{tabular}{| l | l | l | l | l | l | l |}\rowcolor{tableheadcolor}\hline\textbf{Ген} & \textbf{Положение (hg19)} & \textbf{Замена} & \textbf{Генотип} & \textbf{Тип замены} & \textbf{Глубина} & \textbf{Ref/Alt} \\ \hline \hline #1 & #2 & #3 & #4 & #5 & #6 & #7 \\ \hline \hline \multicolumn{2}{| >{\bfseries}p{0.3\textwidth}}{~} & \multicolumn{5}{ >{}p{0.65\textwidth} |}{~} \\ \hline \multicolumn{2}{| >{\bfseries}p{0.3\textwidth}}{OMIM фенотип:} & \multicolumn{5}{ >{}p{0.65\textwidth} |}{#8} \\ \hline \multicolumn{2}{| >{\bfseries}p{0.3\textwidth}}{HGMD:} & \multicolumn{5}{ >{}p{0.65\textwidth} |}{#9} \\ \hline \multicolumn{2}{| >{\bfseries}p{0.3\textwidth}}{Экзон:} & \multicolumn{5}{ >{}p{0.65\textwidth} |}{##1} \\ \hline 
\multicolumn{2}{| >{\bfseries}p{0.3\textwidth}}{Частота аллеля:} & \multicolumn{5}{ >{}p{0.65\textwidth} |}{##2} \\ \hline 
\multicolumn{2}{| >{\bfseries}p{0.3\textwidth}}{Критерии патогенности (согласно российским рекомендациям по интерпретации данных MPS\,\ecitep{Ryzhkova_2017}):} & \multicolumn{5}{ >{}p{0.65\textwidth} |}{##3} \\ \hline
\multicolumn{2}{| >{\bfseries}p{0.3\textwidth}}{Комментарий:} & \multicolumn{5}{ >{}p{0.65\textwidth} |}{##4} \\ \hline \hline\end{tabular}\egroup\end{lrbox}\resizebox{\textwidth}{!}{\usebox{\defaultsavebox}}\end{table}}\vvariant}

\newcommand{\reportgen}[9]{
\begin{tabular}{ >{\bfseries}p{0.35\textwidth} l }
Код пациента: & #1 \\ \hline
Вид материала: & #2 \\ \hline
Дата забора: & #3 \\ \hline
Экзомная панель: & #4 \\ \hline
Количество прочтений: & \numprint[\mln]{#5} \\ \hline
Длина прочтений: & \numprint[bp]{#6} \\ \hline
Выравнивание: & \numprint[\%]{#7} \\ \hline
Покрытие экзома: & ~ \\ \hline
~~~~~~ -- $\geqslant10$: & \numprint[\%]{#8} \\ \hline
~~~~~~ -- среднее: & \numprint[прочтений/позицию]{#9} \\ \hline
\end{tabular}}

\newenvironment{albumtable}[2]
{\begin{sidewaystable}\caption{\label{#2}#1}\vspace{0.5em}\setlength\arrayrulewidth{1pt}\begin{lrbox}{\defaultsavebox}\bgroup\def\arraystretch{1}\rowcolors{2}{tableoddrowcolor}{tableevenrowcolor}}
			{\egroup\end{lrbox}\resizebox{1\textheight}{!}{\usebox{\defaultsavebox}}\end{sidewaystable}}

\newenvironment{albumtablewidth}[2]
{\begin{sidewaystable}\caption{\label{#2}#1}\vspace{0.5em}\setlength\arrayrulewidth{1pt}\begin{lrbox}{\defaultsavebox}\bgroup\def\arraystretch{1}\rowcolors{2}{tableoddrowcolor}{tableevenrowcolor}}
			{\egroup\end{lrbox}\resizebox{0.8\textheight}{!}{\usebox{\defaultsavebox}}\end{sidewaystable}}

% ------ FIGURES ------

\usepackage{graphicx}

\newcommand{\intextfigure}[5]
{\begin{wrapfigure}{#1}{#2\textwidth}\centering\includegraphics[width=#2\textwidth]{#3}\caption{\label{#4}#5}\end{wrapfigure}}

\newcommand{\centerfigure}[5]
{\begin{figure}[#1]\centering\includegraphics[width=#5\textwidth]{#2}\caption{\label{#3}#4}\end{figure}}

% ------ FORMULAE ------

\usepackage{amsmath}
\usepackage{amssymb}

\usepackage{numprint}
\npthousandsep{\,}
\npdecimalsign{,}

\newcommand{\thousands}{тыс.}
\newcommand{\hours}{ч.}
\newcommand{\mln}{млн}
\newcommand{\ml}{мл}
\newcommand{\kilogramm}{кг}
\newcommand{\gramm}{г}
\newcommand{\mug}{мкг}
\newcommand{\grammliter}{г/л}
\newcommand{\mgpl}{мг/л}
\newcommand{\mugpl}{мкг/л}
\newcommand{\KOEpml}{КОЕ/мл}
\newcommand{\mugpdl}{мкг/дл}
\newcommand{\eliter}{Ед/л}
\newcommand{\pliter}{кл/л}
\newcommand{\secs}{сек}
\newcommand{\mmolpl}{ммоль/л}
\newcommand{\mkmolpl}{мкмоль/л}
\newcommand{\pgpml}{пг/мл}
\newcommand{\pmolpl}{пмоль/л}
\newcommand{\nmolpl}{нмоль/л}
\newcommand{\ngpml}{нг/мл}
\newcommand{\ngpdl}{нг/дл}
\newcommand{\mgpkg}{мг/кг}
\newcommand{\mgpdl}{мг/дл}
\newcommand{\mgpd}{мг/сут.}
\newcommand{\mugpd}{мкг/сут.}
\newcommand{\mmolpd}{ммоль/сут}
\newcommand{\mkmolpd}{мкмоль/сут}
\newcommand{\copypml}{копий/мл}
\newcommand{\mugpkg}{мкг/кг}
\newcommand{\pov}{в~п/з}
\newcommand{\tablets}{табл.}
\newcommand{\mg}{мг}
\newcommand{\fliter}{фл}
\newcommand{\hz}{Гц}
\newcommand{\cm}{см}
\newcommand{\mm}{мм}
\newcommand{\months}{мес.}
\newcommand{\weeks}{нед.}
\newcommand{\oCelsius}{\tcdegree{}C}
\newcommand{\bpm}{уд/мин}
\newcommand{\mpm}{/мин}
\newcommand{\torr}{торр}
\newcommand{\mmph}{мм/ч}
\newcommand{\me}{ME/л}
\newcommand{\mme}{мME/л}
\newcommand{\mosmpkg}{мОсм/кг}
\newcommand{\genename}[1]{\textit{#1}}
\newcommand{\utilname}[1]{\textenglish{#1}}

\newcommand{\pdate}[1]{\emph{#1:} }
\newcommand{\cdate}[2]{\strong{#2 от #1:} }
\newcommand{\DS}[2]{[#2] #1}

% ------ BIBLIOGRAPHY ------

\usepackage[square,sort,semicolon,authoryear]{natbib}
% \bibliographystyle{naturemag}
\bibliographystyle{ugost2008ns}
\makeatletter
\renewcommand{\@biblabel}[1]{#1.}
\makeatother

\newcommand{\ecitep}[1]{\textenglish{\citep{#1}}}

% ------ LINKS ------

\usepackage{hyperref}
\definecolor{linkcolor}{RGB}{0,102,153}
\hypersetup{colorlinks=true, linkcolor=linkcolor, citecolor=linkcolor, filecolor=linkcolor, urlcolor=linkcolor}
\renewcommand{\url}[1]{\href{#1}{#1}}

% ------ REFERENCES ------

\newcommand{\picref}[1]{Рис.~\ref{#1}}
\newcommand{\tableref}[1]{Табл.~\ref{#1}}
\newcommand{\formularef}[1]{Формула~\ref{#1}}
\newcommand{\engterm}[1]{англ. \textenglish{\textit{#1}}}

\usepackage{titlesec}
\titleformat{\section}{\newpage\normalfont\huge\bfseries\filcenter}{Глава~\thesection.}{0.5em}{}
\titleformat{\subsection}{\normalfont\Large\bfseries\filcenter}{\thesubsection.}{0.5em}{}
\titleformat{\subsubsection}{\normalfont\large\bfseries\filcenter}{\thesubsubsection.}{0.5em}{}

\begin{document}

\begin{titlepage}
	\centering
	{\par\small{ФЕДЕРАЛЬНОЕ ГОСУДАРСТВЕННОЕ АВТОНОМНОЕ ОБРАЗОВАТЕЛЬНОЕ УЧРЕЖДЕНИЕ ВЫСШЕГО ОБРАЗОВАНИЯ <<НОВОСИБИРСКИЙ НАЦИОНАЛЬНЫЙ ИССЛЕДОВАТЕЛЬСКИЙ ГОСУДАРСТВЕННЫЙ УНИВЕРСИТЕТ>> (НОВОСИБИРСКИЙ ГОСУДАРСТВЕННЫЙ УНИВЕРСИТЕТ, НГУ)}}
	{\par\small{Институт медицины и психологии В.\,Зельмана НГУ}}
	{\par\small{Направление подготовки: 31.05.01 Лечебное дело}}

	\vspace{5cm}

	{\par\Large\textbf{ДНЕВНИК-ОТЧЕТ\\по производственной практике 5 курса}}

	\vspace{0.5cm}

	ФИО: \hrulefill\\
	Группа \hrulefill Срок прохождения практики с \hrulefill по \hrulefill

	\vfill

	{\centering\small{Новосибирск, 2021}}
\end{titlepage}

\newpage

\subsection*{ПРАВИЛА ВЕДЕНИЯ ДНЕВНИКА}

Дневник является официальным документом по производственной практике. 
Он должен быть написан разборчиво, грамотно, медицинским языком.
Студенты работают по пятидневной рабочей неделе.
Записи в дневнике ведутся ежедневно в конце рабочего дня и должны отражать всю 
выполненную работу в подразделениях больницы с примерами.
Выполненная работа ежедневно заверяется подписью медицинской сестры (постовой, процедурного кабинета, перевязочной и т.~д.).
В начале дневника даётся краткая характеристика отделения: профиль отделения, количество коек, штат отделения, наличие вспомогательных кабинетов и пр.
После окончания практики, студент, на основании записей в дневнике, должен заполнить сводный цифровой отчёт о проделанной работе.
Санитарно\hyp{}просветительная работа проводится в форме бесед, санбюллютеней;
ее содержание, место и время проведения должны быть отражены в дневнике и заверены подписью непосредственного руководителя практики.
УИРС выполняется в виде реферата и сдается вместе с дневником для проверки ассистенту\hyp{}руководителю практики.
После окончания практики непосредственный руководитель дает характеристику работы студента и оценивает ее по пятибалльной шкале. 
Характеристика и оценка практики заверяются подписью непосредственного руководителя, главной медицинской сестры и заверяются печатью лечебного учреждения.

Дневник практики должен быть сдан для проверки ассистентуу\hyp{}руководителю практики:

\begin{itemize}
\item Студентами, проходившими практику на клинических базах Новосибирска "--- в последние два дня практики;
\item Студентами, проходившими практику вне клинических баз "--- в течение первой недели, после начала осеннего семестра.
\end{itemize}

Не допускается:

\begin{itemize}
\item Изменение сроков прохождения практики без уважительной причины или без согласования с ответственным руководителем практики.
\item Изменение объема рекомендуемой работы.
\end{itemize}

Итоговая оценка по практике ставится ассистентому\hyp{}руководителем на основании характеристики студента, оценки качества ведения дневника и выполненного объема работы (соответствие программе), результатов зачета.

\subsection*{Инструкция по технике безопасности для обучающихся ИМП НГУ при работе в лечебно-профилактических учреждениях во время прохождения летней производственной практикина 1--5 курсах}

\begin{enumerate}
\item Каждый обучающийся обязан пройти инструктаж по технике безопасности в лечебно-профилактическом учреждении (ЛПУ), перед тем, как приступить к работе.
\item Перед  началом  работы  в  отделении  стационара  необходимо  переодеться.
Форма одежды: медицинский халат, хирургический костюм, медицинская шапочка, медицинская маска, сменная обувь (моющаяся и на устойчивом каблуке).
Ногти должны быть коротко острижены, волосы убраны под шапочку, украшения не должны касаться одежды.
При повреждении кожи рук, места повреждений  должны быть закрыты лейкопластырем или повязкой.
\item Требования безопасности во время работы:

\begin{enumerate}
\item всех  пациентов  необходимо  рассматривать  как  потенциально  инфицированных ВИЧ-инфекцией  и  другими инфекциями,  передающимися  через  кровь.  Следует помнить и применять правила безопасности для защиты кожи и слизистых при контакте  с  кровью  и  жидкими  выделениями  любого  пациента;  все  виды  работ выполняются в перчатках;
\item необходимо мыть руки до и после любого контакта с пациентом;
\item работать с кровью и жидкими выделениями всех пациентов только в перчатках;
\item сразу после проведения инвазивных манипуляций дезинфицировать инструменты, приборы,   материалы   в   соответствии   с   требованиями   санитарно\hyp{}противоэпидемического  режима.  Не  производить  никакие  манипуляции  с использованными иглами и другими режущими и колющими инструментами, сразу после использования "--- дезинфицировать их;
\item пользоваться средствами защиты глаз и масками для предотвращения попадания брызг крови и жидких выделений в лицо (во время манипуляций, катетеризаций и других лечебных процедур);
\item рассматривать всё бельё, загрязнённое кровью или другими жидкими выделениями пациентов, какпотенциально инфицированное;
\item рассматривать  все  образцы  лабораторных  анализов  как  потенциально инфицированные.  Транспортировку  биоматериала  осуществлять  в  специальных контейнерах;
\item разборку,  мойку  и  полоскание  инструментов,  лабораторной  посуды  и  всего, соприкасавшегося  с  кровью  или  другими  жидкими  выделениями  пациента проводить только после дезинфекции, в перчатках;
\item в рабочих помещениях, где существует риск инфицирования, запрещено есть, пить, курить, наносить косметику и брать в руки контактные линзы;
\item пользоваться  электроприборами  и  оборудованием  разрешается  только  после дополнительного  инструктажа  по  технике  безопасности  на  рабочем  месте,  под руководством  непосредственного  руководителя  практики  и  при  условии  полной исправности приборов. В случае обнаружениялюбых неисправностей необходимо срочно сообщить непосредственному  руководителю практики, не предпринимая попыток устранить неисправность;
\item необходимо  использовать  индивидуальные  средства  защиты  при  работе  с дезинфицирующими и моющими средствами (перчатки, халат, маска, респиратор при необходимости, очки);
\item при работе с кислородом:запрещено работать с кислородом лицам моложе 18 лет. При  работе  с  кислородом  руки  должны  быть  сухими,  чистыми,  без  крема. Кислород разрешено давать только с разрешения и в присутствии врача.
\item соблюдать  универсальные  меры  предосторожности  при  работе  с  бьющимися острыми и режущими предметами;
\item соблюдать  правильную  биомеханику  тела  для  предотвращения  травм  опорно\hyp{}двигательного аппарата при транспортировке пациентов и уходе за ними.                                                                                                                                                                                                                                                                                                                                                                                                                                                                                                     \end{enumerate}
\item Требования безопасности по окончании работы:
\begin{enumerate}
\item использованные перчатки подлежат дезинфекции перед утилизацией;
\item сменная рабочая одежда стирается отдельно от другого белья, при максимально допустимом температурном  режиме, желательно кипячение;
\item сменная обувь обрабатывается дезинфицирующим средством,
\item после окончания работы необходимо принять гигиенический душ.                                                                                                                                                                                                                                                                                                                                                                                \end{enumerate}
\item Требования безопасности в аварийной ситуации:

\begin{enumerate}
\item при загрязнении перчаток кровью, необходимо обработать её настолько быстро, насколько позволяет безопасность пациента, затем: перед снятием перчаток с рук необходимо обработать их раствором дезинфектанта, перчатки снять, руки вымыть гигиеническим способом;
\item при повреждении перчаток и кожных покровов: немедленно обработать перчатки раствором дезинфектанта, снять их с рук, не останавливая кровотечение из ранки, вымыть  руки  с  мылом  под  проточной  водой,  затем,  обработать  кожу  70\% раствором спирта или 5\% спиртовым раствором йода. О происшедшем аварийном случае сообщить заведующему, старшей медсестре отделения, ответственному по производственной практике;
\item при попадании крови на кожу рук, немедленно вымыть руки дважды под тёплой проточной водой, затем обработать руки 70\% раствором спирта;
\item при попадании крови на слизистую оболочку глаз –немедленно промыть водой и обработать  1\%  раствором  борной  кислоты  или    0,05\%  раствором  перманганата калия;
\item при попадании крови на слизистую оболочку носа –не заглатывая воду, промыть нос проточной водой, затем закапать 1\% раствор протаргола;
\item при  попадании  крови  на  одежду  место  загрязнения  немедленно  обработать раствором  дезинфектанта,  затем  снять  загрязненную  одежду    погрузить  её  в дезинфицирующий раствор. Кожу рук и других участков тела под загрязненной одеждой  обработать  спиртом.  Обувь  обрабатывается  путём  двукратного протирания ветошью, смоченной в дезинфицирующем растворе;
\item при загрязнении кровью или другими биологическими жидкостями поверхностей необходимо обработать их раствором дезинфектанта;
\item при  попадании  дезинфицирующих  и  моющих  средств  на  кожу  или  слизистые немедленно промыть ихводой. При попадании в дыхательные пути прополоскать рот и носоглотку водой и выйти на свежий воздух.                                                                                                                                                                                                                                                                                                                                                                                                                                                                                                                                                                                                                                                                                                                                                                                                                                                                                                                                                                                                                                                                                                                                                                                                                                                                                                                                                                                                                                                                                                                                                                                                                                                                                                                                                                                                                                                                       \end{enumerate}
\item Требования безопасности при пожаре и аварийной ситуации:

\begin{enumerate}
\item немедленно прекратить работу, насколько это позволяет безопасность пациента;
\item сообщить о случившемся администрации отделения или дежурному персоналу;
\item в кратчайшие сроки покинуть здание.
\end{enumerate}
                                                                                                                                                                                                                                                                                                                                                                                                                                                                                                                                                                                                                                                                                                                                                                                                                                                                                                                                                                                                                                                                                                                                                                                                                                                                                                                                                                                                                                                                                                                                                                                                                                                                                                                                                                                                                                                                                                                                                                                                                                                                                                                                                                                                                                                                                                                                                                                                                                                                                                                                                                                                                                                                                                                                                                                                                                                                                                                                                                                                                                                               \end{enumerate}
С техникой безопасности ознакомлен: \hrulefill (\hrulefill)\\


\newpage
\subsection*{В КАЧЕСТВЕ ПОМОЩНИКА ВРАЧА ТЕРАПЕВТА(2 РАБОЧИХ НЕДЕЛИ, 72 ЧАСА)}


Срок прохождения практики: с \hrulefill по \hrulefill

Место прохождения практики: \hrulefill

Название больницы: \hrulefill

Отделение: \hrulefill

Базовый руководитель (ассистент-доцент НГУ): \hrulefill

~\hrulefill

Непосредственный руководитель (зав.отделения): \hrulefill

~\hrulefill

\subsection*{СХЕМА ИСТОРИИ БОЛЕЗНИ ПО ТЕРАПИИ}

\begin{itemize}
\item Номер истории  болезни,  первая  буква  фамилии  больного,  возраст,  профессия больного, дата госпитализации;
\item клинический  диагноз:  основное  заболевание,  осложнения,  сопутствующие заболевания;
\item жалобы на момент начала курации больного;
\item анамнез заболевания: течение болезни от момента появления первых ее признаков до начала курации, включая период пребывания в стационаре;
\item наиболее существенные для постановки диагноза сведения из анамнеза жизни;
\item объективное состояние больного на момент начала курации (коротко, с несколько более расширенным описанием страдающей системы организма);
\item основные результаты дополнительных методов исследования, подтверждающие диагноз;
\item проводимая  терапия:  режим,  диета,  лекарственные  средства  (по  латыни)  с указанием дозы, другие виды лечения.\end{itemize}

\newpage
\subsection*{ИСТОРИЯ БОЛЕЗНИ}

\begin{booktable}{Содержание выполненной работы}{tab:exoc-enrichment}
	\begin{tabular}{| l |}
		\hline
~~~~~~~~~~~~~~~~~~~~~~~~~~~~~~~~~~~~~~~~~~~~~~~~~~~~~~~~~~~~~~~~~~~~~~~~~~~~~~~~~~~~~~~~~~~~~~~~~~~~~~~~~~~~~~~~~~~ \\ \hline
~ \\ \hline
~ \\ \hline
~ \\ \hline
~ \\ \hline
~ \\ \hline
~ \\ \hline
~ \\ \hline
~ \\ \hline
~ \\ \hline
~ \\ \hline
~ \\ \hline
~ \\ \hline
~ \\ \hline
~ \\ \hline
~ \\ \hline
~ \\ \hline
~ \\ \hline
~ \\ \hline
~ \\ \hline
~ \\ \hline
~ \\ \hline
~ \\ \hline
~ \\ \hline
~ \\ \hline
		\hline
	\end{tabular}
\end{booktable}
Подпись непосредственного руководителя практики \hrulefill
\newpage
\subsection*{ИСТОРИЯ БОЛЕЗНИ}

\begin{booktable}{Содержание выполненной работы}{tab:exoc-enrichment}
	\begin{tabular}{| l |}
		\hline
~~~~~~~~~~~~~~~~~~~~~~~~~~~~~~~~~~~~~~~~~~~~~~~~~~~~~~~~~~~~~~~~~~~~~~~~~~~~~~~~~~~~~~~~~~~~~~~~~~~~~~~~~~~~~~~~~~~ \\ \hline
~ \\ \hline
~ \\ \hline
~ \\ \hline
~ \\ \hline
~ \\ \hline
~ \\ \hline
~ \\ \hline
~ \\ \hline
~ \\ \hline
~ \\ \hline
~ \\ \hline
~ \\ \hline
~ \\ \hline
~ \\ \hline
~ \\ \hline
~ \\ \hline
~ \\ \hline
~ \\ \hline
~ \\ \hline
~ \\ \hline
~ \\ \hline
~ \\ \hline
~ \\ \hline
~ \\ \hline
		\hline
	\end{tabular}
\end{booktable}
Подпись непосредственного руководителя практики \hrulefill
\newpage
\subsection*{ИСТОРИЯ БОЛЕЗНИ}

\begin{booktable}{Содержание выполненной работы}{tab:exoc-enrichment}
	\begin{tabular}{| l |}
		\hline
~~~~~~~~~~~~~~~~~~~~~~~~~~~~~~~~~~~~~~~~~~~~~~~~~~~~~~~~~~~~~~~~~~~~~~~~~~~~~~~~~~~~~~~~~~~~~~~~~~~~~~~~~~~~~~~~~~~ \\ \hline
~ \\ \hline
~ \\ \hline
~ \\ \hline
~ \\ \hline
~ \\ \hline
~ \\ \hline
~ \\ \hline
~ \\ \hline
~ \\ \hline
~ \\ \hline
~ \\ \hline
~ \\ \hline
~ \\ \hline
~ \\ \hline
~ \\ \hline
~ \\ \hline
~ \\ \hline
~ \\ \hline
~ \\ \hline
~ \\ \hline
~ \\ \hline
~ \\ \hline
~ \\ \hline
~ \\ \hline
		\hline
	\end{tabular}
\end{booktable}
Подпись непосредственного руководителя практики \hrulefill
\newpage
\subsection*{ИСТОРИЯ БОЛЕЗНИ}

\begin{booktable}{Содержание выполненной работы}{tab:exoc-enrichment}
	\begin{tabular}{| l |}
		\hline
~~~~~~~~~~~~~~~~~~~~~~~~~~~~~~~~~~~~~~~~~~~~~~~~~~~~~~~~~~~~~~~~~~~~~~~~~~~~~~~~~~~~~~~~~~~~~~~~~~~~~~~~~~~~~~~~~~~ \\ \hline
~ \\ \hline
~ \\ \hline
~ \\ \hline
~ \\ \hline
~ \\ \hline
~ \\ \hline
~ \\ \hline
~ \\ \hline
~ \\ \hline
~ \\ \hline
~ \\ \hline
~ \\ \hline
~ \\ \hline
~ \\ \hline
~ \\ \hline
~ \\ \hline
~ \\ \hline
~ \\ \hline
~ \\ \hline
~ \\ \hline
~ \\ \hline
~ \\ \hline
~ \\ \hline
~ \\ \hline
		\hline
	\end{tabular}
\end{booktable}
Подпись непосредственного руководителя практики \hrulefill

\newpage
\subsection*{Индивидуальный график прохождения практики по терапии (по пятидневной рабочей неделе)}

\begin{booktable}{Индивидуальный график}{tab:exoc-enrichment}
	\begin{tabular}{| l | l | l |}
		\hline
		\rowcolor{tableheadcolor}
		\textbf{Работа в подразделениях больницы} &
		\textbf{Часов} &
		\textbf{Дата} \\ \hline
		\hline
Курация больных & ежедневно & ежедневно \\ \hline
Палата  или  блок  интенсивной  терапии  терапевтического профиля & 3 & ~ \\ \hline
Рентгеновский кабинет & 3 & ~ \\ \hline
Кабинет УЗИ & 3 & ~ \\ \hline
Кабинет ЭКГ & 3 &  \\ \hline
Кабинет функциональной диагностики & 3 & ~ \\ \hline
Физиотерапевтический кабинет & 3 & ~ \\ \hline
		\hline
	\end{tabular}
\end{booktable}


\subsection*{Краткая характеристика отделения}

~\hrulefill

~\hrulefill

~\hrulefill

~\hrulefill

~\hrulefill

~\hrulefill

~\hrulefill

~\hrulefill

~\hrulefill

~\hrulefill

~\hrulefill

~\hrulefill

~\hrulefill

~\hrulefill

~\hrulefill

~\hrulefill

~\hrulefill

~\hrulefill

\newpage
\begin{booktable}{Содержание выполненной работы}{tab:exoc-enrichment}
	\begin{tabular}{| l | l | }
		\hline
		\rowcolor{tableheadcolor}
		\textbf{Дата и время} &
		\textbf{Содержание выполненной работы~~~~~~~~~~~~~~~~~~~~~~~~~~~~~~~~~~}\\
		\hline
~ & ~ \\ \hline
~ & ~ \\ \hline
~ & ~ \\ \hline
~ & ~ \\ \hline
~ & ~ \\ \hline
~ & ~ \\ \hline
~ & ~ \\ \hline
~ & ~ \\ \hline
~ & ~ \\ \hline
~ & ~ \\ \hline
~ & ~ \\ \hline
~ & ~ \\ \hline
~ & ~ \\ \hline
~ & ~ \\ \hline
~ & ~ \\ \hline
~ & ~ \\ \hline
~ & ~ \\ \hline
~ & ~ \\ \hline
~ & ~ \\ \hline
~ & ~ \\ \hline
~ & ~ \\ \hline
~ & ~ \\ \hline
~ & ~ \\ \hline
~ & ~ \\ \hline
~ & ~ \\ \hline
~ & ~ \\ \hline
~ & ~ \\ \hline
~ & ~ \\ \hline

		\hline
	\end{tabular}
\end{booktable}
Подпись непосредственного руководителя практики \hrulefill

\newpage
\begin{booktable}{Содержание выполненной работы}{tab:exoc-enrichment}
	\begin{tabular}{| l | l | }
		\hline
		\rowcolor{tableheadcolor}
		\textbf{Дата и время} &
		\textbf{Содержание выполненной работы~~~~~~~~~~~~~~~~~~~~~~~~~~~~~~~~~~}\\
		\hline
~ & ~ \\ \hline
~ & ~ \\ \hline
~ & ~ \\ \hline
~ & ~ \\ \hline
~ & ~ \\ \hline
~ & ~ \\ \hline
~ & ~ \\ \hline
~ & ~ \\ \hline
~ & ~ \\ \hline
~ & ~ \\ \hline
~ & ~ \\ \hline
~ & ~ \\ \hline
~ & ~ \\ \hline
~ & ~ \\ \hline
~ & ~ \\ \hline
~ & ~ \\ \hline
~ & ~ \\ \hline
~ & ~ \\ \hline
~ & ~ \\ \hline
~ & ~ \\ \hline
~ & ~ \\ \hline
~ & ~ \\ \hline
~ & ~ \\ \hline
~ & ~ \\ \hline
~ & ~ \\ \hline
~ & ~ \\ \hline
~ & ~ \\ \hline
~ & ~ \\ \hline

		\hline
	\end{tabular}
\end{booktable}
Подпись непосредственного руководителя практики \hrulefill

\newpage
\begin{booktable}{Содержание выполненной работы}{tab:exoc-enrichment}
	\begin{tabular}{| l | l | }
		\hline
		\rowcolor{tableheadcolor}
		\textbf{Дата и время} &
		\textbf{Содержание выполненной работы~~~~~~~~~~~~~~~~~~~~~~~~~~~~~~~~~~}\\
		\hline
~ & ~ \\ \hline
~ & ~ \\ \hline
~ & ~ \\ \hline
~ & ~ \\ \hline
~ & ~ \\ \hline
~ & ~ \\ \hline
~ & ~ \\ \hline
~ & ~ \\ \hline
~ & ~ \\ \hline
~ & ~ \\ \hline
~ & ~ \\ \hline
~ & ~ \\ \hline
~ & ~ \\ \hline
~ & ~ \\ \hline
~ & ~ \\ \hline
~ & ~ \\ \hline
~ & ~ \\ \hline
~ & ~ \\ \hline
~ & ~ \\ \hline
~ & ~ \\ \hline
~ & ~ \\ \hline
~ & ~ \\ \hline
~ & ~ \\ \hline
~ & ~ \\ \hline
~ & ~ \\ \hline
~ & ~ \\ \hline
~ & ~ \\ \hline
~ & ~ \\ \hline

		\hline
	\end{tabular}
\end{booktable}
Подпись непосредственного руководителя практики \hrulefill
\newpage
\begin{booktable}{Содержание выполненной работы}{tab:exoc-enrichment}
	\begin{tabular}{| l | l | }
		\hline
		\rowcolor{tableheadcolor}
		\textbf{Дата и время} &
		\textbf{Содержание выполненной работы~~~~~~~~~~~~~~~~~~~~~~~~~~~~~~~~~~}\\
		\hline
~ & ~ \\ \hline
~ & ~ \\ \hline
~ & ~ \\ \hline
~ & ~ \\ \hline
~ & ~ \\ \hline
~ & ~ \\ \hline
~ & ~ \\ \hline
~ & ~ \\ \hline
~ & ~ \\ \hline
~ & ~ \\ \hline
~ & ~ \\ \hline
~ & ~ \\ \hline
~ & ~ \\ \hline
~ & ~ \\ \hline
~ & ~ \\ \hline
~ & ~ \\ \hline
~ & ~ \\ \hline
~ & ~ \\ \hline
~ & ~ \\ \hline
~ & ~ \\ \hline
~ & ~ \\ \hline
~ & ~ \\ \hline
~ & ~ \\ \hline
~ & ~ \\ \hline
~ & ~ \\ \hline
~ & ~ \\ \hline
~ & ~ \\ \hline
~ & ~ \\ \hline

		\hline
	\end{tabular}
\end{booktable}
Подпись непосредственного руководителя практики \hrulefill
\newpage
\subsection*{ОТЧЕТ О ПРОИЗВОДСТВЕННОЙ ПРАКТИКЕ ПО ТЕРАПИИ}

\begin{booktable}{Отчет о практике}{tab:exoc-enrichment}
	\begin{tabular}{| l | l | l | l | l |}
		\hline
		\rowcolor{tableheadcolor}
		\textbf{№ п/п} &
		\textbf{Вид выполненной работы} &
		\headerbigrow{Уровень освоения}{Уровень освоения умения} &
		\headerbigrow{Рекомендуемый}{Рекомендуемый объем} &
		\headerbigrow{Фактическое}{Фактическое выполнение} \\ \hline
		\hline

1 & Курация больных в стационаре & 2-3 & 4-5 & ~ \\ \hline
2 & Заполнение истории болезни & 2-3 & 6-8 & ~ \\ \hline
3 & Оформление эпикриза & 2-3 & 6-8 & ~ \\ \hline
4 & Проведение дежурств & 2 & 24 часа & ~ \\ \hline
5 & Доклад о дежурстве & 2-3 & 2 & ~ \\ \hline
6 & Присутствие на утренней конференции & 2 & 10-12 & ~ \\ \hline
7 & Рентгеновские исследования & 2 & 4-5 & ~ \\ \hline
8 & Запись ЭКГ & 2-3 & 4-6 & ~ \\ \hline
9 & Расшифровка ЭКГ & 2-3 & 10-15 & ~ \\ \hline
10 & Проведение функциональных исследований & 1-2 & 4-6 & ~ \\ \hline
11 & УЗИ & 1-2 & 3-4 & ~ \\ \hline
12 & Физиопроцедуры & 1-2 & 4-6 & ~ \\ \hline
13 & Внутривенные вливания & 3 & 8-10 & ~ \\ \hline
14 & Переливание компонентов крови & 2-3 & 1-2 & ~ \\ \hline
15 & Пункции (плевральные, стернальные и др.) & 1-2 & 1-2 & ~ \\ \hline
16 & \bigrow{Участие в научно-практических конференциях}{Купирование неотложных состояний (приступ стенокардии, отек легких и др.)} & 2-3 & 3-4 & ~ \\ \hline
17 & Участие в научно-практических конференциях & 2-3 & 1-2 & ~ \\ \hline
18 & Патологоанатомическое вскрытие & 1 & 1-2 & ~ \\ \hline
19 & Прочие виды работы & ~ & ~ \\ \hline
		\hline
	\end{tabular}
\end{booktable}

Уровни освоения умений:

\begin{enumerate}
\item Иметь представление, профессионально ориентироваться, знать показания к проведению.
\item Знать, оценить, принять участие.
\item Выполнить самостоятельно.
\end{enumerate}
Подпись непосредственного руководителя практики \hrulefill
\newpage
\subsection*{САНИТАРНО-ПРОСВЕТИТЕЛЬСКАЯ РАБОТА}

\begin{booktable}{Отчет о практике}{tab:exoc-enrichment}
	\begin{tabular}{| l | l | l |}
		\hline
		\rowcolor{tableheadcolor}
		\textbf{Дата и время} &
		\textbf{Название лекции, беседы, санбюллетеня} &
		\headerbigrow{присутствующих}{Количество присутствующих} \\ \hline
		\hline
~ & ~ & ~ \\ \hline
~ & ~ & ~ \\ \hline
~ & ~ & ~ \\ \hline
~ & ~ & ~ \\ \hline
~ & ~ & ~ \\ \hline
~ & ~ & ~ \\ \hline
~ & ~ & ~ \\ \hline
~ & ~ & ~ \\ \hline
~ & ~ & ~ \\ \hline
~ & ~ & ~ \\ \hline
		\hline
	\end{tabular}
\end{booktable}
Подпись непосредственного руководителя практики \hrulefill

\newpage
\subsection*{ХАРАКТЕРИСТИКА СТУДЕНТА}

Студент ИМПЗ НГУ \hrulefill

~\hrulefill 

прошел практику в ~\hrulefill 

~\hrulefill 

в течение \hrulefill недель. За время прохождения практики выполнил весь спектр требуемых работ в полном объеме, а именно: \hrulefill 

~\hrulefill 

~\hrulefill 

~\hrulefill 

~\hrulefill 

~\hrulefill 

~\hrulefill

~\hrulefill

~\hrulefill

~\hrulefill

~

За время прохождения практики студент зарекомендовал себя \hrulefill 

~\hrulefill 

~\hrulefill 

~\hrulefill 

~

Оценка по терапии: \hrulefill

~

Руководитель практики:\hrulefill (\hrulefill)

~

Курирующий врач:\hrulefill (\hrulefill)

~

Место печати лечебного учреждения

\end{document}
