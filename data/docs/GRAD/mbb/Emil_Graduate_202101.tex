\documentclass[a4paper,14pt]{extarticle}

% ------ GLOBAL ------

\usepackage{calc,etoolbox,float,microtype,soul,xspace,textcomp,xltxtra}
\usepackage[table]{xcolor}

% ------ LANGUAGE ------

\usepackage{polyglossia}
\setdefaultlanguage[babelshorthands=true]{russian}
\setotherlanguage{english}
\defaultfontfeatures{Ligatures=TeX,Mapping=tex-text}

% ------ PAGE VIEW ------

\usepackage[left=3cm,right=1.5cm,top=2cm,bottom=2cm]{geometry}
\usepackage{hyphenat}

\setmainfont[Ligatures=TeX]{Liberation Serif}
\setsansfont[Ligatures=TeX]{Liberation Sans}
\setlength{\parindent}{1.27cm}
\linespread{1.3}

\renewcommand{\labelitemi}{$\bullet$}

% ------ HEADER & FOOTER ------

\usepackage{fancyhdr}
\pagestyle{fancy}
\fancyhead{}\renewcommand{\headrulewidth}{0mm}\fancyfoot[CE,CO]{\thepage}
\fancypagestyle{plain}{\fancyhead{}\renewcommand{\headrulewidth}{0mm}\fancyfoot{}}

% ------ TABLES ------

\usepackage{multirow,tabularx,rotating,wrapfig}

\definecolor{tableheadcolor}{RGB}{200,200,200}
\definecolor{tableoddrowcolor}{RGB}{238,238,238}
\definecolor{tableevenrowcolor}{gray}{1.0}

\newsavebox{\defaultsavebox}

\newcommand{\headerbigrow}[2]{\parbox[c][3.8em]{\widthof{\textbf{#1}}}{\textbf{#2}}}
\newcommand{\bigrow}[2]{\parbox[c][3.8em]{\widthof{#1}}{#2}}

\newenvironment{booktable}[2]
{\begin{table}[H]\caption{\label{#2}#1}\vspace{0.5em}\setlength\arrayrulewidth{1pt}\begin{lrbox}{\defaultsavebox}\bgroup\def\arraystretch{1}\rowcolors{2}{tableoddrowcolor}{tableevenrowcolor}}
			{\egroup\end{lrbox}\resizebox{\textwidth}{!}{\usebox{\defaultsavebox}}\end{table}}

\newenvironment{albumtable}[2]
{\begin{sidewaystable}\caption{\label{#2}#1}\vspace{0.5em}\setlength\arrayrulewidth{1pt}\begin{lrbox}{\defaultsavebox}\bgroup\def\arraystretch{1}\rowcolors{2}{tableoddrowcolor}{tableevenrowcolor}}
			{\egroup\end{lrbox}\resizebox{0.8\textheight}{!}{\usebox{\defaultsavebox}}\end{sidewaystable}}

% ------ FIGURES ------

\usepackage{graphicx}

\newcommand{\intextfigure}[5]
{\begin{wrapfigure}{#1}{#2\textwidth}\centering\includegraphics[width=#2\textwidth]{#3}\caption{\label{#4}#5}\end{wrapfigure}}

\newcommand{\centerfigure}[5]
{\begin{figure}[#1]\centering\includegraphics[width=#5\textwidth]{#2}\caption{\label{#3}#4}\end{figure}}

% ------ FORMULAE ------

\usepackage{amsmath}

\usepackage{numprint}
\npthousandsep{\,}
\npdecimalsign{.}
\newcommand{\thousands}{тыс.}
\newcommand{\mln}{млн}
\newcommand{\genename}[1]{\textit{#1}}
\newcommand{\utilname}[1]{\textenglish{#1}}

% ------ BIBLIOGRAPHY ------

\usepackage[square,sort,comma,number]{natbib}
% \bibliographystyle{naturemag}
\bibliographystyle{ugost2008ns}

% ------ LINKS ------

\usepackage{hyperref}
\definecolor{linkcolor}{RGB}{0,102,153}
\hypersetup{colorlinks=true, linkcolor=linkcolor, citecolor=linkcolor, filecolor=linkcolor, urlcolor=linkcolor}

% ------ REFERENCES ------

\newcommand{\picref}[1]{Рис.~\ref{#1}}
\newcommand{\tableref}[1]{Табл.~\ref{#1}}
\newcommand{\formularef}[1]{Формула~\ref{#1}}
\newcommand{\engterm}[1]{англ. \textenglish{\textit{#1}}}

\begin{document}

\begin{titlepage}
	\centering
	{\par\small{ФЕДЕРАЛЬНОЕ ГОСУДАРСТВЕННОЕ АВТОНОМНОЕ ОБРАЗОВАТЕЛЬНОЕ УЧРЕЖДЕНИЕ ВЫСШЕГО ОБРАЗОВАНИЯ <<НОВОСИБИРСКИЙ НАЦИОНАЛЬНЫЙ ИССЛЕДОВАТЕЛЬСКИЙ ГОСУДАРСТВЕННЫЙ УНИВЕРСИТЕТ>> (НОВОСИБИРСКИЙ ГОСУДАРСТВЕННЫЙ УНИВЕРСИТЕТ, НГУ)}}
	{\par\small{Институт медицины и психологии В.\,Зельмана НГУ}}

	\vspace{4cm}

	{\par\LARGE\textbf{КУРСОВАЯ РАБОТА}}

	\vspace{0.5cm}

	Валеев Эмиль Салаватович\\
	Группа 12452\\
	Тема работы: <<Разработка инструментов для поиска клинически значимых полиморфизмов в геноме человека на основе данных секвенирования 3C\hyp{}библиотек>>

	\vfill

	\hfill
	\begin{minipage}{0.57\textwidth}

		\textbf{Научный руководитель:}\\
		Фишман Вениамин Семенович,\\
		к.б.н., ведущий научный сотрудник,\\
		заведующий Сектором геномных\\
		механизмов онтогенеза, ИЦиГ~СО~РАН

		\vspace{1cm}

		ФИО: \hrulefill/\hrulefill\\
		<<\rule{2em}{0.4pt}>>\hrulefill20\rule{2em}{0.4pt}~г.\\
		Оценка: \hrulefill\\
	\end{minipage}

	\vfill

	{\centering\small{Новосибирск, 2021}}
\end{titlepage}

\tableofcontents
\newpage

\section*{Список сокращений}

\begin{description}
	\item[3C] (\engterm{Chromosome Conformation Capture}) "--- захват конформации хромосом
	\item[BAM] (\engterm{Binary sequence Alignment/Map}) "--- бинарный файловый формат, предназначенный для хранения информации о картированных прочтениях
	\item[bp] (\engterm{base pairs}) "--- пары оснований
	\item[BQSR] (\engterm{Base Quality Score Recalibration}) "--- рекалибровка качества прочтений
	\item[cffDNA] (\engterm{Cell-Free Fetal DNA}) "--- свободная ДНК плода
	\item[CGH] (\engterm{Comparative Genomic Hybridization}) "--- сравнительная геномная гибридизация
	\item[CNV] (\engterm{Copy Number Variation}) "--- вариация числа копий
	\item[Exo-C] "--- метод приготовления NGS\hyp{}библиотек, сочетающий таргетное обогащение экзома и технологии захвата конформации хромосом
	\item[FISH] (\engterm{Fluorescence In Situ Hybridization}) "--- флуоресцентная \textit{in situ} гибридизация
	\item[GATK] (\engterm{Genome Analysis ToolKit}) "--- набор инструментов для биоинформационного анализа, созданный Broad Institute
	\item[Hi-C] "--- метод захвата конформации хромосом <<все против всех>>
	\item[kbp] (\engterm{kilo base pairs}) "--- тысячи пар оснований
	\item[LoF] (\engterm{Loss of Function}) "--- потеря функции гена
	\item[MAPQ] (\engterm{MAPping Quality}) "--- качество картирования
	\item[Mbp] (\engterm{mega base pairs}) "--- миллионы пар оснований
	\item[MIP] (\engterm{Molecularly Imprinted Polymers}) "--- молекулярно импринтированные полимеры
	\item[MLPA] (\engterm{Multiplex Ligation-dependent Probe Amplification}) "--- мультиплексная лигаза-зависимая амплификация зонда
	\item[NGS] (\engterm{New Generation Sequencing}) "--- секвенирование нового поколения
	\item[NIPT] (\engterm{Non-Invasive Prenatal Testing}) "--- неинвазивное пренатальное тестирование
	\item[NOR] (\engterm{Nucleolus Organizer Region}) "--- ядрышковый организатор
	\item[PEC] (\engterm{Primer Extension Capture}) "--- захват с помощью расширения праймера
	\item[RG] (\engterm{Read Group}) "--- группа прочтения
	\item[SKY] (\engterm{Spectral Karyotyping}) "--- спектральное кариотипирование
	\item[SMART] "--- анализ транскриптома одной клетки
	\item[SNV] (\engterm{Single Nucleotide Variant}) "--- однонуклеотидный генетический вариант
	\item[UTR] (\engterm{UnTranslated Regions}) "--- нетранслируемая область
	\item[VCF] (\engterm{Variant Call Format}) "--- формат записи генетических вариантов, найденных в результатах секвенирования
	\item[WES] (\engterm{Whole Exome Sequencing}) "--- полноэкзомное секвенирование
	\item[WGS] (\engterm{Whole Genome Sequencing}) "--- полногеномное секвенирование
	\item[БД] "--- база данных
	\item[ВИЧ] "--- вирус иммунодефицита человека
	\item[ДНК] "--- дезоксирибонуклеиновая кислота
	\item[мРНК] "--- матричная РНК
	\item[ПЦР] "--- полимеразная цепная реакция
	\item[РНК] "--- рибонуклеиновая кислота
	\item[ТАД] "--- топологически ассоциированные домены
	\item[ХМА] "--- хромосомный микроматричный анализ
\end{description}

\newpage

\section{Введение}

\subsection{Актуальность}

Наследственные заболевания являются одной из основных причин младенческой и детской смертности в развитых странах.
Взрослые люди с такими патологиями требуют огромных затрат средств на медикаменты, оперативные вмешательства, специальный уход и социальные льготы.
Таким образом, доступные и точные методы диагностики наследственных заболеваний могут помочь в сокращении заболеваемости и смертности, а также повысить экономическое благополучие населения.

Несмотря на то, что в развитии наследственных заболеваний играют роль множество механизмов, в основе их всегда лежат изменения тех или иных участков ДНК.
Эти генетические варианты существенно различаются по размеру, характеру изменения, а также функциональному значению.
Существует множество методов выявления генетических вариантов, каждый метод имеет свои преимущества и границы применения.

Наиболее перспективными в диагностическом и исследовательском плане в настоящее время являются методы секвенирования "--- например, полногеномное и полноэкзомное секвенирование.
В Секторе геномных механизмов онтогенеза ИЦиГ~СО~РАН был разработан новейший метод секвенирования "--- Exo-C, сочетающий технологии экзомного обогащения с захватом конформации хромосом.
Потенциальным преимуществом данного метода может быть возможность поиска как крупных перестроек, так и точечных полиморфизмов в экзоме при относительно небольшой глубине секвенирования, от которой напрямую зависит цена секвенирования.
Широкий спектр применения метода и доступность в финансовом аспекте делают метод Exo-C привлекательным как для медико-биологических научных исследований, так и для внедрения в клиническую практику.

\subsection{Цель}

Целью нашей работы является сравнение эффективности методов Exo-C, полногеномного секвенирования и экзомного секвенирования для поиска точечных полиморфизмов в геномах клеток человека.

\subsection{Задачи}

Основные задачи, которые необходимо решить для достижения поставленной нами цели:

\begin{enumerate}
	\item Разработать биоинформационный протокол анализа данных секвенирования Exo-C\hyp{}библиотек.
	\item Проанализировать доступные данные полногеномного, полноэкзомного, Hi-C и Exo-C\hyp{}секвенирования для иммортализованной клеточной линии человека K562.
	\item Сравнить точечные генетические варианты в геноме клеток K562, детектируемые при использовании полногеномного и экзомного секвенирования, с таковыми, найденными методом Exo-C.
\end{enumerate}

\section{Обзор литературы}

Генетические варианты, их взаимодействие друг с другом и со средой определяет течение болезней.
Существуют генетические варианты, которые определяют предрасположенность и проявляются только во взаимодействии со средой; примером могут служить варианты, определяющие предрасположенность к аддикциям (никотин, героин, алкоголь и пр.)\,\citep{Hiroi_2004}.
Бывают и такие генетические варианты, которые повышают восприимчивость к одному фактору среды и повышают устойчивость к другому, либо дают позитивный эффект в сочетании и негативный по отдельности.
Примером может служить бета-талассемия\,\citep{Galanello_2010}.
Особняком стоят те варианты, которые вне зависимости от средового компонента и генетического окружения приводят к развитию заболевания (например, нейрофиброматоз I типа, который наследуется по аутосомно-доминантному типу и имеет \numprint[\%]{100} пенетрантность "--- \citealp{Jett_2009}).

Генетические заболевания остаются одной из основных причин младенческой и детской смертности в развитых странах.
Врождённые аномалии являются причиной около \numprint[\%]{20} смертности до 1 года, а также порядка \numprint[\%]{10} в возрасте 1--4 года и \numprint[\%]{6} в возрасте 5--9 лет.
Злокачественные новообразования являются причиной смерти в \numprint[\%]{8} случаев в возрасте 1--4 лет, и \numprint[\%]{15} случаев в возрасте 5--9 лет.
Порядка \numprint[\%]{3} от смертности в возрасте 1--9 лет связаны с сердечными патологиями\,\citep{Field_2003}.
Взрослые люди с генетическими патологиями требуют огромных затрат средств "--- на радикальные и паллиативные операции, медикаментозную поддержку (иногда пожизненную), создание условий, учреждений и обучение персонала для обеспечения специализированного ухода.

Таким образом, доступные и точные методы диагностики генетических заболеваний могут помочь в сокращении заболеваемости и смертности, а также повысить экономическое благополучие населения.

\paragraph{Частые и редкие (орфанные) патологии.}
Генетические патологии делятся на группы по частоте встречаемости в популяции.
Выделяют частые и редкие (орфанные) заболевания.
Определения орфанных заболеваний могут различаться "--- например, в США, согласно \textenglish{``Health Promotion and Disease Prevention Amendments of 1984''}, редкими считаются патологии, поражающие менее \numprint[\thousands]{200} населения страны (примерно \numprint{1 : 1630} при текущей численности населения в \numprint[\mln]{326} человек)\,\citep{Herder_2017}.
Европейское Медицинское Агентство определяет границу как \numprint{1 : 2000}.
Систематический анализ показал, что существует более 290 определений, и среднее значение находится в интервале 40--50 на \numprint[\thousands]{100} населения\,\citep{Richter_2015}.

Также сложность в определении орфанных заболеваний представляет неравномерность их распространённости в тех или иных регионах.
Некоторые заболевания могут быть орфанными в одной популяции и частыми в другой (эффект основателя, а также сверхдоминирование).
Частным случаем эффекта основателя является атаксия Каймановых островов, связанная с гипоплазией мозжечка и сопутствующими неврологическими проявлениями (задержка развития, дизартрия, нистагм, интенционное дрожание).
Это аутосомно-рецессивное заболевание распространено исключительно в одном регионе "--- Большой Кайманов остров, гетерозиготные носители составляют около \numprint[\%]{18} местного населения\,\citep{Bomar_2003}.
Примером сверхдоминирования может служить бета-талассемия "--- заболевание, связанное с нарушением структуры гемоглобина.
Несмотря на то, что у эритроцитов носителей в значительной степени снижена способность переносить кислород, дефектный гемоглобин представляет сложность для развития малярийного плазмодия и таким образом повышает устойчивость носителя бета-талассемии к малярии\,\citep{Galanello_2010}.
Соответственно, бета-талассемия распространена в эпидемически опасных по малярии регионах "--- Средиземноморье и Юго-Восточная Азия, наибольшая частота встречаемости наблюдается на Кипре (\numprint[\%]{14}) и Сардинии (\numprint[\%]{10.3}) при средней частоте по земному шару в \numprint[\%]{1.5}.

Несмотря на то, что каждое из орфанных заболеваний само по себе встречается редко, в сумме они поражают значительный процент населения (предположительно \numprint[\%]{5--8} европейской популяции).
Общее число орфанных болезней неизвестно по причине недостатков стандартизации, наиболее частая оценка "--- \numprint[\thousands]{5--8}\,\citep{The_Lancet_Neurology_2011}.
Существуют различные базы данных, собирающие информацию по орфанным заболеваниям, наиболее известными и часто используемыми из них являются:

\begin{enumerate}
	\item Global Genes;
	\item Online Mendelian Inheritance in Man (OMIM\textregistered)\,\citep{Amberger_2014};
	\item Orphanet\,\citep{Orphanet}.
\end{enumerate}

Около \numprint[\%]{80} редких болезней имеют генетическую природу и начинаются в раннем детстве\,\citep{The_Lancet_Neurology_2011}.
Таким образом, ключевым моментом для изучения данных заболеваний является понимание механизмов, лежащих в основе их развития.
Количество орфанных заболеваний делает эту задачу крайне непростой.
Тем не менее, многие механизмы на сегодняшний момент достаточно хорошо изучены.
О них речь пойдёт далее.

\subsection{Механизмы развития генетических патологий}

Механизмы развития генетических патологий делятся на две большие группы.
В первую относят изменения белок-кодирующей последовательности гена, приводящие к прекращению синтеза белка либо к синтезу изменённого полипептида.
Ко второй группе относятся эпигенетические механизмы, не затрагивающие непосредственно белок-кодирующие последовательности генов.

Изменения белок-кодирующей последовательности гена (экзонов и сплайс-сайтов) могут приводить к замене аминокислот, сдвигам рамки считывания, появлению преждевременных стоп-кодонов и нарушениям сплайсинга.
Прекращение синтеза белка снижает дозу гена, а изменённый полипептид способен как потерять свою функцию, снизив таким образом дозу гена, так и приобрести новые свойства (токсичность).
Классическим примером приобретения белком токсичности является известное наследственное нейродегенеративное заболевание "--- аутосомно-доминантный вариант болезни Альцгеймера.
Другое нейродегенеративное заболевание "--- аутосомно-рецессивная болезнь Паркинсона "--- может служить примером потери белком протективной функции\,\citep{Winklhofer_2008}.

Также генетические патологии могут развиваться из-за эпигенетических механизмов, приводящих к изменению экспрессии генов.
К таким механизмам можно отнести метилирование ДНК "--- изменение молекулы ДНК без изменения нуклеотидной последовательности, а также ацетилирование гистонов\,\citep{Handy_2011}.

В частности, нарушение метилирования ДНК ответственно за развитие синдрома Беквита---Видемана.
Экспрессия генов \genename{CDKN1C} и \genename{IGF2} регулируется в зависимости от того, на материнской или отцовской хромосоме они находятся (явление геномного импринтинга).
Потеря импринтинга, вызванная изменениями регуляторного района, ведёт к изменению экспрессии этих генов и, как следствие, к тяжёлым порокам развития, включающим висцеромегалию, виcцеральные грыжи, эмбриональные опухоли, пороки сердца и почек\,\citep{Jin_2018}.
Изменение ацетилирования гистонов некоторых генов в клетках головного мозга связано с развитием такого заболевания, как шизофрения\,\citep{Tang_2011}.

Кроме того, на экспрессию генов в значительной степени влияет трёхмерная структура хроматина.
К примеру, энхансерный район не обязательно находится в непосредственной близости от гена, для его работы необходим физический контакт с промотором гена за счёт выпетливания ДНК.
Белковый комплекс, связанный с энхансером, привлекает в эту область РНК-полимеразу и увеличивает вероятность её связывания с промотором.
Известно, что большая часть промотор-энхансерных взаимодействий находится внутри топологически ассоциированных доменов (ТАДов)\,\citep{Rao_2014}.
В результате разрушения старых или образования новых границ ТАДов формируются структурные варианты, характеризующиеся иными промотор-энхансерными взаимодействиями.
Подобные изменения лежат в основе таких состояний, как FtM-инверсия пола (ген \genename{SOX9}) и синдром Кукса (ген \genename{KCNJ2})\,\citep{Spielmann_2018}.

Несмотря на то, что в развитии наследственных заболеваний эпигенетика безусловно играет важную роль, в основе их всегда лежат изменения тех или иных участков ДНК.
Эти генетические варианты существенно различаются по размеру, характеру изменения, а также функциональному значению, которое напрямую зависит от затрагиваемых вариантом районов генома.

\subsection{Типы генетических аномалий, лежащих в основе генетических патологий}

Генетические аномалии различаются по размеру.
Размер непосредственно влияет на способность исследователя обнаружить эту аномалию.
Самыми крупными являются хромосомные перестройки.
Они делятся на две основных группы "--- сбалансированные (без изменения количества генетической информации) и несбалансированные (с изменением количества генетической информации).

Несбалансированные перестройки в большинстве своём приводят к летальному исходу (в эмбриональном или детском периодах) и грубым изменениям фенотипа.
К несбалансированным относятся:

\begin{itemize}
	\item Анэуплоидии "--- изменение числа хромосом.
	      Примерами анэуплоидий могут служить синдром Дауна (трисомия 21 хромосомы), Эдвардса (трисомия 18 хромосомы), Патау (трисомия 13 хромосомы), а также вариации числа половых хромосом (синдромы Тёрнера, Клайнфельтера и другие).
	      Частичная моносомия "--- синдром кошачьего крика (связан с утратой плеча 5 хромосомы).
	      Прочие анэуплоидии ведут к несовместимым с жизнью нарушениям эмбрионального развития и, как следствие, спонтанным абортам.
	\item Несбалансированные транслокации "--- перемещение фрагмента хромосомы с одного места на другое с изменением количества генетической информации.
	      Несбалансированные транслокации могут приводить к значимым изменениям фенотипа (например, инверсия пола "--- \citealp{Rizvi_2008}) и служить онкогенами\,\citep{O_Connor_2008}.
	\item Вариации числа копий (\engterm{Copy Number Variations, CNV}) "--- дупликации (мультипликации) и делеции хромосомных сегментов размером от тысячи до нескольких миллионов пар оснований.
	      Могут возникнуть из несбалансированных транслокаций, амплификаций и собственно делеций.
	      CNV способны увеличивать или уменьшать дозу гена, в значительной степени влияя на его экспрессию.
	      Различия в количестве копий могут носить как положительный характер, так и отрицательный "--- в частности, дупликации в гене \genename{CCL3L1} способны увеличить устойчивость к ВИЧ\,\citep{Gonzalez_2005}, а крупные CNV в разных частях генома ассоциированы с расстройствами аутического спектра\,\citep{Sebat_2007}.
\end{itemize}

Сбалансированные перестройки чаще всего характеризуются более мягкими фенотипическими проявлениями, а иногда и их отсутствием.
К сбалансированным перестройкам относятся:

\begin{itemize}
	\item Инверсии "--- переворот фрагмента хромосомы.
	      Крупные инверсии могут быть причиной изменения границы ТАД, а также запирания кроссинговера и образования гаплогрупп.
	\item Сбалансированные транслокации "--- перемещение фрагмента хромосомы с одного места на другое без изменения количества генетической информации.
	      В свою очередь они делятся на реципрокные (взаимный обмен участками между негомологичными хромосомами) и Робертсоновские (слияние акроцентрических хромосом с образованием метацентрической или субметацентрической).
	      Сбалансированные транслокации могут как не проявляться в фенотипе (сказываясь только на фертильности "--- \citealp{Dong_2012}), так и приводить к серьёзным последствиям "--- например, синдрому Дауна (робертсоновская транслокация является причиной синдрома Дауна в \numprint[\%]{2--4} случаев "--- \citealp{Asim_2015}).
\end{itemize}

Самыми небольшими "--- но не менее важными "--- являются точечные полиморфизмы (\engterm{Single Nucleotide Variants, SNV}) и короткие инсерции и делеции (\engterm{indels}) размером \numprint[bp]{20--50}.
Чаще всего эти генетические варианты нейтральные и не имеют фенотипических проявлений, но некоторые могут приводить как к генетическим, так и к эпигенетическим изменениям.
Также варианты делятся на наследуемые, которые передаются от родителей к детям, и варианты \textit{de novo}.
Согласно оценкам, предоставленным \citet{Acuna_Hidalgo_2016}, в среднем в каждом поколении у человека возникают 44--82 SNV \textit{de novo}, из них 1--2 приходятся на белок-кодирующие регионы.
Число небольших инсерций и делеций оценивается в \numprint{2.9--9} на геном, крупные перестройки встречаются значительно реже.
Также известно, что количество генетических вариантов \textit{de novo} непрерывно растёт в течение жизни человека.

\subsection{Функциональные классы генетических вариантов}

Как уже было упомянуто выше, значение генетических вариантов напрямую зависит от их положения относительно функциональных частей генома.
Варианты могут находиться как внутри генов, так и вне их.

Области гена, в которые может попасть генетический вариант:

\begin{itemize}
	\item Экзоны, непосредственно отвечающие за последовательность белка.
	      Генетические варианты в экзонах могут быть синонимичными (без замены аминокислоты) и несинонимичными "--- миссенс (замена на другую аминокислоту), нонсенс (замена на стоп-кодон) либо сдвиг рамки считывания, приводящий к изменению значительной части белковой молекулы.
	      Миссенс-варианты редко приводят к утрате функции белка, но они могут повлиять на экспрессию гена, если замена пришлась на регуляторный мотив\,\citep{j_Brea_Fernandez_2011}.
	\item Интроны, которые содержат регуляторные области и сплайс-сайты, необходимые для процессинга транскрипта в готовую мРНК, а также 3'\hyp{}нетранслируемая область (\engterm{3'\hyp{}untranslated region, 3'UTR}) и 5'\hyp{}нетранслируемая область (\engterm{5'\hyp{}untranslated region, 5'UTR}), вовлечённые в регуляцию транскрипции, трансляции и деградации транскрипта.
	      В частности, в 5'UTR находится так называемая консенсусная последовательность Козак, важная для инициации трансляции мРНК\,\citep{Kozak_1987}.
	      Также известно, что в 5'UTR могут находиться открытые рамки считывания, которые влияют на поведение рибосомы "--- могут вызывать её торможение, диссоциацию, либо перекрывать основной старт-кодон гена\,\citep{Young_2016}.
	      Генетические варианты могут как разрушать канонические сплайс-сайты, так и способствовать образованию новых внутри интронных участков\,\citep{Abramowicz_2018}.
	      Влияние генетических вариантов в этих областях недостаточно изучено, и их связь с конкретной патологией у пациента порой достаточно трудно доказать.
	      Тем не менее, существуют специальные инструменты, позволяющие оценить патогенность таких вариантов.
	      Интронные и UTR генетические варианты обычно рассматриваются в случае, если иного объяснения фенотипу пациента не было найдено.
\end{itemize}

Внегенные варианты могут приходиться на различные регуляторные последовательности, например, энхансеры, сайленсеры, а также сайты связывания белков, отвечающих за процессы метилирования или трёхмерную организацию хроматина.

Как мы видим, типов генетических вариантов существует огромное множество, они в значительной степени различаются между собой, и их определение может представлять трудность для исследователя.
На сегодняшний день разработано множество методик, облегчающих эту задачу.
О них речь пойдёт ниже.

\subsection{Методы детекции генетических вариантов}

\paragraph{Кариотипирование.}
Данный метод представляет собой микроскопическое исследование клеток, синхронизированных на стадии метафазы митоза.
Однако простое микроскопическое исследование хромосом плохо подходит для обнаружения генетических вариантов, поэтому были разработаны различные методы окраски (бэндинга), позволяющие отдифференцировать отдельные хромосомы и хромосомные регионы\,\citep{Schreck_2001}:

\begin{enumerate}
	\item Q-окрашивание "--- позволяет отдифференцировать все хромосомы, применяется для исследования Y-хромосомы (быстрое определение генетического пола, выявление мозаицизма по Y-хромосоме, транслокаций между Y-хромосомой и другими хромосомами).
	      Окрашивание легко снимается, что позволяет использовать этот метод для последовательной окраски и изучения хромосом;
	\item G-окрашивание "--- наиболее часто используемый метод.
	      Позволяет отдифференцировать все хромосомы, гарантирует стойкое окрашивание, легко поддаётся фотографированию.
	\item R-окрашивание "--- визуализирует концы хромосом, а также специфические именно для этого окрашивания бэнды (так называемые R\hyp{}позитивные бэнды).
	\item C-окрашивание "--- применяется для анализа вариабельной дистальной части Y-хромосомы, а также центромерных регионов прочих хромосом, содержащих конститутивный гетерохроматин.
	      Хорошо подходит для выявления перестроек, затрагивающих гетерохроматиновые регионы.
	      Кроме того, C-окрашиванием хорошо определяются кольцевые и дицентрические хромосомы;
	\item NOR-окрашивание "--- визуализирует ядрышковые организаторы (\engterm{Nucleolus Organizer Region, NOR}), богатые рибосомальными генами;
	\item DA--DAPI-окрашивание "--- применяется для идентификации центромерных гетерохроматизированных районов.
\end{enumerate}

Окрашенные хромосомы далее изучаются на предмет формы, количества и наличия перестроек.

Кариотипирование "--- рутинная методика при диагностике врождённых патологий, аутопсии мертворожденных и злокачественных образований кроветворного ряда.
Преимущества кариотипирования в том, что данным методом можно охватить весь геном, визуализации поддаются отдельные клетки и отдельные хромосомы.
Ограничения "--- обязательно требуются живые клетки, также на эффективность влияет размер перестроек (не менее \numprint[Mbp]{1--5}) и процент поражённых клеток в образце (минимум \numprint[\%]{5--10})\,\citep{Sampson_2014}.

В целом классический метод кариотипирования, достаточно дешёвый и простой в исполнении, требует от исследователя значительного опыта при интерпретации.
Более поздние методы изучения хромосом, как будет показано далее, развивались не только в направлении увеличения разрешающей способности, но и облегчения интерпретации полученных данных.

\paragraph{Флуоресцентная \textit{in situ} гибридизация} (\engterm{Fluorescence In Situ Hybridization, FISH}).
Основой является гибридизация нуклеиновых кислот образца и комплементарных им проб, содержащих флуоресцентную метку.
Гибридизация может производиться с ДНК (метафазные или интерфазные хромосомы) или с РНК.
FISH позволяет определить число исследуемых локусов в геноме (при использовании метода 3D-FISH) или последовательность расположения на метафазной хромосоме.
Метод является <<золотым стандартом>> в определении хромосомных патологий "--- как в клетках с врождёнными перестройками, так и в клетках опухолей.

Данные при помощи метода FISH можно получить, анализируя отсутствие или присутствие сигналов от использованных флюорофоров.
Количество различимых цветовых меток равно $(2^x - 1)$, где  $x$ "--- количество флюорофоров.
Это позволяет реализовать, например, спектральное кариотипирование (\engterm{Spectral Karyotyping, SKY}), при котором каждая хромосома окрашивается в свой собственный цвет и межхромосомные перестройки видны даже начинающему специалисту\,\citep{Guo_2014}.
Тем не менее, лимитирующими факторами остаются:

\begin{itemize}
	\item потребность в хорошо обученном персонале.
	      Относительная простота интерпретации результатов сочетается со сложностью протокола приготовления образца, который зависит от характера пробы и образца, и должен быть настроен эмпирически;
	\item цена реактивов;
	\item время гибридизации.
	      Кинетика реакций гибридизации в ядре изучена недостаточно, и требуется достаточно долгое время, чтобы получить сигналы, которые можно измерить и сравнить между собой.
	\item разрешение.
	      Детектировать сигнал от одной молекулы флюорофора очень сложно, такими молекулами должен быть покрыт протяжённый участок ДНК.
	      Поэтому детектировать изменения участков размером менее \numprint[kbp]{100} достаточно затруднительно.
\end{itemize}

В настоящее время методика FISH значительно усложнилась.
Биотехнологические компании предлагают панели олигонуклеотидов, определяющие специфические участки размером от десятков тысяч до миллиона пар оснований, а также олигонуклеотиды с высокой чувствительностью, позволяющие определить сплайс-варианты и даже SNV.
Разрабатываются технологии micro-FISH ($\mu$FISH), сочетающие FISH с микрофлюидными технологиями (проведение реакций в микроскопических объёмах жидкости).
При этом процесс удешевляется, автоматизируется, ускоряется (за счёт уменьшения объёмов, а соответственно, и времени гибридизации) и упрощается для использования в обширных исследованиях и для внедрения в клинику\,\citep{Huber_2018}.

\paragraph{Сравнительная геномная гибридизация} (\engterm{Comparative Genomic Hybridization, CGH}).
Как и в случае с методом FISH, основой данного метода является флуоресцентная гибридизация.
Однако CGH использует два образца генома "--- тестовый и контрольный, каждый из которых метится флюорофором, а затем гибридизуется в соотношении \numprint[]{1 : 1}.
Таким образом в тестовом образце можно обнаружить CNV и перестройки.

В отличие от FISH, CGH проверяет весь геном на наличие перестроек и не требует знаний о целевом регионе.
К ограничениям анализа относится невозможность выявления полиплоидии, мозаицизма и сбалансированных транслокаций.

В настоящее время CGH используется в виде array-CGH (aCGH), или хромосомного микроматричного анализа (ХМА), при котором CGH комбинируется с микрочиповой гибридизацией\,\citep{Theisen_2008}.
ДНК-микрочипы, или микроматрицы, представляют собой сотни тысяч или миллионы однонитевых фрагментов ДНК (зондов), которые ковалентно пришиты к основанию (микрочипу).
При ХМА на микрочип наносятся контрольные фрагменты генома либо контрольные последовательности генов, которые могут быть связаны с конкретной патологией.
Порядок зондов на чипе строго определён, что упрощает локализацию и определение характера перестройки.

С помощью сравнительной гибридизации геномов могут быть обнаружены самые разные структурные вариации "--- CNV, инверсии, хромосомные транслокации и анэуплоидии.
Для этого используются длинные зонды, которые позволяют проводить гибридизацию последовательностей, имеющих некоторые различия.
Когда пробы ДНК короткие, эффективность гибридизации очень чувствительна к несовпадениям; такие зонды облегчают сравнение геномов на нуклеотидном уровне (поиск SNV).

Микроматрицы предлагают относительно недорогие и эффективные средства сравнения всех известных типов генетических вариаций.
Однако для таких целей, как обнаружение неизвестных или часто повторяющихся последовательностей, эти методы не подходят\,\citep{Gresham_2008}.

\paragraph{Мультиплексная лигаза-зависимая амплификация зонда} (\engterm{Multiplex Ligation-dependent Probe Amplification, MLPA}).
Основой MLPA является ПЦР\hyp{}амплификация специальных проб, гибридизующихся с целевыми районами ДНК.
Каждая проба представляет собой пару полу-проб;
каждая полу-проба имеет комплементарную геному часть и технические последовательности "--- праймер для ПЦР и вставки, обеспечивающие большой размер продукта амплификации.
Если полу-пробы гибридизуются с геномом без зазора, они лигируются и впоследствии амплифицируются;
лигированные пробы отличаются от полу-проб с праймером по длине.
Длину готового ПЦР\hyp{}продукта определяют методом электрофореза.

Данная методика подходит для определения CNV, включающих целые гены, а также аномалий метилирования ДНК.
Во втором случае используют метил-чувствительные рестриктазы "--- ферменты, которые по определённым сайтам гидролизуют исключительно метилированную ДНК.
Для определения этих участков также применяют электрофорез, т.к. не подвергшаяся гидролизу ДНК по длине значительно превосходит фрагменты гидролизованных рестриктазой метилированных регионов.

Слабым местом MLPA остаётся интерпретация результатов.
Определение гомозиготных CNV не представляет труда "--- их распознают по наличию/отсутствию пика в сравнении с контрольным образцом.
Гетерозиготные CNV видны как пики отличающейся высоты, и их поиск требует серьёзную биоинформационную обработку с учётом особенностей конкретной ПЦР\hyp{}реакции и различий между образцами\,\citep{Stuppia_2012}.

~

Как мы видим, перечисленные методы имеют один серьёзный недостаток "--- они могут определить наличие или отсутствие, совпадение или несовпадение, но не способны прочитать априори неизвестную последовательность ДНК.
Специально для этого были разработаны методы секвенирования.

\paragraph{Секвенирование по Сэнгеру.}
Исторический метод, позволяющий с высокой точностью анализировать короткие (до \numprint[kbp]{1}) фрагменты ДНК\,\citep{Sanger_1977}.
Суть его состоит в проведении обычной реакции амплификации ДНК, только в смесь дезоксирибонуклеотидов (dNTP) добавлены дидезоксирибонуклеотиды (ddNTP), которые при присоединении к ДНК обрывают синтез и имеют флуоресцентную или радиоактивную метку (соотношение примерно \numprint{100 : 1} соответственно).
Таким образом, в процессе амплификации в пробирках образуется смесь из меченых цепей разной длины.
При разделении этой смеси на электрофорезе проявляется характерная <<лестница>>, последовательность флуоресцентных сигналов в которой совпадает с последовательностью исследуемой ДНК.

Основным недостатком секвенирования по Сэнгеру является ограничение длины исследуемого фрагмента ДНК.

В настоящее время метод Сэнгера используется для подтверждения вариантов, найденных с помощью методов секвенирования нового поколения.

\paragraph{Секвенирование нового поколения} (\engterm{New Generation Sequencing, NGS}).
Это комплекс технологий, позволяющих прочитать за сравнительно небольшое время миллионы последовательностей ДНК.
Благодаря этому единовременно можно проанализировать несколько генов, либо весь геном.

В методах NGS наблюдается развитие двух основных парадигм, различающихся по длине прочтений.
Секвенирование короткими прочтениями характеризуется меньшей ценой и более качественными данными, что позволяет применять данные методы в популяционных исследованиях и клинической практике (поиск патогенных генетических вариантов).
Секвенирование длинными прочтениями хорошо подходит для сборки новых геномов и изучения отдельных изоформ генов\,\citep{Goodwin_2016}.
Количество различных методов в настоящее время значительно, но самым часто используемым является метод Illumina (короткие прочтения).

Основные проблемы данных NGS:

\begin{itemize}
	\item Финансовые вложения и время, затраченные на секвенирование и анализ данных.
	      По-прежнему остаются лимитирующим фактором применения NGS в клинической практике;
	\item Ошибки секвенирования и ПЦР.
	      Их значимость уменьшается с увеличением покрытия, но не исчезает полностью;
	\item Неоднородность покрытия генома или таргетных регионов прочтениями.
	      Это может быть связано как с недостатками приготовления библиотеки, так и с проблемами картирования.
\end{itemize}

\subsection{Виды NGS}

\paragraph{Полногеномное секвенирование} (\engterm{Whole Genome Sequencing, WGS}).
Приготовление библиотек при полногеномном секвенировании производится из всего клеточного материала, либо только из ядер.
ДНК фрагментируется таким образом, что достигается относительно ровное покрытие генома.

WGS при достаточной глубине покрытия вполне пригодно для поиска SNV, небольших делеций и инсерций.
Полногеномное секвенирование со слабым покрытием может быть использовано для определения CNV "--- например, при неинвазивном пренатальном тестировании (\engterm{Non-Invasive Prenatal Testing, NIPT}), когда используется свободная ДНК плода (\engterm{Cell-Free Fetal DNA, cffDNA}), циркулирующая в крови матери\,\citep{Yu_2019}.

\paragraph{Таргетные панели.}
Основой данных методов является обогащение целевых регионов генома.
Методов обогащения существует достаточно много, но все они делятся на четыре основные категории\,\citep{Teer_2010}:

\begin{enumerate}
	\item Твердофазная гибридизация.
	      Для этого используют комплементарные целевым регионам короткие ДНК-пробы, зафиксированные на твёрдом основании (микрочипе).
	      После гибридизации нецелевую ДНК вымывают, а целевые фрагменты остаются на чипе.
	\item Жидкофазная гибридизация.
	      Эти методы характеризуются тем, что ДНК-пробы находятся в растворе и помечены специальной молекулой (например, биотином).
	      После гибридизации с целевой ДНК пробы вылавливают бусинами, поверхность которых способна связывать молекулы биотина.
	\item Полимеразно-опосредованный захват.
	      В этих методах ПЦР производят на стадии обогащения.
	      Например, методы молекулярно импринтированных полимеров (\engterm{Molecularly Imprinted Polymers, MIP}) и анализа транскриптома одной клетки (\engterm{SMART}) используют длинные пробы, содержащие как праймер, так и регион для остановки элонгации и инициации лигирования.
	      После элонгации и лигирования получаются кольцевые молекулы, содержащие целевой регион;
	      линейные молекулы в последующем удаляют из раствора.
	      Метод захвата с помощью расширения праймера (\engterm{Primer Extension Capture, PEC}) использует биотинилированные праймеры, которые гибридизуются с целевыми регионами и элонгируются;
	      далее их вылавливают бусинами, как в методах жидкофазной гибридизации.
	\item Захват регионов.
	      Включает в себя сортировку и микродиссекцию хромосом, благодаря чему можно обогатить библиотеку фрагментов последовательностями отдельной хромосомы или даже её части.
	      Это методы, требующие чрезвычайно сложных техник и хорошо обученный персонал, но очень полезные в отдельных ситуациях.
\end{enumerate}

Данный вид тестов позволяет анализировать гены, ответственные за отдельные группы заболеваний "--- например, существуют таргетные панели для иммунодефицитов, почечных, неврологических болезней, болезней соединительной ткани, сетчатки, а также предрасположенности к отдельным видам онкологических заболеваний.
Таргетные панели позволяют анализировать и клетки опухолей "--- некоторые приспособлены к выявлению общих для многих раковых линий мутаций, другие же разработаны для специфического типа опухолей\,\citep{Yohe_2017}.

\paragraph{Полноэкзомное секвенирование} (\engterm{Whole Exome Sequencing, WES}).
Техника заключается в секвенировании обогащённого экзома "--- совокупности белок-кодирующих последовательностей клетки.
Для этого используют специальные экзомные таргетные панели.
Несмотря на то, что существует множество методов таргетного обогащения, конкретно для WES могут быть использованы лишь немногие из них, а именно "--- твердофазная и жидкофазная гибридизация\,\citep{Teer_2010}.

У человека экзом составляет примерно \numprint[\%]{1} от генома, или примерно \numprint[Mbp]{30} (суммарно).
При этом более \numprint[\%]{80} генетических вариантов, которые представлены в базе данных известных геномных вариантов CLINVAR\,\citep{Landrum_2017}, и из них более \numprint[\%]{89} вариантов, которые отмечены как <<патогенные>>, относятся к белок-кодирующим областям генома;
эта цифра приближается к \numprint[\%]{99}, если учитывать ближайшие окрестности экзонов\,\citep{Barbitoff_2020}.
Таким образом, полноэкзомное секвенирование намного лучше подходит для обычной клинической практики, нежели полногеномное.
Кроме того, полноэкзомное секвенирование значительно дешевле, что увеличивает его доступность и позволяет, например, произвести тестирование ребёнка и родителей (так называемый трио-тест) и, как следствие, улучшить интерпретацию вариантов\,\citep{Yohe_2017}.

\paragraph{Технологии захвата конформации хромосом} (\engterm{Chromosome Conformation Capture, 3C}).
Данные методики позволяют определить расстояние в 3D-пространстве ядра между двумя точками генома.
Принцип состоит в том, что интактное ядро фиксируют формальдегидом, ДНК гидролизуют, лигируют, затем продукты лигазной реакции секвенируют при помощи NGS.
Во время лигирования ковалентно связанными могут оказаться только те участки, которые физически находятся близко друг от друга.
Картирование химерных прочтений с помощью специальных инструментов позволяет узнать, какие именно участки генома были связаны, а значит, распологались близко друг к другу в пространстве ядра\,\citep{Lieberman_Aiden_2009}.
При обработке большого количества 3C-данных геном разделяют на районы фиксированной длины, называемые бинами.
Длина бинов называется разрешением; чем меньше длина, тем более высоким считается разрешение.
Прочтение, части которого были картированы на два разных бина, называется контактом между этими районами.
Практическое значение имеет информация об относительной частоте контактов между бинами.

В настоящее время существует множество вариантов протокола 3C.
Самым известным и широко применяемым является метод Hi-C, сочетающий 3C с методами массового параллельного секвенирования.
С его помощью можно подсчитать количество контактов во всём геноме "--- как внутри-, так и межхромосомные контакты\,\citep{Oluwadare_2019}.

~

Результаты NGS представляют собой гигантские блоки данных, содержащие всевозможные ошибки.
Обработка данных секвенирования "--- это высокотехнологичная отрасль, которая позволяет получить из этих данных практически значимую информацию и минимизировать влияние ошибок на эту информацию.

\subsection{Базовая схема обработки результатов высокопроизводительного секвенирования для поиска и клинической интерпретации однонуклеотидных полиморфизмов}

\paragraph{Демультиплексикация.}
В процессе приготовления NGS\hyp{}библиотеки к целевым фрагментам ДНК лигируют так называемые адаптерные последовательности, или адаптеры.
Очень часто потенциальное количество прочтений, которое способен выдать секвенатор за один запуск, значительно превышает требуемое количество прочтений для отдельной библиотеки, поэтому из соображений экономии и повышения производительности на одном чипе секвенируют сразу несколько библиотек.
Для этого в адаптеры вставляют баркоды "--- последовательности, с помощью которых можно отличить прочтения, относящиеся к разным библиотекам или образцам.
Процесс сортировки данных секвенирования по баркодам называется демультиплексикацией.

\centerfigure{h}{Adapters.pdf}{fig:adapters}{Результаты секвенирования библиотек, содержащих короткие последовательности, могут быть контаминированы адаптерами. На рисунке показаны результаты секвенирования последовательностей, длина которых превышает количество циклов секвенирования (сверху), либо значительно меньше количества циклов (снизу).}{0.7}

\paragraph{Удаление адаптерных последовательностей.}
Если целевой фрагмент ДНК короче длины прочтения, то фрагменты адаптерной последовательности могут попасть в готовые данные (\picref{fig:adapters}).
Это замедляет работу алгоритма картирования, а порой в значительной степени ухудшает его результаты, поэтому встаёт вопрос об удалении адаптерных последовательностей.
Также присутствие адаптера в прочтениях может быть признаком контаминации библиотеки, и такие прочтения следует исключить из дальнейшего анализа\,\citep{Martin_2011}.

\paragraph{Картирование прочтений.}
Как уже упоминалось выше, результаты NGS "--- это прочтения, содержащие небольшие (в пределах \numprint[bp]{200}) фрагменты генома.
Извлечение информации из необработанных результатов секвенирования затруднительно, так как эти фрагменты содержат много ошибок (как в результате ПЦР\hyp{}реакции, так и допущенные в процессе секвенирования) и не имеют никакой информации о регионе, из которого они произошли.
Поэтому прочтения необходимо картировать на некую референсную геномную последовательность.
Алгоритм картирования представляет собой очень сложную систему, которая учитывает последовательность букв в прочтении и качество прочтения.
Качество прочтения отражает вероятность того, что буква, прочитанная секвенатором, совпадает с реальным нуклеотидом в данной позиции.
Обычно качество прочтения записывается в шкале Phred, к которой приводится формулой \begin{equation}Q = -10\log_{10}P,\label{eq:qual}\end{equation} где $P$ "--- вероятность того, что нуклеотид прочтен правильно.
Было разработано множество алгоритмов картирования, но в настоящее время <<золотым стандартом>> являются утилиты, использующие алгоритм Берроуса---Уиллера\,\citep{Burrows_1994}.

Обычно алгоритм картирования выставляет выравниванию коэффициент, называемый качеством выравнивания (\engterm{MAPping Quality, MAPQ}).
MAPQ отражает вероятность правильности картирования и также записывается в шкале Phred (\formularef{eq:qual}).
В силу размеров референсной последовательности в ней существует огромное множество повторов и похожих регионов.
Современные алгоритмы могут находить несколько потенциальных мест картирования для одного прочтения, и их количество влияет на качество выравнивания.

Также алгоритмы способны разделять прочтение на участки, которые могут быть картированы в разные места генома.
По этому признаку прочтения делятся на линейные и химерные.
В линейных прочтениях не может быть изменения направления картирования, т.е. картированная часть может иметь только прямое направление, либо только обратное направление относительно генома.
Химерные прочтения имеют картированные части с разным направлением.
Эти участки могут перекрываться, и количество перекрытий также влияет на MAPQ.

Исходя из особенностей алгоритмов картирования, выравнивания делятся на следующие классы:

\begin{itemize}
	\item Первичное выравнивание (\engterm{primary}) "--- выравнивание наиболее крупного (и содержащего наименьшее количество перекрытий, в случае химерного прочтения) фрагмента прочтения с наиболее высоким MAPQ.
	      Первичное выравнивание только одно.
	      Первичное выравнивание химерного прочтения называется репрезентативным;
	\item Вторичное выравнивание (\engterm{secondary}) "--- выравнивание наиболее крупного фрагмента прочтения с меньшим MAPQ.
	      Вторичных выравниваний может быть несколько (в зависимости от выставленного нижнего порога MAPQ);
	\item Добавочное выравнивание (\engterm{supplementary}) "--- выравнивание менее крупных (либо содержащих большее количество перекрытий) фрагментов прочтения.
	      Добавочные выравнивания характерны только для химерных прочтений.
\end{itemize}

Картированный участок может содержать в себе несовпадения с референсной последовательностью, инсерции и делеции.
Это могут быть как ошибки, так и генетические варианты, поэтому данная информация безусловно важна при анализе данных.
Также в частично картированных прочтениях могут присутствовать некартируемые участки с 3'- или 5'-конца.
В отличие от делеций внутри картированных участков, некартированные концы обычно подвергаются так называемому клипированию и в дальнейшем не учитываются при анализе.
Клипирование бывает двух типов:

\begin{itemize}
	\item Мягкое клипирование (\engterm{soft-clip}) "--- отсечение невыравненного конца прочтения с сохранением полной последовательности прочтения.
	      В отсечённых методом мягкого клипирования регионах могут быть адаптерные последовательности, а также часть химерного прочтения (в репрезентативном выравнивании).
	\item Жёсткое клипирование (\engterm{hard-clip}) "--- отсечение невыравненного конца прочтения без сохранения его последовательности.
	      В регионах, подвергшихся жёсткому клипированию, обычно находятся репрезентативные участки химерных прочтений (в добавочных выравниваниях).
\end{itemize}

Основные проблемы картирования:

\begin{itemize}
	\item Высоковариативные регионы.
	      Алгоритм картирования разработан для поиска наиболее полных соответствий, и при большом количестве несовпадений прочтение просто не сможет быть картировано на нужный регион генома;
	\item Вырожденные (неуникальные) регионы.
	      Соответствие между регионами может привести к неправильному распределению прочтений между ними, а значит "--- и неправильному картированию генетических вариаций.
	      Кроме того, генетические варианты в регионах с короткими повторами в принципе невозможно картировать точно, поэтому обычной практикой является левое смещение (\engterm{left-align}).
	\item Регионы с инсерциями и делециями.
	      Помимо того, что сами по себе эти варианты сильно ухудшают картирование, содержащие их прочтения могут быть картированы неправильно (из-за того, что алгоритмы картирования используют случайно выбранные позиции в геноме для начала поиска соответствий).
	      Из-за этого могут возникать ложные SNP, а пропорции аллелей могут быть посчитаны неправильно.
	      Пример показан на \picref{fig:indels}.
\end{itemize}

\centerfigure{h}{Indels.pdf}{fig:indels}{Неоптимальное картирование прочтения, содержащего делецию. (1) "--- референсная последовательность, (2) "--- последовательность прочтения, (3) "--- картирование, произведённое алгоритмом, включающее две SNV и одну делецию, (4) "--- оптимальное местоположение делеции}{0.7}

\paragraph{Удаление дубликатов.}
Так как молекулы ДНК очень малы, вероятность их разрушения или возникновения в них ошибок велика, а полученные от них сигналы находятся за пределами чувствительности многих современных приборов.
Решением этих проблем является амплификация молекул ДНК.
Амплификация может быть как на стадии приготовления библиотеки (ПЦР), так и на стадии секвенирования.
При секвенировании амплификация и последующее объединение ампликонов в кластер производятся для усиления сигнала и нивелирования ошибок, происходящих на каждом цикле секвенирования с отдельными молекулами.
Соответственно, в процессе секвенирования возникают дубликатные прочтения, которые могут быть как ПЦР\hyp{}дубликатами библиотеки, так и возникать из-за ошибок распознавания кластеров амплификации (оптические дубликаты).
Согласно принятой практике, дубликаты должны быть удалены или помечены для улучшения поиска генетических вариантов\,\citep{Auwera_2013}.

Однако было показано, что для WGS-данных удаление дубликатов имеет минимальный эффект на улучшение поиска полиморфизмов "--- приблизительно \numprint[\%]{92} из более чем \numprint[\mln]{17} вариантов были найдены вне зависимости от наличия этапа удаления дубликатов и использованных инструментов для поиска дубликатов\,\citep{Ebbert_2016}.
Учитывая, что удаление дубликатов может занимать значительную часть потраченного на обработку данных времени, следует взвесить пользу и затраты данного этапа для конкретной прикладной задачи.

\paragraph{Рекалибровка качества прочтений} (\engterm{Base Quality Score Recalibration, BQSR}).
В приборной оценке качества прочтений всегда имеют место систематические ошибки.
Это связано как с особенностями физико-химических реакций в секвенаторе, так и с техническими недостатками оборудования.
Вычисление качества прочтения "--- сложный алгоритм, защищённый авторскими правами производителя секвенатора.
Вместе с тем от качества прочтений напрямую зависит алгоритм поиска вариантов "--- он использует данный коэффициент как вес в пользу присутствия или отсутствия генетического варианта в конкретной точке генома.

Решением является рекалибровка качества прочтений, представляющая собой корректировку систематических ошибок, исходя из известных паттернов зависимости случайных величин.
Следует заметить, что рекалибровка не помогает определить, какой нуклеотид в реальности находится в данной позиции "--- она лишь указывает алгоритму поиска генетических вариантов, выше или ниже вероятность правильного прочтения нуклеотида секвенатором.

Первоочередное влияние на ошибки оказывают:

\begin{enumerate}
	\item Собственно прибор (секвенатор) и номер запуска.
	      Большая часть секвенаторов выставляет прочтению более высокое качество прочтения по сравнению с ожидаемым, гораздо реже встречаются модели, занижающие качество прочтения\,\citep{Auwera_2013}.
	      Каждый отдельный запуск может различаться по параметрам чипа и химических реагентов;
	\item Цикл секвенирования.
	      Качество прочтения уменьшается с каждым циклом за счёт накопления ошибок в кластере амплификации;
	\item Нуклеотидный контекст.
	      Систематические ошибки, связанные с физико-химическими процессами, влияют на качество прочтения нуклеотида в зависимости от предшествующего ему динуклеотида.
\end{enumerate}

Кроме того, алгоритм рекалибровки учитывает изменчивость каждого отдельного сайта, используя базы данных известных генетических вариантов.
Высокая изменчивость повышает вероятность правильного прочтения нуклеотида, не совпадающего с референсным в данной позиции генома.

Институт Броуд (\engterm{Broad Institute of MIT and Harvard}) рекомендует BQSR к использованию для любых данных секвенирования\,\citep{Auwera_2013}.

\paragraph{Поиск генетических вариантов.}
Невозможно точно сказать, какой нуклеотид находится в каждой позиции генома.
Анализ производит специальный алгоритм, который оценивает качество прочтения, качество выравнивания и процент букв в данной позиции на картированных прочтениях.
Отличие генома образца от референсного генома называется генетическим вариантом.
Алгоритм выставляет каждому генетическому варианту коэффициент качества варианта (\engterm{VCF QUAL}), записываемый в шкале Phred (\formularef{eq:qual}).
Помимо определения генетического варианта, алгоритм может определять его зиготность.

Также важным этапом поиска вариантов является уже упомянутое выше левое выравнивание.
Варианты в повторяющихся последовательностях с длиной менее длины одного прочтения невозможно точно локализовать, поэтому они всегда сдвигаются как можно левее относительно последовательности генома.
Это чрезвычайно важно при аннотации генетических вариантов, так как все БД используют данные с левым выравниванием, и неправильная локализация может привести к отсеиванию потенциально патогенного варианта.

~

После того, как генетические варианты найдены, можно приступать к поиску тех, которые связаны с конкретной патологией у пациента.
Однако только в кодирующих областях генома количество генетических вариантов достигает \numprint[\thousands]{100} (из них около \numprint[\%]{86} SNV, \numprint[\%]{7} инсерций и \numprint[\%]{7} делеций)\,\citep{Supernat_2018}, из них с патологиями связаны единицы.
Даже после жёсткой фильтрации приходится работать минимум с сотней подходящих генетических вариантов.
Это делает серьёзной проблемой поиск нужного варианта и интерпретацию полученных результатов.

\subsection{Аннотация, фильтрация и интерпретация результатов}

Первое, что следует сделать "--- это определить, насколько генетический вариант значим для нашего исследования, то есть аннотировать его.
Существуют две основных парадигмы аннотации генетического варианта "--- это аннотация по региону и аннотация по координате.

Основные методы аннотации по региону:

\begin{enumerate}
	\item Функциональный класс.
	      Для определения функционального класса генетического варианта существуют три основных базы данных: knownGene, refGene и ensGene.
	      Они содержат информацию о генах, их частях и транскриптах "--- координаты, направление, а также номера экзонов и интронов.
	      Координаты в этих базах данных могут различаться\,\citep{McCarthy_2014}, поэтому, во избежание ошибок, рекомендуется использовать их все.
	      Это особенно важно при дифферециации генетических вариантов с высокой вероятностью повреждающего эффекта (сдвиги рамок считывания, нонсенс-кодоны).
	      Кроме того, различаются алгоритмы определения функционального класса в различных утилитах аннотации, что также создаёт определённые трудности\,\citep{Jesaitis_2014}.

	\item Клиническая значимость гена.
	      Количество генетических вариантов для поиска можно сузить, зная, какие именно гены могут быть связаны с наблюдаемым у пациента фенотипом.
	      Для поиска генов по клинической значимости существуют такие базы данных, как OMIM\,\citep{Amberger_2014} и OrphaData\,\citep{Orphanet}.

	\item Потеря функции (\engterm{Loss of Function, LoF}).
	      Различные показатели, отражающие устойчивость функции гена, основанные на данных о стоп-кодонах, сдвигах рамки считывания и сплайс-вариантах.
	      Одним из таких показателей является pLI.

	      Основные проблемы pLI\,\citep{Ziegler_2019}:

	      \begin{itemize}
		      \item Плохо приспособлен к распознаванию аутосомно-рецессивных вариантов (из-за того, что частота повреждающих вариантов в популяции может быть высокой) и X-сцепленных рецессивных вариантов (из-за наличия в популяции здоровых гетерозиготных носителей).
		      \item Плохо приспособлен к распознаванию генетических вариантов в генах, ответственных за патологии, не влияющие на взросление и воспроизводство.
		            Их частота в популяции также может быть высокой.
		            К таким относятся варианты в генах \genename{BRCA1} и \genename{BRCA2}, ответственных за рак молочной железы.
		      \item Сплайс-варианты априори рассматриваются как повреждающие, несмотря на то, что вариант в сайте сплайсинга может не иметь эффекта на сплайсинг, либо приводить к появлению изоформы белка без потери функции.
		      \item Высокая частота распространения заболевания в контрольной группе.
		            Пример "--- шизофрения.
		      \item К миссенс-вариантам pLI применять следует с осторожностью, и без клинических данных следует исключить из анализа.
		      \item Также следует отнестись с осторожностью к нонсенс-вариантам и сдвигам рамки считывания в последнем экзоне либо в C\hyp{}терминальной части предпоследнего.
		            Такие транскрипты избегают нонсенс-индуцированной деградации РНК и могут в результате как не привести к каким-либо функциональным изменениям, так и привести к образованию мутантного белка, обладающего меньшей активностью по сравнению с исходным, либо токсичного для клетки.
		      \item В некоторых случаях соотношение pLI с гаплонедостаточностью конкретного гена в принципе сложно объяснить.
	      \end{itemize}

	      Таким образом, высокое значение pLI можно считать хорошим показателем LoF, низкое "--- с осторожностью.                                                                                                       \end{enumerate}

Аннотация по координате обычно предназначена для миссенс-, интронных и сплайс-вариантов, связь которых с патологическим состоянием значительно сложнее выявить и доказать.

\begin{enumerate}
	\item Частота аллеля в популяции.
	      Многие тяжёлые генетические патологии испытывают на себе давление отбора, а значит, вызывающие их генетические варианты не могут иметь высокую частоту в популяции.
	      Фильтрация по частоте является одним из базовых способов фильтрации генетических вариантов.
	      Следует заметить, однако, что низкая частота генетического варианта далеко не всегда связана с его патогенностью, поэтому рассматривать низкую частоту как доказательство патогенности некорректно.

	      По мере развития методов NGS и увеличения их доступности, начали появляться базы данных, агрегирующие результаты секвенирования различных популяций, а значит "--- способные определить частоту генетических вариантов в популяции.
	      В настоящее время наиболее крупной является gnomAD\,\citep{Karczewski_2020}, поглотившая существовавший ранее ExAC, содержавший исключительно экзомные данные.
	      Она содержит частоты генетических вариантов для всех основных рас, а также некоторых условно-здоровых групп.

	      Несмотря на то, что были созданы базы данных для всех рас, очень часто этого недостаточно и необходимо учитывать частоты в популяциях отдельных народов и стран.
	      Такими базами данных являются GME\,\citep{Scott_2016}, в которой отражены частоты по популяции Ближнего Востока, ABraOM\,\citep{Naslavsky_2017}, предоставляющая частоты генетических вариантов среди практически здорового пожилого населения Бразилии.
	      Также для анализа берутся популяции, в которых велика доля близкородственных связей, например, пакистанская\,\citep{Saleheen_2017}.

	\item Клинические данные из БД и статей.
	      Наиболее достоверным источником данных о патогенности генетического варианта являются семейные и популяционные исследования конкретной патологии, а также базы данных, агрегирующие информацию из подобных статей.
	      Наиболее используемыми в настоящее время являются HGMD\,\citep{Stenson_2017} и CLINVAR\,\citep{Landrum_2017}.
	      Тем не менее, CLINVAR считается лишь дополнительным источником, так как часто содержит информацию низкого качества\,\citep{Ryzhkova_2017}.

	\item Анализ и предсказание функционального эффекта \textit{in silico}.
	      \textit{In silico} методы появились в ответ на необходимость как-то классифицировать генетические варианты, по которым недостаточно клинической информации.
	      Существует множество способов проверить патогенность таких вариантов \textit{in vitro}, но проверять таким образом все нецелесообразно, а иногда и невозможно.
	      Даже в хорошо изученных генах варианты с неопределённой клинической значимостью могут занимать большую долю "--- например, в \genename{BRCA1} и \genename{BRCA2} это \numprint[\%]{33} и \numprint[\%]{50} соответственно.
	      Менее изученные гены, а также пациенты, принадлежащие к популяциям с плохо изученным составом генетических вариантов, представляют ещё большую проблему.

	      Поэтому были разработаны инструменты на основе машинного обучения, предсказывающие консервативность районов и патогенность генетических вариантов на основе имеющихся данных "--- положения относительно гена и его функциональных элементов, характера замены, а также клинической информации об известных заменах\,\citep{j_Brea_Fernandez_2011}.
	      Предсказательная способность отдельных инструментов оставляет желать лучшего, поэтому чаще всего в клинической практике используются агрегаторы, собирающие предсказания с большого числа известных \textit{in silico} инструментов.
\end{enumerate}

Значимость вклада каждого отдельного фактора достаточно сложно оценить.
Эту проблему решают калькуляторы патогенности, которые по специальным критериям присваивают генетическому варианту ранг, отражающий вероятность повреждающего действия\,\citep{Ryzhkova_2017}.

\paragraph{Когортный и семейный анализ.}
В случае, если исследователь имеет доступ к группе, представители которой связаны узами крови с пациентом, есть возможность провести семейный анализ.
Семейный анализ нужен для установления путей наследования тех или иных генетических вариантов в родословной.
Это позволяет уточнить их связь с фенотипом.
Также анализ нескольких родственных образцов помогает определить зиготность варианта, обнаружить генетические варианты \textit{de novo}, либо импутировать район с недостаточным покрытием.

Если же в распоряжении исследователя находится группа, связанная одной патологией или вариантом фенотипа, можно провести когортный анализ.
Когортный анализ позволяет, например, оценить частоты генетических вариантов в исследуемой и контрольной группе.
Кроме того, когортный анализ образцов в конкретной лаборатории помогает детектировать систематические отклонения покрытия и артефакты выравнивания, связанные с конкретными районами генома и/или особенностями приготовления библиотек.

% \paragraph{Cлучайные находки.}
% Несмотря на то, что точность определения патогенности вариантов достаточно невысокая, этические правила, регламентирующие работу врача-генетика, рекомендуют сообщать о потенциально патогенных вариантах в некоторых генах, даже если они не связаны с текущим состоянием пациента.
% К таким генам относятся, например, BRCA1 и BRCA2, связанные с раком молочной железы.
% 
% ~
% 
% Описанная выше схема характерна для поиска генетических вариантов во всех видах NGS\hyp{}данных.
% Тем не менее, частности могут различаться.
% Это связано с особенностями покрытия генома, наличием технических последовательностей в результатах секвенирования и многими другими факторами.
% Таким образом, новые методы приготовления NGS\hyp{}библиотек часто требуют соответствующей доработки биоинформационных методов, а иногда и разработки новых.

\subsection{Exo-C: суть метода}

Как уже упоминалось выше, одним из основных ограничений NGS\hyp{}технологий в настоящее время является их цена, напрямую зависящая от глубины секвенирования библиотеки.
Есть ограничения и по возможностям поиска тех или иных генетических вариантов.
3C-методы на сегодняшний момент являются наиболее перспективным способом обнаружения хромосомных перестроек\,\citep{Melo_2020}, но при небольшой глубине секвенирования в них обнаружение точечных полиморфизмов затруднительно\,\citep{Sims_2014}.
WGS способно обнаруживать большую часть SNV, небольших инсерций и делеций, но требует большую глубину секвенирования\,\citep{Sims_2014}; WES, с другой стороны, позволяет выявить генетические варианты при небольшой глубине секвенирования, но только в экзоме.
Возможности обнаружения хромосомных перестроек для последних двух методов ограничены.

Компромиссом между ценой и возможностями поиска генетических вариантов может служить новейший метод Exo-C, сочетающий технологии таргетного обогащения с 3C.
Суть его заключается в приготовлении Hi-C\hyp{}библиотеки и последующем обогащении только тех последовательностей, которые связаны с экзомом.
Таким образом, с его помощью можно как искать точечные варианты в обогащённых регионах (за счёт большой глубины покрытия в них), так и хромосомные перестройки во всём геноме (за счёт Hi-C, дающей относительно небольшое, но доступное для анализа покрытие всего генома)\,\citep{Mozheiko_2019}.

Тем не менее, как выяснилось, уже существующие биоинформационные методы следует модифицировать для корректной обработки данных Exo-C.
Это связано в первую очередь с особенностями протокола Hi-C, к примеру, наличием технических последовательностей (бридж-адаптеров), которые приводят к появлению ложных SNV в экзомных регионах.
Таргетное обогащение, со своей стороны, вносит определённые помехи в Hi-C-данные, так как изменяется представленность регионов генома в библиотеке, а значит, и пропорции контактов между регионами.

Данная работа посвящена разработке биоинформационных методов для поиска точковых генетических вариантов в Exo-C-данных и последующего сравнения Exo-C с методами полногеномного и полноэкзомного секвенирования.

\section{Материалы и методы}

\paragraph{Данные секвенирования.}
Поиск данных секвенирования производился в базах данных NCBI (GEO DataSets, SRA, PubMed) и ENCODE с использованием ключевых слов ``K562'', ``K562+WGS'', ``K562+WES'', ``K562+Hi-C''.

\paragraph{Контроль качества NGS\hyp{}данных.}
Для контроля качества прочтений мы использовали утилиту \utilname{FastQC}\,\citep{FastQC}, способную оценивать наличие адаптерных последовательностей, распределение прочтений по длине, GC-состав прочтений, а также производить анализ зависимости нуклеотидного состава от позиции в прочтении.
Критерии качества были использованы согласно протоколу разработчика\,\citep{FastQC}.

\paragraph{Удаление адаптерных последовательностей.}
Удаление адаптерных последовательностей производилось с помощью утилиты \utilname{cutadapt}\,\citep{Martin_2011}.
В \citet{Auwera_2013} рекомендуется использовать в качестве входных данных некартированный BAM-файл (\engterm{Unmapped Binary sequence Alignment/Map, uBAM}), а для удаления адаптеров использовать их собственный инструмент "--- \utilname{MarkIlluminaAdapters}, так как это позволяет сохранить важные метаданные.
Тем не менее, был сделан акцент на том, что uBAM должен использоваться как выходной формат на уровне секвенатора, что не является общепринятой практикой.

Мы использовали данные секвенирования в формате FastQ.
Пребразование FastQ-файлов в uBAM не предотвращает потерю метаданных, но значительно увеличивает время обработки данных.
Сравнение эффективности \utilname{cutadapt} и \utilname{MarkIlluminaAdapters} в процессе удаления адаптеров не показало каких-либо значимых различий.

\paragraph{Картирование.}
Картирование производилось с помощью инструментов \utilname{Bowtie2}\,\citep{Langmead_2012} и \utilname{BWA}\,\citep{Li_2009}.
\utilname{BWA} показал лучшие результаты;
кроме того, он значительно более эффективно работает с химерными ридами, что немаловажно для используемого нами метода Exo-C.

Для картирования был взят геном GRCh37/hg19, предоставленный NCBI.
Из него были удалены так называемые неканоничные хромосомы (некартированные/вариативные референсные последовательности), что позволило улучшить качество выравнивания и значительно упростить работу с готовыми данными.

Кроме того, для правильного функционирования инструментов на дальнейших этапах был разработан скрипт, создающий метку группы прочтений (\engterm{Read Group tag, RG}) для каждого файла.
Конкретных рекомендаций по составлению RG не существует, поэтому мы разработали собственные, основанные на следующих требованиях\,\citep{Auwera_2013}:

\begin{itemize}
	\item Поле SM является уникальным для каждого биологического образца и используется при поиске вариантов.
	      Несколько SM в одном файле могут быть использованы при когортном анализе.
	\item Поле ID является уникальным для каждого RG в BAM-файле.
	      BQSR использует ID как идентификатор самой базовой технической единицы секвенирования.
	\item Поле PU не является обязательным.
	      Рекомендации GATK советуют помещать в него информацию о чипе секвенирования (баркод чипа), ячейке и баркоде (номере) образца.
	      Во время BQSR поле PU является приоритетным по отношению к ID.
	\item Поле LB является уникальным для каждой библиотеки, приготовленной из биологического образца.
	      Оно отражает различия в количестве ПЦР\hyp{}дубликатов и потому используется инструментом \utilname{MarkDuplicates}.
\end{itemize}

Объединение BAM-файлов производилось инструментом \utilname{MergeSamFiles}.
Сбор статистики по картированию мы осуществляли с помощью инструмента \utilname{SAMTools flagstat}\,\citep{Li_2009_SAMTools}.

\paragraph{Удаление ПЦР\hyp{}дубликатов.}
Для улучшения данных экзомного секвенирования в пайплайн был включён этап удаления ПЦР\hyp{}дубликатов.
Обычно этот процесс занимает много времени, но количество образцов у нас было относительно небольшим, и мы были заинтересованы в максимально качественной подготовке данных.

Удаление дубликатов производилось инструментом \utilname{MarkDuplicates} от Picard\,\citep{PicardTools}, интегрированным в \utilname{GATK}.
Оптимальные показатели скорости \utilname{MarkDuplicates} достигаются при запуске \utilname{Java} с параллелизацией сборщиков мусора и количеством сборщиков мусора равным двум\,\citep{Heldenbrand_2019}.
Также, согласно рекомендациям разработчиков, прочтения были предварительно отсортированы по именам, чтобы удалению подверглись не только первичные, но и добавочные выравнивания\,\citep{Auwera_2013}.

\paragraph{Рекалибровка качества прочтений (BQSR).}
Рекалибровка производилась с помощью инструментов \utilname{GATK BaseRecalibrator} и \utilname{GATK ApplyBQSR}.
Для обучения машинной модели требуются генетические варианты в VCF-формате (согласно рекомендациям для \textit{Homo sapiens} "--- dbSNP v132+).

К сожалению, предоставленная Broad Institute база данных оказалась сильно устаревшей и не вполне подходила для сделанной нами геномной сборки, поэтому было решено подвергнуть обработке dbSNP v150, предоставленную NCBI\,\citep{Sherry_2001}.
База данных потребовала замену и сортировку контигов в соответствии с референсным геномом, а также удаление <<пустых>> вариантов, содержащих точки в полях REF и ALT.
Далее база данных была архивирована с помощью \utilname{bgzip}, а затем проиндексирована \utilname{GATK IndexFeatureFile} (этот же инструмент одновременно проверяет БД на пригодность для BQSR).

В \citet{Heldenbrand_2019} было показано, что оптимальные показатели скорости \utilname{BaseRecalibrator} достигаются, как и в случае с \utilname{MarkDuplicates}, запуском \utilname{Java} с двумя параллельными сборщиками мусора;
кроме того, \utilname{BaseRecalibrator} поддаётся внешнему распараллеливанию путём разделения картированных прочтений на хромосомные группы.
Хромосомные группы формировались вручную для используемой сборки генома, каждая запускалась с помощью \utilname{bash}-скрипта.
Нам удалось усовершенствовать данный этап "--- запуск \utilname{BaseRecalibrator} производился с помощью библиотеки \utilname{Python} \utilname{subprocess}, а параллелизация осуществлялась библиотекой \utilname{multiprocessing}, таким образом, можно было делить файл с картированными прочтениями по хромосомам и обрабатывать их отдельно, так как \utilname{multiprocessing} автоматически распределяет процессы по имеющимся потокам.
Также для повышения отказоустойчивости скрипта у \utilname{BaseRecalibrator} и \utilname{ApplyBQSR} была устранена разница в фильтрации прочтений, из-за которой при малых размерах библиотек пайплайн экстренно завершал работу.

\paragraph{Оценка покрытия и обогащения.}
Покрытие и обогащение в экзоме оценивались с помощью скрипта на основе \utilname{BEDTools}\,\citep{Quinlan_2010}.

\paragraph{Поиск вариантов.}
Поиск вариантов производился с помощью инструмента \utilname{GATK HaplotypeCaller}.
Инструмент запускался с дополнительным параметром \verb|--dont-use-soft-clipped-bases|, который не позволял использовать для поиска генетических вариантов клипированные химерные части и адаптеры.

Как и в случае с \utilname{BaseRecalibrator}, \utilname{HaplotypeCaller} поддаётся внешнему распараллеливанию\,\citep{Heldenbrand_2019}.
Мы также осуществили параллелизацию с помощью сочетания \utilname{subprocess} и \utilname{multiprocessing}, достигнув 10--12-кратного ускорения по сравнению с запуском на одном потоке.

\paragraph{Рекалибровка и ранжирование вариантов.}
В GATK также присутствуют инструменты для рекалибровки и ранжирования вариантов, с использованием моделей машинного обучения и баз данных с частыми вариантами (\utilname{CNNScoreVariants} и \utilname{FilterVariantTranches}).

Анализ показал, что при наличии этапа рекалибровки вариантов время обработки результатов секвенирования увеличивается почти вдвое.
Между тем, рекалибровка и ранжирование с помощью инструментов GATK не исключают необходимость фильтрации генетических вариантов.
Таким образом, от этого этапа решено было отказаться.

\paragraph{Аннотация вариантов.}
Аннотация вариантов производилась вначале с помощью инструмента \utilname{Ensembl~VEP}\,\citep{McLaren_2016}, затем мы мигрировали на \utilname{ANNOVAR}\,\citep{Wang_2010}.

Используемые базы данных:

\begin{enumerate}
	\item Human Gene Mutation Database (HGMD\textregistered)\,\citep{Stenson_2017}
	\item Online Mendelian Inheritance in Man (OMIM\textregistered)\,\citep{Amberger_2014}
	\item GeneCards\textregistered: The Human Gene Database\,\citep{Stelzer_2016}
	\item CLINVAR\,\citep{Landrum_2017}
	\item dbSNP\,\citep{Sherry_2001}
	\item Genome Aggregation Database (gnomAD)\,\citep{Karczewski_2020}
	\item 1000Genomes Project\,\citep{Auton_2015}
	\item Great Middle East allele frequencies (GME)\,\citep{Scott_2016}
	\item dbNSFP: Exome Predictions\,\citep{Liu_2016}
	\item dbscSNV: Splice site prediction\,\citep{Jian_2013}
	\item RegSNPIntron: intronic SNVs prediction\,\citep{Lin_2019}
\end{enumerate}

\paragraph{Фильтрация генетических вариантов.}
Аннотации были агрегированы для удобства использования.
Так, агрегации подверглись:

\begin{itemize}
	\item Имена генов по разным БД "--- для облегчения поиска;
	\item Описания функциональных классов из разных БД "--- для устранения несоответствий между ними;
	\item Ранги инструментов, предсказывающих патогенность генетического варианта.
	      Трёхранговые системы (патогенный, вероятно патогенный и безвредный) были сведены к двухранговой (патогенный и безвредный).
	      Отдельно были агрегированы предсказательные инструменты для экзонов, инструменты для интронов и сплайс-вариантов также учитывались отдельно;
	\item Ранги инструментов, предсказывающих консервативность нуклеотида.
	      Эмпирическим путём было подобрано пороговое значение \numprint{0.7} "--- нуклеотид считался консервативным, если его предсказанная консервативность была выше, чем у \numprint[\%]{70} всех нуклеотидов.
	      Это максимальное пороговое значение, которое обеспечивает распределение балла агрегатора от минимального до максимального (от 0 до 7 баз данных, считающих данный нуклеотид консервативным);
	\item Популяционные частоты "--- из всех имеющихся в базах данных по конкретному генетическому варианту была выбрана максимальная частота.
\end{itemize}

Фильтрация происходила в две стадии:
\begin{enumerate}
	\item Фильтрация отдельных генетических вариантов на основе имеющихся аннотаций.
	      Самая жёсткая фильтрация, которой подвергались все варианты:
	      \begin{itemize}
		      \item По глубине покрытия.
		            Генетический вариант считался существующим, если он присутствовал в двух перекрывающихся парных прочтениях, либо в чётырёх независимых прочтениях;
		      \item Частота генетического варианта в популяции не более \numprint[\%]{3}\,\citep{Ryzhkova_2017}.
	      \end{itemize}

	      Прочие фильтры были мягкими "--- генетический вариант отсеивался только в случае несоответствия всем указанным критериям:

	      \begin{itemize}
		      \item Присутствие описания связанной с геном патологии в базе данных OMIM;
		      \item Присутствие генетического варианта в базе данных HGMD;
		      \item Балл агрегатора патогенности экзомных вариантов не менее 3\,\citep{Ryzhkova_2017};
		      \item Ранг <<патогенный>> у агрегаторов интронных или сплайс-вариантов;
		      \item Ранги <<патогенный>> и <<возможно патогенный>> по базе данных CLINVAR;
		      \item По функциональному классу: сдвиги рамки считывания, потери стоп- и старт-кодонов, нонсенс- и сплайс-варианты.
	      \end{itemize}

	\item Фильтрация значимых вариантов на основе аннотаций гена.
	      Все эти фильтры были мягкими "--- ген мог соответствовать одному любому из перечисленных критериев:

	      \begin{itemize}
		      \item Значение pLI более \numprint{0.9}, согласно рекомендациям в оригинальной статье\,\citep{Lek_2016};
		      \item Наследование в гене значится как <<доминантное>> по базе данных OMIM, либо информации о доминантности нет;
		      \item Любой значимый вариант в гене является гомозиготным;
		      \item В гене более одного значимого варианта (вероятность цис-транс-положения).
	      \end{itemize}
\end{enumerate}

\paragraph{Интерпретация.}
Интерпретация данных и составление отчёта производилось в соответствии с рекомендациями Американского колледжа медицинской генетики и геномики (\engterm{American College of Medical Genetics, Bethesda, MD, USA}) и Ассоциации молекулярной патологии\,\citep{Richards_2015}.
% В среднем на каждый образец в данных Exo-C приходилось порядка 1--2 \thousands значимых вариантов, затрагивающих около 100--150 генов.
% Порядка 100--200 вариантов были результатом систематических ошибок, возникших в ходе приготовления библиотеки или обработки данных.

\section{Результаты}

На сегодняшний день были выполнены следующие этапы работы:

\begin{enumerate}
	\item Создание контрольной выборки генетических вариантов, с помощью которой будет проведена оценка пригодности Exo-C\hyp{}библиотек к поиску генетических вариантов;
	\item Проверка качества данных, полученных в результате массового параллельного секвенирования Exo-C\hyp{}библиотек;
	\item Разработка, отладка и тестирование автоматизированного инструмента для обработки данных секвенирования Exo-C\hyp{}библиотек.
\end{enumerate}

\subsection{Результаты секвенирования Exo-C\hyp{}библиотек}

Несмотря на то, что составляющие протокола Exo-C "--- таргетное обогащение и Hi-C "--- в настоящее время достаточно отработаны, сочетание этих методик имеет свои подводные камни.
Было разработано две вариации протокола Exo-C (ExoC-19 и ExoC-20), обе этих вариации были использованы для приготовления библиотек клеточной линии K562\,\citep{Ma_2018,Ramani_2016,Gridina_Forthcoming}.
Критическим различием протоколов является использование дополнительных адаптеров в протоколе ExoC-19.
Результаты секвенирования этих библиотек проверялись биоинформационными методами.

Базовыми параметрами качества библиотек были приняты:

\begin{itemize}
	\item Доля дубликатов, отражающая качество стадии ПЦР;
	\item Доля участков, в которых покрытие прочтениями отсутствует, а также тех, в которых оно превышает минимальный порог для анализа (10 прочтений);
	\item Отношение среднего покрытия вне и внутри экзома, которое можно считать показателем качества таргетного обогащения.
\end{itemize}

Данные по качеству Exo-C\hyp{}библиотек представлены в \tableref{tab:exoc-enrichment}.

\begin{booktable}{Данные по обогащению Exo-C\hyp{}библиотек}{tab:exoc-enrichment}
	\begin{tabular}{| l | r | r | r | r | r | r | r | r |}
		\hline
		\rowcolor{tableheadcolor}
		\textbf{Название}                                                                   &
		\headerbigrow{ прочтений}{Глубина, прочтений}                                       &
		\textbf{Доля дубликатов, \%}                                                        &
		\headerbigrow{Доля экзома с глубиной}{Доля экзома с глубиной покрытия более 10, \%} &
		\headerbigrow{Среднее покрытие}{Среднее покрытие в экзоме}                          &
		\headerbigrow{Среднее покрытие}{Среднее покрытие вне экзома}                        &
		\headerbigrow{Обогащение}{Обогащение экзома, раз}                                   &
		\headerbigrow{регионов в экзоме, \%}{Доля непокрытых\newline регионов в экзоме, \%} &
		\headerbigrow{регионов вне экзома, \%}{Доля непокрытых регионов вне экзома, \%}                                                                                                                                                               \\
		\hline
		ExoC-19                                                                             & \numprint{136609179} & \numprint{18.86} & \numprint{91.68} & \numprint{60.51} & \numprint{5.56} & \numprint{10.89} & \numprint{1.75} & \numprint{28.12} \\
		ExoC-20                                                                             & \numprint{109486529} & \numprint{15.00} & \numprint{72.58} & \numprint{14.88} & \numprint{7.74} & \numprint{1.92}  & \numprint{1.66} & \numprint{11.62} \\
		\hline
	\end{tabular}
\end{booktable}

\subsection{Автоматизация обработки данных секвенирования}

При обработке данных секвенирования приходится сталкиваться с проблемами различного характера.
Одними из ключевых являются проблемы использования ресурсов компьютера.
Результаты секвенирования даже в сжатом виде занимают десятки и сотни гигабайт дискового пространства, и многие инструменты создают файлы с промежуточными результатами, которые занимают дисковое пространство, не неся никакой практической пользы для исследования.
Кроме того, из-за вычислительной сложности обработка таких больших блоков данных может занимать дни, недели и даже месяцы работы вычислительного кластера.

Вторая, не менее важная группа проблем, связана с используемыми для обработки инструментами.
Как было показано выше, стадий у обработки значительное количество, и не все стадии нужны при обработке конкретного блока данных секвенирования.
Ручная настройка и контроль процесса отнимают значительное количество времени исследователя;
таким образом, встаёт вопрос стандартизации и автоматизации процесса обработки данных секвенирования.

Существующие инструменты для обработки данных секвенирования были разработаны независимыми группами людей.
Эти инструменты различаются по многим аспектам.
% 
% \begin{itemize}
% \item Язык программирования.
% Каждый язык имеет свои особенности взаимодействия с вычислительной техникой "--- использование памяти, потребность в специальных окружениях и т.п.;
% \item Консольный интерфейс.
% Каждый инструмент имеет свои собственный интерфейс взаимодействия с пользователем.
% Этот интерфейс может содержать недокументированные или неправильно документированные возможности;
% \item Способность к взаимодействию с потоками данных.
% Большая часть инструментов может использовать стандартные потоки ввода/вывода (\verb|stdin|, \verb|stdout|), но некоторые в силу особенностей алгоритма (например, требующего обработку не одной строки, а всего файла целиком) взаимодействуют исключительно с файловой системой.
% \item Настраиваемость.
% Отдельные инструменты не предоставляют пользователю возможность настроить нужные параметры, и для этого требуются дополнительные надстройки.
% \item Требования к входным данным.
% Несмотря на то, что большая часть используемых форматов стандартизированы, в них могут быть вариабельные и необязательные блоки данных, к которым у конкретного инструмента могут быть свои требования.
% \end{itemize}
%
Так как разработка каждого отдельного инструмента является сложным и трудоёмким процессом, целесообразно использовать их как есть, а несоответствия устранять с помощью специально разработанной надстройки.
Таким образом, для нами был создан пайплайн, интегрирующий все стадии обработки данных секвенирования.
Блок-схема пайплайна представлена на \picref{fig:pipeline}.

\centerfigure{h}{BlockScheme.pdf}{fig:pipeline}{Принципиальная схема пайплайна для обработки Exo-C-данных}{1}

Решённые задачи:

\begin{itemize}
	\item Отказоустойчивость: максимально устранены несоответствия форматов входных и выходных данных; процесс разделён на стадии, и в случае экстренного прерывания вычислений (программного или аппаратного) предусмотрен автоматический откат.
	\item Оптимизация, параллелизация и масштабируемость: все процессы, которые способны использовать стандартные потоки ввода/вывода, объединены вместе, поддающиеся внешнему распараллеливанию были распараллелены, также были подобраны оптимальные параметры запуска приложений, использующих машину Java.
	      Пайплайн может быть использован как на кластерах с большим количеством ядер и оперативной памяти, так и на относительно небольших мощностях офисных компьютеров;
	\item Значительно упрощены процессы развёртки и использования пайплайна: автоматизировано индексирование референсной последовательности, настройки вынесены в специальный конфигурационный файл, есть возможность обработки пула данных, используя один короткий сценарий;
\end{itemize}

Код пайплайна доступен на GitHub\,\citep{Scissors}.

\subsection{Сравнение данных секвенирования клеточной линии K562}

Следующим важным этапом работы была проверка эффективности поиска генетических вариантов в Exo-C\hyp{}библиотеках.
Было решено использовать для этого распространённую иммортализованную клеточную линию K562, полученную от пациентки с хроническим миелолейкозом\,\citep{Lozzio_1975}.
Данная клеточная линия была многократно секвенирована различными лабораториями с использованием различных методик приготовления библиотек.
Таким образом, несмотря на то, что в этой клеточной линии наблюдается некоторая гетерогенность между лабораториями из-за большого количества пассажей, несмотря на наличие систематических ошибок при использовании разных методов секвенирования и приготовления библиотек, по K562 существует достаточное количество данных, чтобы использовать эту клеточную линию как стандарт для поиска генетических вариантов.

Результаты секвенирования клеточной линии K562 были взяты из публичных источников\,\citep{Banaszak_2018,Belaghzal_2017,Dixon_2018,Moquin_2017,Rao_2014,Ray_2019,Wang_2020,Zhou_2019}.
Использованные в этих статьях методики включают WGS, WES, Hi-C и Repli-seq.
Из данных полноэкзомного секвенирования в дальнейшем были исключены все генетические варианты в интервале \numprint{chr2:25455845-25565459} с фланкированием \numprint[kbp]{1} (ген \genename{DNMT3A}), так как в одной из работ использовали генетически модифицированную линию с вариантами в данном гене\,\citep{Banaszak_2018}.
В качестве тестовых Exo-C-образцов мы использовали данные, полученные на основе клеточной линии K562, имеющейся в Институте Цитологии и Генетики СО~РАН.
Технические данные контроля качества по тестовым и контрольным образцам представлены в \tableref{appendix:control-libs} и \tableref{appendix:control-samples}.

В общей сложности, объединив варианты из всех контрольных образцов, мы получили \numprint{5496486} различных генетических вариантов.
Также в библиотеках было найдено некоторое количество уникальных генетических вариантов, встречающихся в одной библиотеке и не встречающихся в остальных (\tableref{tab:unique-controls}).
Наибольший процент уникальных вариантов найден в данных \citeauthor{Banaszak_2018}

\begin{booktable}{Уникальные генетические варианты в данных секвенирования контрольных образцов клеточной линии K562}{tab:unique-controls}
	\begin{tabular}{| l | l | r | r | r | r |}
		\hline
		\rowcolor{tableheadcolor}
		\textbf{Название}                                                         &
		\textbf{Протокол}                                                         &
		\headerbigrow{Глубина секвенирования,}{Глубина секвенирования, прочтений} &
		\headerbigrow{Общее число}{Общее число вариантов}                         &
		\headerbigrow{Уникальные}{Уникальные варианты}                            &
		\headerbigrow{Доля уникальных}{Доля уникальных вариантов, \%}
		\\
		\hline
		\citeauthor{Banaszak_2018}                                                & WES       & \numprint{254983225}  & \numprint{408008}  & \numprint{41830}  & \numprint{10.25} \\
		\citeauthor{Belaghzal_2017}                                               & Hi-C      & \numprint{72914268}   & \numprint{1399457} & \numprint{27365}  & \numprint{1.95}  \\
		\citeauthor{Dixon_2018}                                                   & WGS       & \numprint{366291496}  & \numprint{4649012} & \numprint{327184} & \numprint{7.03}  \\
		\citeauthor{Moquin_2017}                                                  & Hi-C      & \numprint{256500659}  & \numprint{2365361} & \numprint{67678}  & \numprint{2.86}  \\
		\citeauthor{Rao_2014}                                                     & Hi-C      & \numprint{1366228845} & \numprint{4218233} & \numprint{320508} & \numprint{7.59}  \\
		\citeauthor{Ray_2019}                                                     & Hi-C      & \numprint{428306794}  & \numprint{1789324} & \numprint{89624}  & \numprint{5.00}  \\
		\citeauthor{Wang_2020}                                                    & Repli-seq & \numprint{301663640}  & \numprint{2207451} & \numprint{37578}  & \numprint{1.70}  \\
		\citeauthor{Zhou_2019}                                                    & WGS       & \numprint{2621311293} & \numprint{4412455} & \numprint{166451} & \numprint{3.77}  \\
		\hline
	\end{tabular}
\end{booktable}

\numprint{75328} генетических вариантов были найдены в данных из всех восьми статей "--- их было решено использовать как <<золотой стандарт>>.
Сразу можно внимание на то, что это составляет лишь \numprint[\%]{1.37} геномных SNV клеток K562.
Такая ситуация может возникнуть в следующих случаях:

\begin{enumerate}
	\item В одной или нескольких работах обнаружено очень много уникальных вариантов, которые дают существенный вклад в общее число вариантов, но не пересекаются с результатами других исследований;
	\item В одной или нескольких работах не найдено подавляющее большинство вариантов, найденных во всех остальных работах;
	\item Распределение уникальных вариантов и число общих вариантов между парами работ относительно равномерно, и низкое число общих для всех восьми работ вариантов не может объясняться особенностями какого-то одного или нескольких исследований.
\end{enumerate}

Чтобы проверить, не связана ли низкая доля общих генетических вариантов с особенностями какого-то одного из использованных наборов данных, мы протестировали все комбинации из семи и шести работ.
Результаты представлены на \picref{fig:exclusion}.

\centerfigure{hp!}{Exclusion_6.pdf}{fig:exclusion}{Исключение отдельных образцов из контрольной выборки позволило увеличить количество генетических вариантов, которые можно использовать как стандарт. На рисунке показано суммарное количество вариантов в выборке (светло-зелёный) и процент общих для этой выборки вариантов (тёмно-зелёный) для выборок размером в 6 и 7 образцов. Слева указаны названия исключённых из выборки образцов.}{0.7}

При исключении из выборки данных \citeauthor{Banaszak_2018} и \citeauthor{Belaghzal_2017} общими являются \numprint{1091331} (\numprint[\%]{19.85}) вариантов.
Их решено было использовать как добавочный (<<серебряный>>) стандарт.
Далее мы использовали варианты <<серебряного>> и <<золотого>> стандартов для того, чтобы определить точность поиска генетических вариантов в наших Exo-C\hyp{}библиотеках.
Для этого мы оценили количество генетических вариантов, являющихся общими для <<серебряного>> и <<золотого>> стандартов и наших Exo-C\hyp{}библиотек, их долю от общего числа вариантов в Exo-C\hyp{}библиотеках, а также количество и долю ложноположительных (отсутствующих в контрольных образцах) генетических вариантов.
Также было решено проверить эффективность использованного нами базового фильтра "--- удаление всех генетических вариантов, в которых глубина альтернативного аллеля составляет менее 4.
Поиск вариантов <<серебряного>> и <<золотого>> стандартов в наших библиотеках был произведён до и после фильтрации.
Результаты показаны в \tableref{tab:filtration-efficiency}.

\begin{booktable}{Параметры Exo-C\hyp{}библиотек. (F--) "--- до фильтрации по глубине альтернативного аллеля, (F+) "--- после фильтрации, ($\Delta$) "--- изменение параметра после фильтрации в процентах}{tab:filtration-efficiency}
	\begin{tabular}{| l | r | r | r | r | r | r | r | r | r | r | r | r |}
		\hline
		\rowcolor{tableheadcolor}
		\textbf{Параметр}                                                                                                                        &
		\multicolumn{3}{ c |}{\textbf{ExoC-19}}                                                                                                  &
		\multicolumn{3}{ c |}{\textbf{ExoC-20}}                                                                                                  &
		\multicolumn{3}{ c |}{\textbf{В обеих}}                                                                                                  &
		\multicolumn{3}{ c |}{\textbf{Ни в одной}}
		\\
		\rowcolor{tableheadcolor}
		~                                                                                                                                        &
		\textbf{F--}                                                                                                                             &
		\textbf{F+}                                                                                                                              &
		\textbf{$\Delta$, \%}                                                                                                                    &
		\textbf{F--}                                                                                                                             &
		\textbf{F+}                                                                                                                              &
		\textbf{$\Delta$, \%}                                                                                                                    &
		\textbf{F--}                                                                                                                             &
		\textbf{F+}                                                                                                                              &
		\textbf{$\Delta$, \%}                                                                                                                    &
		\textbf{F--}                                                                                                                             &
		\textbf{F+}                                                                                                                              &
		\textbf{$\Delta$, \%}                                                                                                                                                                                                                                                                                                                                                                        \\
		\hline
		Общее число вариантов в библиотеке                                                                                                       & \numprint{3173343} & \numprint{1396525} & \numprint{-55.99} & \numprint{3750319} & \numprint{2577934} & \numprint{-31.26} & ---               & ---               & ---               & ---              & ---               & ---                \\
		Вариантов <<золотого стандарта>>                                                                                                         & \numprint{62335}   & \numprint{52732}   & \numprint{-15.41} & \numprint{72705}   & \numprint{67270}   & \numprint{-7.48}  & \numprint{60728}  & \numprint{48840}  & \numprint{-19.58} & \numprint{1016}  & \numprint{4166}   & \numprint{+310.04} \\
		Доля вариантов <<золотого стандарта>>, \%                                                                                                & \numprint{82.75}   & \numprint{70.00}   & ---               & \numprint{96.52}   & \numprint{89.30}   & ---               & \numprint{80.62}  & \numprint{64.84}  & ---               & \numprint{1.35}  & \numprint{5.53}   & ---                \\
		Вариантов <<серебряного стандарта>>                                                                                                      & \numprint{616375}  & \numprint{391273}  & \numprint{-36.52} & \numprint{982858}  & \numprint{821991}  & \numprint{-16.37} & \numprint{580351} & \numprint{340833} & \numprint{-41.27} & \numprint{72449} & \numprint{218900} & \numprint{+202.14} \\
		Доля вариантов <<серебряного стандарта>>, \%                                                                                             & \numprint{56.48}   & \numprint{35.85}   & ---               & \numprint{90.06}   & \numprint{75.32}   & ---               & \numprint{53.18}  & \numprint{31.23}  & ---               & \numprint{6.64}  & \numprint{20.06}  & ---                \\
		\bigrow{Доля вариантов <<серебряного стандарта>>, \%}{Количество вариантов библиотеки,\newline отсутствующих в контрольных образцах}     & \numprint{1130049} & \numprint{84770}   & \numprint{-92.50} & \numprint{354044}  & \numprint{41719}   & \numprint{-88.22} & \numprint{14455}  & \numprint{2981}   & \numprint{-79.38} & ---              & ---               & ---                \\
		\bigrow{Доля вариантов <<серебряного стандарта>>, \%}{Доля вариантов, отсутствующих в контрольных образцах, от вариантов библиотеки, \%} & \numprint{35.61}   & \numprint{6.07}    & \numprint{-82.95} & \numprint{9.44}    & \numprint{1.62}    & \numprint{-82.86} & ---               & ---               & ---               & ---              & ---               & ---                \\
		\hline
	\end{tabular}
\end{booktable}


\section{Обсуждение результатов}

\subsection{Контрольные образцы}

<<Золотой стандарт>> с учётом подбора библиотек скорее всего является набором генетических вариантов, относящихся к экзомным регионам, так как одна из библиотек представляла собой результаты WES.
Их было обнаружено \numprint[\thousands]{75}, что соответствует оценкам среднего количества генетических вариантов в кодирующих регионах у человека "--- \numprint[\thousands]{100}\,\citep{Supernat_2018}.
Общее число несоответствий с референсным геномом у среднего человека составляет \numprint[\mln]{4.1--5}\,\citep{Auton_2015}, что с учётом гетерогенности клеточной линии K562 перекликается с общим количеством найденных нами генетических вариантов (\numprint[\mln]{5.5}).

Как видно из представленных выше данных, образец \citeauthor{Banaszak_2018} содержит наибольшее число уникальных вариантов (\numprint[\%]{10.25}).
Это может быть связано с тем, что это данные полноэкзомного секвенирования, с высоким покрытием в экзонах, где и были найдены уникальные варианты.
В качестве дополнительной гипотезы можно предположить, что в этой работе использовались линии клеток, в значительной степени отличающиеся от классической линии K562.

Прослеживается ожидаемая положительная связь между глубиной секвенирования Hi-C\hyp{}библиотек и количеством уникальных вариантов в них.
В двух WGS\hyp{}библиотеках подобной связи не наблюдается.
Вероятнее всего, это также связано с отличиями использованных линий K562.

\subsection{Оценка результатов секвенирования Exo-C\hyp{}библиотек}

В Exo-C\hyp{}библиотеках глубина секвенирования составляет \numprint[\mln]{136.6} прочтений (\numprint[bp]{2.05e10}) и \numprint[\mln]{109.4} прочтений (\numprint[bp]{1.64e10}), а среднее покрытие в экзоме "--- \numprint{60.51} и \numprint{14.88} прочтений для ExoC-19 и ExoC-20 соответственно.
Глубину покрытия более 10 прочтений имеют \numprint[\%]{91.68} и \numprint[\%]{72.58} экзома для ExoC-19 и ExoC-20 соответственно.
Согласно \citet{Sims_2014}, для репрезентативных результатов экзомного секвенирования необходима глубина секвенирования не менее чем в \numprint[bp]{e10}, а для Hi-C "--- не менее чем \numprint[\mln]{100} прочтений.
Минимальным порогом глубины для возможности поиска генетических вариантов считается 10 прочтений, практически все гомозиготные SNV могут быть найдены при глубине в 15 прочтений, а гетерозиготные требуют глубину прочтений не менее 33.
Приемлемая доля экзома с репрезентативным покрытием (более 10 прочтений) составляет \numprint[\%]{90}.
Таким образом, можно утверждать, что ExoC-19 отвечает требованиям для поиска SNV, а ExoC-20, во-первых, пригодна к поиску только гомозиготных генетических вариантов, а во-вторых, имеет недостаточно хорошее покрытие в экзоме.

<<Золотой стандарт>> покрыт нашими библиотеками на \numprint[\%]{82.75} и \numprint[\%]{96.52}, <<серебряный стандарт>> "--- на \numprint[\%]{56.48} и \numprint[\%]{90.06} (библиотеки ExoC-19 и ExoC-20 соответственно).
Различия объясняются протоколами приготовления: у библиотеки ExoC-20 выше глубина покрытия в экзоме, в 6 раз выше обогащение в экзомных районах (критерий Манна---Уитни $p = 0.0003$).
Кроме того, в библиотеке ExoC-19 были использованы адаптерные последовательности, дающие большое количество шума.

Одним из базовых методов фильтрации генетических вариантов является фильтрация по глубине альтернативного аллеля.
Сразу можно обратить внимание на следующее:

\begin{itemize}
	\item В библиотеке ExoC-19 потеряна большая доля вариантов, чем в ExoC-20 "--- как относительно общего числа, так и относительно вариантов <<золотого>> и <<серебряного>> стандартов.
	\item Доля ложноположительных (отсутствующих в контрольных образцах) генетических вариантов снизилась в 5 раз.
\end{itemize}

Всё это можно объяснить наличием в библиотеке ExoC-19 большого количества регионов с низким покрытием, генетические варианты в которых были отсеяны фильтрацией по глубине.
То есть, фильтрация по глубине является эффективным способом улучшения данных низкого качества.

\section{Предварительные выводы}

Таким образом, из приведённых нами данных можно сделать следующие выводы:

\begin{enumerate}
	\item Пайплайн, созданный нами с учётом актуальных рекомендаций для биоинформационной обработки, позволяет обрабатывать данные Exo-C\hyp{}секвенирования, а также находить в этих данных SNV.
	\item Использование разработанного конвейера биоинформационных инструментов позволило обнаружить около \numprint[\mln]{5.5} генетических вариантов в контрольных данных клеточной линии K562 (что сопоставимо со средним количеством точечных полиморфизмов в геноме человека), из которых наличие \numprint[\thousands]{75} подтвердилось в восьми независимых исследованиях, а \numprint[\mln]{1} "--- в шести независимых исследованиях, не включающих экзомные данные.
	\item Сравнение генетических вариантов, полученных из контрольных образцов и Exo-C\hyp{}библиотек, позволяет утверждать, что метод Exo-C способен детектировать около \numprint[\%]{75--90} SNV, обнаруживаемых другими методами.
\end{enumerate}

\section{План работы}

В следующем семестре мы планируем:

\begin{enumerate}
	\item Произвести анализ генетических вариантов в контрольных и наших образцах по следующим параметрам:
	      \begin{enumerate}
		      \item тип;
		      \item количество альтернативных аллелей;
		      \item распределение в геноме (в том числе с учётом проблемных регионов);
		      \item глубина покрытия;
		      \item зиготность.
	      \end{enumerate}

	\item Произвести анализ данных Exo-C на предмет систематических ошибок поиска генетических вариантов.

	\item Произвести анализ результатов секвенирования Exo-C\hyp{}библиотек у реальных пациентов.

\end{enumerate}

\newpage

\selectlanguage{english}
\setcitestyle{numbers}
\bibliography{Sources}
\newpage

\appendix

\begin{albumtable}{Библиотеки данных секвенирования клеточной линии K562}{appendix:control-libs}
	\begin{tabular}{| l | l | l | l | l | l | r | r | r | r | r | r | r | r | r | r | r |}
		\hline
		\rowcolor{tableheadcolor}
		\textbf{Библиотека}                                                        &
		\textbf{Статья}                                                            &
		\textbf{Репозиторий}                                                       &
		\textbf{Коды доступа}                                                      &
		\textbf{Тип данных}                                                        &
		\textbf{Тип прочтений}                                                     &
		\headerbigrow{ прочтений}{Глубина, прочтений}                              &
		\headerbigrow{Общее число}{Общее число прочтений}                          &
		\headerbigrow{Доля картированных,}{Доля картированных, \% от общего числа} &
		\headerbigrow{\% от общего числа}{Доля добавочных, \% от общего числа}     &
		\headerbigrow{Картированные}{Картированные PE прочтения}                   &
		\headerbigrow{Картированные}{Картированные синглетоны}                     &
		\headerbigrow{PE прочтений}{Дубликаты\newline PE прочтений}                &
		\headerbigrow{синглетонов }{Дубликаты синглетонов}                         &
		\textbf{Доля дубликатов, \%}                                               &
		\headerbigrow{Оценка размера}{Оценка размера библиотеки}
		\\
		\hline
		\multicolumn{16}{| c |}{\headerbigrow{\Large Контрольные данные}{\Large Контрольные данные}}                                                                                                                                                                                                                                                                                                                                                                                                                          \\
		\hline
		GSM1551618\_HIC069                                                         & \citeauthor{Rao_2014}       & GEO    & SRR1658693                                                                                                                              & Hi-C                & PE & \numprint{456757799}  & \numprint{1001169248} & \numprint{96.57} & \numprint{8.755}  & \numprint{424945100} & \numprint{29290805}   & \numprint{17848021} & \numprint{13182626}  & \numprint{5.56}  & \numprint{4916114832}  \\
		GSM1551619\_HIC070                                                         & \citeauthor{Rao_2014}       & GEO    & SRR1658694                                                                                                                              & Hi-C                & PE & \numprint{591854553}  & \numprint{1314487595} & \numprint{98.7}  & \numprint{9.949}  & \numprint{575565379} & \numprint{15452072}   & \numprint{98778796} & \numprint{8811532}   & \numprint{17.69} & \numprint{1478944337}  \\
		GSM1551620\_HIC071                                                         & \citeauthor{Rao_2014}       & GEO    & \parbox[c][3.8em]{\widthof{ENCFF004THU  }}{SRR1658695\newline SRR1658696}                                                               & Hi-C                & PE & \numprint{79905895}   & \numprint{173931529}  & \numprint{98.81} & \numprint{8.118}  & \numprint{77880938}  & \numprint{1975600}    & \numprint{486893}   & \numprint{269138}    & \numprint{0.79}  & \numprint{6202732721}  \\
		GSM1551621\_HIC072                                                         & \citeauthor{Rao_2014}       & GEO    & \parbox[c][3.8em]{\widthof{ENCFF004THU  }}{SRR1658697\newline SRR1658698}                                                               & Hi-C                & PE & \numprint{79578049}   & \numprint{159160116}  & \numprint{98.38} & \numprint{0.003}  & \numprint{77155821}  & \numprint{2265995}    & \numprint{366805}   & \numprint{285395}    & \numprint{0.65}  & \numprint{8088955029}  \\
		GSM1551622\_HIC073                                                         & \citeauthor{Rao_2014}       & GEO    & \parbox[c][3.8em]{\widthof{ENCFF004THU  }}{SRR1658699\newline SRR1658700}                                                               & Hi-C                & PE & \numprint{77353816}   & \numprint{154710364}  & \numprint{98.33} & \numprint{0.002}  & \numprint{74866287}  & \numprint{2383970}    & \numprint{240304}   & \numprint{293115}    & \numprint{0.51}  & \numprint{11637260975} \\
		GSM1551623\_HIC074                                                         & \citeauthor{Rao_2014}       & GEO    & \parbox[c][3.8em]{\widthof{ENCFF004THU  }}{SRR1658702\newline SRR1658701}                                                               & Hi-C                & PE & \numprint{80778733}   & \numprint{175291763}  & \numprint{98.65} & \numprint{7.835}  & \numprint{78467294}  & \numprint{2254814}    & \numprint{644986}   & \numprint{321965}    & \numprint{1.01}  & \numprint{4746870162}  \\
		ENCSR025GPQ                                                                & \citeauthor{Zhou_2019}      & ENCODE & \parbox[c][5.5em]{\widthof{ENCFF004THU  }}{ENCFF574YLG\newline ENCFF921AXL\newline ENCFF590SSX}                                         & WGS                 & SE & \numprint{258022356}  & \numprint{260044021}  & \numprint{85.39} & \numprint{0.777}  & ---                  & \numprint{220029156}  & ---                 & \numprint{50689083}  & \numprint{23.04} & ---                    \\
		ENCSR053AXS                                                                & \citeauthor{Zhou_2019}      & ENCODE & \parbox[c][8.8em]{\widthof{ENCFF004THU  }}{ENCFF004THU\newline ENCFF066GQD\newline ENCFF313MGL\newline ENCFF506TKC\newline ENCFF080MQF} & WGS                 & SE & \numprint{1472492722} & \numprint{1592540515} & \numprint{91.19} & \numprint{7.538}  & ---                  & \numprint{1332175586} & ---                 & \numprint{496237198} & \numprint{37.25} & ---                    \\
		ENCSR711UNY                                                                & \citeauthor{Zhou_2019}      & ENCODE & \parbox[c][5.5em]{\widthof{ENCFF004THU  }}{ENCFF471WSA\newline ENCFF826SYZ\newline ENCFF590SSX}                                         & WGS                 & SE & \numprint{890796215}  & \numprint{899473769}  & \numprint{99.72} & \numprint{0.965}  & ---                  & \numprint{888239055}  & ---                 & \numprint{203498352} & \numprint{22.91} & ---                    \\
		SRX3358201                                                                 & \citeauthor{Dixon_2018}     & GEO    & SRR6251264                                                                                                                              & WGS                 & PE & \numprint{366291496}  & \numprint{737534099}  & \numprint{99.72} & \numprint{0.671}  & \numprint{364794328} & \numprint{923254}     & \numprint{73018048} & \numprint{406066}    & \numprint{20.05} & \numprint{785091005}   \\
		GSE148362\_G1                                                              & \citeauthor{Wang_2020}      & GEO    & SRR11518301                                                                                                                             & Repli-seq           & SE & \numprint{24804095}   & \numprint{24804396}   & \numprint{96.39} & \numprint{0.001}  & ---                  & \numprint{23909072}   & ---                 & \numprint{921353}    & \numprint{3.85}  & ---                    \\
		GSE148362\_G2                                                              & \citeauthor{Wang_2020}      & GEO    & SRR11518308                                                                                                                             & Repli-seq           & SE & \numprint{33032314}   & \numprint{33033010}   & \numprint{97.61} & \numprint{0.002}  & ---                  & \numprint{32241907}   & ---                 & \numprint{3881991}   & \numprint{12.04} & ---                    \\
		GSE148362\_S1                                                              & \citeauthor{Wang_2020}      & GEO    & SRR11518302                                                                                                                             & Repli-seq           & SE & \numprint{30884788}   & \numprint{30885298}   & \numprint{98.7}  & \numprint{0.002}  & ---                  & \numprint{30481936}   & ---                 & \numprint{2156480}   & \numprint{7.07}  & ---                    \\
		GSE148362\_S2                                                              & \citeauthor{Wang_2020}      & GEO    & SRR11518303                                                                                                                             & Repli-seq           & SE & \numprint{45359273}   & \numprint{45360305}   & \numprint{98.39} & \numprint{0.002}  & ---                  & \numprint{44630884}   & ---                 & \numprint{1939846}   & \numprint{4.35}  & ---                    \\
		GSE148362\_S3                                                              & \citeauthor{Wang_2020}      & GEO    & SRR11518304                                                                                                                             & Repli-seq           & SE & \numprint{49807076}   & \numprint{49807988}   & \numprint{98.79} & \numprint{0.002}  & ---                  & \numprint{49205535}   & ---                 & \numprint{2889464}   & \numprint{5.87}  & ---                    \\
		GSE148362\_S4                                                              & \citeauthor{Wang_2020}      & GEO    & SRR11518305                                                                                                                             & Repli-seq           & SE & \numprint{44149029}   & \numprint{44149770}   & \numprint{98.46} & \numprint{0.002}  & ---                  & \numprint{43469002}   & ---                 & \numprint{2678091}   & \numprint{6.16}  & ---                    \\
		GSE148362\_S5                                                              & \citeauthor{Wang_2020}      & GEO    & SRR11518306                                                                                                                             & Repli-seq           & SE & \numprint{38424060}   & \numprint{38424835}   & \numprint{97.96} & \numprint{0.002}  & ---                  & \numprint{37640056}   & ---                 & \numprint{3600260}   & \numprint{9.57}  & ---                    \\
		GSE148362\_S6                                                              & \citeauthor{Wang_2020}      & GEO    & SRR11518307                                                                                                                             & Repli-seq           & SE & \numprint{35203005}   & \numprint{35203676}   & \numprint{97.51} & \numprint{0.002}  & ---                  & \numprint{34324742}   & ---                 & \numprint{4177438}   & \numprint{12.17} & ---                    \\
		INSITU\_HS1                                                                & \citeauthor{Ray_2019}       & GEO    & SRR9019504                                                                                                                              & Hi-C                & PE & \numprint{86294895}   & \numprint{172589790}  & \numprint{93.3}  & 0                 & \numprint{75521119}  & \numprint{9982274}    & \numprint{1841061}  & \numprint{1615286}   & \numprint{3.29}  & \numprint{1523677153}  \\
		INSITU\_HS2                                                                & \citeauthor{Ray_2019}       & GEO    & SRR9019505                                                                                                                              & Hi-C                & PE & \numprint{127093919}  & \numprint{254187838}  & \numprint{93.36} & 0                 & \numprint{111730240} & \numprint{13858195}   & \numprint{1923146}  & \numprint{3048273}   & \numprint{2.91}  & \numprint{3208280267}  \\
		INSITU\_NHS1                                                               & \citeauthor{Ray_2019}       & GEO    & SRR9019506                                                                                                                              & Hi-C                & PE & \numprint{86445594}   & \numprint{172891188}  & \numprint{93.43} & 0                 & \numprint{75893138}  & \numprint{9737847}    & \numprint{1903981}  & \numprint{1649376}   & \numprint{3.38}  & \numprint{1487154386}  \\
		INSITU\_NHS2                                                               & \citeauthor{Ray_2019}       & GEO    & SRR9019507                                                                                                                              & Hi-C                & PE & \numprint{128472386}  & \numprint{256944772}  & \numprint{93.27} & 0                 & \numprint{112615319} & \numprint{14417076}   & \numprint{1961996}  & \numprint{3196535}   & \numprint{2.97}  & \numprint{3194317878}  \\
		PDDE\_TRANSIENT                                                            & \citeauthor{Moquin_2017}    & GEO    & \parbox[c][3.8em]{\widthof{ENCFF004THU  }}{SRR5470541\newline SRR5470540}                                                               & Hi-C                & PE & \numprint{55158049}   & \numprint{110319638}  & \numprint{95.6}  & \numprint{0.003}  & \numprint{51158920}  & \numprint{3140556}    & \numprint{3917308}  & \numprint{721938}    & \numprint{8.11}  & \numprint{316780447}   \\
		PD\_STABLE\_REP1                                                           & \citeauthor{Moquin_2017}    & GEO    & \parbox[c][3.8em]{\widthof{ENCFF004THU  }}{SRR5470535\newline SRR5470534}                                                               & Hi-C                & PE & \numprint{67172619}   & \numprint{134347099}  & \numprint{97.58} & \numprint{0.001}  & \numprint{64767511}  & \numprint{1565427}    & \numprint{5573966}  & \numprint{376260}    & \numprint{8.79}  & \numprint{354373851}   \\
		PD\_STABLE\_REP2                                                           & \citeauthor{Moquin_2017}    & GEO    & \parbox[c][3.8em]{\widthof{ENCFF004THU  }}{SRR5470536\newline SRR5470537}                                                               & Hi-C                & PE & \numprint{52872167}   & \numprint{105745908}  & \numprint{98.23} & \numprint{0.001}  & \numprint{51442087}  & \numprint{993483}     & \numprint{2058449}  & \numprint{217598}    & \numprint{4.17}  & \numprint{625522723}   \\
		PD\_TRANSIENT                                                              & \citeauthor{Moquin_2017}    & GEO    & \parbox[c][3.8em]{\widthof{ENCFF004THU  }}{SRR5470539\newline SRR5470538}                                                               & Hi-C                & PE & \numprint{81297824}   & \numprint{162600928}  & \numprint{95.28} & \numprint{0.003}  & \numprint{75141163}  & \numprint{4639787}    & \numprint{7298377}  & \numprint{1339404}   & \numprint{10.29} & \numprint{361336652}   \\
		GSM2588815\_R1                                                             & \citeauthor{Belaghzal_2017} & GEO    & SRR5479813                                                                                                                              & Hi-C                & PE & \numprint{72914268}   & \numprint{172533452}  & \numprint{99.39} & \numprint{15.478} & \numprint{72067575}  & \numprint{648294}     & \numprint{9694590}  & \numprint{210273}    & \numprint{13.54} & \numprint{243264112}   \\
		GSM2536769\_WT                                                             & \citeauthor{Banaszak_2018}  & GEO    & SRR5345331                                                                                                                              & WES\footnotemark[1] & PE & \numprint{39211303}   & \numprint{78464649}   & \numprint{99.46} & \numprint{0.054}  & \numprint{38914993}  & \numprint{171253}     & \numprint{7821960}  & \numprint{91145}     & \numprint{20.17} & \numprint{83342746}    \\
		\footnotetext[1]{Варианты в гене DNMT3A были исключены из выборки.}
		GSM2536770\_WT\_TF                                                         & \citeauthor{Banaszak_2018}  & GEO    & SRR5345332                                                                                                                              & WES\footnotemark[1] & PE & \numprint{49394206}   & \numprint{98820633}   & \numprint{99.54} & \numprint{0.033}  & \numprint{49068605}  & \numprint{193565}     & \numprint{10478814} & \numprint{114795}    & \numprint{21.43} & \numprint{97869629}    \\
		GSM2536771\_MT2                                                            & \citeauthor{Banaszak_2018}  & GEO    & SRR5345333                                                                                                                              & WES\footnotemark[1] & PE & \numprint{42020936}   & \numprint{84093776}   & \numprint{99.63} & \numprint{0.062}  & \numprint{41772436}  & \numprint{189177}     & \numprint{8755216}  & \numprint{104927}    & \numprint{21.04} & \numprint{85177326}    \\
		GSM2536772\_MT3                                                            & \citeauthor{Banaszak_2018}  & GEO    & SRR5345334                                                                                                                              & WES\footnotemark[1] & PE & \numprint{43669613}   & \numprint{87375385}   & \numprint{99.6}  & \numprint{0.041}  & \numprint{43414109}  & \numprint{164448}     & \numprint{9489133}  & \numprint{93601}     & \numprint{21.92} & \numprint{84242110}    \\
		GSM2536773\_MT4                                                            & \citeauthor{Banaszak_2018}  & GEO    & SRR5345335                                                                                                                              & WES\footnotemark[1] & PE & \numprint{39879263}   & \numprint{79788847}   & \numprint{99.53} & \numprint{0.038}  & \numprint{39609943}  & \numprint{166651}     & \numprint{8590165}  & \numprint{90809}     & \numprint{21.76} & \numprint{77577055}    \\
		GSM2536774\_MT5                                                            & \citeauthor{Banaszak_2018}  & GEO    & SRR5345336                                                                                                                              & WES\footnotemark[1] & PE & \numprint{40807904}   & \numprint{81649292}   & \numprint{99.59} & \numprint{0.041}  & \numprint{40559969}  & \numprint{163957}     & \numprint{8801283}  & \numprint{91545}     & \numprint{21.77} & \numprint{79383290}    \\
		\hline
		\multicolumn{16}{| c |}{\headerbigrow{\Large Тестовые данные}{\Large Тестовые данные}}                                                                                                                                                                                                                                                                                                                                                                                                                                \\
		\hline
		FG\_ExoCBel-001                                                            & ExoC-19                     & ---    & ---                                                                                                                                     & Exo-C               & PE & \numprint{136609179}  & \numprint{359215777}  & \numprint{99.31} & \numprint{23.940} & \numprint{135150334} & \numprint{443409}     & \numprint{25453568} & \numprint{159152}    & \numprint{18.86} & \numprint{319784450}   \\
		FG\_Quarantine-A                                                           & ExoC-20                     & ---    & ---                                                                                                                                     & Exo-C               & PE & \numprint{53598130}   & \numprint{140214460}  & \numprint{99.79} & \numprint{23.150} & \numprint{53598130}  & \numprint{259561}     & \numprint{7809282}  & \numprint{68779}     & \numprint{14.60} & \numprint{193853459}   \\
		FG\_Quarantine-B                                                           & ExoC-20                     & ---    & ---                                                                                                                                     & Exo-C               & PE & \numprint{55279173}   & \numprint{144641130}  & \numprint{99.76} & \numprint{23.108} & \numprint{55279173}  & \numprint{310369}     & \numprint{8808307}  & \numprint{90489}     & \numprint{15.97} & \numprint{177375163}   \\
		\hline
	\end{tabular}
\end{albumtable}

\begin{albumtable}{Образцы данных секвенирования клеточной линии K562}{appendix:control-samples}
	\begin{tabular}{| l | l | l | r | r | r | r | r | r | r | r | r | r |}
		\hline
		\rowcolor{tableheadcolor}
		\textbf{Образец}                                                                                              &
		\textbf{Тип данных}                                                                                           &
		\textbf{Тип прочтений}                                                                                        &
		\headerbigrow{ прочтений}{Глубина, прочтений}                                                                 &
		\headerbigrow{Общее число}{Общее число прочтений}                                                             &
		\headerbigrow{Доля картированных,}{Доля картированных, \% от общего числа}                                    &
		\headerbigrow{\% от общего числа}{Доля добавочных, \% от общего числа}                                        &
		\headerbigrow{\% от картированных}{FR PE прочтения,\newline \% от картированных}                              &
		\headerbigrow{Картированные}{Картированные PE прочтения}                                                      &
		\headerbigrow{Картированные}{Картированные синглетоны}                                                        &
		\headerbigrow{Картированные на разные хромосомы}{Картированные на разные хромосомы пары, \% от картированных} &
		\headerbigrow{Картированные на разные хромосомы пары (QMAP 4+),}{Картированные на разные хромосомы пары (QMAP 4+), \% от картированных на разные хромосомы}
		\\
		\hline
		\multicolumn{12}{| c |}{\headerbigrow{\Large Контрольные данные}{\Large Контрольные данные}}                                                                                                                                                                                                                                 \\
		\hline
		\citeauthor{Rao_2014}                                                                                         & Hi-C      & PE & \numprint{1366228845} & \numprint{2978750615} & \numprint{97.95} & \numprint{8.268}  & \numprint{27.04} & \numprint{2617761638} & \numprint{53623256} & \numprint{21.03} & \numprint{84.23} \\
		\citeauthor{Zhou_2019}                                                                                        & WGS       & SE & \numprint{2621311293} & \numprint{2752058305} & \numprint{93.43} & \numprint{4.751}  & ---              & ---                   & ---                 & ---              & ---              \\
		\citeauthor{Dixon_2018}                                                                                       & WGS       & PE & \numprint{366291496}  & \numprint{737534099}  & \numprint{99.72} & \numprint{0.671}  & \numprint{97.16} & \numprint{729588656}  & \numprint{923254}   & \numprint{1.25}  & \numprint{51.22} \\
		\citeauthor{Wang_2020}                                                                                        & Repli-seq & SE & \numprint{301663640}  & \numprint{301669278}  & \numprint{98.09} & \numprint{0.002}  & ---              & ---                   & ---                 & ---              & ---              \\
		\citeauthor{Ray_2019}                                                                                         & Hi-C      & PE & \numprint{428306794}  & \numprint{856613588}  & \numprint{93.33} & 0                 & \numprint{35.92} & \numprint{751519632}  & \numprint{47995392} & \numprint{22.77} & \numprint{76.00} \\
		\citeauthor{Moquin_2017}                                                                                      & Hi-C      & PE & \numprint{256500659}  & \numprint{513013573}  & \numprint{96.56} & \numprint{0.002}  & \numprint{46.64} & \numprint{485019362}  & \numprint{10339253} & \numprint{17.76} & \numprint{75.56} \\
		\citeauthor{Belaghzal_2017}                                                                                   & Hi-C      & PE & \numprint{72914268}   & \numprint{172533452}  & \numprint{99.39} & \numprint{15.478} & \numprint{24.77} & \numprint{144135150}  & \numprint{648294}   & \numprint{34.02} & \numprint{88.02} \\
		\citeauthor{Banaszak_2018}                                                                                    & WES       & PE & \numprint{254983225}  & \numprint{510192582}  & \numprint{99.56} & \numprint{0.044}  & \numprint{99.41} & \numprint{506680110}  & \numprint{1049051}  & \numprint{0.11}  & \numprint{81.38} \\
		\hline
		\rowcolor{white}
		\multicolumn{12}{| c |}{\headerbigrow{\Large Тестовые данные}{\Large Тестовые данные}}                                                                                                                                                                                                                                       \\
		\hline
		ExoC-19                                                                                                       & Exo-C     & PE & \numprint{136609179}  & \numprint{359215777}  & \numprint{99.31} & \numprint{23.94}  & \numprint{89.22} & \numprint{270300668}  & \numprint{443409}   & \numprint{5.01}  & \numprint{66.93} \\
		ExoC-20                                                                                                       & Exo-C     & PE & \numprint{109486529}  & \numprint{284855590}  & \numprint{99.77} & \numprint{23.13}  & \numprint{70.02} & \numprint{217754606}  & \numprint{569930}   & \numprint{5.87}  & \numprint{78.00} \\
		\hline
	\end{tabular}
\end{albumtable}

\end{document}
